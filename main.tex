%Este trabalho está licenciado sob a Licença Creative Commons Atribuição-CompartilhaIgual 3.0 Não Adaptada. Para ver uma cópia desta licença, visite http://creativecommons.org/licenses/by-sa/3.0/ ou envie uma carta para Creative Commons, PO Box 1866, Mountain View, CA 94042, USA.

\documentclass[12pt,a4paper]{book}

\input preamble.tex

\begin{document}

\frontmatter

%titlepage
\title{Cálculo Numérico\\\small{Um Livro Colaborativo}}
\author{}
\date{\today}

\AddToShipoutPicture*{\BackgroundPic}
\maketitle


\chapter*{Autores}
\addcontentsline{toc}{chapter}{Autores}

Lista alfabética de autores:
\begin{itemize}
\item[] Dagoberto Adriano Rizzotto Justo - UFRGS
\item[] Esequia Sauter - UFRGS
\item[] Fabio Souto de Azevedo - UFRGS
\item[] Pedro Henrique de Almeida Konzen - UFRGS
\end{itemize}

\chapter*{Licença}
\addcontentsline{toc}{chapter}{Licença}

Este trabalho está licenciado sob a Licença Creative Commons Atribuição-CompartilhaIgual 3.0 Não Adaptada. Para ver uma cópia desta licença, visite http://creativecommons.org/licenses/by-sa/3.0/ ou envie uma carta para Creative Commons, PO Box 1866, Mountain View, CA 94042, USA.
 
\chapter*{Nota dos autores}
\addcontentsline{toc}{chapter}{Nota dos autores}

Este livro vem sendo construído de forma colaborativa desde 2011. Nosso intuito é de melhorá-lo, expandi-lo e adaptá-lo às necessidades de um curso de cálculo numérico em nível de graduação.

Caso queira colaborar, encontrou erros, tem sugestões ou reclamações, entre em contato conosco pelo endereço de e-mail:
\begin{center}
\url{livro_colaborativo@googlegroups.com}  
\end{center}
Alternativamente, abra um chamado no repositório GitHub do projeto:
\begin{center}
\url{https://github.com/livroscolaborativos/CalculoNumerico}  
\end{center}


\chapter*{Prefácio}
\addcontentsline{toc}{chapter}{Prefácio}

Este livro busca abordar os tópicos de um curso de introdução ao cálculo numérico moderno oferecido a estudantes de matemática, física, engenharias e outros. A ênfase é colocada na formulação de resolução de problemas, implementação em computador e interpretação de resultados. Pressupõe-se que o estudante domine conhecimentos e habilidades típicas desenvolvidas em cursos de graduação de cálculo, álgebra linear e equações diferenciais. Conhecimentos prévios em linguagem de computadores é fortemente recomendável, embora apenas técnicas elementares de programação sejam realmente necessárias.

\ifisscilab
Ao longo do livro, fazemos ênfase na utilização do \emph{software} livre \verb+Scilab+ para a implementação dos métodos numéricos abordados. Assim, recomendamos que o leitor tenha a sua disposição um computador com o \verb+Scilab+ instalado. Não é necessário estar familiarizado com a linguagem \verb+Scilab+, mas recomendamos a leitura do Apêndice~\ref{cap:scilab}, no qual apresentamos uma rápida introdução a este pacote computacional. Alternativamente, existem algumas soluções em nuvem que fornecem acesso ao Scilab via internet. Por exemplo, a plataforma virtual rollApp (\url{https://www.rollapp.com/app/scilab}).
\fi

\tableofcontents

\mainmatter

\subfile{./cap_intro/cap_intro.tex}
\subfile{./cap_aritmetica/cap_aritmetica.tex}
\subfile{./cap_equacao1d/cap_equacao1d.tex}
\subfile{./cap_linsis/cap_linsis.tex}
\subfile{./cap_nlinsis/cap_nlinsis.tex}
\subfile{./cap_aproxfun/cap_aproxfun.tex}
\subfile{./cap_derint/cap_derint.tex}
\subfile{./cap_pvi/cap_pvi.tex}

\ifisscilab
\appendix
\subfile{./cap_scilab/cap_scilab.tex}
\fi

%resposta dos exercícios
\chapter*{Resposta dos Exercícios}
\addcontentsline{toc}{chapter}{Respostas dos Exercícios}
Recomendamos ao leitor o uso criterioso das respostadas aqui apresentadas. Devido a ainda muito constante atualização do livro, as respostas podem conter imprecisões e errors.
\shipoutAnswer  

%references
\nocite{*}
\bibliographystyle{plain}
\begingroup
\chapter*{Referências Bibliográficas}
\renewcommand{\chapter}[2]{}
\addcontentsline{toc}{chapter}{Referências Bibliográficas}
\bibliography{main}
\endgroup

\clearpage
\addcontentsline{toc}{chapter}{\'{I}ndice Remessivo}
\printindex

\end{document}