%Este trabalho está licenciado sob a Licença Creative Commons Atribuição-CompartilhaIgual 3.0 Não Adaptada. Para ver uma cópia desta licença, visite http://creativecommons.org/licenses/by-sa/3.0/ ou envie uma carta para Creative Commons, PO Box 1866, Mountain View, CA 94042, USA.

\chapter{Rápida Introdução à Python}\label{cap:python}\index{Python}

Neste apêndice, discutiremos os principais aspectos da linguagem computacional \verb+Python+ que são essenciais para uma boa leitura desta versão do livro. O material aqui apresentado, é uma adaptação livre do Apêndice A de \cite{CALSCI}.

\section{Sobre a linguagem Python}\index{Python!sobre}

\href{https://www.python.org/}{Python} é uma linguagem de programação de alto nível, interpretada e multi-paradigma. Lançada por \href{https://gvanrossum.github.io//}{Guido van Rossum}\footnote{Guido van Rossum, nascido em 1956, programador de computadores dos Países Baixos.} em 1991 é, atualmente, mantida de forma colaborativa e aberta.

Para mais informações, consulte:
\begin{itemize}
\item Página oficial da linguagem Python: \url{https://www.python.org/}
\item Comunidade Python Brasil: \url{http://wiki.python.org.br/}
\end{itemize}

Para iniciantes, recomendamos o curso EAD gratuíto no site \href{https://www.codecademy.com/}{Codecademy}:
\begin{center}
  \url{https://www.codecademy.com/learn/python}
\end{center}

\subsection{Instalação e Execução}\index{Python!instalação e execução}

Para executar um código \verb+Python+ é necessário ter instalado um interpretador para a linguagem. No \href{https://www.python.org/}{site oficial do Python} estão disponíveis para {\it download} os interpretadores \verb+Python 2.7+ e \verb+Python 3+ para vários sistemas operacionais, como \verb+Linux+, \verb+Mac OS+ e \verb+Windows+. Muitas distribuições de \verb+Linux+ (\verb+Linux Mint+, \verb+Ubuntu+, etc.) têm o \verb+Python+ no seu sistema de pacotes (incluindo documentação em várias línguas).

Ao longo do texto, assumiremos que o leitor estaja usando um computar rodando \verb+Linux+. Para outros sistemas, pode ser necessário fazer algumas adaptações.

\subsection{Usando Python}\index{Python!usando}

O uso do Python pode ser feito de três formas básicas:
\begin{itemize}
\item usando um {\bf console Python} de modo iterativo;
\item executando um código \verb+codigo.py+ no console \verb+Python+;
\item executando um código \verb+Python+ \verb+codigo.py+ diretamente em terminal;
\end{itemize}

\begin{ex}
  Considere o seguinte pseudocódigo:
\begin{verbatim}
s = "Olá Mundo!". (Sem imprimir na tela o resultado.)
saída(s). (Imprime na tela.)
\end{verbatim}
Implemente este pseudocódigo em \verb+Python+: a) usando diretamente um console; b) digitando seu código em um arquivo separado e executando-o no console Python com a função \verb+execfile+.b) digitando seu código em um arquivo separado e executando-o em terminal com o comando \verb+python+.
\end{ex}
\begin{sol} Seguem as soluções de cada item:
  \begin{itemize}
  \item[a)]  No console temos:
\begin{verbatim}
>>> s = "Olá, Mundo!"
>>> print(s)
Olá, Mundo!
\end{verbatim}
Para sair do console, digite:
\begin{verbatim}
>>> quit()
\end{verbatim}
  \item[b)] Abra o editor de texto de sua preferência e digite o código:
\begin{verbatim}
#!/usr/bin/env python
# -*- coding: utf-8 -*-

s = 'Olá'
print(s)
\end{verbatim}
Salve o arquivo como, por exemplo, \verb+ola.py+. No console \verb+Python+, digite:
\begin{verbatim}
>>> execfile("ola.py")
\end{verbatim}
  \item[c)] Abra o editor de texto de sua preferência e digite o código:
\begin{verbatim}
#!/usr/bin/env python
# -*- coding: utf-8 -*-

s = 'Olá'
print(s)
\end{verbatim}
Salve o arquivo como, por exemplo, \verb+ola.py+. Em um terminal, digite:
\begin{verbatim}
$ python ola.py
\end{verbatim}
\end{itemize}
\end{sol}

\section{Elementos da linguagem}\index{Python!elementos da linguagem}

\verb+Python+ é uma linguagem de alto nível, interpretada e dinâmica. Uma variável é criada quando um valor é atribuído a ela. Por exemplo:
\begin{verbatim}
>>> x=1
>>> y = x * 2.0
\end{verbatim}
a variável \verb+x+ recebe o valor \verb+int+ $1$ e, logo após, na segunda linha de comando, a variável \verb+y+ recebe o valor \verb+double+ $2$. Observamos que o símbolo \verb+=+ significa o operador de atribuição não o de igualdade. O operador lógico de igualdade no \verb+Python+ é \verb+==+. Veja os seguintes comandos:
\begin{verbatim}
>>> print(x,y)
(1, 2.0)
>>> type(x), type(y)
(<type 'int'>, <type 'float'>)
\end{verbatim}

Comentários e continuação de linha de comando são usados como no seguinte exemplo:
\begin{verbatim}
>>> #isto é um comentário
... 
>>> x = 1 \
... + 2
>>> print(x)
3
\end{verbatim}

\subsection{Operações matemáticas elementares}\index{Python!operações matemáticas}

Em \verb+Python+, os operadores matemáticos elementares são os seguintes:
\begin{verbatim}
  + adição
  - subtração
  * multiplicação
  / divisão
  ** potenciação
\end{verbatim}

Atenção, a operação de divisão se comporta diferente nas versões \verb+Python 2.7+ e \verb+Python 3+. Em \verb+Python 3+, temos:
\begin{verbatim}
>>> 1/2
0.5
\end{verbatim}
Já, em \verb+Python 2.7+:
\begin{verbatim}
>>> 1/2
0
>>> from __future__ import division
>>> 1/2
0.5
\end{verbatim}

\subsection{Funções e constantes elementares}\index{Python!funções e constantes (math)}

Várias funções e constantes elementares estão disponíveis no pacote módulo Python \href{https://docs.python.org/3/library/math.html?highlight=math#module-math}{math}. Por exemplo:
\begin{verbatim}
>>> import math as math
>>> math.cos(pi)
-1.0
>>> math.exp(1)
2.718281828459045
>>> math.log(math.exp(1))
1.0
\end{verbatim}

Observamos que \verb+math.log+ é a função logarítmo natural, isto é, $f(x) = \ln(x)$, enquanto que a implementação Python de $f(x) = \log(x)$ é:
\begin{verbatim}
>>> math.log10(10)
1.0
\end{verbatim}
Veja mais na documentação do \href{https://docs.python.org/3/library/math.html?highlight=math#module-math}{módulo math}:
\begin{verbatim}
>>> help(math)
\end{verbatim}

\subsection{Operadores lógicos}\index{Python!operadores lógicos}

Em \verb+Python+, o valor lógico verdadeiro é escrito como \verb+True+ e o valor lógico falso como \verb+False+. Temos os seguintes operadores lógicos disponíveis:
\begin{verbatim}
and  e lógico
or  ou lógico
not  negação
== igualdade
!= diferente
<  menor que
>  maior que
<= menor ou igual que
>= maior ou igual que
\end{verbatim}

\begin{ex}
  Se $x=2$, então $x$ é maior ou igual a 1 e menor que 3? 
\end{ex}
\begin{sol}
  Em \verb+Python+, temos:
\begin{verbatim}
>>> x=2
>>> (x >= 1) and (x < 3)
True
\end{verbatim}
\end{sol}

\section{Matrizes}\index{Python!matrizes (numpy)}

Em \verb+Python+, temos um ótimo suporte para computação científica com o pacote \href{http://www.numpy.org/}{numpy}. Uma matriz $A = [a_{i,j}]_{i,j=1}^{m,n}$ em Python é definida usando-se a seguinte sintaxe:
\begin{verbatim}
>>> import numpy as np
>>> A = np.array([[ a11 , a12 , ... , a1n], [...]. [am1 , am2 , ... , amn]])
\end{verbatim}

\begin{ex}
  Defina a matriz:
  \begin{equation*}
    A = \left[
      \begin{array}{ccc}
        1 & 2 & 3\\
        4 & 5 & 6
      \end{array}
\right]
  \end{equation*}
\end{ex}
\begin{sol}
  Em \verb+Python+, digitamos:
\begin{verbatim}
>>> import numpy as np
>>> A = np.array([[1,2,3],
...               [4,5,6]])
>>> print(A)
[[1 2 3]
 [4 5 6]]
\end{verbatim}
\end{sol}

A seguinte lista contém uma série de funções que geram matrizes particulares:
\begin{verbatim}
numpy.eye      matrix identidade
numpy.linspace vetor de elementos linearmente espaçados
numpy.ones     matriz cheia de uns
numpy.zeros    matriz nula
\end{verbatim}

\subsection{Obtendo dados de uma matriz}

A função \verb+numpy.shape+ retorna o tamanho de uma matriz, por exemplo:
\begin{verbatim}
>>> A = np.ones((3,2))
>>> print(A)
[[ 1.  1.]
 [ 1.  1.]
 [ 1.  1.]]
>>> nl, nc = np.shape(A)
>>> print(nl,nc)
(3, 2)
\end{verbatim}
informando que a matriz \verb+A+ tem três linhas e duas colunas.

Existem vários métodos para se acessar os elementos de uma matriz dada \verb+A+:
\begin{itemize}
\item a matriz inteira acessa-se com a sintaxe:
\begin{verbatim}
A
\end{verbatim}
\item o elemento da $i$-ésima linha e $j$-ésima coluna acessa-se usando a sintaxe:
\begin{verbatim}
A[i,j]
\end{verbatim}
\item o bloco formado pelas linhas $i_1$, $i_2$ e pelas colunas $j_1$, $j_2$ obtém-se usando a sintaxe:
\begin{verbatim}
A[i1:i2, j1:j2]
\end{verbatim}
\end{itemize}

\begin{ex}
  Veja as seguintes linhas de comando:
\begin{verbatim}
>>> from numpy import random
>>> A = np.random.random((3,4))
>>> A
array([[ 0.39235668,  0.30287204,  0.24379253,  0.98866709],
       [ 0.72049734,  0.99300252,  0.14232844,  0.25604346],
       [ 0.61553036,  0.80615392,  0.22418474,  0.13685148]])
>>> A[2,3]
0.13685147547025989
>>> A[1:3,1:4]
array([[ 0.99300252,  0.14232844,  0.25604346],
       [ 0.80615392,  0.22418474,  0.13685148]])
\end{verbatim}
\end{ex}

Definida uma matriz $A$ em Python, as seguintes sintaxes são bastante úteis:
\begin{verbatim}
A[:,:]   toda a matriz
A[i:j,k] os elementos das linhas i até j (exclusive) da k-ésima coluna
A[i,j:k] os elementos da i-ésina linha das colunas j até k (exclusive)
A[i,:]   a i-ésima linha da matriz
A[:,j]   a j-ésima coluna da matriz
\end{verbatim}

Atenção, os índices em \verb+Python+ iniciam-se em $0$. Assim, o comando \verb+A[1:3,1:4]+ retorna o bloco da matriz $A$ compreendido da segunda à terceira linha e da segunda a quarta coluna desta matriz. 

\begin{ex}
Veja as seguintes linhas de comando:
\begin{verbatim}
>>> B = np.random.random((4,4))
>>> B
array([[ 0.94313432,  0.72650883,  0.55487089,  0.18753526],
       [ 0.02094937,  0.45726099,  0.51925464,  0.8535878 ],
       [ 0.75948469,  0.95362926,  0.77942318,  0.06464183],
       [ 0.91243198,  0.22775889,  0.04061536,  0.14908227]])
>>> aux = np.copy(B[:,2])
>>> B[:,2] = np.copy(B[:,3])
>>> B[:,3] = np.copy(aux)
>>> B
array([[ 0.94313432,  0.72650883,  0.18753526,  0.55487089],
       [ 0.02094937,  0.45726099,  0.8535878 ,  0.51925464],
       [ 0.75948469,  0.95362926,  0.06464183,  0.77942318],
       [ 0.91243198,  0.22775889,  0.14908227,  0.04061536]])
\end{verbatim}
\end{ex}

\subsection{Operações matriciais e elemento-a-elemento}

Em \verb+Python+ com \verb+numpy+, o operador \verb+*+ opera elemento a elemento. Por exemplo:
\begin{verbatim}
>>> A = np.array([[1,2],[2,1]]); print(A)
[[1 2]
 [2 1]]
>>> B = np.array([[2,1],[2,1]]); print(B)
[[2 1]
 [2 1]]
>>> print(A*B)
[[2 2]
 [4 1]]
\end{verbatim}
A multiplicação matricial obtemos com:
\begin{verbatim}
>>> C = A.dot(B)
>>> print(C)
[[6 3]
 [6 3]]
\end{verbatim}

Aqui, temos as sintaxes análogas entre operações elemento-a-elemento:
\begin{verbatim}
+ adição       
- subtração    
* multiplicação 
/ divisão
** potenciação
\end{verbatim}

\begin{ex}
  Veja as seguintes linhas de comando:
\begin{verbatim}
>>> A = np.ones((2,2))
>>> A
array([[ 1.,  1.],
       [ 1.,  1.]])
>>> B = 2 * np.ones((2,2))
>>> B
array([[ 2.,  2.],
       [ 2.,  2.]])
>>> A*B
array([[ 2.,  2.],
       [ 2.,  2.]])
>>> A.dot(B)
array([[ 4.,  4.],
       [ 4.,  4.]])
>>> A/B
array([[ 0.5,  0.5],
       [ 0.5,  0.5]])
\end{verbatim}
\end{ex}

\section{Estruturas de ramificação e repetição}\index{Python!ramificação e repetição}

A linguagem \verb+Python+ contém estruturas de repetição e ramificação padrões de linguagens estruturadas.

\subsection{A instrução de ramificação ``if''}\index{Python!if}

A instrução ``if'' permite executar um pedaço do código somente se uma dada condição for satisfeita.

\begin{ex}
  Veja o seguinte código Scilab:
\begin{verbatim}
#!/usr/bin/env python
# -*- coding: utf-8 -*-

i = 2
if (i == 1):
    print("Olá!")
elif (i == 2):
    print("Hallo!")
elif (i == 3):
    print("Hello!")
else:
    print("Ça Va!")
\end{verbatim}
Qual é a saída apresentada pelo código? Porquê?
\end{ex}

Observamos que, em \verb+Python+, a identação é obrigatória, pois é ela que defini o escopo da instrução.

\subsection{A instrução de repetição ``for''}\index{Python!for}

A instrução \verb+for+ permite que um pedaço de código seja executado repetidamente.

\begin{ex}
  Veja o seguinte código:
\begin{verbatim}
for i in range(6):
    print(i)
\end{verbatim}
Qual é a saída deste código? Porquê?
\end{ex}

\begin{ex}
  Veja o seguinte código:
\begin{verbatim}
import numpy as np
for i in np.arange(1,8,2):
    print(i)
\end{verbatim}
Qual é a saída deste código? Porquê?
\end{ex}

\begin{ex}
  Veja o seguinte código:
\begin{verbatim}
for k = 10:-3:1
    disp(k)
end
\end{verbatim}
O que é mostrado no console do Scilab?
\end{ex}

\begin{ex}
  Veja o seguinte código:
\begin{verbatim}
import numpy as np
for i in np.arange(10,1,-3):
    print(i)
\end{verbatim}
O que é mostrado no console do Scilab?
\end{ex}

\subsection{A instrução de repetição ``while''}\index{Python!while}

A instrução \verb+while+ permite que um pedaço de código seja executado repetidamente até que uma dada condição seja satisfeita.

\begin{ex}
Veja o seguinte código Scilab:
\begin{verbatim}
s = 0
i = 1
while (i <= 10):
    s = s + i
    i = i + 1
\end{verbatim}
Qual é o valor de \verb+s+ ao final da execução? Porquê?
\end{ex}

\section{Funções}\index{Python!funções}

Além das muitas funções disponíveis em \verb+Python+ (e os tantos muitos pacotes livers disponíveis), podemos definir nossas próprias funções. Para tanto, existe a instrução \verb+def+. Veja os seguintes exemplos:

\begin{ex}
  O seguinte código:
\begin{verbatim}
def f(x): 
    return x + np.sin(x)
\end{verbatim}
define a função $f(x) = x + \sen x$.

Observe que $f(\pi) = \pi$. Confirme isso computando:
\begin{verbatim}
>>> f(np.pi)
\end{verbatim}
\end{ex}

\begin{ex}
  O seguinte código em \verb+Python+:
\begin{verbatim}
def h(x,y):
    if (x < y):
        return y - x
    else:
        return x - y
\end{verbatim}
define a função:
\begin{equation*}
  h(x,y) = \left\{
    \begin{array}{ll}
      y - x &, x < y\\
      x - y &, x \geq y
    \end{array}
\right.
\end{equation*}
\end{ex}

\begin{ex}
  O seguinte código:
\begin{verbatim}
def J(x):
    y = np.zeros((2,2))
    y[0,0] = 2*x[0]
    y[0,1] = 2*x[1]

    y[1,0] = -x[1]*np.sin(x[0]*x[1])
    y[1,1] = -x[0]*np.sin(x[0]*x[1])

    return y
\end{verbatim}
define a matriz jacobiana $J(x_1,x_2) := \frac{\p(f_1,f_2)}{\p(x_1,x_2)}$ da função:
\begin{equation*}
  \pmb{f}(x_1,x_2) = (x_1^2 + x_2^2,~\cos(x_1x_2)).
\end{equation*}
\end{ex}

\section{Gráficos}\index{Python!gráficos}

Para criar um esboço do gráfico de uma função de uma variável real $y = f(x)$, podemos usar a biblioteca \verb+Python+ \href{http://matplotlib.org/}{mathplotlib}.A função \verb+matplotlib.pyplot.plot+ faz uma representação gráfica de um conjunto de pontos $\{(x_i, y_i)\}$ fornecidos. Existe uma série de opções para esta função de forma que o usuário pode ajustar várias questões de visualização. Veja a \href{http://matplotlib.org/api/pyplot_api.html?highlight=pyplot.plot#matplotlib.pyplot.plot}{documentação}.

\begin{ex}
  Veja as seguintes linhas de código:
\begin{verbatim}
>>> import numpy as np
>>> import matplotlib.pyplot as plt
>>> def f(x): return x**3 + 1
... 
>>> x = np.linspace(-2,2)
>>> plt.plot(x, f(x))
[<matplotlib.lines.Line2D object at 0x7f4f6d153510>]
>>> plt.grid()
>>> plt.show()
\end{verbatim}
\end{ex}