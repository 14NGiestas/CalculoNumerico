%Este trabalho está licenciado sob a Licença Creative Commons Atribuição-CompartilhaIgual 3.0 Não Adaptada. Para ver uma cópia desta licença, visite http://creativecommons.org/licenses/by-sa/3.0/ ou envie uma carta para Creative Commons, PO Box 1866, Mountain View, CA 94042, USA.

%%%%%%%%%%%%%%%%%%%%%%%%%%%%%%%%%%%%%%%%%
% ATENÇÃO
%
% POR SEGURANÇA, NÃO EDITE ESTE ARQUIVO.
%
%%%%%%%%%%%%%%%%%%%%%%%%%%%%%%%%%%%%%%%%

%%%%%%%%%%%%%%%%%%%%%%%%%%%%%%%%%
%   Predefinicoes
%%%%%%%%%%%%%%%%%%%%%%%%%%%%%%%%%

\newif\ifisbook        % O layout será book?
\newif\ifishtml        % O layout será html?
\newif\ifisslide       % O layout será slide?

\newif\ifisscilab      % As notas incluirão scilab? 
\newif\ifisoctave      % As notas incluirão octave?
\newif\ifispython      % As notas incluirão python?

\def\tfn{main.knd}     % Arquivo que guarda as definições do tipo de saída
\def \tdata{}          % Definições do tipo de saída: book, slide ou html.

\openin1=\tfn\relax    % Leitura das definições de saída
\read1 to \tdata
\closein1

\tdata                 % Definições de saída

%%%%%%%%%%%%%%%%%%%%%%%%%%%%%%%%%
%   Opcões de Linguagem
%%%%%%%%%%%%%%%%%%%%%%%%%%%%%%%%%
\usepackage[brazil]{babel}
\usepackage[utf8]{inputenc}
\usepackage[T1]{fontenc}
%\usepackage{xunicode} é o pacote necessário para a codificação UTF-8 no XeTeX

\usepackage{fancyhdr}
\pagestyle{fancy}
\fancyhf{}
\fancyhead[RE]{Cálculo Numérico}
\fancyhead[LO]{\rightmark}
\fancyhead[LE,RO]{\thepage}

%license footnote
\cfoot{\tiny{Licença CC-BY-SA-3.0. Contato: \url{livro_colaborativo@googlegroups.com}}}

%%%% no blank pages between chapters %%%%
\let\cleardoublepage\clearpage

%%%% independent chapters %%%%
\usepackage{subfiles}

%%%% ams-latex %%%%
\usepackage{amsmath}
\usepackage{amssymb}
\usepackage{amsthm}
\usepackage{mathtools}

%%%% graphics %%%%
\usepackage{graphics}
\usepackage{graphicx}
%\usepackage{caption}

%%%% links %%%%
\usepackage[pdfborder={0 0 0 [0 0]},colorlinks=true,linkcolor=blue,citecolor=blue,filecolor=blue,urlcolor=blue]{hyperref}

%%%% copy and paste from PDF (correctly) %%%%
\usepackage{upquote}
\usepackage{lmodern}

%%%% code insert (verbatim) %%%%
\usepackage{verbatim}
\usepackage{listings}

%%%% indent first line %%%%
\usepackage{indentfirst}

%%%% comma as a decimal separator %%%%
\usepackage{icomma}

%%%% citation %%%%
\usepackage{cite}

%%%% lists %%%%
\usepackage{enumerate}

%%%% miscellaneous %%%%
\usepackage{multicol}
\usepackage{multirow}
\usepackage[normalem]{ulem}
\renewcommand{\arraystretch}{1.5} %space between rows in tables
\usepackage{array,booktabs}

%%%%%%%%%%%%%%%%%%%%%%%%%%%%%%%%%
%   Formatacoes de estilo
%%%%%%%%%%%%%%%%%%%%%%%%%%%%%%%%%
\usepackage{xcolor}
\newcommand{\RED}[1]{{\color{red}{#1}}}
\newcommand{\BLU}[1]{{\color{blue}{#1}}}
\newcommand{\GRE}[1]{{\color{darkgreen}{#1}}}

\usepackage{tikz}
\newcommand*\circled[1]{\tikz[baseline=(char.base)]{
            \node[shape=circle,draw,inner sep=1pt] (char) {#1};}}

%emphasis \emph
\DeclareTextFontCommand{\emph}{\bfseries}

\newcommand{\sen}{\operatorname{sen}\,}
\newcommand{\senh}{\operatorname{senh}\,}
\renewcommand{\sin}{\operatorname{sen}\,}
\renewcommand{\sinh}{\operatorname{senh}\,}
\newcommand{\tg}{\operatorname{tg}\,}

\newcommand{\p}{\partial}
\newcommand{\Dom}{\operatorname{Dom}\,}
\newcommand{\diag}{\operatorname{diag}\,}

%%%% indexing %%%%
\usepackage{makeidx}
\makeindex

%%%%% macros  %%%%%%%%%%%%%
\newcommand{\matdd}[4]{\begin{bmatrix} #1&#2\\#3&#4 \end{bmatrix}}
\newcommand{\matddd}[9]{\begin{bmatrix} #1&#2&#3 \\ #4&#5&#6 \\ #7&#8&#9 \end{bmatrix}}
\newcommand{\vetdd}[2]{\begin{bmatrix} #1 \\#2 \end{bmatrix}}
\newcommand{\vetddd}[3]{\begin{bmatrix} #1 \\ #2\\ #3 \end{bmatrix}}
\newcommand{\field}[1]{\mathbb{#1}}

\newcommand{\emconstrucao}{
Em construção ... Gostaria de colaborar na escrita deste livro? 
Veja como em:
\begin{center}
\url{http://www.ufrgs.br/numerico}  
\end{center}}

%E = 10^
\def\E#1{\mathrm{E}\!#1\!}

\ifisslide
\input preambulo_slide.tex
\else
\ifishtml
\input preambulo_html.tex
\else
\input preambulo_book.tex
\fi
\fi
