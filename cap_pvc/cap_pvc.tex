%Este trabalho está licenciado sob a Licença Creative Commons Atribuição-CompartilhaIgual 3.0 Não Adaptada. Para ver uma cópia desta licença, visite http://creativecommons.org/licenses/by-sa/3.0/ ou envie uma carta para Creative Commons, PO Box 1866, Mountain View, CA 94042, USA.

%\documentclass[main.tex]{subfiles}
%\begin{document}

\chapter{Problemas de Valores de Controno}\index{Problemas de valores de contorno}

Neste capítulo, discutimos sobre métodos numéricos para resolver equações diferenciais ordinárias com condições de controno.

%%%%%%%%%%%%%%%%%%%%
% python
\ifispython
Nos códigos \verb+Python+ apresentados, assumimos que as seguintes bibliotecas e módulos estão carregados:
\begin{verbatim}
>>> from __future__ import division
>>> import numpy as np
>>> from numpy import linalg
>>> import matplotlib.pyplot as plt
\end{verbatim}
\fi
%%%%%%%%%%%%%%%%%%%%

\section{Método de Diferenças Finitas}

Nesta seção, discutimos os fundamentos do método de diferenças finitas (MDF) para problemas de valores de contorno. Este método consiste na reformulação do problema contínuo em um problema discreto definido sobre uma malha apropriada.

Para introduzir os conceitos principais, consideramos o seguinte problema de valor de contorno de Dirichlet:
\begin{eqnarray}
    -u_{xx} &=& f(x,u),\quad a < x < b,\label{eq:pvc1}\\
    u(a) &=& u_a,\label{eq:pvc1-bc1}\\
    u(b) &=& u_b,\label{eq:pvc1-bc2}
\end{eqnarray}

Resolver numericamente o problema acima exige uma discretização do domínio $[a,b]$, ou seja, dividir o domínio em $N$ partes iguais, definindo
$$
h=\frac{b-a}{N}
$$
O conjunto de abcissas $x_i$, $i=1,...,N+1$ formam uma malha para o problema discreto. Nosso objetivo é encontrar as ordenadas $u_i=u(x_i)$ que satisfazem a versão discreta:
$$\left\{\begin{array}{l}-\frac{u_{i+1}-2u_i+u_{i-1}}{h^2}=f(x_i,u_i),~~ 2\leq i\leq N.\\
u_1=u_a\\
u_{N+1}=u_b\end{array}
\right.
$$
O vetor solução $(u_i)_{i=1}^{N+1}$ do problema é solução do sistema acima, que é linear se $f$ for linear em $u$ e não linear caso contrário.

\begin{ex}Encontre uma solução numérica para o problema de contorno:
$$\left\{\begin{array}{l}-u_{xx}+u=e^{-x},~~ 0<x<1.\\
u(0)=1\\
u(1)=2\end{array}
\right.
$$
\end{ex}
\begin{sol}
Observe que
$$
h=\frac{1}{N}
$$
e a versão discreta da equação é
$$\left\{\begin{array}{l}-\frac{u_{i+1}-2u_i+u_{i-1}}{h^2}+u_i=e^{-x_i},~~ 2\leq i\leq N.\\
u_1=1\\
u_{N+1}=2\end{array}
\right.
$$
ou seja,
$$\left\{\begin{array}{l}u_1=1\\-u_{i+1}+(2+h^2)u_i-u_{i-1}=h^2e^{-x_i},~~ 2\leq i\leq N.\\
u_{N+1}=2\end{array}
\right.
$$
que é um sistema linear. A sua forma matricial é:
$$
\left[\begin{array}{ccccccc}
1&0&0&\cdots&0&0&0\\
-1&2+h^2&-1&\cdots&0&0&0\\
0&-1&2+h^2&\cdots&0&0&0\\
\vdots&&&&\ddots&&\\
0&0&0&\cdots&-1&2+h^2&-1\\
0&0&0&\cdots&0&0&1\\
\end{array}\right]
\left[\begin{array}{c}u_1\\u_2\\u_3\\ \vdots\\ u_{N}\\u_{N+1}\end{array}\right]=
\left[\begin{array}{c}1\\h^2e^{-x_2}\\h^2e^{-x_3}\\ \vdots\\ h^2e^{-x_N}\\2\end{array}\right]
$$
Para $N=10$, temos a seguinte solução:
$$
\left[\begin{array}{c} 1,000000\\  1,0735083  \\1,1487032 \\ 1,2271979\\  1,3105564\\  1,4003172\\  1,4980159\\  1,6052067\\  1,7234836\\  1,8545022\\2,000000\end{array}\right]
$$  
\end{sol}

\subsection*{Exercícios}

\begin{Exercise}
 Considere o seguinte problema de valor de contorno para a equação de calor no estado estacionário:
$$\left\{\begin{array}{l}-u_{xx}=32,~~ 0<x<1.\\
u(0)=5\\
u(1)=10\end{array}
\right.
$$

Defina $u_j=u(x_j)$ onde $x_j={(j-1)}{h}$ e $j=1,\ldots,5$. Aproxime a derivada segunda por um esquema de segunda ordem e transforme a equação diferencial em um sistema de equações lineares. Escreva este sistema linear na forma matricial e resolva-o. Faça o mesmo com o dobro de subintervalos, isto é, com malha de 9 pontos. 
\end{Exercise}
\begin{Answer}
\begin{tiny} 
 $$\left[
  \begin{array}{ccccc}
         1 & 0& 0& 0& 0\\
         -1 & 2 & -1 &0&0\\
         0&-1 & 2 & -1 &0\\
         0&0&-1 & 2 & -1 \\
         0 & 0& 0& 0& 1\\
        \end{array}
\right]
\left[
  \begin{array}{c}
     u_1\\ u_2\\u_3\\u_4 \\ u_5
   \end{array}
\right]
=
\left[
  \begin{array}{c}
     5\\ 2\\2\\2 \\ 10
   \end{array}
\right]
$$


Solução:  [5, 9.25, 11.5, 11.75, 10]    

$$\left[
  \begin{array}{ccccccccc}
         1 & 0& 0& 0& 0& 0& 0& 0& 0\\
         -1 & 2 & -1 &0&0& 0& 0& 0& 0\\
         0&-1 & 2 & -1 &0& 0& 0& 0& 0\\
         0&0&-1 & 2 & -1 & 0& 0& 0& 0\\
         0&0&0&-1 & 2 & -1 & 0& 0& 0\\
         0&0&0&0&-1 & 2 & -1 & 0& 0\\
         0&0&0&0&0&-1 & 2 & -1 & 0\\
         0&0&0&0&0&0&-1 & 2 & -1\\
         0 & 0& 0& 0& 0& 0& 0& 0& 1\\
        \end{array}
\right]
\left[
  \begin{array}{c}
     u_1\\ u_2\\u_3\\u_4 \\u_5\\ u_6\\u_7\\u_8\\u_9
   \end{array}
\right]
=
\left[
  \begin{array}{c}
     5\\ 0.5\\0.5\\0.5\\ 0.5\\0.5\\0.5\\0.5 \\ 10
   \end{array}
\right]
$$

Solução:  $[5, 7.375, 9.25, 10.625, 11.5, 11.875, 11.75, 1.125, 10]$
\end{tiny}
\end{Answer}


\begin{Exercise} Considere o seguinte problema de valor de contorno para a equação de calor no estado estacionário:
$$\left\{\begin{array}{l}-u_{xx}=200e^{-(x-1)^2},~~ 0<x<2.\\
u(0)=120\\
u(2)=100\end{array}
\right.
$$
Defina $u_j=u(x_j)$ onde $x_j={(j-1)}{h}$ e $j=1,\ldots,21$. Aproxime a derivada segunda por um esquema de segunda ordem e transforme a equação diferencial em um sistema de equações lineares. Resolva o sistema linear obtido.


\end{Exercise}
\begin{Answer}
  \begin{tiny}
120.    133.56    146.22    157.83    168.22    177.21    184.65    190.38    194.28    196.26    196.26    194.26    190.28    184.38    176.65    167.21  156.22    143.83    130.22    115.56    100.    
  \end{tiny}
\end{Answer}



\begin{Exercise} Considere o seguinte problema de valor de contorno para a equação de calor no estado estacionário:
$$\left\{\begin{array}{l}-u_{xx}=200e^{-(x-1)^2},~~ 0<x<2.\\
u'(0)=0\\
u(2)=100\end{array}
\right.
$$
Defina $u_j=u(x_j)$ onde $x_j={(j-1)}{h}$ e $j=1,\ldots,21$. Aproxime a derivada segunda por um esquema de segunda ordem, a derivada primeira na fronteira por um esquema de primeira ordem e transforme a equação diferencial em um sistema de equações lineares. Resolva o sistema linear obtido.
\end{Exercise}

\begin{Answer}
  \begin{tiny}
391.13    391.13    390.24    388.29    385.12    380.56    374.44    366.61    356.95    345.38    331.82    316.27    298.73    279.27    257.99    234.99    210.45    184.5    157.34    129.11    100.    
  \end{tiny}
\end{Answer}


\begin{Exercise} Considere o seguinte problema de valor de contorno para a equação de calor no estado estacionário com um termo não linear de radiação:
$$\left\{\begin{array}{l}-u_{xx}=100- \frac{u^4}{10000},~~ 0<x<2.\\
u(0)=0\\
u(2)=10\end{array}
\right.
$$
Defina $u_j=u(x_j)$ onde $x_j={(j-1)}{h}$ e $j=1,\ldots,21$. Aproxime a derivada segunda por um esquema de segunda ordem e transforme a equação diferencial em um sistema de equações não lineares. Resolva o sistema  obtido. Expresse  a solução com dois algarismos depois do separador decimal. Dica: Veja problema 38 da lista 2, seção de sistemas não lineares.
\end{Exercise}

\begin{Answer}
  \begin{tiny}
0.,    6.57,    12.14,    16.73,    20.4,    23.24,    25.38,    26.93 ,   28,    28.7,    29.06,    29.15,    28.95,    28.46, 27.62 ,   26.36,    24.59,    22.18,    19.02,    14.98,    10.    
  \end{tiny}
\end{Answer}


\begin{Exercise} Considere o seguinte problema de valor de contorno para a equação de calor no estado estacionário com um termo não linear de radiação e um termo de convecção:
$$\left\{\begin{array}{l}-u_{xx}+3u_x=100- \frac{u^4}{10000},~~ 0<x<2.\\
u'(0)=0\\
u(2)=10\end{array}
\right.
$$
Defina $u_j=u(x_j)$ onde $x_j={(j-1)}{h}$ e $j=1,\ldots,21$. Aproxime a derivada segunda por um esquema de segunda ordem, a derivada primeira na fronteira por um esquema de primeira ordem, a derivada primeira no interior por um esquema de segunda ordem e transforme a equação diferencial em um sistema de equações não lineares. Resolva o sistema  obtido.
\end{Exercise}
\begin{Answer}
  \begin{tiny}
u(0)=31.62, u(1)=31.50, u(1.9)=18.17    
  \end{tiny}
\end{Answer}

\begin{Exercise} Considere o seguinte problema de valor de contorno:
$$\left\{\begin{array}{l}-u''+2u'=e^{-x}- \frac{u^2}{100},~~ 1<x<4.\\
u'(1)+u(1)=2\\
u'(4)=-1\end{array}
\right.
$$
Defina $u_j=u(x_j)$ onde $x_j=1+{(j-1)}{h}$ e $j=1,\ldots,101$. Aproxime a derivada segunda por um esquema de segunda ordem, a derivada primeira na fronteira por um esquema de primeira ordem, a derivada primeira no interior por um esquema de segunda ordem e transforme a equação diferencial em um sistema de equações não lineares. Resolva o sistema  obtido.
\end{Exercise}
\begin{Answer}
  \begin{tiny}
u(1)=1.900362, u(2.5)=1.943681, u(4)=1.456517    
  \end{tiny}
\end{Answer}

%\end{document} 