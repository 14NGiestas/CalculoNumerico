%Este está licenciado sob a Licença Creative Commons Atribuição-CompartilhaIgual 3.0 Não Adaptada. Para ver uma cópia desta licença, visite http://creativecommons.org/licenses/bu-sa/3.0/ ou envie uma carta para Creative Commons, PO Box 1866, Mountain View, CA 94042, USA.

%\documentclass[main.tex]{subfiles}
%\begin{document}

\chapter{Problemas de valor inicial}\index{problema de valor inicial}
Neste capítulo, vamos desenvolver técnicas numéricas para aproximar a solução do problema de valor inicial (PVI) dado pela equação diferencial ordinária (EDO) de primeira ordem
\begin{subequations}\label{PVI}
\begin{eqnarray}
u'(t)&=&f(t,u(t))\label{PVI_EDO}\\
u(t_1)&=&a ~~ \hbox{(condição inicial)}.\label{PVI_CI}
\end{eqnarray}
\end{subequations}

A incógnita de um problema de valor inicial é uma função que satisfaz a equação diferencial (\ref{PVI_EDO})  e a condição inicial (\ref{PVI_CI}).

Considere os próximos três exemplos:
\begin{ex}
\begin{eqnarray}
   \frac{du}{dt} &=t\\
            u(0) &= a
\end{eqnarray}
\end{ex}

\begin{ex}
\begin{eqnarray}
   \frac{du}{dt} &=u\\
            u(0) &= a
\end{eqnarray}
\end{ex}

\begin{ex}
\begin{eqnarray}
   \frac{du}{dt} &=&\sin(u^2+\sin(t))\\
            u(0) &=& a
\end{eqnarray}
\end{ex}

A solução do primeiro exemplo é $u(t)=t^2/2+a$ pois satisfaz a equação diferencial e a condição inicial.

A solução do segundo exemplo é fácil de ser obtida: $u(t)=ae^t$. Porém como podemos resolver o terceiro problema?


% 
% \begin{ex}
% Considere o seguinte problema de valor inicial
% \begin{subequations}\label{exemplo_u_2u}
% \begin{eqnarray}
% u'(t)&=&2u(t),\\
% u(0)&=&1.
% \end{eqnarray}
% \end{subequations}
% A solução desta equação é dada pela função $u(t)=e^{2t}$ pois $u'(t)=2e^{2t}=2u(t)$ e $u(0)=e^0=1$.
% \end{ex}


Muitos problemas de valor inicial da forma (\ref{PVI}) não podem ser resolvidos exatamente, ou seja, sabe-se que a solução existe e é única, porém não podemos expressá-la em termos de funções elementares. Por isso é necessário calcular aproximações numéricas. Diversos métodos completamente diferentes estão disponíveis para aproximar uma função real. 

Existem várias maneiras de obter aproximações para a solução deste problema. Nos limitaremos a estudar métodos que aproximam $u(t)$ em um conjunto finito de valores de $t$ chamado \emph{malha} que será denotado por  $\{t_i\}_{i=1}^N=\{t_1, t_2, t_3,\ldots, t_N\}$. Desta forma, aproximamos a solução $u(t_i)$ por $u_i$ em cada ponto da malha usando diferentes esquemas numéricos.


\section{Teoria de equações diferenciais}
Uma questão fundamental é analisar se um dado PVI é um problema \emph{bem posto}. Ou seja,
\begin{itemize}
 \item Existe uma solução para o $PVI$?
 \item A solução é única?
 \item A solução do PVI é pouco sensível a pequenas perturbações nas condições iniciais?
\end{itemize}


\begin{defn}
A função $f(t,u)$ é Lipschitz em $u$ se existe uma constante $L$, tal que $\forall t \in [a,b]$ e $u,v \in \mathbb R$,
$$ |f(t,u)-f(t,v)| \leq L|u(t)-v(t)|. $$
\end{defn}


\begin{teo}
Seja $f(t,u)$ contínua em $t$ e Lipschitz em $u$. Então existe uma única solução para o PVI
\begin{eqnarray}
  u'(t)  &=& f(t,u(t)) \\
  u(t_1) &=& a.
\end{eqnarray}
\end{teo}

\begin{defn}
  \emph{Estabilidade dinâmica} refere-se a propriedade de pequenas perturbações sobre o estado inicial de um sistema gerarem pequenas variações no estado final deste sistema (haverá decaimento nas variações, ou pelo menos não crescimento, quanto $t$ cresce).
\end{defn}

\begin{teo}[Dependência na condição inicial]
Se $u(t)$ e $v(t)$ são soluções do PVI com $f$ Lipschitz com $u(t_1)=u_1$, $v(t_1)=v_1$, então
$$ |u(t)-v(t)| \leq  e^{L(t-t_1)}|u_1-v_1| . $$
\end{teo}

















\section{Método de Euler}\index{método!de Euler}
Considere o PVI dado por
\begin{eqnarray}\label{EDO1}
  u'(t)  &=& f(t,u(t)) \\
  u(t_1) &=& a
\end{eqnarray}
Ao invés de solucionar o problema para qualquer $t>t_1$, (encontrar $u(t)$), iremos aproximar $u(t)$ em $t_2=t_1+h$.

Integrando \eqref{EDO1} de $t_1$ até $t_2$,
\begin{eqnarray}
  \int_{t_1}^{t_2} u'(t) \;dt &=& \int_{t_1}^{t_2} f(t,u(t)) \; dt\\
  u(t_2)-u(t_1)               &=& \int_{t_1}^{t_2} f(t,u(t)) \; dt\\
  u(t_2)                      &=& u(t_1) +  \int _{t_1}^{t_2} f(t,u(t)) \; dt
\end{eqnarray}

Seja $u_n$ a aproximação de $u(t_n)$. Para obter o método numérico mais simples aproximamos $f$ em $[t1,t2]$ pela função constante $f(t,u(t)) \approx  f(t_1,u_1)$,
\begin{eqnarray}
  u_2 &=&  u_1 +   f(t_1,u_1) \int _{t_1}^{t_2}  \; dt \\
  u_2 &=&  u_1 +   f(t_1,u_1) (t_2-t_1) \\
  u_2 &=&  u_1 + h f(t_1,u_1)
\end{eqnarray}

Este procedimento pode ser estendido para $t_3,t_4,\ldots $, onde
$$ t_{n+1}=t_n + h=t_1+n h, \quad  n=1,2,\ldots $$
e $h$ é o passo do método, ou espaçamento, que consideraremos constante.

Obtendo assim o \emph{método de Euler},
\begin{eqnarray}\label{euler}
u_{n+1}=u_n + h\;f(t_n,u_n).
\end{eqnarray}


Podemos também  obter o método de Euler a partir da aproximação de $u'(t)$ por um esquema de primeira ordem do tipo
$$u'(t)=\frac{u(t+h)-u(t)}{h}+\mathcal{O}(h),~~ h>0.$$
Substituindo na EDO temos
\begin{eqnarray}
\frac{u(t+h)-u(t)}{h}&=&f(t,u(t))+\mathcal{O}(h)\\
u(t+h)&=&u(t)+hf(t,u(t))+O(h^2).{\label{erro_local}}
\end{eqnarray}

Sendo $u_n$ a aproximação de $u$ em $t_n$ produzida pelo método de Euler, obtemos
\begin{eqnarray}\label{PVI_EULER}
u_{n+1}&=&u_n+hf(t_n,u_n),\\
u_1    &=&a.
\end{eqnarray}


% Note que o método numérico pode ser escrito como $R_h(u)_n=f(t_n,u_n)$ onde, para o método de Euler,
% $$R_h(u)_n=\frac{u_{n+1}-u_n}{h}$$


\begin{ex}
Considere o problema de valor inicial
\begin{eqnarray*}
u'(t)&=&2u(t)\\
u(0)&=&1
\end{eqnarray*}

cuja solução é $u(t)=e^{2t}$. O método de Euler aplicado a este problema produz o  esquema:
\begin{eqnarray*}
u_{k+1}&=&u_k+2hu_k=(1+2h)u_k\\
u_1&=&1,
\end{eqnarray*}
Suponha que queremos calcular o valor aproximado de $u(1)$ com $h=0,2$. Então os pontos $t^{(1)}=0$, $t^{(2)}=0,2$, $t^{(3)}=0,4$, $t^{(4)}=0,6$, $t^{(5)}=0,8$ e $t^{(6)}=1,0$ formam os seis pontos da malha. As aproximações para a solução nos pontos da malha usando o método de Euler são:
\begin{eqnarray*}
  u(0)  &\approx &u^{(1)}=1\\
  u(0,2)&\approx &u^{(2)}=(1+2h) u^{(1)}=1,4 u^{(1)}=1,4\\
  u(0,4)&\approx &u^{(3)}=1,4 u^{(2)}=1,96\\
  u(0,6)&\approx &u^{(4)}=1,4 u^{(3)}=2,744\\
  u(0,8)&\approx &u^{(5)}=1,4 u^{(4)}=3,8416\\
  u(1,0)&\approx &u^{(6)}=1,4 u^{(5)}=5,37824
\end{eqnarray*}
Essa aproximação é bem grosseira quando comparamos com a solução do problema em $t=1$: $u(1)=e^{2}\approx 7,38906$.
% Observe que a solução da relação de recorrência \eqref{exemplo_y_2y_euler} é dada por
% $$u^{(k)}=(1+2h)^{k-1}.$$
% Em um ponto genérico da malha $t\approx t^{(k)}=(k-1)h$ a solução aproximada pelo Método de Euler é
% $$u(t)\approx \tilde{u}(t)= (1+2h)^{\frac{t}{h}}.$$
% Observe que $\tilde{u}(t) \neq u(t)$, mas se $h$ é pequeno, a aproximação é boa, pois
% $$\lim_{h\to 0+} (1+2h)^{\frac{t}{h}}= e^{2t}.$$
\end{ex}


\begin{ex}
Aproxime a solução do PVI
\begin{eqnarray}
   \frac{du}{dt} &=& -0.5u+2+t\\
            u(0) &=&  8
\end{eqnarray}
Teste para $h=1.6, 0.8, 0.4, 0.2, 0.1$.

Note que a solução exata do problema é
\begin{equation}
     u(t) = 2t+8e^{-t/2}
\end{equation}

Itere a fórmula
\begin{eqnarray}
  u_{n+1}=u_n + h( -0.5u_n+2+t_n), \quad  u_1=8
\end{eqnarray}
através do código abaixo:

\begin{verbatim}
%---------------------------
function [u,t]=euler(h,Tmax)
  u(1)= 8;
  t(1)= 0;
  itmax = Tmax/h;
  for n=1:itmax
    t(n+1)= t(n) + h;
    u(n+1)= u(n) + h*(-0.5*u(n)+2+t(n));
  end
  plot(t,u,'g*-');
%---------------------------
\end{verbatim}

% Veja o gráfico da solução para $h=1, 0.5, 0.1, 0.05$:
% \begin{figure}
% \includegraphics[width=\textwidth]{euler.eps}
% \end{figure}

\end{ex}


Vamos agora, analisar o desempenho do método de Euler usando um exemplo mais complicado, porém ainda simples suficiente para que possamos obter a solução exata:  
\begin{ex}\label{ex_euler_1}
Considere o problema de valor inicial relacionado à equação logística\index{equação!logística}:
\begin{eqnarray*}
u'(t)&=&u(t)(1-u(t))\\
u(0)&=&1/2
\end{eqnarray*}
\end{ex}
Podemos obter a solução exata desta equação usando o método de separação de variáveis\index{método!de separação de variáveis} e o método das frações parciais\index{método das frações parciais}. Para tal escrevemos:
\begin{equation*}
\frac{du(t)}{u(t)(1-u(t))}=dt
\end{equation*}
O termo $\frac{1}{u(1-u)}$ pode ser decomposto em frações parciais como $\frac{1}{u}-\frac{1}{1-u}$ e chegamos na seguinte equação diferencial:
\begin{equation*}
\left(\frac{1}{u}+\frac{1}{1-u}\right)du=dt.
\end{equation*}
Integrando termo-a-termo, temos a seguinte equação algébrica relacionando $u(t)$ e $t$:
\begin{equation*}
\ln(u)-\ln(1-u)=t+C
\end{equation*}
Onde $C$ é a constante de integração, que é definida pela condição inicial, isto é, $u=1/2$ em $t=0$. Substituindo, temos $C=0$. O que resulta em:
\begin{equation*}
\ln\left(\frac{u}{1-u}\right)=t
\end{equation*}
Equivalente a
\begin{equation*}
\frac{u}{1-u}=e^{t}
\end{equation*}
e
\begin{equation*}
u=(1-u)e^{t} 
\end{equation*}
Colocando o termo $u$ em evidência, encontramos:
\begin{equation}
(1+e^t)u=e^{t} 
\end{equation}
E, finalmente, encontramos a solução exata dada por $u(t)=\frac{e^t}{1+e^{t}}$.

Vejamos, agora, o esquema iterativo produzido pelo método de Euler:
\begin{eqnarray*}
u_{k+1}&=& u_k+h u_k(1-u_k), \\
u_1&=& 1/2.
\end{eqnarray*}

Para fins de comparação, calculamos a solução exata e aproximada para alguns valores de $t$ e de passo $h$ e resumimos na Tabela~\ref{tab:log}.

\begin{table}
  \caption{Tabela comparativa entre método de Euler e solução exata para problema \ref{ex_euler_1}.}
  \label{tab:log}
  \begin{tabular}{|c|c|c|c|}\hline
    $t$ & $\hbox{Exato}$ & $\hbox{Euler}~~ h=0,1$ & $\hbox{Euler}~~ h=0,01$\\\hline
    $0$ & $1/2$ & $0,5$ & $0,5$\\\hline
    $1/2$ & $\frac{e^{1/2}}{1+e^{1/2}}\approx 0,6224593$ & $0,6231476$ & $0,6225316$\\\hline
    $1$ & $\frac{e}{1+e}\approx 0,7310586$ & $0,7334030$ & $0,7312946$\\\hline
    $2$ & $\frac{e^2}{1+e^2}\approx  0,8807971$ & $0,8854273$  & $0,8812533$ \\\hline
    $3$ & $\frac{e^3}{1+e^3}\approx   0,9525741$  & $0,9564754$ & $0,9529609$ \\\hline
  \end{tabular}
\end{table}


No exemplo a seguir, apresentamos um problema envolvendo uma equação não-autônoma\index{equação diferencial!não autônoma}, isto é, quando a função $f(u)$ depende explicitamente do tempo.

\begin{ex}
Resolva o problema de valor inicial
  \begin{eqnarray*}
    u'(t)&=&-u(t)+t\\
    u(0)&=&1,
  \end{eqnarray*}
cuja solução exata é $u(t)=2e^{-t}+t-1$.
\end{ex}
O esquema recursivo de Euler fica:
\begin{eqnarray*}
  u_{k+1}&=&u_k+h(-u_k+t_k)\\
  u_1&=&1
\end{eqnarray*}

Comparação
\begin{center}
\begin{tabular}{|c|c|c|c|}\hline
$t$ &  Exato & Euler~~ $h=0,1$ & Euler~~ $h=0,01$\\\hline
$0$ &  $1$ & $1$ & $1$\\\hline
$1$ &   $2e^{-1}\approx 0,7357589$ & $0,6973569$   &   $0,7320647$  \\\hline
$2$ &   $2e^{-2}+1\approx  1,2706706$ & $ 1,2431533 $   &  $ 1,2679593$     \\\hline
$3$ &   $2e^{-3}+2\approx 2,0995741$  & $ 2,0847823$ & $2,0980818$   \\\hline
\end{tabular}    
\end{center}



\subsection*{Exercícios}
\begin{exer}Resolva o problema de valor inicial dado por
\begin{eqnarray*}
u'&=& -2u + \sqrt{u}\\
u(0)&=&1
\end{eqnarray*}
com passo $h=0,1$ e $h=0,01$ para obter aproximações para $u(1)$. Compare com a solução exata dada por $u(t) =  \left({1+2 e^{-t}+e^{-2 t}}\right)/{4}$
\end{exer}
\begin{resp}
  
 $0,4496$ com $h=0,1$ e $0,4660$ com $h=0,01$. A solução exata vale $u(1)=\frac{1+2e^{-1}+e^{-2}}{4}= \left(\frac{1+e^{-1}}{2}\right)^2\approx 0,4678$    
  
\end{resp}


\begin{exer}Resolva o problema de valor inicial dado por
\begin{eqnarray*}
u'&=& -2u + \sqrt{z}\\
z'&=& -z + u\\
u(0)&=&0\\
z(0)&=&2\\
\end{eqnarray*}
com passo $h=0,2$, $h=0,02$, $h=0,002$ e $h=0,0002$ para obter aproximações para $u(2)$ e $z(2)$.
\end{exer}
\begin{resp}
  
$u(2)\approx 0,430202$ e $z(2)=0,617294$ com $h=0,2$, 
$u(2)\approx 0,435506$ e $z(2)=0,645776$ com $h=0,02$,
$u(2)\approx 0,435805$ e $z(2)=0,648638$ com $h=0,002$ e 
$u(2)\approx 0,435832$ e $z(2)=0,648925$ com $h=0,0002$.     
  
\end{resp}

\begin{exer}Resolva o problema de valor inicial dado por
\begin{eqnarray*}
u'&=& \cos(tu(t))\\
u(0)&=&1\\
\end{eqnarray*}
com passo $h=0,1$, $h=0.01$, $h=0,001$, $h=0,0001$ e $0,00001$ para obter aproximações para $u(2)$. 
\end{exer}
\begin{resp}
  
$u(2)\approx 1,161793$ com $h=0,1$, 
$u(2)\approx 1,139573$ com $h=0,01$,
$u(2)\approx 1,137448$ com $h=0,001$,
$u(2)\approx 1,137237$ com $h=0,0001$,
$u(2)\approx 1,137216$ com $h=0,00001$
\end{resp}















\subsection{Ordem de precisão}

A \emph{precisão} de um método numérico que aproxima a solução de um PVI é dada pela ordem do erro acumulado ao calcular a aproximação em um ponto $t_{n+1}$ em função do espaçamento da malha $h$. 

Se $u(t_{n+1})$ for aproximado por $u_{n+1}$ com erro da ordem $O(h^{p+1})$ dizemos que o método tem \textbf{ordem de precisão $p$}\index{método!de Euler!ordem de precisão}.


Queremos obter a ordem de precisão do método de Euler. Para isso, substituímos a EDO $u'=f(t,u)$ na expansão em série de Taylor
\begin{eqnarray}\label{taylor}
   u(t_{n+1})=u(t_n)+hu'(t_n)+h^2u''(t_n)/2+ \mathcal O(h^3)
\end{eqnarray}
e obtemos
\begin{eqnarray}\label{tayloreuler}
 u(t_{n+1})=u(t_n)+hf(t_n,u(t_n))+h^2u''(t_n)/2+ \mathcal O(h^3)
\end{eqnarray}
Subtraindo \eqref{tayloreuler} do método de Euler
\begin{eqnarray}
    u_{n+1}=u_n + h\;f(t_n,u_n)
\end{eqnarray}
obtemos
\begin{eqnarray}
   e_{n+1}   &=& u_{n+1}-u(t_{n+1}) \\
             &=&u_n - u(t_n)  +h(f(t_n,u(t_n)+e_n)- f(t_n,u(t_n))) +\\
             &+&\frac{h^2}{2}u''_n+\mathcal O(h^3)
\end{eqnarray}
Defina o \emph{erro numérico} como $e_n=u_n-u(t_n)$ onde $u(t_n)$ é a solução exata e $u_n$ é a solução aproximada. Assim
\begin{eqnarray}
   e_{n+1}    =&e_n + h(f(t_n,u(t_n)+e_n)- f(t_n,u(t_n))) +\frac{h^2}{2}u''_n+\mathcal O(h^3)
\end{eqnarray}
Usando a condição de Lipschitz em $f$  temos
\begin{eqnarray}
   |e_{n+1}|      &\le &  |e_n| + h|f(t_n,u(t_n)+e_n)- f(t_n,u(t_n))|+\frac{h^2}{2}|u''_n|+\mathcal O(h^3)\\
                  &\le &  |e_n| + hL |u(t_n)+e_n- u(t_n)|+\frac{h^2}{2}|u''_n|+\mathcal O(h^3)\\
                  &\le &  |e_n| + hL |e_n|+\frac{h^2}{2}|u''_n|+\mathcal O(h^3)\\
                  &\le &  (1+ hL) |e_n|+\frac{h^2}{2}|u''_n|+\mathcal O(h^3)
\end{eqnarray}

\subsection{Erro de truncamento Local}

O \emph{Erro de Truncamento Local} é o erro cometido em \textbf{uma} iteração do método numérico supondo que a solução exata é conhecida no passo anterior.

Assim, supondo que a solução é exata em $t_n$ ($|e_n|=0$), obtemos que o ETL é
$$ETL_{Euler}^{n+1}= h^2/2|u''|+ \mathcal O(h^3) = \mathcal O(h^2)$$

Como o $ETL=\mathcal O(h^2)$ temos que o método de Euler possui ordem $1$.


\subsection{Erro de truncamento Global}
O \emph{Erro de Truncamento Global} é o erro cometido durante \textbf{várias} iterações do método numérico.

Supondo que a solução exata é conhecida em $t_1$ ($\|e_1\|=0$), então realizando $n=\frac{T}{h}$ iterações obtemos
\begin{eqnarray}
   ETG &=& nETL \\
       &=& n[h^2/2|u''|+ \mathcal O(h^3)] \\
       &=& Th/2|u''|+ \mathcal O(h^2)
\end{eqnarray}
ou seja
$$ETG_{Euler}^{n+1} = \mathcal \mathcal{O}(h)$$



\section{Convergência, consistência e estabilidade}
Nesta seção veremos três conceitos fundamentais em análise numérica: convergência, consistência e estabilidade.

\subsection{Convergência}
Um método é dito \emph{convergente} se para toda EDO com $f$ Lipschitz e todo $t>0$ temos que
$$ \lim_{h \rightarrow 0} |u_n - u(t_n)| =0, \quad \quad \forall n$$
Convergência significa que a solução numérica tende a solução do PVI.


\begin{teo}
O método de Euler é convergente.
\end{teo}

De fato, se $f$ Lipschitz e $|e_0|=0$, temos que
\begin{eqnarray}
 \lim_{h\rightarrow 0} |e_{n+1}|  &= \lim_{h\rightarrow 0} \mathcal \mathcal{O}(h) = 0
\end{eqnarray}






\subsection{Consistência}
\begin{defn}
Dizemos que um método numérico $R_h(u_n)=f$ é consistente com o PVI $u'(t)=f$ se para qualquer $u(t)$
\begin{eqnarray}
  \lim_{h \rightarrow 0} |u'(t_n)-R_h(u_n)| = 0, \quad  \forall n
\end{eqnarray}
\end{defn}

Isto é equivalente a
\begin{eqnarray}
  \lim_{h \rightarrow 0} \frac{ETL}{h} = 0
\end{eqnarray}



%\novapagina
%\section{Estabilidade}
%\begin{defn}\label{def:estnum}
% Um método numérico $P_h(u_n)$ para um PVI é \emph{estável} numa região $\Lambda $ se  $\exists  J$ inteiro tal que $\forall T>0$, $\exists  C_T$ tal que
% \begin{eqnarray}
% |u_n|  \leq   C_T |u_0|
% \end{eqnarray}
%para $0\leq nh\leq T$, com $h\in \Lambda $.
%\end{defn}
%
%Isto significa que para ser estável a solução em $t\in [0,T]$ deve permanecer limitada por $C_T$ vezes a norma dos $J+1$ dados iniciais ($J=0$ para métodos de passo simples e $J>0$ para passo múltiplo).
%




\subsection{Estabilidade}
\begin{defn}
Um método numérico é \emph{estável} se
$$ |u_n-v_n| \leq  C_1|u_1-v_1|, \quad  \forall n$$
\end{defn}
Isto significa que dadas duas condições iniciais $u_1$ e $v_1$, teremos que as soluções $u_n$ e $v_n$ estarão a uma distância limitada  por uma constante $C_1$ vezes $|u_1-v_1|$. Se $u_1$ e $v_1$ estiverem próximas então $u_n$ e $v_n$ estão também próximas dependendo da constante $C_1$ (obviamente $C_1$ depende da função $f$).


Considere o PVI linear bem-posto
\begin{eqnarray}\label{EDO4.7}
  u'(t)= \lambda u(t), \quad  u(0)=1,
\end{eqnarray}
onde $\lambda  \in  \mathbb{C}$. Note que:
\begin{itemize}
\item Possui solução exata $u(t)=e^{\lambda t}.$
\item O PVI é \emph{assintoticamente estável}, isto é, $\lim_{t\rightarrow \infty }u(t)=0$, se e somente se $\Re{\lambda }<0$.
\
\end{itemize}
%Não estamos interessados no momento em soluções que crescem rapidamente ($\Re{\lambda }>0$).

%\begin{defn}
%O PVI linear é \emph{assintoticamente estável} se e somente se $\Re{\lambda }<0$.
%\end{defn}



\begin{defn}
O \emph{domínio de estabilidade linear} $\mathcal D$ do método numérico é o conjunto de todos $h\lambda  \in  \mathbb{C}$ tal que $\lim_{n\rightarrow \infty }u_n=0$.
\end{defn}

Ou seja, $\mathcal D$ é o conjunto de todos $h\lambda $ para o qual o correto comportamento assintótico de \eqref{EDO4.7} seja recuperado.

%tal que essa equação seja estável.


\begin{ex}
Utilizando o \textbf{Método de Euler} para solucionar \eqref{EDO4.7} obtemos ($u_1=1$)
\begin{eqnarray}
 u_{n+1}   & =& u_n+h\lambda u_n, \\
 u_{n+1}   & =& (1+h\lambda )u_n, \\
 u_{n+1}   & =& (1+h\lambda )^2u_{n-1}, \\
 u_{n+1}   & =& (1+h\lambda )^{n+1}u_1 \\
 u_{n+1}   & =& (1+h\lambda )^{n+1}  , \quad  n=0,1,\ldots 
\end{eqnarray}
Para que o método de Euler seja estável, é necessário que $h$ seja escolhido tal que $|1+h\lambda |<1$. Ou seja, $h\lambda $ deve estar em $\mathcal D_{Euler}$ onde
\begin{eqnarray}
 \mathcal D_{Euler} = \{z \in  \mathbb{C}: |1+z|<1\}
\end{eqnarray}
é o interior de um disco no plano complexo de raio $1$ e centro em $z=-1$ como na Fig.\ref{RegiaoEuler}.
% \begin{figure}[htp]
% \begin{center}
%   \includegraphics[width=8cm]{RegiaoEuler.eps}\\
%   \caption{Região de estabilidade para o método de Euler, $|1+z|<1$ }\label{RegiaoEuler}
% \end{center}
% \end{figure}
\end{ex}


Tal análise pode ser facilmente estendida para $u'=\lambda u+b$ (veja exercícios).

Para o caso EDO não-linear, seja
\begin{eqnarray}
u'= f(t,u), \quad  t\geq t_0, \quad  u(t_0)=u_0
\end{eqnarray}
é comum requerer que $h\lambda _{n,k} \in \mathcal D$ onde $\lambda _{n,k}$ são os autovalores da matriz Jacobiana $J_n := \frac{\partial f}{\partial u}|_{(t_n,u_n)},$ baseado na hipótese que o comportamento local da EDO é modelado por
\begin{eqnarray}
  u'= u_n + J_n(u-u_n)
\end{eqnarray}
Esta prática não é exata e fornece apenas uma ideia local do comportamento da EDO (podendo levar a conclusões errôneas).


Um dos teoremas mais importantes em análise numérica é o seguinte:

\begin{teo}
Um método numérico \emph{consistente} para um PVI bem-posto é \emph{convergente} se e somente se ele é \emph{estável}.
\end{teo}


Ele também é usado da seguinte forma:

\begin{teo}
Se um método numérico é \emph{consistente} e \emph{estável} em $[a,b]$ então ele é \emph{convergente}.
\end{teo}











\section{O método de Euler implícito}
Integrando o PVI
\begin{eqnarray}
  u'(t)  &=& f(t,u(t)) \\
  u(t_1) &=& a
\end{eqnarray}
de $t_1$ até $t_2$ obtemos (como feito anteriormente)
\begin{eqnarray}
  u(t_2)      &=& u(t_1) +  \int_{t_1}^{t_2} f(t,u(t)) \; dt
\end{eqnarray}

Entretanto se aproximarmos a função $f$ por uma função constante $f(t,u(t)) \approx  f(t_2,u_2)$, obteremos um novo método
\begin{eqnarray}
  u_2 &=&  u_1 + f(t_2,u_2) \int _{t_1}^{t_2}  \; dt \\
  u_2 &=&  u_1 + h f(t_2,u_2)
\end{eqnarray}


Generalizando este procedimento para $t_n$ obtemos o \emph{método de Euler implícito}
\begin{eqnarray}
u_{n+1}=u_n + h\;f(t_{n+1},u_{n+1}).
\end{eqnarray}

Note que este método é \emph{implícito} (a equação é implícita) pois depende de $u_{n+1}$ dos dois lados da equação. Se a função $f$ for simples o suficiente, podemos resolver a equação isolando o termo $u_{n+1}$. Se isso não for possível, devemos usar um dos métodos vistos anteriormente para calcular as raízes da equação (por exemplo, método da bissecção e método de Newton).



Pode ser mostrado que o erro de truncamento local é
$$ETL_{EulImp}^{n+1}= \mathcal{O}(h^2).$$
portanto o método é de ordem $1$. E o erro de truncamento global é
$$ETG_{EulImp}^{n+1}= \mathcal{O}(h).$$



\begin{ex}
Utilizando o \textbf{método de Euler implícito} para solucionar \eqref{EDO4.7} obtemos
\begin{eqnarray}
 u_{n+1}      &=& u_n+h\lambda u_{n+1}, \\
 (1-h\lambda )u_{n+1} & =& u_n, \\
       u_{n+1} & =& \left(\frac{1}{1- h\lambda }\right)u_n, \\
       u_{n+1} & =& \left(\frac{1}{1- h\lambda }\right)^2u_{n-1}, \\
       u_{n+1} & =& \left(\frac{1}{1- h\lambda }\right)^{n+1}, \quad  n=0,1,\ldots 
\end{eqnarray}
onde $u_1=1$.
Concluímos então que
\begin{eqnarray}
 \mathcal D_{EulImp} = \{z \in  \field{C}:  \left|\frac{1}{1- z}\right|<1\}
\end{eqnarray}
ou ainda,
\begin{eqnarray}
 \mathcal D_{EulImp} = \{z \in  \field{C}:  |1- z|>1\}
\end{eqnarray}

Para que o método de Euler implícito seja estável, é necessário que $h$ seja escolhido tal que $\left|\frac{1}{1- h\lambda }\right|<1$, ou ainda, $|1-h\lambda |>1$. Ou seja, $h\lambda $ deve estar em $\mathcal D_{EulImp}$ onde
\begin{eqnarray}
 \mathcal D_{EulImp} = \{z \in  \field{C}: |1-z|>1\}
\end{eqnarray}
é o exterior de um disco no plano complexo de raio $1$ e centro em $z=1$.

Note que $\mathcal D_{EulImp}$ inclui todo o semiplano negativo. Portanto o método de Euler implícito imita a estabilidade assintótica da EDO linear sem restrição no passo $h$.

\end{ex}

\begin{defn}
Um método numérico é chamado \emph{A-estável} ou \emph{incondicionalmente estável} se incluir todo o semiplano complexo com parte real negativa,
$$    \{z \in \field{C}: \Re{z}<0\} \subseteq \mathcal D$$
\end{defn}


Portanto o método de Euler implícito é $A$-estável (incondicionalmente estável).






















\section{Método Trapezoidal}\index{método!trapezoidal}
O método de Euler aproxima $f$ como uma constante no intervalo $[t_1,t_2]$. Podemos melhorar isso usando a regra do trapézio,

\begin{eqnarray}
  u(t_2) &=& u(t_1) +  \int _{t_1}^{t_2}  f(t,u(t)) \; dt \\
  u_2    &=&   u_1  +  (t_2-t_1)\left(\frac{1}{2}f(t_1,u_1)+\frac{1}{2}f(t_2,u_2)\right)
\end{eqnarray}
motivando o \emph{método trapezoidal}
\begin{eqnarray}
  u_{n+1} &=& u_n +  \frac{h}{2} \left(f(t_n,u_n)+f(t_{n+1},u_{n+1})\right)
\end{eqnarray}
O método trapezoidal é dito \textbf{implícito}, pois para obter $u_{n+1}$ é necessário calcular $f(t_{n+1},u_{n+1})$.

Entretanto, pode ser mostrado que o erro de truncamento local é
$$ETL_{Trap}^{n+1}= O(h^3)$$
portanto o método é de ordem $2$. E o erro de truncamento global é
$$ETG_{Trap}^{n+1}= O(h^2)$$


\begin{ex}
Utilizando o \textbf{método trapezoidal} para solucionar \eqref{EDO4.7} obtemos
\begin{eqnarray}
 u_{n+1} = \left(\frac{1+ h\lambda /2}{1- h\lambda /2}\right)^{n+1}, \quad  n=0,1,\ldots 
\end{eqnarray}
Concluímos então que
\begin{eqnarray}
 \mathcal D_{Tr} = \{z \in  \field{C}:  \left|\frac{1+ z/2}{1- z/2}\right|<1\}
\end{eqnarray}
Note que $\mathcal D_{Tr}=\field{C}^-$, o semiplano negativo. Portanto o método do trapézio imita a estabilidade assintótica da EDO linear sem restrição no passo $h$.
\end{ex}




\section{O método de Heun}
Também chamado de método de \emph{Euler Modificado}. A ideia é calcular primeiramente um valor intermediário $\tilde{u}$ usando o método de Euler expl\'icito e usar esse valor na equação para o método do Trapézio. Ou seja, o \emph{método de Heun} é
\begin{eqnarray}
  \tilde{u} &=& u_n +   h f(t_n,u_n) \\
  u_{n+1}   &=& u_n +  \frac{h}{2} \left(f(t_n,u_n)+f(t_{n+1},\tilde{u})\right)
\end{eqnarray}

Este é um exemplo de um método preditor-corretor.

Felizmente o erro de truncamento local continua sendo
$$ETL_{Heun}^{n+1}= O(h^3)$$
e o erro de truncamento global é
$$ETG_{Heun}^{n+1}= O(h^2)$$















\subsection*{Exercícios}

\begin{exer} Use o método de Euler melhorado para obter uma aproximação numérica do valor de $u(1)$ quando $u(t)$ satisfaz o seguinte problema de valor inicial
\begin{eqnarray*}
 u'(t)&=&-u(t)+ e^{u(t)},\\
 u(0)&=&0,
\end{eqnarray*}
usando passos $h=0,1$ e $h=0,01$.
\end{exer}
\begin{resp}
  
 $u(1)\approx 1,317078$ quando $h=0,1$ e $u(1)\approx 1,317045$.    
  
\end{resp}


\begin{exer}
Use o método de Euler e o método de Euler melhorado para obter aproximações numéricas para a solução do seguinte problema de valor inicial para $t\in[0,1]$:
\begin{eqnarray*}
 u'(t)&=&-u(t)- u(t)^2,\\
 u(0)&=&1,
\end{eqnarray*}
usando passo $h=0,1$. Compare os valores da solução exata dada por $u(t)=\frac{1}{2e^t-1}$ com os numéricos nos pontos $t=0$, $t=0.1$, $t=0.2$, $t=0.3$, $t=0.4$, $t=0.5$, $t=0.6$, $t=0.7$, $t=0.8$, $t=0.9$, $t=1.0$.
\end{exer}
\begin{resp}

 $$\begin{array}{|c|c|c|c|c|c|}
\hline
t &  \hbox{Exato} & \hbox{Euler} & \hbox{Euler melhorado} & \hbox{Erro Euler} & \hbox{Erro Euler melhorado}\\
\hline
0.0&    1.          &1.          &1.          &0.          &0.       \\ 
0.1&    0.826213    &0.8         &0.828       &0.026213    &0.001787\\  
0.2&    0.693094    &0.656       &0.695597    &0.037094    &0.002502  \\
0.3&    0.588333    &0.547366    &0.591057    &0.040967    &0.002724  \\
0.4&    0.504121    &0.462669    &0.506835    &0.041453    &0.002714  \\
0.5&    0.435267    &0.394996    &0.437861    &0.040271    &0.002594  \\
0.6&    0.378181    &0.339894    &0.380609    &0.038287    &0.002428  \\
0.7&    0.330305    &0.294352    &0.332551    &0.035953    &0.002246  \\
0.8&    0.289764    &0.256252    &0.291828    &0.033512    &0.002064  \\
0.9&    0.255154    &0.224061    &0.257043    &0.031093    &0.001889  \\
1.0&    0.225400    &0.196634    &0.227126    &0.028766    &0.001726\\

\hline
\end{array}
$$

\ifisscilab
      No Scilab, esta tabela pode ser produzida com o código:
      \begin{verbatim}
       deff('du=f(u)','du=-u-u^2')
       sol_Euler=Euler(f,0,1,10,1)'
       sol_Euler_mod=Euler_mod(f,0,1,10,1)'
       deff('u=u_exata(t)','u=1/(2*exp(t)-1)')
       t=[0:.1:1]'
       sol_exata=feval(t,u_exata)
       tabela=[t sol_exata sol_Euler sol_Euler_mod abs(sol_exata-sol_Euler) abs(sol_exata-sol_Euler_mod)]
      \end{verbatim}

    \fi
 
\end{resp}





\section{O método theta}
Tanto o método de Euler quanto o método trapezoidal se encaixam no método
\begin{eqnarray}
  u_{n+1} &= u_n +  h (\theta f(t_n,u_n)+(1-\theta )f(t_{n+1},u_{n+1}))
\end{eqnarray}
com $\theta =1$ e $\theta =\frac{1}{2}$ respectivamente. O método é expl\'icito somente para $\theta =1$. Para $\theta =0$, obtemos o método impl\'icito de Euler.








\section{O método de Taylor}
Uma maneira simples de aumentar a ordem do método é utilizar diretamente a série de Taylor.
Considere a expansão
\begin{eqnarray}
 u(t+h)=u(t) +h u'(t)+ \frac{h^2}{2!}u''(t)+\frac{h^3}{3!}u'''(t)+\ldots 
\end{eqnarray}

Utilizando dois termos temos o método de Euler. Utilizando os três primeiros termos da série e substituindo $u'(t)=f(t,x)$ e $u''(t)=\frac{\partial f}{\partial t}(t,x)$ temos o \emph{método de Taylor de ordem $2$}
\begin{eqnarray}
   u_{n+1}=u_n +h f(t_n,u_n)+ \frac{h^2}{2!} \frac{\partial f}{\partial t}(t_n,u_n)
\end{eqnarray}


O método de Taylor de ordem $3$ é
\begin{eqnarray*}
   u_{n+1}=u_n +h f(t_n,u_n)+ \frac{h^2}{2!}\frac{\partial f}{\partial t}(t_n,u_n)+\frac{h^3}{3!}\frac{\partial^2 f}{\partial t^2}(t_n,u_n)
\end{eqnarray*}




\section{Estabilidade dos métodos de Taylor}
\begin{ex}
Prove que para um método de Taylor de ordem $p$ para a EDO \eqref{EDO4.7} temos
\begin{eqnarray}
  p(z)= 1 + z+ \frac{z^2}{2!} +\frac{z^3}{3!}+\ldots +\frac{z^p}{p!}
\end{eqnarray}
onde  $u_n = (p(z))^nu_0$ e a região de estabilidade é dada por
\begin{eqnarray}
 \mathcal D_{T} = \{z \in  \field{C}:  \left|p(z)\right|<1\}
\end{eqnarray}

Plote as regiões de estabilidade para o método de Taylor para $p=1,\ldots ,6$ no mesmo gráfico.
\end{ex}

% \begin{figure}
% \begin{center}
%   \includegraphics[width=8cm]{RegiaoTaylor.eps}\\
%   \caption{Região de estabilidade paras os métodos de Taylor de ordem $1,\ldots ,4$ (interior as curvas). A curva mais interna é para $p=1$}\label{RegiaoTaylor}
% \end{center}
% \end{figure}


\begin{ex}
Aproxime a solução do PVI
\begin{eqnarray}
   \frac{du}{dt} &=\sin{t}\\
            u(0) &= 1
\end{eqnarray}
para  $t\in [0,10]$.

\begin{enumerate}
\item [a.] Plote a solução para $h=0.16,0.08, 0.04, 0.02, 0.01$ para o método de Taylor de ordem $1$, $2$ e $3$. (Plote todos de ordem $1$ no mesmo gráfico, ordem $2$ em outro gráfico e ordem $3$ outro gráfico separado.)

\item [b.] Utilizando a solução exata, plote um gráfico do erro em escala logar\'itmica.
Comente os resultados (novamente, em cada gráfico separado para cada método repita os valores acima)

\item [c.] Fixe agora o valor $h=0.02$ e plote no mesmo gráfico uma curva para cada método.

\item [d.] Plote em um gráfico o erro em $t=10$ para cada um dos métodos (uma curva para cada ordem) a medida que $h$ diminui. (Use escala \verb#loglog#)
\end{enumerate}
\end{ex}











% \section{Ordem de precisão}\index{ordem de precisão}
% Considere o problema de valor inicial dado por
% \begin{eqnarray*}
% u'(t)&=&f(t,u(t)),\\
% u(0)&=&u_0.
% \end{eqnarray*}
% Nessa seção vamos definir a precisão de um método numérico pela ordem do erro acumulado ao calcular o valor da função em um ponto $t_N$ em função do espaçamento da malha $h$. Se $u(t_n)$ pode ser aproximado por uma expressão que depende de $f$, $h$, $u(t_0)$, $u(t_1)$, $\cdots$, $u(t_n)$, com erro da ordem de $O(h^{p+1})$, ou seja,
% \begin{equation}{\label{erro_local_1}}
% u(t_{n+1})=\mathcal{F}(f, h, u(t_n), u(t_{n-1}), \cdots, u_0) + O(h^{p+1})
% \end{equation}
% para cada função analítica $f$, dizemos que o método tem erro de truncamento da ordem de $O(h^{p})$ ou {\bf ordem de precisão $p$}\index{método!de Euler!ordem de precisão}. Essa afirmação faz sentido quando fazemos a seguinte análise informal: para aproximar $u_1$, acumulamos erros da ordem $O(h^{p+1})$, para calcular $u_2$ acumulamos os erros de $u_1$ e novos erros $O(h^{p+1})$. Para calcular $u_N$, acumulamos todos os erros até $t_N$, ou seja, $N$ vezes $O(h^{p+1})$. Como $N=O(1/h)$, temos que os erros ao calcular $u_N$ são da ordem $O(h^p)$. É verdade que essa análise só vale quando impomos condições de suavidade para $f$ e condições adequada para a expressão $\mathcal{F}(f, h, u(t_n), u(t_{n-1}), \cdots, u_0)$. Para explicar melhor esse pequeno texto, fazemos em detalhes essa operação para o método de Euler na seção \ref{sec_pre_euler}.
% 
% \subsection{Ordem de precisão do método de Euler}{\label{sec_pre_euler}}
% Primeiro lembramos da expressão (\ref{erro_local}) que origina a seguinte relação de recorrência:
% \begin{eqnarray}{\label{es_Euler}}
% u(t_{n+1})&=&u(t_n)+hf(t,u(t_n),t_n)+O(h^2).
% \end{eqnarray}
% Para entender melhor o motivo de na expressão (\ref{es_Euler}) aparecer $O(h^2)$ e o método ser de precisão 1, vamos a seguinte análise informal: observemos que
% \begin{eqnarray*}
%  u(t_1)&=&u(t_0)+hf(t,u(t_0),t_0)+O(h^2)\\
%  &=&u_0+hf(u_0,t_0)+O(h^2)=u_1+O(h^2)
% \end{eqnarray*}
% onde $u_i$ é a aproximação pelo método de Euler para o valor exato $u(t_i)$. Subsequentemente, temos
% \begin{eqnarray*}
%  u(t_2)&=&u(t_1)+hf(t,u(t_1),t_1)+O(h^2)\\
%  &=&u(t_1)+hf(u_1+O(h^2),t_1)+O(h^2)\\
%  &=&u(t_1)+hf(u_1,t_1) +O(h^2)\\
%  &=&u_1+O(h^2)+hf(u_1,t_1) +O(h^2)= u_2+O(h^2)+O(h^2).
% \end{eqnarray*}
% onde usamos o primeiro termo da série de Taulor $hf(u_1+O(h^2),t_1)=hf(u_1,t_1)+O(h^3)$ na passagem da segunda para terceira linha. Repetindo sucessivamente o passo anterior, obtemos uma expressão geral para o valor exato $u(t_N)$ em termos do valor aproximado $u_N$:
% \begin{eqnarray*}
%  u(t_N)=u_N+N O(h^2) .
% \end{eqnarray*}
% Como $N=(t_f-t_0)/h$, temos
% \begin{eqnarray}{\label{euler_precisao}}
%  u(t_N)&=&=u_{N}+\frac{t-t_0}{h}O(h^2)=u_{N}+\mathcal{O}(h),
% \end{eqnarray}
% ou seja, o erro entre o valor exato e o aproximado é de ordem $h$. Uma demonstração mais formal que garante que o erro é limitado por uma expressão que é proporcional a $h$ está discutido na seção \ref{sec_conv_Euler}.
% 
% \subsection{Ordem de precisão do método de Euler melhorado}
% Para obter o erro de precisão do método de Euler melhorado vamos calcular o erro de truncamento do método, ou seja, precisamos demonstrar que:
% \begin{equation}\label{es_Euler_melhorado}
% u(t+h)=u(t)+\frac{h}{2} f(t,u(t))+\frac{h}{2} f(t,u(t)+hf(t,u(t))),t+h)+O(h^3)
% \end{equation}
% De fato, tomando a diferença do termo da esquerda o os termos da direita, temos:
% \begin{eqnarray*}
% &&u(t+h)-\left(u(t)+\frac{h}{2} f(t,u(t))+\frac{h}{2} f(t,u(t)+hf(t,u(t))),t+h)\right)\\
% &&=u(t)+hu'(t)+\frac{h^2}{2}u''(t)+O(h^3)\\
% &&-\left(u(t)+\frac{h}{2} u'(t)+\frac{h}{2} f(t,u(t)+hf(t,u(t))),t+h)\right),
% \end{eqnarray*}
% onde usamos uma expansão em série de Taulor para $u(t+h)$ e a equação diferencial $u'(t)=f(t,u(t))$. Portanto,
% \begin{eqnarray*}
% &&u(t+h)-\left(u(t)+\frac{h}{2} f(t,u(t))+\frac{h}{2} f(t,u(t)+hf(t,u(t))),t+h)\right)\\
% &&=\frac{h}{2}u'(t)+\frac{h^2}{2}u''(t)-\frac{h}{2} f(t,u(t)+hf(t,u(t))),t+h)+O(h^3).
% \end{eqnarray*}
% Agora, usamos a série de Taulor de $f(t,u(t)+hf(t,u(t)),t+h)$ e, torno de $(t,u)$:
% \begin{eqnarray*}
% &&u(t+h)-\left(u(t)+\frac{h}{2} f(t,u(t))+\frac{h}{2} f(t,u(t)+hf(t,u(t))),t+h)\right)\\
% &&=\frac{h}{2}u'(t)+\frac{h^2}{2}u''(t)+O(h^3)\\
% &&-\frac{h}{2}\left(f(t,u(t))+\frac{\partial f(t,u(t)) }{\partial t}h +\frac{\partial f(t,u(t))}{\partial u} hf(t,u(t))+O(h^2)\right).
% \end{eqnarray*}
% Usando a equação diferencial $u'(t)=f(t,u(t))$ obtemos 
% $$
% u''(t)=\frac{f(t,u(t))}{\partial t}+\frac{f(t,u(t))}{\partial u}u'(t)=\frac{f(t,u(t))}{\partial t}+\frac{f(t,u(t))}{\partial u}f(t,u(t)).
% $$
% Logo,
% \begin{eqnarray*}
% &&u(t+h)-\left(u(t)+\frac{h}{2} f(t,u(t))+\frac{h}{2} f(t,u(t)+hf(t,u(t))),t+h)\right)\\
% &&=\frac{h}{2}u'(t)+\frac{h^2}{2}u''(t)+O(h^3)\\
% &&-\frac{h}{2}\left(f(t,u(t))+hu''(t)+O(h^2)\right)\\
% &&=\frac{h}{2}u'(t)+\frac{h^2}{2}u''(t)\\
% &&-\frac{h}{2}\left(u'(t)+hu''(t)\right)+O(h^3)=O(h^3)
% \end{eqnarray*}
% Portanto, a expressão (\ref{es_Euler_melhorado}) é válida. Logo, usando uma discussão análoga aquela feita na seção \ref{sec_pre_euler} para o método de Euler, concluímos que o método de Euler melhorado possui ordem de precisão 2.

% \section{Convergência}
% 
% \emconstrucao
% 
% \subsection{Convergência do método de Euler}\label{sec_conv_Euler}
% 
% \emconstrucao
% 
% \subsection{Convergência do método de Euler melhorado}
% 
% \emconstrucao






\section{Métodos de Runge-Kutta}\label{sec_RK}\index{método!de Runge-Kutta}

Os métodos de Runge-Kutta consistem em iterações do tipo:
$$u_{k+1}=u_k+w_1 k_1 + \ldots + w_n k_n$$
onde
\begin{eqnarray*}
k_1&=&hf(u_k,t_k)\\
k_2&=&hf(u_k+\alpha_{2,1}k_1,t_k+\beta_{2}h)\\
k_3&=&hf(u_k+\alpha_{3,1}k_1+\alpha_{3,2}k_2,t_k+\beta_{3}h)\\
&\vdots&\\
k_n&=&hf(u_k+\alpha_{n,1}k_1+\alpha_{n,2}k_2+\ldots \alpha_{n,n-1}k_{n-1},t_k+\beta_{n}h)\\
\end{eqnarray*}

Os coeficientes são escolhidos de forma que a expansão em Taulor de $u_{k+1}$ e $u_k+w_1 k_1 + \ldots + w_n k_n$ coincidam até ordem $n+1$.

\begin{ex} O método de Euler melhorado é um exemplo de Runge-Kutta de segunda ordem
$$u^{(n+1)}=u^{(n)}+\frac{k_1+k_2}{2}$$
onde $k_1=hf(u^{(n)},t^{(n)})$ e $k_2=hf(u^{(n)}+k_1,t^{(n)}+h)$.
\end{ex}

\subsection{Métodos de Runge-Kutta - quarta ordem}\index{método!de Runge-Kutta!de quarta ordem}

$$u^{(n+1)}=u^{(n)}+\frac{k_1+2k_2+2k_3+k_4}{6}$$
onde
\begin{eqnarray*}
k_1&=&hf(u^{(n)},t^{(n)})\\
k_2&=&hf(u^{(n)}+k_1/2,t^{(n)}+h/2)\\
k_3&=&hf(u^{(n)}+k_2/2,t^{(n)}+h/2)\\
k_4&=&hf(u^{(n)}+k_3,t^{(n)}+h)\\
\end{eqnarray*}
Este método tem ordem de precisão 4. Uma discussão heurística usando método de Simpson pode ajudar a compreender os estranhos coeficientes:
\begin{eqnarray*}
u({t^{(n+1)}})-u({t^{(n)}})&=&\int_{t^{(n)}}^{t^{(n+1)}}f(t,u(s),s)ds \\
&\approx& \frac{h}{6}\left[ f\left(u(t^{(n)}),t^{(n)}\right)+4f\left(u(t^{(n)}+h/2),t^{(n)}+h/2\right)\right.\\
&+&\left.f\left(u(t^{(n)}+h),t^{(n)}+h\right)\right]\\
&\approx& \frac{k_1+4(\frac{k_2+k_3}{2})+k_4}{6}
\end{eqnarray*}
onde $k_1$ e $k_4$ representam as inclinações nos extremos e $k_2$ e $k_3$ são duas aproximações diferentes para a inclinação no meio do intervalo.


\section{Métodos de passo múltiplo - Adams-Bashforth}\index{método!de passo múltiplo!Adams-Bashforth}

O método de Adams-Bashforth consiste de um esquema recursivo do tipo:
$$u^{(n+1)}=u^{(n)}+\sum_{j=0}^k w_jf(u^{(n-j)},t^{(n-j)})$$

\begin{ex} Adams-Bashforth de segunda ordem
$$u^{(n+1)}=u^{(n)}+\frac{h}{2}\left[3f\left(u^{(n)},t^{(n)}\right)-f\left(u^{(n-1)},t^{(n-1)}\right)\right]$$
\end{ex}

\begin{ex} Adams-Bashforth de terceira ordem
$$u^{(n+1)}=u^{(n)}+\frac{h}{12}\left[23f\left(u^{(n)},t^{(n)}\right)-16f\left(u^{(n-1)},t^{(n-1)}\right)+5f\left(u^{(n-2)},t^{(n-2)}\right)\right]$$
\end{ex}

\begin{ex} Adams-Bashforth de quarta ordem
  \begin{equation*}
    \begin{split}
      u^{(n+1)} &= u^{(n)} + \frac{h}{24}\left[55f\left(u^{(n)},t^{(n)}\right)-59f\left(u^{(n-1)},t^{(n-1)}\right)\right.\\
        &+\left. 37f\left(u^{(n-2)},t^{(n-2)}\right)-9f\left(u^{(n-3)},t^{(n-3)}\right)\right]    
    \end{split}
  \end{equation*}
\end{ex}
Os métodos de passo múltiplo evitam os múltiplos estágios do métodos de Runge-Kutta, mas exigem ser "iniciados" com suas condições iniciais.

\section{Métodos de passo múltiplo - Adams-Moulton}\index{método de passo múltiplo!Adams-Moulton}

O método de Adams-Moulton consiste de um esquema recursivo do tipo:
$$u^{(n+1)}=u^{(n)}+\sum_{j=-1}^k w_jf(u^{(n-j)},t^{(n-j)})$$

\begin{ex} Adams-Moulton de quarta ordem
  \begin{equation*}
    \begin{split}
      u^{(n+1)} &= u^{(n)} + \frac{h}{24}\left[9f\left(u^{(n+1)},t^{(n+1)}\right) + 19f\left(u^{(n)},t^{(n)}\right) \right.\\
      &-\left. 5f\left(u^{(n-1)},t^{(n-1)}\right) + f\left(u^{(n-2)},t^{(n-2)}\right)\right]      
    \end{split}
  \end{equation*}
\end{ex}
O método de Adams-Moulton é implícito, ou seja, exige que a cada passo, uma equação em $u^{(n+1)}$ seja resolvida.

\section{Estabilidade}\index{estabilidade}

Consideremos o seguinte problema de teste:
$$\left\{\begin{array}{rcl}u'&=&-\alpha u\\u(0)&=&1\end{array}\right.$$
cuja solução exata é dada por $u(t)=e^{-\alpha t}$.

Considere agora o método de Euler aplicado a este problema com passa $h$:
$$\left\{\begin{array}{rcl}u_{k+1}&=&u_k-\alpha h u_k\\u_1&=&1\end{array}\right.$$
A solução exata do esquema de Euler é dada por
$$u_{k+1}=(1-\alpha h)^{k}$$
e, portanto,
$$\tilde{u}(t)=u_{k+1}=(1-\alpha h)^{t/h}$$

Fixamos um $\alpha>0$, de forma que $u(t)\to 0$. Mas observamos que $\tilde{u}(t)\to 0$ somente quando $|1-\alpha h|<1$ e solução positivas somente quando $\alpha h<1$.

{\bf Conclusão:} Se o passo $h$ for muito grande, o método pode se tornar instável, produzindo solução espúrias.







\subsection*{Exercícios}

\begin{exer} Resolva o problema 1 pelos diversos métodos e verifique heuristicamente a estabilidade para diversos valores de $h$.
\end{exer}











\section{Sistemas de equações diferenciais e equações de ordem superior}
O problema \eqref{PVI} pode ser um sistema de equações de primeira ordem, isto é, a incógnita $y(t)$ pode ser um vetor de funções, como mostra o exemplo~\ref{eq_exemplo_sistema}
\begin{ex}\label{exemplo_sistema_PVI}O problema de valor inicial
\begin{subequations}\label{eq_exemplo_sistema}
\begin{eqnarray}
u'(t)&=&v(t),\\
v'(t)&=&u(t),\\
u(0)&=&1.\\
v(0)&=&0.
\end{eqnarray}
\end{subequations}
pode ser escrito na forma \eqref{PVI} com $y(t)$
\end{ex}

No exemplo \ref{sys_edo}, mostramos como o Método de Euler pode ser facilmente estendido para problemas envolvendo sistemas de equações diferenciais.\index{sistemas!de equações diferenciais}.
\begin{ex}\label{sys_edo} Escreva o processo iterativo de Euler para resolver numericamente o seguinte sistema de equações diferenciais
\begin{eqnarray*}
x'&=&-y\\
y'&=&x\\
x(0)&=&1\\
y(0)&=&0,\\
\end{eqnarray*}
cuja solução exata é $x(t)=\cos(t)$ e $y(t)=\sin(t)$.
\end{ex}
Para aplicar o Método de Euler a um sistema, devemos encarar as diversas incógnitas do sistema como formando um vetor, neste caso, escrevemos: 
 $$z(t)=\left[\begin{array}{c}x(t)\\y(t)\end{array}\right].$$
 O sistema é igualmente escrito na forma vetorial:
\begin{eqnarray*}
\left[\begin{array}{c}x^{(k+1)}\\y^{(k+1)}\end{array}\right]=\left[\begin{array}{c}x^{(k)}\\y^{(k)}\end{array}\right]+h\left[\begin{array}{c}-y^{(k)}\\x^{(k)}\end{array}\right].
\end{eqnarray*}
Observe que este processo iterativo é equivalente a:
\begin{eqnarray*}
x^{(k+1)}&=&x^{(k)}-hy^{(k)}\\
y^{(k+1)}&=&y^{(k)}+hx^{(k)}.
\end{eqnarray*}


\begin{ex} Escreva o problema de valor inicial de segunda ordem dado por
\begin{eqnarray*}
y''+y'+y&=&\cos(t),\\
y(0)&=&1,\\
y'(0)&=&0,
\end{eqnarray*}
como um problema envolvendo um sistema de primeira ordem.
\end{ex}
A fim de transformar a equação diferencial dada em um sistema de equações de primeira ordem, introduzimos a substituição $w=y'$, de forma que obteremos o sistema:
\begin{eqnarray*}
y'&=&w\\
w'&=&-w-y+\cos(t)\\
y(0)&=&1\\
w(0)&=&0
\end{eqnarray*}
Portanto, o Método de Euler produz o seguinte processo iterativo:
\begin{eqnarray*}
y^{(k+1)}&=&y^{(k)}+hw^{(k)},\\
w^{(k+1)}&=&w^{(k)}-hw^{(k)}-hy^{(k)}+h\cos(t^{(k)}),\\
y^{(1)}&=&1,\\
w^{(1)}&=&0.
\end{eqnarray*}


\subsection*{Exercícios}

\begin{exer}Resolva o problema de valor inicial dado por
\begin{eqnarray*}
y'&=& -2y + \sqrt{z}\\
z'&=& -z + y\\
y(0)&=&0\\
z(0)&=&2\\
\end{eqnarray*}
com passo $h=0,2$, $h=0,02$, $h=0,002$ e $h=0,0002$ para obter aproximações para $y(2)$ e $z(2)$.
\end{exer}



\section{Exercícios finais}

\begin{exer} Considere o seguinte modelo para o crescimento de uma colônia de bactérias:
$$\frac{du}{dt}=\alpha u (A-u)$$
onde $u$ indica a densidade de bactérias em unidades arbitrárias na colônia e $\alpha$ e $A$ são constantes positivas.
Pergunta-se:
\begin{itemize}
\item[a)] Qual a solução quando a condição inicial $u(0)$ é igual a $0$ ou $A$?
\item[b)] O que acontece quando a condição inicial $u(0)$ é um número entre $0$ e $A$?
\item[c)] O que acontece quando a condição inicial $u(0)$ é um número negativo?
\item[d)] O que acontece quando a condição inicial $u(0)$ é um número positivo maior que A?
\item[e)] Se $A=10$ e $\alpha=1$ e $u(0)=1$, use métodos numéricos para obter tempo necessário para que a população dobre?
\item[f)] Se $A=10$ e $\alpha=1$ e $u(0)=4$, use métodos numéricos para obter tempo necessário para que a população dobre?
\end{itemize}
\end{exer}
\begin{resp}
  
Os valores exatos para os itens e e f são:$\frac{1}{10}\ln\left(\frac{9}{4}\right)$ e $\frac{1}{10}\ln\left(6\right)$    
  
\end{resp}

\begin{exer} Considere o seguinte modelo para a evolução da velocidade de um objeto em queda (unidades no SI):
$$v'=g-\alpha v^2$$
Sabendo que $g=9,8$ e $\alpha=10^{-2}$ e $v(0)=0$. Pede-se a velocidade ao tocar o solo, sabendo que a altura inicial era 100.

\end{exer}
\begin{resp}
  
O valor exato é $\sqrt{\frac{g}{\alpha}\left[1-e^{{-200\alpha}}\right]}$ em $t=\frac{1}{\sqrt{g\alpha}}\tanh^{-1}\left(\sqrt{1-e^{{-200\alpha}}}\right)$    
  
\end{resp}


\begin{exer} Considere o seguinte modelo para o oscilador não-linear de Van der Pol:
$$u''(t) - \alpha (A-u(t)^2)u'(t) + w_0^2u(t)=0$$
onde $A$, $\alpha$ e $w_0$ são constantes positivas.
\begin{itemize}
\item Encontre a frequência e a amplitude de oscilações quando $w_0=1$, $\alpha=.1$ e $A=10$. (Teste diversas condições iniciais)
\item Estude a dependência da frequência e da amplitude com os parâmetros  $A$, $\alpha$ e $w_0$. (Teste diversas condições iniciais)
\item Que diferenças existem entre esse oscilador não-linear e o oscilador linear?
\end{itemize}
\end{exer}

\begin{exer} Considere o seguinte modelo para um oscilador não-linear:
\begin{eqnarray*}
u''(t)-\alpha(A-z(t))u'(t)+w_0^2 u(t)&=&0\\
Cz'(t)+z(t)&=&u(t)^2
\end{eqnarray*}
onde $A$, $\alpha$, $w_0$ e $C$ são constantes positivas.
\begin{itemize}
\item Encontre a frequência e a amplitude de oscilações quando $w_0=1$, $\alpha=.1$, $A=10$ e $C=10$. (Teste diversas condições iniciais)
\item Estude a dependência da frequência e da amplitude com os parâmetros  $A$, $\alpha$, $w_0$ e $C$. (Teste diversas condições iniciais)
\end{itemize}
\end{exer}

\begin{exer} Considere o seguinte modelo para o controle de temperatura em um processo químico:
\begin{eqnarray*}
CT'(t)+T(t)&=&\kappa P(t)+T_{ext}\\
P'(t)&=&\alpha(T_{set}-T(t))
\end{eqnarray*}
onde $C$, $\alpha$ e $\kappa$ são constantes positivas e $P(t)$ indica o potência do aquecedor. Sabendo que $T_{set}$ é a temperatura desejada, interprete o funcionamento esse sistema de controle.
\begin{itemize}
\item Calcule a solução quando a temperatura externa $T_{ext}=0$, $T_{set}=1000$, $C=10$, $\kappa=.1$ e $\alpha=.1$. Considere condições iniciais nulas.
\item Quanto tempo demora o sistema para atingir a temperatura 900K?
\item Refaça os dois primeiros itens com $\alpha=0.2$ e $\alpha=1$
\item Faça testes para verificar a influência de $T_{ext}$, $\alpha$ e $\kappa$ na temperatura final.
\end{itemize}
\end{exer}

\begin{exer} Considere a equação do pêndulo dada por:
$$\frac{d^2\theta(t)}{dt^2}+\frac{g}{l}\sin(\theta(t))=0$$
onde $g$ é o módulo da aceleração da gravidade e $l$ é o comprimento da haste.
\begin{itemize}
\item Mostre analiticamente que a energia total do sistema dada por
$$\frac{1}{2}\left(\frac{d\theta(t)}{dt}\right)^2-\frac{g}{l}\cos(\theta(t))$$
é mantida constante.
\item Resolva numericamente esta equação para $g=9,8m/s^2$ e $l=1m$ e as seguintes condições iniciais:
\subitem $\theta(0)=0.5$ e $\theta'(0)=0$.
\subitem $\theta(0)=1.0$ e $\theta'(0)=0$.
\subitem $\theta(0)=1.5$ e $\theta'(0)=0$.
\subitem $\theta(0)=2.0$ e $\theta'(0)=0$.
\subitem $\theta(0)=2.5$ e $\theta'(0)=0$.
\subitem $\theta(0)=3.0$ e $\theta'(0)=0$.
\end{itemize}
Em todos os casos, verifique se o método numérico reproduz a lei de conservação de energia e calcule período e amplitude.
\end{exer}

\begin{exer} Considere o modelo simplificado de FitzHugh-Nagumo para o potencial elétrico sobre a membrana de um neurônio:
\begin{eqnarray*}
\frac{d V}{dt}& = &  V-V^3/3 - W +  I  \\
\frac{d W}{dt} & = & 0.08(V+0.7 - 0.8W)
\end{eqnarray*}
onde $I$ é a corrente de excitação.
\begin{itemize}
\item Encontre o único estado estacionário $\left(V_0,W_0\right)$ com $I=0$.
\item Resolva numericamente o sistema com condições iniciais dadas por $\left(V_0,W_0\right)$ e
\subitem $I=0$
\subitem $I=0.2$
\subitem $I=0.4$
\subitem $I=0.8$
\subitem $I=e^{-t/200}$
\end{itemize}
\end{exer}


\begin{exer} Considere o problema de valor inicial dado por
\begin{eqnarray*}
\frac{d u(t)}{dt} &=& -u(t) + e^{-t} \\
u(0)&=&0
\end{eqnarray*}
Resolva analiticamente este problema usando as técnicas elementares de equações diferenciais ordinárias. A seguir encontre aproximações numéricas usando os métodos de Euler, Euler modificado, Runge-Kutta Clássico e Adams-Bashforth de ordem 4 conforme pedido nos itens.
\begin{itemize}
\item[a)]  Construa uma tabela apresentando valores com 7 algarismos significativos para comparar a solução analítica com as aproximações numéricas produzidas pelos métodos sugeridos. Construa também uma tabela para o erro absoluto obtido por cada método numérico em relação à solução analítica. Nesta última tabela, expresse o erro com 2 algarismos significativos em formato científico. Dica: $format('e',8)$ para a segunda tabela.
\begin{center}
\begin{tabular}{|c|c|c|c|c|c|}
\hline
&0.5&1.0&1.5&2.0&2.5\\
\hline
Analítico&&&&&\\
\hline
Euler&&&&&\\
\hline
Euler modificado&&&&&\\
\hline
Runge-Kutta Clássico&&&&&\\
\hline
Adams-Bashforth ordem 4&&&&&\\
\hline
\end{tabular}
\end{center}

\begin{center}
\begin{tabular}{|c|c|c|c|c|c|}
\hline
&0.5&1.0&1.5&2.0&2.5\\
\hline
Euler&&&&&\\
\hline
Euler modificado&&&&&\\
\hline
Runge-Kutta Clássico&&&&&\\
\hline
Adams-Bashforth ordem 4&&&&&\\
\hline
\end{tabular}
\end{center}

\item[b)] Calcule o valor produzido por cada um desses método para $u(1)$ com passo $h=0.1$, $h=0.05$, $h=0.01$, $h=0.005$ e $h=0.001$. Complete a tabela com os valores para o erro absoluto encontrado.
\begin{center}
\begin{tabular}{|c|c|c|c|c|c|}
\hline
&0.1&0.05&0.01&0.005&0.001\\
\hline
Euler&&&&&\\
\hline
Euler modificado&&&&&   \\
\hline
Runge-Kutta Clássico&&&&&\\
\hline
Adams-Bashforth ordem 4&&&&&\\
\hline
\end{tabular}
\end{center}

\end{itemize}

\end{exer}


\begin{resp}
  
\begin{center}
\begin{tabular}{|c|c|c|c|c|c|}
\hline
&0.5&1.0&1.5&2.0&2.5\\
\hline
Analítico&  0.3032653 &   0.3678794  &  0.3346952  &  0.2706706 &   0.2052125  \\
\hline
Euler& 0.3315955 &   0.3969266 &   0.3563684 &   0.2844209  &  0.2128243\\
\hline
Euler modificado &0.3025634 &   0.3671929 &   0.3342207 &   0.2704083  &  0.2051058 \\
\hline
Runge-Kutta Clássico& 0.3032649  &  0.3678790  &  0.3346949  &  0.2706703  &  0.2052124\\
\hline
Adams-Bashforth ordem 4& 0.3032421  &  0.3678319 &   0.3346486  &  0.2706329  &  0.2051848  \\
\hline
\end{tabular}
\end{center}


\begin{center}
\begin{tabular}{|c|c|c|c|c|c|}
\hline
&0.5&1.0&1.5&2.0&2.5\\
\hline
Euler& 2.8D-02  &  2.9D-02  &  2.2D-02  &  1.4D-02 &   7.6D-03\\
\hline
Euler modificado& 7.0D-04  &  6.9D-04   & 4.7D-04 &   2.6D-04 &   1.1D-04\\
\hline
Runge-Kutta Clássico& 4.6D-07 &   4.7D-07    &3.5D-07  &  2.2D-07 &   1.2D-07\\
\hline
Adams-Bashforth ordem 4&  2.3D-05 &   4.8D-05  &  4.7D-05  &  3.8D-05  &  2.8D-05 \\
\hline
\end{tabular}
\end{center}

\begin{center}
\begin{tabular}{|c|c|c|c|c|c|}
\hline
&0.1&0.05&0.01&0.005&0.001\\
\hline
Euler&2.9D-02  &  5.6D-03 &   2.8D-03 &   5.5D-04 &   2.8D-04\\
\hline
Euler modificado&6.9D-04 &   2.5D-05  &  6.2D-06 &   2.5D-07 &   6.1D-08   \\
\hline
Runge-Kutta Clássico& 4.7D-07 &   6.9D-10 &   4.3D-11   & 6.8D-14  &  4.4D-15\\
\hline
Adams-Bashforth ordem 4&4.8D-05 &   9.0D-08 &   5.7D-09 &   9.2D-12 &   5.8D-13  \\
\hline
\end{tabular}
\end{center}    
  
\end{resp}

%\end{document} 
