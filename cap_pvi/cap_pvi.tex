%Este trabalho está licenciado sob a Licença Creative Commons Atribuição-CompartilhaIgual 3.0 Não Adaptada. Para ver uma cópia desta licença, visite http://creativecommons.org/licenses/by-sa/3.0/ ou envie uma carta para Creative Commons, PO Box 1866, Mountain View, CA 94042, USA.

\documentclass[main.tex]{subfiles}

\begin{document}

\chapter{Problemas de valor inicial}\index{problema de valor inicial}
Neste capítulo, desenvolveremos técnicas numérica para aproximar a solução de problemas de valor inical da forma
\begin{subequations}\label{PVI}
\begin{eqnarray}
y'(t)&=&f(y(t),t)\label{PVI_EDO}\\
y(t_0)&=&y_0 ~~ \hbox{(condição inicial)}.\label{PVI_CI}
\end{eqnarray}
\end{subequations}

A ingógnita de um problema de valor inicial é uma função que satisfaz a equação diferencial (\ref{PVI_EDO})  e a condição inicial (\ref{PVI_CI}).
\begin{ex}Considere o seguinte problema de valor inicial
\begin{subequations}
\begin{eqnarray}
y'(t)&=&2y(t),\\
y(t_0)&=&1.
\end{eqnarray}
\end{subequations}
A solução desta equação é dada pela função $y(t)=e^{2y}$ pois $y'(t)=2e^{2t}=2y(t)$ e $y(0)=e^0=1$.
\end{ex}


Muito problemas de valor inicial da forma (\ref{PVI}) não podem ser resolvidos exatamente, ou seja, sabe-se que a solução existe e é única, porém não podemos expressá-la em termos de funções elementares. Por isso é necessário calcular aproximações numéricas. Diversos métodos completamente diferentes estão disponíveis para aproximar uma função real. Aqui nos limitaremos a estudar métodos que se fundamentam em tentar calcular $y(t)$ em um conjunto finito de valores de $t$. Esse conjunto de valores para $t$ será denotado por  $\{t_i\}_{i=1}^N$, isto é $\{t_1, t_2, t_3,\ldots, t_N\}$ e calculamos o valor aproximado da função solução $y(t_i)$ em cada ponto da malha usando esquemas numéricos.


\section{Método de Euler}\index{método!de Euler}

Seja o problema de valor inicial

\begin{eqnarray*}
y'(t)&=&f(y(t),t)\\
y(0)&=&y_0 ~~ \hbox{condição inicial}
\end{eqnarray*}

Aproximamos a derivada $y'(t)$ por um esquema de primeira ordem do tipo
$$y'(t)=\frac{y(t+h)-y(t)}{h}+O(h),~~ h>0$$
assim temos
\begin{eqnarray*}
\frac{y(t+h)-y(t)}{h}&=&f(y(t),t)+O(h)\\
y(t+h)&=&y(t)+hf(y(t),t)+O(h^2)
\end{eqnarray*}

Definindo $y^{(k)}$ como uma aproximação para $y\left((k-1)h\right)$ e $t^{(k)}=(k-1)h$, temos
\begin{eqnarray*}
y^{(k+1)}&=&y^{(k)}+hf(y^{(k)},t^{(k)})\\
y^{(1)}&=&y_0~~ \hbox{condição inicial}
\end{eqnarray*}

\begin{ex} Considere o problema de valor inicial:
\begin{eqnarray*}
y'(t)&=&2y(t)\\
y(0)&=&1
\end{eqnarray*}
\end{ex}
Sabemos da teoria de equação diferenciais ordinárias, que a solução exata deste problema é única e é dada por
$$y(t)=e^{2t}.$$

O método de Euler aplicado a este problema produz o seguinte esquema:
\begin{eqnarray*}
y^{(k+1)}&=&y^{(k)}+2hy^{(k)}=(1+2h)y^{(k)}\\
y^{(1)}&=&1,
\end{eqnarray*}
cuja solução é dada por
$$y^{(k)}=(1+2h)^{k-1}.$$
Como $t=(k-1)h$, a solução aproximada é
$$y(t)\approx \tilde{y}(t)= (1+2h)^{\frac{t}{h}}.$$
Daí, vemos que se $h\to 0$, então
$$\tilde{y}(t)\to e^{2t}.$$



\begin{ex}{\label{ex_euler_1}} Considere o problema de valor inicial:
\begin{eqnarray*}
y'(t)&=&y(t)(1-y(t))\\
y(0)&=&1/2
\end{eqnarray*}
\end{ex}
É fácil encontrar a solução exata desta equação pois

\begin{eqnarray*}
\frac{dy(t)}{y(t)(1-y(t))}&=&dt\\
\left(\frac{1}{y}+\frac{1}{1-y}\right)dy&=&dt\\
\ln(y)-\ln(1-y)&=&t+C\\
\ln\left(\frac{y}{1-y}\right)&=&t+C\\
\frac{y}{1-y}&=&e^{t+C}\\
y&=&e^{t+C}(1-y)\\
y(1+e^{t+C})&=&e^{t+C}\\
y&=&\frac{e^{t+C}}{1+e^{t+C}}
\end{eqnarray*}
ainda $y(0)=\frac{e^C}{1+e^C}=1/2$, temos $e^C=1$ e, portanto,  $C=0$.

Assim, a solução exata é dada por $y=\frac{e^t}{1+e^{t}}$

O método de Euler produz o seguinte esquema iterativo:
\begin{eqnarray*}
y^{(k+1)}&=&y^{(k)}+h y^{(k)}(1-y^{(k)})\\
y^{(1)}&=&1/2
\end{eqnarray*}
Comparação
$$\begin{array}{|c|c|c|c|}
\hline
t &  \hbox{Exato} & \hbox{Euler}~~ h=0,1 & \hbox{Euler}~~ h=0,01\\
\hline
&&&\\
0 &  1/2 & 0,5 & 0,5\\
&&&\\
\hline
&&&\\
1/2 &   \frac{e^{1/2}}{1+e^{1/2}}\approx 0,6224593 & 0,6231476 & 0,6225316\\
&&&\\
\hline
&&&\\
1 &   \frac{e}{1+e}\approx 0,7310586 & 0,7334030 & 0,7312946\\
&&&\\
\hline
&&&\\
2 &   \frac{e^2}{1+e^2}\approx  0,8807971 & 0,8854273  & 0,8812533 \\
&&&\\
\hline
&&&\\
3 &   \frac{e^3}{1+e^3}\approx   0,9525741   & 0,9564754   & 0,9529609 \\
&&&\\
\hline
\end{array}
$$

\begin{ex} Resolva o problema de valor inicial
\begin{eqnarray*}
y'&=&-y+t\\
y(0)&=&1,
\end{eqnarray*}
cuja solução exata é $y(t)=2e^{-t}+t-1$.
\end{ex}
O esquema recursivo de Euler fica:
\begin{eqnarray*}
y^{(k+1)}&=&y({k})- hy({k})+ht^{(k)}\\
y(0)&=&1
\end{eqnarray*}

Comparação
$$\begin{array}{|c|c|c|c|}
\hline
t &  \hbox{Exato} & \hbox{Euler}~~ h=0,1 & \hbox{Euler}~~ h=0,01\\
\hline
&&&\\
0 &  1 & 1 & 1\\
&&&\\
\hline
&&&\\
1 &   2e^{-1}\approx 0,7357589 & 0,6973569   &   0,7320647  \\
&&&\\
\hline
&&&\\
2 &   2e^{-2}+1\approx  1,2706706 &  1,2431533    &   1,2679593     \\
&&&\\
\hline
&&&\\
3 &   2e^{-3}+2\approx 2,0995741  &  2,0847823 & 2,0980818   \\
&&&\\
\hline
\end{array}
$$

\newpage
\begin{ex} Resolva o seguinte sistema de equações diferenciais:
\begin{eqnarray*}
x'&=&-y\\
y'&=&x\\
x(0)&=&1\\
y(0)&=&0,\\
\end{eqnarray*}
cuja solução exata é $x(t)=\cos(t)$ e $y(t)=\sin(t)$.
\end{ex}
Escreva $$z(t)=\left[\begin{array}{c}x(t)\\y(t)\end{array}\right]$$  e temos
\begin{eqnarray*}
\left[\begin{array}{c}x^{(k+1)}\\y^{(k+1)}\end{array}\right]=\left[\begin{array}{c}x^{(k)}\\y^{(k)}\end{array}\right]+h\left[\begin{array}{c}-y^{(k)}\\x^{(k)}\end{array}\right]
\end{eqnarray*}

Equivalente a
\begin{eqnarray*}
x^{(k+1)}&=&x^{(k)}-hy^{(k)}\\
y^{(k+1)}&=&y^{(k)}+hx^{(k)}\\
\end{eqnarray*}

\begin{ex} Resolva o problema de valor inicial de segunda ordem dado por
\begin{eqnarray*}
y''+y'+y&=&\cos(t)\\
y(0)&=&1\\
y'(0)&=&0
\end{eqnarray*}
e compare com a solução exata para $h=0,1$ e $h=0,01$.
\end{ex}
Procedemos com a substituição $w=y'$, de forma que obtermos o sistema:
\begin{eqnarray*}
y'&=&w\\
w'&=&-w-y+\cos(t)\\
y(0)&=&1\\
w(0)&=&0
\end{eqnarray*}

\begin{eqnarray*}
y^{(k+1)}&=&y^{(k)}+hw^{(k)}\\
w^{(k+1)}&=&w^{(k)}-hw^{(k)}-hy^{(k)}+h\cos(t^{(k)})\\
y^{(1)}&=&1\\
w^{(1)}&=&0
\end{eqnarray*}

\subsection{Método de Euler melhorado}\index{método!de Euler melhorado}

No método de Euler, usamos a seguinte iteração:
\begin{eqnarray*}
y^{(k+1)}&=&y^{(k)}+hf(y^{(k)},t^{(k)})\\
y^{(1)}&=&y_i~~ \hbox{condição inicial}
\end{eqnarray*}
A ideia do método de Euler melhorado é substituir a declividade  $f(y^{(k)},t^{(k)})$ pela média aritmética entre $f(y^{(k)},t^{(k)})$ e $f(y^{(k+1)},t^{(k+1)})$.

No entanto, não dispomos do valor de $y^{(k+1)}$ pelo que aproximamos por
$$\tilde{y}^{(k+1)}=y^{(k)}+hf(y^{(k)},t^{(k)}).$$

\begin{eqnarray*}
\tilde{y}^{(k+1)}&=&y^{(k)}+hf(y^{(k)},t^{(k)})\\
y^{(k+1)}&=&y^{(k)}+\frac{h}{2}\left[f(y^{(k)},t^{(k)})+f(\tilde{y}^{(k+1)},t^{(k+1)})\right]\\
y^{(1)}&=&y_i~~ \hbox{condição inicial}
\end{eqnarray*}

\section*{Exercícios}

\begin{Exercise}
Refaça o exemplo \ref{ex_euler_1} via método de Euler melhorado.
\end{Exercise}

\subsection{Ordem de precisão}\index{ordem de precisão}

Considere o problema de valor inicial dado por
\begin{eqnarray*}
y'(t)&=&f(y(t),t)\\
y(0)&=&y_i
\end{eqnarray*}

No método de Euler, aproximamos a derivada $y'(t)$ por um esquema de primeira ordem do tipo
$$y'(t)=\frac{y(t+h)-y(t)}{h}+O(h),~~ h>0$$
de forma que tínhamos
\begin{eqnarray*}
y(t+h)&=&y(t)+hf(y(t),t)+O(h^2)
\end{eqnarray*}
Se fixarmos um instante de tempo $t=Nh$, temos:
\begin{eqnarray*}
y(t)&=&\left[y(0)+hf(y(0),0)+O(h^2)\right]+\left[y(h)+hf(y(h),h)+O(h^2)\right]\\
&+&\ldots \left[y(t-h)+hf(y(t-h),t-h)+O(h^2)\right]\\
&=&y^{k}+\sum_{j=0}^{N-1}O(h^2)=y^{k}+O(h)
\end{eqnarray*}
Por isso, o método de Euler é dito ter ordem global de precisão $h$.

\section{Métodos de Runge-Kutta}\index{método!de Runge-Kutta}

Os métodos de Runge-Kutta consistem em iterações do tipo:
$$y^{(k+1)}=y^{(k)}+w_1 k_1 + \ldots + w_n k_n$$
onde
\begin{eqnarray*}
k_1&=&hf(y^{(k)},t^{(k)})\\
k_2&=&hf(y^{(k)}+\alpha_{2,1}k_1,t^{(k)}+\beta_{2}h)\\
k_3&=&hf(y^{(k)}+\alpha_{3,1}k_1+\alpha_{3,2}k_2,t^{(k)}+\beta_{3}h)\\
&\vdots&\\
k_n&=&hf(y^{(k)}+\alpha_{n,1}k_1+\alpha_{n,2}k_2+\ldots \alpha_{n,n-1}k_{n-1},t^{(k)}+\beta_{n}h)\\
\end{eqnarray*}

Os coeficientes são escolhidos de forma que a expansão em Taylor de $y^{(k+1)}$ e $y^{(k)}+w_1 k_1 + \ldots + w_n k_n$ coincidam até ordem $n+1$.

\begin{ex} O método de Euler melhorado é um exemplo de Runge-Kutta de segunda ordem
$$y^{(n+1)}=y^{(n)}+\frac{k_1+k_2}{2}$$
onde $k_1=hf(y^{(n)},t^{(n)})$ e $k_2=hf(y^{(n)}+k_1,t^{(n)}+h)$
\end{ex}

\subsection{Métodos de Runge-Kutta - Quarta ordem}\index{método!de Runge-Kutta!de quarta ordem}

$$y^{(n+1)}=y^{(n)}+\frac{k_1+2k_2+2k_3+k_4}{6}$$
onde
\begin{eqnarray*}
k_1&=&hf(y^{(n)},t^{(n)})\\
k_2&=&hf(y^{(n)}+k_1/2,t^{(n)}+h/2)\\
k_3&=&hf(y^{(n)}+k_2/2,t^{(n)}+h/2)\\
k_4&=&hf(y^{(n)}+k_3,t^{(n)}+h)\\
\end{eqnarray*}
Este método tem ordem de truncamento local de quarta ordem. Uma discussão heurística usando método de Simpson pode ajudar a compreender os estranhos coeficientes:
\begin{eqnarray*}
y({t^{(n+1)}})-y({t^{(n)}})&=&\int_{t^{(n)}}^{t^{(n+1)}}f(y(s),s)ds \\
&\approx& \frac{h}{6}\left[ f\left(y(t^{(n)}),t^{(n)}\right)+4f\left(y(t^{(n)}+h/2),t^{(n)}+h/2\right)\right.\\
&+&\left.f\left(y(t^{(n)}+h),t^{(n)}+h\right)\right]\\
&\approx& \frac{k_1+4(\frac{k_2+k_3}{2})+k_4}{6}
\end{eqnarray*}
onde $k_1$ e $k_4$ representam as inclinações nos extremos e $k_2$ e $k_3$ são duas aproximações diferentes para a inclinação no meio do intervalo.


\section{Métodos de passo múltiplo - Adams-Bashforth}\index{método!de passo múltiplo!Adams-Bashforth}

O método de Adams-Bashforth consiste de um esquema recursivo do tipo:
$$y^{(n+1)}=y^{(n)}+\sum_{j=0}^k w_jf(y^{(n-j)},t^{(n-j)})$$

\begin{ex} Adams-Bashforth de segunda ordem
$$y^{(n+1)}=y^{(n)}+\frac{h}{2}\left[3f\left(y^{(n)},t^{(n)}\right)-f\left(y^{(n-1)},t^{(n-1)}\right)\right]$$
\end{ex}
\begin{ex} Adams-Bashforth de terceira ordem
$$y^{(n+1)}=y^{(n)}+\frac{h}{12}\left[23f\left(y^{(n)},t^{(n)}\right)-16f\left(y^{(n-1)},t^{(n-1)}\right)+5f\left(y^{(n-2)},t^{(n-2)}\right)\right]$$
\end{ex}
\begin{ex} Adams-Bashforth de quarta ordem
  \begin{equation*}
    \begin{split}
      y^{(n+1)} &= y^{(n)} + \frac{h}{24}\left[55f\left(y^{(n)},t^{(n)}\right)-59f\left(y^{(n-1)},t^{(n-1)}\right)\right.\\
        &+\left. 37f\left(y^{(n-2)},t^{(n-2)}\right)-9f\left(y^{(n-3)},t^{(n-3)}\right)\right]    
    \end{split}
  \end{equation*}
\end{ex}
Os métodos de passo múltiplo evitam os múltiplos estágios do métodos de Runge-Kutta, mas exigem ser "iniciados" com suas condições iniciais.

\section{Métodos de passo múltiplo - Adams-Moulton}\index{método de passo múltiplo!Adams-Moulton}

O método de Adams-Moulton consiste de um esquema recursivo do tipo:
$$y^{(n+1)}=y^{(n)}+\sum_{j=-1}^k w_jf(y^{(n-j)},t^{(n-j)})$$

\begin{ex} Adams-Moulton de quarta ordem
  \begin{equation*}
    \begin{split}
      y^{(n+1)} &= y^{(n)} + \frac{h}{24}\left[9f\left(y^{(n+1)},t^{(n+1)}\right) + 19f\left(y^{(n)},t^{(n)}\right) \right.\\
      &-\left. 5f\left(y^{(n-1)},t^{(n-1)}\right) + f\left(y^{(n-2)},t^{(n-2)}\right)\right]      
    \end{split}
  \end{equation*}
$$$$
\end{ex}
O método de Adams-Moulton é implícito, ou seja, exige que a cada passo, uma equação em $y^{(n+1)}$ seja resolvida.

\section{Estabilidade}\index{estabilidade}

Consideremos o seguinte problema de teste:
$$\left\{\begin{array}{rcl}y'&=&-\alpha y\\y(0)&=&1\end{array}\right.$$
cuja solução exata é dada por $y(t)=e^{-\alpha t}$.

Considere agora o método de Euler aplicado a este problema com passa $h$:
$$\left\{\begin{array}{rcl}y^{(k+1)}&=&y^{(k)}-\alpha h y^{(k)}\\y^{(1)}&=&1\end{array}\right.$$
A solução exata do esquema de Euler é dada por
$$y^{(k+1)}=(1-\alpha h)^{k}$$
e, portanto,
$$\tilde{y}(t)=y^{(k+1)}=(1-\alpha h)^{t/h}$$

Fixamos um $\alpha>0$, de forma que $y(t)\to 0$. Mas observamos que $\tilde{y}(t)\to 0$ somente quando $|1-\alpha h|<1$ e solução positivas somente quando $\alpha h<1$.

{\bf Conclusão:} Se o passo $h$ for muito grande, o método pode se tornar instável, produzindo solução espúrias.

\section*{Exercícios}

\begin{Exercise} Resolva o problema 1 pelos diversos métodos e verifique heuristicamente a estabilidade para diversos valores de $h$.
\end{Exercise}

\section*{Exercícios finais}

\begin{Exercise} Considere o seguinte modelo para o crescimento de uma colônia de bactérias:
$$\frac{dy}{dt}=\alpha y (A-y)$$
onde $y$ indica a densidade de bactérias em unidades arbitrárias na colônia e $\alpha$ e $A$ são constantes positivas.
Pergunta-se:
\begin{itemize}
\item[a)] Qual a solução quando a condição inicial $y(0)$ é igual a $0$ ou $A$?
\item[b)] O que acontece quando a condição inicial $y(0)$ é um número entre $0$ e $A$?
\item[c)] O que acontece quando a condição inicial $y(0)$ é um número negativo?
\item[d)] O que acontece quando a condição inicial $y(0)$ é um número positivo maior que A?
\item[e)] Se $A=10$ e $\alpha=1$ e $y(0)=1$, use métodos numéricos para obter tempo necessário para que a população dobre?
\item[f)] Se $A=10$ e $\alpha=1$ e $y(0)=4$, use métodos numéricos para obter tempo necessário para que a população dobre?
\end{itemize}
\end{Exercise}
\begin{Answer}
  \begin{tiny}
Os valores exatos para os itens e e f são:$\frac{1}{10}\ln\left(\frac{9}{4}\right)$ e $\frac{1}{10}\ln\left(6\right)$    
  \end{tiny}
\end{Answer}

\begin{Exercise} Considere o seguinte modelo para a evolução da velocidade de um objeto em queda (unidades no SI):
$$v'=g-\alpha v^2$$
Sabendo que $g=9,8$ e $\alpha=10^{-2}$ e $v(0)=0$. Pede-se a velocidade ao tocar o solo, sabendo que a altura inicial era 100.

\end{Exercise}
\begin{Answer}
  \begin{tiny}
O valor exato é $\sqrt{\frac{g}{\alpha}\left[1-e^{{-200\alpha}}\right]}$ em $t=\frac{1}{\sqrt{g\alpha}}\tanh^{-1}\left(\sqrt{1-e^{{-200\alpha}}}\right)$    
  \end{tiny}
\end{Answer}


\begin{Exercise} Considere o seguinte modelo para o oscilador não-linear de Van der Pol:
$$y''(t) - \alpha (A-y(t)^2)y'(t) + w_0^2y(t)=0$$
onde $A$, $\alpha$ e $w_0$ são constantes positivas.
\begin{itemize}
\item Encontre a frequência e a amplitude de oscilações quando $w_0=1$, $\alpha=.1$ e $A=10$. (Teste diversas condições iniciais)
\item Estude a dependência da frequência e da amplitude com os parâmetros  $A$, $\alpha$ e $w_0$. (Teste diversas condições iniciais)
\item Que diferenças existem entre esse oscilador não-linear e o oscilador linear?
\end{itemize}
\end{Exercise}

\begin{Exercise} Considere o seguinte modelo para um oscilador não-linear:
\begin{eqnarray*}
y''(t)-\alpha(A-z(t))y'(t)+w_0^2 y(t)&=&0\\
Cz'(t)+z(t)&=&y(t)^2
\end{eqnarray*}
onde $A$, $\alpha$, $w_0$ e $C$ são constantes positivas.
\begin{itemize}
\item Encontre a frequência e a amplitude de oscilações quando $w_0=1$, $\alpha=.1$, $A=10$ e $C=10$. (Teste diversas condições iniciais)
\item Estude a dependência da frequência e da amplitude com os parâmetros  $A$, $\alpha$, $w_0$ e $C$. (Teste diversas condições iniciais)
\end{itemize}
\end{Exercise}

\begin{Exercise} Considere o seguinte modelo para o controle de temperatura em um processo químico:
\begin{eqnarray*}
CT'(t)+T(t)&=&\kappa P(t)+T_{ext}\\
P'(t)&=&\alpha(T_{set}-T(t))
\end{eqnarray*}
onde $C$, $\alpha$ e $\kappa$ são constantes positivas e $P(t)$ indica o potência do aquecedor. Sabendo que $T_{set}$ é a temperatura desejada, interprete o funcionamento esse sistema de controle.
\begin{itemize}
\item Calcule a solução quando a temperatura externa $T_{ext}=0$, $T_{set}=1000$, $C=10$, $\kappa=.1$ e $\alpha=.1$. Considere condições iniciais nulas.
\item Quanto tempo demora o sistema para atingir a temperatura 900K?
\item Refaça os dois primeiros itens com $\alpha=0.2$ e $\alpha=1$
\item Faça testes para verificar a influência de $T_{ext}$, $\alpha$ e $\kappa$ na temperatura final.
\end{itemize}
\end{Exercise}

\begin{Exercise} Considere a equação do pêndulo dada por:
$$\frac{d^2\theta(t)}{dt^2}+\frac{g}{l}\sin(\theta(t))=0$$
onde $g$ é o módulo da aceleração da gravidade e $l$ é o comprimento da haste.
\begin{itemize}
\item Mostre analiticamente que a energia total do sistema dada por
$$\frac{1}{2}\left(\frac{d\theta(t)}{dt}\right)^2-\frac{g}{l}\cos(\theta(t))$$
é mantida constante.
\item Resolva numericamente esta equação para $g=9,8m/s^2$ e $l=1m$ e as seguintes condições iniciais:
\subitem $\theta(0)=0.5$ e $\theta'(0)=0$.
\subitem $\theta(0)=1.0$ e $\theta'(0)=0$.
\subitem $\theta(0)=1.5$ e $\theta'(0)=0$.
\subitem $\theta(0)=2.0$ e $\theta'(0)=0$.
\subitem $\theta(0)=2.5$ e $\theta'(0)=0$.
\subitem $\theta(0)=3.0$ e $\theta'(0)=0$.
\end{itemize}
Em todos os casos, verifique se o método numérico reproduz a lei de conservação de energia e calcule período e amplitude.
\end{Exercise}

\begin{Exercise} Considere o modelo simplificado de FitzHugh-Nagumo para o potencial elétrico sobre a membrana de um neurônio:
\begin{eqnarray*}
\frac{d V}{dt}& = &  V-V^3/3 - W +  I  \\
\frac{d W}{dt} & = & 0.08(V+0.7 - 0.8W)
\end{eqnarray*}
onde $I$ é a corrente de excitação.
\begin{itemize}
\item Encontre o único estado estacionário $\left(V_0,W_0\right)$ com $I=0$.
\item Resolva numericamente o sistema com condições iniciais dadas por $\left(V_0,W_0\right)$ e
\subitem $I=0$
\subitem $I=0.2$
\subitem $I=0.4$
\subitem $I=0.8$
\subitem $I=e^{-t/200}$
\end{itemize}
\end{Exercise}


\begin{Exercise} Considere o problema de valor inicial dado por
\begin{eqnarray*}
\frac{d u(t)}{dt} &=& -u(t) + e^{-t} \\
u(0)&=&0
\end{eqnarray*}
Resolva analiticamente este problema usando as técnicas elementares de equações diferenciais ordinárias. A seguir encontre aproximações numéricas usando os métodos de Euler, Euler modificado, Runge-Kutta Clássico e Adams-Bashforth de ordem 4 conforme pedido nos itens.
\begin{itemize}
\item[a)]  Construa uma tabela apresentando valores com 7 algarismos significativos para comparar a solução analítica com as aproximações numéricas produzidas pelos métodos sugeridos. Construa também uma tabela para o erro absoluto obtido por cada método numérico em relação à solução analítica. Nesta última tabela, expresse o erro com 2 algarismos significativos em formato científico. Dica: $format('e',8)$ para a segunda tabela.
\begin{center}
\begin{tabular}{|c|c|c|c|c|c|}
\hline
&0.5&1.0&1.5&2.0&2.5\\
\hline
Analítico&&&&&\\
\hline
Euler&&&&&\\
\hline
Euler modificado&&&&&\\
\hline
Runge-Kutta Clássico&&&&&\\
\hline
Adams-Bashforth ordem 4&&&&&\\
\hline
\end{tabular}
\end{center}

\begin{center}
\begin{tabular}{|c|c|c|c|c|c|}
\hline
&0.5&1.0&1.5&2.0&2.5\\
\hline
Euler&&&&&\\
\hline
Euler modificado&&&&&\\
\hline
Runge-Kutta Clássico&&&&&\\
\hline
Adams-Bashforth ordem 4&&&&&\\
\hline
\end{tabular}
\end{center}

\item[b)] Calcule o valor produzido por cada um desses método para $u(1)$ com passo $h=0.1$, $h=0.05$, $h=0.01$, $h=0.005$ e $h=0.001$. Complete a tabela com os valores para o erro absoluto encontrado.
\begin{center}
\begin{tabular}{|c|c|c|c|c|c|}
\hline
&0.1&0.05&0.01&0.005&0.001\\
\hline
Euler&&&&&\\
\hline
Euler modificado&&&&&   \\
\hline
Runge-Kutta Clássico&&&&&\\
\hline
Adams-Bashforth ordem 4&&&&&\\
\hline
\end{tabular}
\end{center}

\end{itemize}

\end{Exercise}


\begin{Answer}
  \begin{tiny}
\begin{center}
\begin{tabular}{|c|c|c|c|c|c|}
\hline
&0.5&1.0&1.5&2.0&2.5\\
\hline
Analítico&  0.3032653 &   0.3678794  &  0.3346952  &  0.2706706 &   0.2052125  \\
\hline
Euler& 0.3315955 &   0.3969266 &   0.3563684 &   0.2844209  &  0.2128243\\
\hline
Euler modificado &0.3025634 &   0.3671929 &   0.3342207 &   0.2704083  &  0.2051058 \\
\hline
Runge-Kutta Clássico& 0.3032649  &  0.3678790  &  0.3346949  &  0.2706703  &  0.2052124\\
\hline
Adams-Bashforth ordem 4& 0.3032421  &  0.3678319 &   0.3346486  &  0.2706329  &  0.2051848  \\
\hline
\end{tabular}
\end{center}


\begin{center}
\begin{tabular}{|c|c|c|c|c|c|}
\hline
&0.5&1.0&1.5&2.0&2.5\\
\hline
Euler& 2.8D-02  &  2.9D-02  &  2.2D-02  &  1.4D-02 &   7.6D-03\\
\hline
Euler modificado& 7.0D-04  &  6.9D-04   & 4.7D-04 &   2.6D-04 &   1.1D-04\\
\hline
Runge-Kutta Clássico& 4.6D-07 &   4.7D-07    &3.5D-07  &  2.2D-07 &   1.2D-07\\
\hline
Adams-Bashforth ordem 4&  2.3D-05 &   4.8D-05  &  4.7D-05  &  3.8D-05  &  2.8D-05 \\
\hline
\end{tabular}
\end{center}

\begin{center}
\begin{tabular}{|c|c|c|c|c|c|}
\hline
&0.1&0.05&0.01&0.005&0.001\\
\hline
Euler&2.9D-02  &  5.6D-03 &   2.8D-03 &   5.5D-04 &   2.8D-04\\
\hline
Euler modificado&6.9D-04 &   2.5D-05  &  6.2D-06 &   2.5D-07 &   6.1D-08   \\
\hline
Runge-Kutta Clássico& 4.7D-07 &   6.9D-10 &   4.3D-11   & 6.8D-14  &  4.4D-15\\
\hline
Adams-Bashforth ordem 4&4.8D-05 &   9.0D-08 &   5.7D-09 &   9.2D-12 &   5.8D-13  \\
\hline
\end{tabular}
\end{center}    
  \end{tiny}
\end{Answer}

\end{document} 