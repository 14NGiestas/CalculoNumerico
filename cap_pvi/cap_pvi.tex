%Este está licenciado sob a Licença Creative Commons Atribuição-CompartilhaIgual 3.0 Não Adaptada. Para ver uma cópia desta licença, visite http://creativecommons.org/licenses/bu-sa/3.0/ ou envie uma carta para Creative Commons, PO Box 1866, Mountain View, CA 94042, USA.

%\documentclass[main.tex]{subfiles}
%\begin{document}

\chapter{Problemas de valor inicial}\index{problema de valor inicial}
Neste capítulo, vamos estudar metodologias numéricas para aproximar a solução de problema de valor inicial (problema de valor inicial) para equações diferenciais ordinárias. Primeiramente, daremos atenção aos problemas de primeira ordem e, depois, mostraremos que estas técnicas podem ser estendidas para problemas e sistemas de ordem superior. Considere um problema de valor inicial de primeira ordem dado por:
\begin{subequations}\label{eq:PVI}
\begin{eqnarray}
  u'(t) &=& f(t, u(t)),~~t>t^{(1)}\label{eq:PVI_EDO}\\
  u(t^{(1)}) &=& a ~~ \text{(condição inicial)}.\label{eq:PVI_CI}
\end{eqnarray}
\end{subequations}

A incógnita de um problema de valor inicial é uma função que satisfaz a equação diferencial \eqref{eq:PVI_EDO} e a condição inicial \eqref{eq:PVI_CI}.

Considere os próximos três exemplos:
\begin{ex} O seguinte problema é linear não homogêneo:
\begin{eqnarray}
   u'(t) &=t\\
   u(0) &= 2
\end{eqnarray}
\end{ex}

\begin{ex} O seguinte problema é linear homogêneo:
\begin{eqnarray}
   u'(t) &=u(t)\\
            u(0) &= 1
\end{eqnarray}
\end{ex}

\begin{ex} O seguinte problema é não linear e não homogêneo:
\begin{eqnarray}
   u'(t) &=&\sin(u(t)^2+\sin(t))\\
            u(0) &=& a
\end{eqnarray}
\end{ex}

A solução do primeiro exemplo é $u(t)=t^2/2+2$ pois satisfaz a equação diferencial e a condição inicial. A solução do segundo também é facilmente obtida: $u(t)=e^t$. Porém como podemos resolver o terceiro problema?

Para muitos problemas de valor inicial da forma \eqref{eq:PVI}, não é possível encontrar uma expressão analítica fechada, ou seja, sabe-se que a solução existe e é única, porém não podemos expressá-la em termos de funções elementares. Por isso é necessário calcular aproximações numéricas para a solução. 

Existem uma enorme família de metodologias para construir soluções numéricas para problemas de valor inicial. Aqui, vamos nos limitar a estudar métodos que aproximam $u(t)$ em um conjunto finito de valores de $t$. Este conjunto de valores será chamado de \emph{malha} e será denotado por  $\{t^{(i)}\}_{i=1}^N=\{t^{(1)}, t^{(2)}, t^{(3)},\ldots, t^{(N)}\}$. Desta forma, aproximamos a solução $u(t^{(i)})$ por $u^{(i)}$ em cada ponto da malha usando diferentes esquemas numéricos.

%%%%%%%%%%%%%%%%%%%%
% python
%%%%%%%%%%%%%%%%%%%%
\ifispython
Nos códgos em \verb+Python+ apresentados neste capítulo, assumiremos que as seguintes bibliotecas e módulos estão importados:
\begin{verbatim}
from __future__ import division
import numpy as np
from numpy import linalg
import matplotlib.pyplot as plt
\end{verbatim}
\fi
%%%%%%%%%%%%%%%%%%%%e

\section{Rudimentos da teoria de problemas de valor inicial}
Uma questão fundamental no estudo dos problemas de valor iniciais consiste em analisar se um dado problema é um problema \emph{bem posto}. Ou seja,
\begin{itemize}
 \item Existe uma solução para o problema de valor inicial?
 \item A solução é única?
 \item A solução do problema de valor inicial é pouco sensível a pequenas perturbações nas condições iniciais?
\end{itemize}

A fim de responder tais questões, precisamos definir o conceito de função Lipschitz contínua, ou simplesmente, função Lipschitz \index{função!Lipschitz}
\begin{defn}
Uma função $f(t, u)$ é Lipschitz contínua em um intervalo $I$ em $u$ se existe uma constante $L$, tal que $\forall t \in [a, b]$ e $u, v \in \mathbb R$,
$$ |f(t, u)-f(t, v)| \leq L|u-v|,~~\forall t\in I. $$
\end{defn}

O seguinte resultado estabelece a existência e unicidade de solução para determinada classe de problemas de valor inicial:
\begin{teo}[Teorema de Picard-Lindelöf] Seja $f(t, u)$ contínua em $t$ e Lipschitz em $u$. Então o seguinte problema de valor inicial\index{teorema!de Picard-Lindelöf}
\begin{equation}\label{eq:pvi_picard}
  \begin{split}
    u'(t)  &= f(t, u(t)), \\
    u(t^{(1)}) &= a,
  \end{split}
\end{equation}
Admite uma única solução em um intervalo $[t^{(1)},t^{(f)})$ com $t^{(f)}>t^{(1)}.$
\end{teo}

% \begin{defn}
%   \emph{Estabilidade dinâmica} refere-se a propriedade de pequenas perturbações sobre o estado inicial de um sistema gerarem pequenas variações no estado final deste sistema (haverá decaimento nas variações, ou pelo menos não crescimento, quanto $t$ cresce).
% \end{defn}

\begin{teo}[Dependência contínua na condição inicial]
Se $u(t)$ e $v(t)$ são soluções do problema de valor inicial \eqref{eq:pvi_picard}, isto é, com $f(t, u)$ contínua em $t$ e Lipschitz em $u$ Lipschitz com $u(a)=u^{(1)}$, $v(a)=v^{(1)}$, então
$$ |u(t)-v(t)| \leq  e^{L(t-t^{(1)})}|u^{(1)}-v_1|.$$
\end{teo}

\subsection*{Exercícios resolvidos}

\begin{exeresol} A função $f(t,u)=\sqrt{u},~~u\geq 0$ não é uma função Lipschitz em $u$, pois
$$\lim_{u\to 0+} \frac{|f(t,u)-f(t,0)|}{|u-0|}=\lim_{u\to 0+} \frac{\sqrt{u}}{u}=\lim_{u\to 0+} \frac{1}{\sqrt{u}}=\infty$$
 Mostre que o seguinte problema de valor inicial não admite solução única:
\begin{eqnarray}
\frac{du}{dt} &=&\sqrt{u},~~u>0,\\
u(0) &=& 0.
\end{eqnarray}
\end{exeresol}
\begin{resol}
 A função identicamente nula, $u(t)=0$, satisfaz a equação diferencial e a condição de contorno, logo é uma solução do problema de valor inicial. No entanto, a função\footnote{Esta solução pode ser obtida por separação de variáveis.} $u(t)=\frac{t^2}{4}$ satisfaz a condição inicial, pois $u(0)=0$ e a equação diferencial pois $\frac{du}{dt}=\frac{t}{2}=\sqrt{\frac{t^2}{4}}$.
 
 De fato, qualquer função do tipo
 $$u(t)=\left\{
 \begin{array}{ll}
  0,&0\leq t \leq t_0\\
  \frac{(t-t_0)^2}{4},& t >t_0\\
 \end{array}
 \right.$$
 é solução do problema de valor inicial dado.
\end{resol}


\section{Método de Euler}\index{método!de Euler}\label{sec:euler}
Nesta seção, contruiremos o mais simples dos métodos para resolver problemas de valor inicial: o método de Euler com passo constante.\index{Passo} Por passo constante, queremos dizer que os pontos da malha estão todos igualmente espaçados, isto é:
$$t^{(i)}=(i-1)h,~~~i=1,2,\ldots,N.$$
onde $h$ é passo, ou seja, a distância entre dois pontos da malha.

Considere então o problema de valor inicial dado por:
\begin{equation}\label{eq:EDO1}
  \begin{split}
    u'(t)  &= f(t,u(t)),~~t>t^{(1)} \\
    u(t^{(1)}) &= a.    
  \end{split}
\end{equation}
Ao invés de tentar solucionar o problema para qualquer $t>t^{(1)}$, iremos aproximar $u(t)$ em $t=t^{(2)}$.

Integrando \eqref{eq:EDO1} de $t^{(1)}$ até $t^{(2)}$, obtemos:
\begin{eqnarray}
  \int_{t^{(1)}}^{t^{(2)}} u'(t) \;dt &=& \int_{t^{(1)}}^{t^{(2)}} f(t,u(t)) \; dt\\
  u(t^{(2)})-u(t^{(1)})               &=& \int_{t^{(1)}}^{t^{(2)}} f(t,u(t)) \; dt\\
  u(t^{(1)})                      &=& u(t^{(1)}) +  \int _{t^{(1)}}^{t^{(2)}} f(t,u(t)) \; dt
\end{eqnarray}

Seja $u_n$ a aproximação de $u(t_n)$. Para obter o método numérico mais simples aproximamos $f$ em $[t^{(1)},t^{(2)}]$ pela função constante $f(t,u(t)) \approx  f(t^{(1)},u^{(1)})$,
\begin{eqnarray}
  u^{(2)} &=&  u^{(1)} +   f(t^{(1)},u^{(1)}) \int _{t^{(1)}}^{t^{(2)}}  \; dt \\
  u^{(2)} &=&  u^{(1)} +   f(t^{(1)},u^{(1)}) (t^{(2)}-t^{(1)}) \\
  u^{(2)} &=&  u^{(1)} + h f(t^{(1)},u^{(1)})
\end{eqnarray}

Este procedimento pode ser repetido para  $t^{(3)}$, $t^{(4)}$, $\ldots$, obtendo, assim, o chamado \emph{método de Euler}:
\begin{equation}\label{eq:euler}
  \begin{split}
    u^{(n+1)}&= u^{(n)} + h\;f(t^{(n)},u^{(n)}),\\
    u^{(1)}&= u^{(1)}=u(t^{(1)})~ \hbox{(condição inicial)}.
  \end{split}
\end{equation}


\begin{ex}
Considere o problema de valor inicial
\begin{eqnarray*}
  u'(t)&=&2u(t)\\
  u(0)&=&1
\end{eqnarray*}
cuja solução é $u(t)=e^{2t}$. O método de Euler aplicado a este problema produz o  esquema:
\begin{equation}\label{eq:exemplo_y_2y_euler}
  \begin{split}
    u^{(k+1)}&= u^{(k)}+2hu^{(k)}=(1+2h)u^{(k)}\\
    u^{(1)}&= 1,
  \end{split}
\end{equation}
Suponha que queremos calcular o valor aproximado de $u(1)$ com $h=0,2$. Então os pontos $t^{(1)}=0$, $t^{(2)}=0,2$, $t^{(3)}=0,4$, $t^{(4)}=0,6$, $t^{(5)}=0,8$ e $t^{(6)}=1,0$ formam os seis pontos da malha. As aproximações para a solução nos pontos da malha usando o método de Euler são:
\begin{eqnarray*}
  u(0)  &\approx &u^{(1)}=1\\
  u(0,2)&\approx &u^{(2)}=(1+2h) u^{(1)}=1,4 u^{(1)}=1,4\\
  u(0,4)&\approx &u^{(3)}=1,4 u^{(2)}=1,96\\
  u(0,6)&\approx &u^{(4)}=1,4 u^{(3)}=2,744\\
  u(0,8)&\approx &u^{(5)}=1,4 u^{(4)}=3,8416\\
  u(1,0)&\approx &u^{(6)}=1,4 u^{(5)}=5,37824
\end{eqnarray*}
Essa aproximação é bem grosseira quando comparamos com a solução do problema em $t=1$: $u(1)=e^{2}\approx 7,38906$. Não obstante, se tivéssemos escolhido um passo menor, teríamos obtido uma aproximação melhor. Veja tabela abaixo com valores obtidos com diferentes valores de passo $h$.

\begin{tabular}{|l|l|l|l|l|l|l|l|}%\label{pvi:tab_euler}
\hline
   h&$10^{-1}$&$10^{-2}$&$10^{-3}$&$10^{-4}$&$10^{-5}$&$10^{-6}$&$10^{-7}$\\
   \hline
   $u^{(N)}$& 6,1917 &  6,7275 &  7,0400 &  7,2096  & 7,2980 &  7,3432  & 7,3660\\
   \hline
\end{tabular}
 
De fato, podemos mostrar que quando $h$ se aproxima de $0$, a solução aproximada via método de Euler converge para a solução exata $e^2$. Para isto, basta observar que a solução da relação de recorrência \eqref{eq:exemplo_y_2y_euler} é dada por
 $$u^{(k)}=(1+2h)^{k-1}.$$
 Como $t^{(k)}=(k-1)h$ e queremos a solução em $t=2$, a solução aproximada pelo método de Euler com passo $h$ em é dada por:
 $$u^{(k)}= (1+2h)^{k-1}= (1+2h)^{\frac{2}{h}}.$$
Aplicando o limite $h\to 0+$, temos:
  $$\lim_{h\to 0+} (1+2h)^{\frac{2}{h}}= e^{2}.$$

%%%%%%%%%%%%%%%%%%%%
% python
%%%%%%%%%%%%%%%%%%%%
\ifispython
Em \verb+Python+, podemos computar a solução numérica deste problema de valor inicial via o método de Euler com o seguite código:
\begin{verbatim}
#define f(t,u)
def f(t,u):
    return 2*u

#tamanho e num. de passos
h = 0.2
N = 6

#cria vetor t e u
t = np.empty(N)
u = np.copy(t)

#C.I.
t[0] = 0
u[0] = 1

#iteracoes
for i in np.arange(N-1):
    t[i+1] = t[i] + h
    u[i+1] = u[i] + h*f(t[i],u[i])

#imprime
for i,tt in enumerate(t):
    print("%1.1f %1.4f" % (t[i],u[i]))
\end{verbatim}
\fi
%%%%%%%%%%%%%%%%%%%%
\end{ex}

\subsection*{Exercícios Resolvidos}

\begin{exeresol}\label{exeresol:exeresol1} Aproxime a solução do problema de valor inicial
\begin{eqnarray}
     u'(t)&=& -0,5u(t)+2+t\\
            u(0) &=&  8
\end{eqnarray}
Usando os seguinte passos: $h=10^{-1}$, $h=10^{-2}$, $h=10^{-3}$, $h=10^{-4}$ e $h=10^{-5}$ e compare a solução aproximada em $t=1$ com a solução exata dada por:
\begin{equation}
     u(t) = 2t+8e^{-t/2} \Longrightarrow u(1)=2+8e^{-1/2} \approx 6,85224527770107
\end{equation}
\end{exeresol}
\begin{resol} Primeramente itentificamos $f(t,u)=-0,5u+2+t$ e construímos o processo iterativo do método de Euler: 
\begin{eqnarray}
  u^{(n+1)}&=&u^{(n)} + h( -0,5u^{(n)}+2+t^{(n)}),~~n=1,2,3,\ldots\\
  u^{(1)}&=&8
\end{eqnarray}

\ifisscilab
O seguinte código pode ser usado para implementar no \verb+Scilab+ a recursão acima:
\begin{verbatim}
function u=euler(h,Tmax)
  u= 8;
  t= 0;
  itmax = Tmax/h;
  for n=1:itmax
    t= (n-1)*h;
    u= u + h*(-0.5*u+2+t);
   end
endfunction
\end{verbatim}
o qual pode ser invocado da seguinte forma:
\begin{verbatim}
-->euler(1e-1,1)
 ans  =
 
    6.7898955 
\end{verbatim}
Podemos construir um vetor com as cinco soluções da seguinte forma:
\begin{verbatim}
 -->S=[euler(1e-1,1) euler(1e-2,1) euler(1e-3,1) euler(1e-4,1) euler(1e-5,1)]
 S  =
 
    6.7898956   6.846163    6.851639    6.852185    6.8522392  
\end{verbatim}

\fi

\ifisoctave
O seguinte código pode ser usado para implementar no \verb+Octave+ a recursão acima:
\begin{verbatim}
function u=euler(h,Tmax)
  u= 8;
  t= 0;
  itmax = Tmax/h;
  for n=1:itmax
    t= (n-1)*h;
    u= u + h*(-0.5*u+2+t);
   end	
endfunction
\end{verbatim}
o qual pode ser invocado da seguinte forma:
\begin{verbatim}
>> euler(1e-1,1)
ans =  6.8799
\end{verbatim}
Podemos construir um vetor com as cinco soluções da seguinte forma:
\begin{verbatim}
 >> S=[euler(1e-1,1) euler(1e-2,1) euler(1e-3,1) euler(1e-4,1) euler(1e-5,1)]
S =

   6.7899   6.8462   6.8516   6.8522   6.8522
\end{verbatim}
\fi


\ifispython
O seguinte código pode ser usado para implementar no \verb+Python+ a recursão acima:

\begin{verbatim}
def euler(h,Tmax):
	u=8
  	itmax = Tmax/h;
	for i in np.arange(0,itmax):
		t=i*h
		k1 = -0.5*u+2+t
		u = u + h*k1
	return u

	
sol_exata = 2+8*np.exp(-.5)
h=1e-1
for i in np.arange(1,5):
	sol_euler=euler(h,1);
	erro_relativo = np.fabs((sol_euler-sol_exata)/sol_exata)
	print("h=%1.0e - u(1) =~ %1.7f - erro_relativo = %1.1e" % (h, sol_euler, erro_relativo) )
	h=h/10
\end{verbatim}
\fi
A seguinte tabela resume os resultados obtidos:
\begin{center}
 \begin{tabular}{|l|l|l|l|l|l|l|l|}%\label{pvi:tab_euler}
\hline
   h&$10^{-1}$&$10^{-2}$&$10^{-3}$&$10^{-4}$&$10^{-5}$\\
   \hline
  Euler & 6,7898955 &  6,8461635  &  6,8516386  &  6,8521846  &  6,8522392  \\
   \hline
   $\varepsilon_{rel}$ &9,1e-03 &  8,9e-04  & 8,9e-05&   8,9e-06 &  8,9e-07\\
    \hline
  \end{tabular}
\end{center}


% Veja o gráfico da solução para $h=1, 0.5, 0.1, 0.05$:
% \begin{figure}
% \includegraphics[width=\textwidth]{euler.eps}
% \end{figure}

\end{resol}


Vamos agora, analisar o desempenho do método de Euler usando um exemplo mais complicado, porém ainda simples suficiente para que possamos obter a solução exata:
\begin{ex}\label{ex:ex_euler_1}
Considere o problema de valor inicial relacionado à equação logística\index{equação!logística}:
\begin{equation}\label{eq:logistica}
  \begin{split}
    u'(t)&= u(t)(1-u(t))\\
    u(0)&= 1/2
  \end{split}
\end{equation}
\end{ex}
Podemos obter a solução exata desta equação usando o método de separação de variáveis\index{método!de separação de variáveis} e o método das frações parciais\index{método das frações parciais}. Para tal escrevemos:
\begin{equation*}
\frac{du(t)}{u(t)(1-u(t))}=dt
\end{equation*}
O termo $\frac{1}{u(t)(1-u(t))}$ pode ser decomposto em frações parciais como $\frac{1}{u}+\frac{1}{1-u}$ e chegamos na seguinte equação diferencial:
\begin{equation*}
\left(\frac{1}{u(t)}+\frac{1}{1-u(t)}\right)du=dt.
\end{equation*}
Integrando termo-a-termo, temos a seguinte equação algébrica relacionando $u(t)$ e $t$:
\begin{equation*}
\ln(u(t))-\ln\left(1-u(t)\right)=t+C
\end{equation*}
Onde $C$ é a constante de integração, que é definida pela condição inicial, isto é, $u=1/2$ em $t=0$. Substituindo, temos $C=0$. O que resulta em:
\begin{equation*}
\ln\left(\frac{u(t)}{1-u(t)}\right)=t
\end{equation*}
Equivalente a
\begin{equation*}
\frac{u(t)}{1-u(t)}=e^{t} \Longrightarrow u(t)=(1-u(t))e^{t} \Longrightarrow (1+e^t)u(t)=e^{t}
\end{equation*}
E, finalmente, encontramos a solução exata dada por $u(t)=\frac{e^t}{1+e^{t}}$.

Vejamos, agora, o esquema iterativo produzido pelo método de Euler:
\begin{eqnarray*}
u^{(k+1)}&=& u^{(k)}+h u^{(k)}(1-u^{(k)}), \\
u^{(1)}&=& 1/2.
\end{eqnarray*}


\ifisscilab
O seguinte código pode ser usado para implementar no \verb+Scilab+ a recursão acima:
\begin{verbatim}
function u=euler(h,Tmax)
  u= .5;
  itmax = Tmax/h;
  for n=1:itmax
    u= u + h*u*(1-u);
   end
endfunction
\end{verbatim}
o qual pode ser invocado da seguinte forma $\left(h=1e-1, t=2\right)$:
\begin{verbatim}
-->euler(1e-1,2)
 ans  =
 
    0.8854273  
\end{verbatim}

\fi

\ifisoctave
O seguinte código pode ser usado para implementar no \verb+Octave+ a recursão acima:
\begin{verbatim}
function u=euler(h,Tmax)
  u= .5;
  itmax = Tmax/h;
  for n=1:itmax
    u= u + h*u*(1-u);
   end
endfunction
\end{verbatim}
o qual pode ser invocado da seguinte forma $\left(h=1e-1, t=2\right)$:
\begin{verbatim}
>> euler(1e-1,1)
ans =  0.88543
\end{verbatim}
\fi


\ifispython
O seguinte código pode ser usado para implementar no \verb+Python+ a recursão acima:

\begin{verbatim}
def euler(h,Tmax):
	u=.5
  	itmax = Tmax/h;
	for i in np.arange(itmax):
		u = u + h*u*(1-u)
	return u

	
h=1e-1
for t in [.5, 1, 2, 3]:
	sol_euler=euler(h,t);
	sol_exata=1/(1+np.exp(-t))
	erro_relativo = np.fabs((sol_euler-sol_exata)/sol_exata)
	print("h=%1.0e - u(%1.1f) =~ %1.7f - erro_relativo = %1.1e" % (h, t, sol_euler, erro_relativo) )

h=1e-2
print;
for t in [.5, 1, 2, 3]:
	sol_euler=euler(h,t);
	sol_exata=1/(1+np.exp(-t))
	erro_relativo = np.fabs((sol_euler-sol_exata)/sol_exata)
	print("h=%1.0e - u(%1.1f) =~ %1.7f - erro_relativo = %1.1e" % (h, t, sol_euler, erro_relativo) )

\end{verbatim}
\fi
Para fins de comparação, calculamos a solução exata e aproximada para alguns valores de $t$ e de passo $h$ e resumimos na tabela abaixo:

%\begin{table}[h]
 % \caption{Tabela comparativa entre método de Euler e solução exata para Problema~\ref{ex_euler_1}.}
 % \label{tab:log}
  \begin{tabular}{|c|c|c|c|}\hline
    $t$ & $\text{Exato}$ & $\text{Euler}~~ h=0,1$ & $\text{Euler}~~ h=0,01$\\\hline
    $0$ & $1/2$ & $0,5$ & $0,5$\\\hline
    $1/2$ & $\frac{e^{1/2}}{1+e^{1/2}}\approx 0,6224593$ & $0,6231476$ & $0,6225316$\\\hline
    $1$ & $\frac{e}{1+e}\approx 0,7310586$ & $0,7334030$ & $0,7312946$\\\hline
    $2$ & $\frac{e^2}{1+e^2}\approx  0,8807971$ & $0,8854273$  & $0,8812533$ \\\hline
    $3$ & $\frac{e^3}{1+e^3}\approx   0,9525741$  & $0,9564754$ & $0,9529609$ \\\hline
  \end{tabular}
%\end{table}

\subsection*{Exercícios}
\begin{exer} Resolva o problema de valor inicial a seguir envolvendo uma equação não autônoma\index{equação diferencial!não autônoma}, isto é, quando a função $f(t,u)$ depende explicitamente do tempo. Use passo $h=0,1$ e $h=0,01$. Depois compare com a solução exata dada por $u(t)=2e^{-t}+t-1$ nos intantes $t=0$, $t=1$, $t=2$ e $t=3$.
  \begin{eqnarray*}
    u'(t)&=&-u(t)+t,\\
    u(0)&=&1.
  \end{eqnarray*}
\end{exer}
\begin{resp}
% O esquema recursivo de Euler fica dado por:
% \begin{eqnarray*}
%   u^{(k+1)}&=&u^{(k)}+h(-u^{(k)}+t^{(k)}),\\
%   u^{(1)}&=&1.
% \end{eqnarray*}
\begin{center}
  \begin{tabular}{|c|c|c|c|}\hline
    $t$ &  Exato & Euler~~ $h=0,1$ & Euler~~ $h=0,01$\\\hline
    $0$ &  $1$ & $1$ & $1$\\\hline
    $1$ &   $2e^{-1}\approx 0,7357589$ & $0,6973569$   &   $0,7320647$  \\\hline
    $2$ &   $2e^{-2}+1\approx  1,2706706$ & $ 1,2431533 $   &  $ 1,2679593$     \\\hline
    $3$ &   $2e^{-3}+2\approx 2,0995741$  & $ 2,0847823$ & $2,0980818$   \\\hline
  \end{tabular}
\end{center}
\end{resp}

\begin{exer} Resolva o prolema de valor inicial envolvendo uma equação não linear\index{Problema de valor inicial!não linear} usando passo $h=0,1$ e $h=0,01$. 
 \begin{eqnarray*}
    u'(t)&=&\cos(u(t))\\
    u(0)&=&0,
  \end{eqnarray*}

Depois compare com a solução exata dada por 
$$u(t)=\tan^{-1} \left( \frac {e^{2t}-1}{{2 e^t}}
 \right).
$$
nos intantes $t=0$, $t=1$, $t=2$ e $t=3$.
\end{exer}
\begin{resp}
%  \begin{eqnarray*}
%   u^{(k+1)}&=&u^{(k)}+h\cos(u^{(k)})\\
%   u^{(1)}&=&0
% \end{eqnarray*}
% Comparação:
\begin{center}
  \begin{tabular}{|c|c|c|c|}\hline
    $t$ &  Exato & Euler~~ $h=0,1$ & Euler~~ $h=0,01$\\\hline
    $0$ &  $0$ & $0$ & $0$\\\hline
    $1$ &   $0,8657695 $ & $ 0.8799602$   &   $0.8671764 $  \\\hline
    $2$ &   $1,3017603 $ & $ 1.3196842 $   &  $  1.3035243$     \\\hline
    $3$ &   $1,4713043 $  & $ 1.4827638 $ & $1.4724512 $   \\\hline
  \end{tabular}
\end{center}
\end{resp}

\begin{exer} Resolva a seguinte problema de valor inicial linear com passo $h=10^{-4}$ via método de Euler e compare a solução obtida com o valor exato $y(t)=e^{\sin(t)}$ em $t=2$:
 \begin{eqnarray*}
    y'(t)&=&\cos(t)y(t)\\
    y(0)&=&1.
  \end{eqnarray*}
\end{exer}  
\begin{resp}
 Aproximação via Euler: $2,4826529 $, exata: $e^{\sin(2)}\approx 2,4825777 $. Erro relativo aproximado: $ 3\times 10^{-5}$.
\end{resp}

\section{Método de Euler melhorado}\index{Método! de Euler melhorado}\label{sec:sec_euler_mod} 
O método de Euler estudado na Seção~\ref{sec:euler} é aplicação bastante restrita devido à sua pequena precisão, isto é, normalmente precisamos escolher um passo $h$ muito pequeno para obter soluções de boa qualidade, o que implica um número elevado de passos e, consequentemente, alto custo computacional. 

Nesta seção, contruiremos o \emph{método de Euler melhorado} ou \emph{método de Euler modificado} ou, ainda, \emph{método de Heun}. Para tal, considere o problema de valor inicial dado por:
\begin{equation}\label{eq:EDO2}
  \begin{split}
    u'(t)  &= f(t,u(t)),~~t>t^{(1)} \\
    u(t^{(1)}) &= a.
  \end{split}
\end{equation}

Assim como fizemos para o método de Euler, integramos \eqref{eq:EDO2} de $t^{(1)}$ até $t^{(2)}$ e obtemos:
\begin{eqnarray}
  \int_{t^{(1)}}^{t^{(2)}} u'(t) \;dt &=& \int_{t^{(1)}}^{t^{(2)}} f(t,u(t)) \; dt\\
  u(t^{(2)})-u(t^{(1)})               &=& \int_{t^{(1)}}^{t^{(2)}} f(t,u(t)) \; dt\\
  u(t^{(2)})                      &=& u(t^{(1)}) +  \int _{t^{(1)}}^{t^{(2)}} f(t,u(t)) \; dt
\end{eqnarray}
A invés de aproximar $f(t,u(t))$ como uma constante igual ao seu valor em $t=t^{(1)}$, aplicamos a regra do trapézio (ver \ref{sec:trapezio}) à integral envolvida no lado direito da expressão, isto é:
\begin{equation}\label{eq:euler_mel_eq}
\int _{t^{(1)}}^{t^{(2)}} f(t,u(t)) \; dt = \left[\frac{f\left(t^{(1)},u(t^{(1)})\right)+f\left(t^{(2)},u(t^{(2)})\right)}{2}\right]h + O(h^3)
\end{equation}
onde $h=t^{(2)}-t^{(1)}$.
Como o valor de $u(t^{(2)})$ não é conhecido antes de o passo ser realizado, aproximamos seu valor aplicando o método de Euler:
\begin{eqnarray}
\tilde{u}(t^{(2)})= u(t^{(1)})+h f\left(t^{(1)},u(t^{(1)})\right)
\end{eqnarray}
Assim obtemos:
\begin{eqnarray}
  u(t^{(2)})&=& u(t^{(1)}) +  \int _{t^{(1)}}^{t^{(2)}} f(t,u(t)) \; dt\\
  &\approx& u(t^{(1)}) +\left[\frac{f\left(t^{(1)},u(t^{(1)})\right)+f\left(t^{(2)},u(t^{(2)})\right)}{2}\right]h\\
  &\approx& u(t^{(1)}) +\left[\frac{f\left(t^{(1)},u(t^{(1)})\right)+f\left(t^{(2)},\tilde{u}(t^{(2)})\right)}{2}\right]h\\
 % &=& u(t^{(1)}) + \frac{k_1+k_2}{2}h
\end{eqnarray}
onde 
\begin{eqnarray}
k_1&=&f\left(t^{(1)},u(t^{(1)})\right)\\
\hbox{e}\\
k_2&=&f\left(t^{(2)},\tilde{u}(t^{(2)})\right)=f\left(t^{(2)}, u(t^{(1)})+hk_1 \right )
\end{eqnarray}

Portanto, o método recursivo de Euler melhorado assume a seguinte forma:
\begin{eqnarray}
\tilde{u}^{(k+1)}&=&u^{(k)}+h f(t^{(k)},u^{(k)}),\\
u^{(k+1)}&=&u^{(k)}+\frac{h}{2}\left(f(t^{(k)},u^{(k)})+f(t^{(k)},\tilde{u}^{(k)})\right),\\
u^{(1)}&=&a~~ \hbox{(condição inicial)}.
\end{eqnarray}

Que pode ser escrito equivalentemente como:
\begin{eqnarray*}
k_1&=&f(t^{(k)},u^{(k)}),\\
k_2&=&f(t^{(k+1)},u^{(k)}+k_1),\\
u^{(k+1)}&=&u^{(k)}+h\frac{k_1+k_2}{2},\\
u^{(1)}&=&a~~ \hbox{(condição inicial)}.
\end{eqnarray*}
Aqui $k_1$ e $k_2$ são variáveis auxiliares que representam as inclinações e devem ser calculadas a cada passo. Esta notação é compatível com a notação usada nos métodos de Runge-Kutta, uma família de esquemas iterativos para aproximar problemas de valor inicial, da qual o método de Euler e o método de Euler melhorado são casos particulares. Veremos os métodos de Runge-Kutta na Seção~\ref{sec:sec_RK}.

\subsection*{Exercícios Resolvidos}
\begin{exeresol}\label{exeresol:exeresol1_euler_melhorado} Resolva pelo método de Euler melhorado problema de valor inicial do Exercício Resolvido~\ref{exeresol:exeresol1}:
\begin{eqnarray}
     u'(t)&=& -0,5u(t)+2+t\\
            u(0) &=&  8
\end{eqnarray}

Usando os seguinte passos: $h=10^{-1}$, $h=10^{-2}$, $h=10^{-3}$, $h=10^{-4}$ e $h=10^{-5}$ e compare a solução aproximada em $t=1$ com a solução obtida pelo método de Euler e a solução exata dada por:
\begin{equation*}
     u(t) = 2t+8e^{-t/2} \Longrightarrow u(1)=2+8e^{-1/2} \approx 6,85224527770107
\end{equation*}
\end{exeresol}
\begin{resol} Primeramente itentificamos $f(t,u)=-0,5u+2+t$ e construímos o processo iterativo do método de Euler melhorado: 
\begin{eqnarray*}
  k_1&=&  f(t^{(n)},u^{(n)})=-0,5u^{(n)}+2+t^{(n)}\\
  \tilde{u}&=& u^{(n)} + hk_1\\
  k_2&=&  f(t^{(n+1)},\tilde{u})=-0,5\tilde{u}+2+t^{(n+1)}\\
  u^{(n+1)}&=&u^{(n)} + h( k_1+k_2),~~n=1,2,3,\ldots\\
  u^{(1)}&=&8 ~~ \hbox{(condição inicial)}.
\end{eqnarray*}

\ifisscilab
O seguinte código pode ser usado para implementar no \verb+Scilab+ a recursão acima:
\begin{verbatim}
function u=euler_mod(h,Tmax)
  u= 8;
  itmax = Tmax/h;

  for n=1:itmax
    t=(n-1)*h;
    k1 = (-0.5*u + 2 + t);
    u_til = u + h*k1;
    k2 = (-0.5*u_til + 2 + t + h);
    u= u + h * (k1 + k2)/2;
  end
endfunction
\end{verbatim}
o qual pode ser invocado da seguinte forma:
\begin{verbatim}
-->euler_mod(1e-1,1)
 ans  =
 
    6.8532941 
\end{verbatim}
Podemos construir um vetor com as cinco soluções da seguinte forma:
\begin{verbatim}
 -->S=[euler_mod(1e-1,1) euler_mod(1e-2,1) euler_mod(1e-3,1) euler_mod(1e-4,1) euler_mod(1e-5,1)]
S  =
 
    6.8532949    6.8522554    6.8522454    6.8522453    6.8522453 
 \end{verbatim}
\fi
\ifisoctave
O seguinte código pode ser usado para implementar no \verb+Octave+ a recursão acima:
\begin{verbatim}
function u=euler_mod(h,Tmax)
  u= 8;
  itmax = Tmax/h;

  for n=1:itmax
    t=(n-1)*h;
    k1 = (-0.5*u + 2 + t);
    u_til = u + h*k1;
    k2 = (-0.5*u_til + 2 + t + h);
    u= u + h * (k1 + k2)/2;
  end
endfunction
\end{verbatim}
o qual pode ser invocado da seguinte forma:
\begin{verbatim}
>> euler_mod(1e-1,1)
ans =  6.8701
\end{verbatim}
Podemos construir um vetor com as cinco soluções da seguinte forma:
\begin{verbatim}
 >> S=[euler_mod(1e-1,1) euler_mod(1e-2,1) euler_mod(1e-3,1) euler_mod(1e-4,1) euler_mod(1e-5,1)]
S =

    6.8533   6.8523   6.8522   6.8522   6.8522
\end{verbatim}
\fi
\ifispython
O seguinte código pode ser usado para implementar no \verb+Python+ a recursão acima:

\begin{verbatim}
def euler_mod(h,Tmax):
	u=8
  	itmax = Tmax/h;
	for i in np.arange(0,itmax):
		t=i*h
		k1 = -0.5*u+2+t
		u_til = u + h*k1
		k2 = -0.5*u_til+2+(t+h)
		u=u+h*(k1+k2)/2
	return u

	

sol_exata = 2+8*np.exp(-1/2)
for h in [1e-1, 1e-2, 1e-3, 1e-4, 1e-5]:
	sol_euler=euler_mod(h,1);
	erro_relativo = np.fabs((sol_euler-sol_exata)/sol_exata)
	print("h=%1.0e - u(1) =~ %1.7f - erro_relativo = %1.1e" % (h, sol_euler, erro_relativo) )

\end{verbatim}
\fi
A seguinte tabela resume os resultados obtidos:
\begin{center}
 \begin{tabular}{|l|l|l|l|l|l|l|l|}%\label{pvi:tab_euler}
\hline
   h&$10^{-1}$&$10^{-2}$&$10^{-3}$&$10^{-4}$&$10^{-5}$\\
   \hline
   Euler & 6,7898955 &  6,8461635  &  6,8516386  &  6,8521846  &  6,8522392  \\
   \hline
   $\varepsilon_{rel}$ &9,1e-03 &  8,9e-04  & 8,9e-05&   8,9e-06 &  8,9e-07\\
   \hline
  Euler mod. & 6,8532949 &  6,8522554  &  6,8522454  &  6,8522453  &  6,8522453 \\
   \hline
   $\varepsilon_{rel}$ &1,5e-04 &  1,5e-06  & 1,5e-08&   1,5e-10 &  1,5e-12\\
   \hline
 
   \end{tabular}
\end{center}
% Veja o gráfico da solução para $h=1, 0.5, 0.1, 0.05$:
% \begin{figure}
% \includegraphics[width=\textwidth]{euler.eps}
% \end{figure}
\end{resol}

\section{Solução de sistemas de equações diferenciais}\index{Sistemas de equações diferenciais}\label{sec:solsistema}

Nas seções \ref{sec:euler} e \ref{sec:sec_euler_mod}, construimos dois métodos numéricos para resolver problemas de valor inicial. Nestas seções, sempre consideremos problemas envolvendo equações diferenciais ordinárias de primeira ordem, isto é:
\begin{eqnarray*}
  u'(t)  &=& f(t,u(t)),~~t>t^{(1)} \\
  u(t^{(1)}) &=& a.
\end{eqnarray*}
Estas técnicas podem ser diretamente estendidas para resolver numericamente problemas de valor inicial envolvendo sistemas de equações diferenciais ordinárias de primeira ordem, isto é:
\begin{equation}\label{eq:exemplo_sistema}
  \begin{split}
    u_1'(t)  &= f_1(t,u_1(t), u_2(t), u_3(t),\ldots, u_n(t)),~~t>t^{(1)} ,\\
    u_2'(t)  &= f_2(t,u_1(t), u_2(t), u_3(t),\ldots, u_n(t)),~~t>t^{(1)} ,\\
    u_3'(t)  &= f_3(t,u_1(t), u_2(t), u_3(t),\ldots, u_n(t)),~~t>t^{(1)} ,\\
    &\vdots \\
    u_n'(t)  &= f^{(n)}(t,u_1(t), u_2(t), u_3(t),\ldots, u_n(t)),~~t>t^{(1)} ,\\
    u(t^{(1)}) &= a_1 ,\\
    u(t^{(2)}) &= a_2 ,\\
    u(t^{(3)}) &= a_3 ,\\
    &\vdots\\
    u(t^{(n)}) &= a_n.
  \end{split}
\end{equation}
O Problema~\eqref{eq:exemplo_sistema} pode ser escrito como um problema de primeira ordem envolvendo uma única incógnita, $u(t)$, dada como um vetor de funções $u_j(t)$, isto é:
$$u(t)=\left[
\begin{array}{c}
 u_1(t)\\
 u_2(t)\\
 u_3(t)\\
 \vdots\\
 u_n(t)
\end{array}
\right]$$
De forma que o Problema~\eqref{eq:exemplo_sistema} assuma a seguinte forma:
\begin{eqnarray*}
  u'(t)  &=& f(t,u(t)),~~t>t^{(1)} \\
  u(t^{(1)}) &=& a.
\end{eqnarray*}
onde
$$f(t,u(t))=\left[
\begin{array}{c}
  f_1(t,u_1(t), u_2(t), u_3(t),\ldots, u_n(t))\\
  f_2(t,u_1(t), u_2(t), u_3(t),\ldots, u_n(t))\\
  f_3(t,u_1(t), u_2(t), u_3(t),\ldots, u_n(t))\\  
  \vdots\\
  f^{(n)}(t,u_1(t), u_2(t), u_3(t),\ldots, u_n(t))\\
  \end{array}
\right]$$
e
$$a=\left[
\begin{array}{c}
 a_1\\
 a_2\\
 a_3\\
 \vdots\\
 a_n
\end{array}
\right]$$

Veja o  o Exemplo~\ref{eq:eq_exemplo_sistema}
\begin{ex}\label{ex:exemplo_sistema_PVI}Considere o problema de resolver numericamente pelo método de Euler o seguinte sistema de equações diferenciais ordinárias com valores iniciais:
\begin{subequations}\label{eq:eq_exemplo_sistema}
\begin{eqnarray}
x'(t)&=&-y(t),\\
y'(t)&=&x(t),\\
u(0)&=&1.\\
v(0)&=&0.
\end{eqnarray}
\end{subequations}
%cuja solução exata é $x(t)=\cos(t)$ e $y(t)=\sin(t)$.
Para aplicar o método de Euler a este sistema, devemos encarar as duas incógnitas do sistema como entradas de um vetor, ou seja, escrevemos:
 $$u(t)=\left[\begin{array}{c}x(t)\\y(t)\end{array}\right].$$
 e, portanto, o sistema pode ser escrito como:
\begin{eqnarray*}
\left[\begin{array}{c}x^{(k+1)}\\y^{(k+1)}\end{array}\right]=\left[\begin{array}{c}x^{(k)}\\y^{(k)}\end{array}\right]+h\left[\begin{array}{c}-y^{(k)}\\x^{(k)}\end{array}\right].
\end{eqnarray*}
Observe que este processo iterativo é equivalente a discretiza as equações do sistema uma-a-uma, isto é:
\begin{eqnarray*}
x^{(k+1)}&=&x^{(k)}-hy^{(k)},\\
y^{(k+1)}&=&y^{(k)}+hx^{(k)},\\
x^{(1)}&=&1,\\
y^{(1)}&=&0,\\
\end{eqnarray*}
\end{ex}

\subsection*{Exercícios Resolvidos}
\begin{exeresol} Resolva pelo método de Euler melhorado o seguinte problema de valor inicial para aproximar o valor de $x$ e $y$ entre $t=0$ e $t=1$:
\begin{eqnarray*}
x'(t)&=&x(t)-y(t),\\
y'(t)&=&x(t)-y(t)^3,\\
u(0)&=&1.\\
v(0)&=&0.
\end{eqnarray*}
\end{exeresol}
\begin{resol}
 Primeiramente, identificamos $u(t)$ como o vetor incógnita:
 $$u(t)=\left[\begin{array}{c}x(t)\\y(t)\end{array}\right].$$
Depois aplicamos a recursão do método de Euler melhorado dada por:
 \begin{eqnarray*}
\tilde{u}^{(k+1)}&=&u^{(k)}+hf(t^{(k)},u^{(k)}),\\
u^{(k+1)}&=&u^{(k)}+\frac{h}{2}\left(f(t^{(k)},u^{(k)})+f(t^{(k)},\tilde{u}^{(k)})\right),\\
\end{eqnarray*}
isto é:
 \begin{eqnarray*}
\tilde{x}^{(k+1)}&=&x^{(k)}+h\left(x^{(k)}-y^{(k)}\right)\\
\tilde{y}^{(k+1)}&=&y^{(k)}+h\left(x^{(k)}-{y^{(k)}}^3\right)\\
{x}^{(k+1)}&=&x^{(k)}+\frac{h}{2}\left[\left({x}^{(k)}-{y}^{(k)}\right)+\left(\tilde{x}^{(k)}-\tilde{y}^{(k)}\right)\right]\\
{y}^{(k+1)}&=&y^{(k)}+\frac{h}{2}\left[\left({x}^{(k)}-{{y}^{(k)}}^3\right)+ \left(\tilde{x}^{(k)}-\left.{\tilde{y}^{(k)}}\right.^3\right)\right]\\
 \end{eqnarray*}

 A tabela a seguir resume os resultados obtidos:
 
 
 \begin{center}
 \begin{tabular}{|l|l|l|l|l|l|l|}%\label{pvi:tab_euler}
\hline
   h&&$t=0,2$&$t=0,4$&$t=0,6$&$t=0,8$&$t=1,0$\\
   \hline
   \multirow{2}{*}{$10^{-2}$} &x  &1.1986240 &1.3890564 &1.5654561 &1.7287187 &1.8874532 \\
			       &y&0.2194288 &0.4692676 &0.7206154 &0.9332802 &1.0850012 \\			
   \hline
  \multirow{2}{*}{$10^{-3}$} &x  &1.1986201 &1.3890485 &1.5654455 &1.7287066 &1.8874392\\
			       &y &0.2194293 &0.4692707 &0.7206252 &0.9332999 &1.0850259\\
			       \hline
 
 
   \multirow{2}{*}{$10^{-4}$} &x  &1.1986201 &1.3890484 &1.5653609 &1.7287065 &1.8874390\\
			       &y&  0.2194293 &0.4692707 &0.7205062 &0.9333001 &1.0850262 \\			
   \hline
   \end{tabular}
\end{center}
\ifispython
A seguinte rotina pode ser usada para implementar a solução do sistema:
\begin{verbatim}
def euler_mod(h,Tmax,u1):
  	itmax = Tmax/h;
	x=np.empty(itmax+1)
	y=np.empty(itmax+1)
	x[0]=u1[0]
	y[0]=u1[1]

	for i in np.arange(0,itmax):
		t=i*h
		kx1 = (x[i]-y[i])
		ky1 = (x[i]-y[i]**3)

		x_til = x[i] + h*kx1
		y_til = y[i] + h*ky1

		kx2 = (x_til-y_til)
		ky2 = (x_til-y_til**3)

		x[i+1]=x[i]+h*(kx1+kx2)/2
		y[i+1]=y[i]+h*(ky1+ky2)/2

	return [x,y]


Tmax=1 			#tempo maximo de simulacao
u1=np.asarray([1,0])	#condicoes iniciais na forma vetorial
h=1e-4			#passo
sol_euler=euler_mod(h,Tmax,u1);

itmax=Tmax/h

for t in [0, .2, .4, .6, .8, 1]:
	k=t/h
	print("h=%1.0e - x(%1.1f) =~ %1.6f - y(%1.1f) =~ %1.6f" % (h, t, sol_euler[0][k], t, sol_euler[1][k]) )

\end{verbatim}
\fi

\ifisoctave
A seguinte rotina pode ser usada para implementar a solução do sistema:
\begin{verbatim}
function [x,y]=euler_mod(h,Tmax,u1)
  itmax = Tmax/h;
  x=zeros(itmax+1);
  y=zeros(itmax+1);
  x(1)=u1(1);
  y(1)=u1(2);
  
  for n = 1:itmax
    t=(n-1)*h;
    kx1 = (x(n)-y(n));
    ky1 = (x(n)-y(n)^3);
    x_til = x(n) + h*kx1;
    y_til = y(n) + h*ky1;
    
    kx2 = (x_til-y_til);
    ky2 = (x_til-y_til^3);

    x(n+1)=x(n)+h*(kx1+kx2)/2;
    y(n+1)=y(n)+h*(ky1+ky2)/2;
  end
endfunction 
\end{verbatim}
Que pode ser invocada como:
\begin{verbatim}
h=1e-2
Tmax=1
itmax=Tmax/h
[x,y]=euler_mod(h,Tmax,[1,0]);
[x(1) x(1+itmax*.2) x(1+itmax*.4) x(1+itmax*.6) x(1+itmax*.8) x(1+itmax)]
[y(1) y(1+itmax*.2) y(1+itmax*.4) y(1+itmax*.6) y(1+itmax*.8) y(1+itmax)]
 \end{verbatim}

\fi


\ifisscilab
A seguinte rotina pode ser usada para implementar a solução do sistema:
\begin{verbatim}
function [x,y]=euler_mod(h,Tmax,u1)
  itmax = Tmax/h;
  x=zeros(itmax+1);
  y=zeros(itmax+1);
  x(1)=u1(1);
  y(1)=u1(2);
  
  for n = 1:itmax
    t=(n-1)*h;
    kx1 = (x(n)-y(n));
    ky1 = (x(n)-y(n)^3);
    x_til = x(n) + h*kx1;
    y_til = y(n) + h*ky1;
    
    kx2 = (x_til-y_til);
    ky2 = (x_til-y_til^3);

    x(n+1)=x(n)+h*(kx1+kx2)/2;
    y(n+1)=y(n)+h*(ky1+ky2)/2;
  end
endfunction 
\end{verbatim}
Que pode ser invocada como:
\begin{verbatim}
h=1e-2
Tmax=1
itmax=Tmax/h
[x,y]=euler_mod(h,Tmax,[1,0]);
[x(1) x(1+itmax*.2) x(1+itmax*.4) x(1+itmax*.6) x(1+itmax*.8) x(1+itmax)]
[y(1) y(1+itmax*.2) y(1+itmax*.4) y(1+itmax*.6) y(1+itmax*.8) y(1+itmax)]
 \end{verbatim}
\fi
\end{resol}


\section{Solução de equações e sistemas de ordem superior}

Na Seção~\ref{eq:solsistema}, estendemos os métodos de Euler e Euler melhorado visto nas seções \ref{sec:euler} e \ref{sec:sec_euler_mod} para resolver numericamente problemas de valor inicial envolvendo sistemas de equações diferenciais ordinárias de primeira ordem. Nesta seção, estenderemos estas técnicas para resolver alguns tipos de problemas de ordem superior. Para tal, converteremos a equação diferencial em um sistema, incluindo as derivadas da incógnita como novas incógnitas. Vejamos um exemplo:

\begin{ex} Resolva o problema de valor inicial de segunda ordem dado por
\begin{eqnarray*}
y''+y'+y&=&\cos(t),\\
y(0)&=&1,\\
y'(0)&=&0,
\end{eqnarray*}
A fim de transformar a equação diferencial dada em um sistema de equações de primeira ordem, introduzimos a substituição $w=y'$, de forma que obteremos o sistema:
\begin{eqnarray*}
y'&=&w\\
w'&=&-w-y+\cos(t)\\
y(0)&=&1\\
w(0)&=&0
\end{eqnarray*}
Este sistema pode ser resolvido usando as técnicas da Seção~\ref{sec:solsistema}.
%Portanto, o método de Euler produz o seguinte processo iterativo:
%\begin{eqnarray*}
%y^{(k+1)}&=&y^{(k)}+hw^{(k)},\\
%w^{(k+1)}&=&w^{(k)}-hw^{(k)}-hy^{(k)}+h\cos(t^{(k)}),\\
%y^{(1)}&=&1,\\
%w^{(1)}&=&0.
%\end{eqnarray*}
\end{ex}

\subsection*{Exercícios resolvidos}
\begin{exeresol} Considere o seguinte sistema envolvendo uma equação de segunda ordem e uma de primeira ordem:
\begin{eqnarray*}
x''(t)-(1- 0,1 z(t))x'(t)+ x(t)&=&0\\
10 z'(t)+z(t)&=&x(t)^2
\end{eqnarray*}
sujeito a condições iniciais dadas por:
\begin{eqnarray*}
x(0)&=&3\\
x'(0)&=&0\\
z(0)&=&10\\
\end{eqnarray*}
Rescreva este sistema como um sistema de três equações de primeira ordem.
\end{exeresol}
\begin{resol}
 Definimos $y(t)=x'(t)$, pelo que o sistema se torna:
\begin{eqnarray*}
x'(t)&=&y(t)\\
y'(t)-(1- 0,1 z(t))y(t)+ x(t)&=&0\\
10 z'(t)+z(t)&=&x(t)^2
\end{eqnarray*}
 defina o vetor $u(t)$ como:
$$u(t)=\left[
\begin{array}{c}
 x(t)\\y(t)\\z(t)
\end{array}
\right]$$ 
De forma que:
\begin{eqnarray*}
u'(t)&=&\left[
\begin{array}{c}
 x'(t)\\y'(t)\\z'(t)
\end{array}
\right]=\left[
\begin{array}{c}
 y(t)\\
 \left[1- 0,1 z(t)\right]y(t)- x(t)\\
 \left[x(t)^2-z(t)\right]/10
\end{array}
\right]\\
&\hbox{ou}&\\
u'(t)&=&\left[
\begin{array}{c}
 u_1'(t)\\u_2'(t)\\u_3'(t)
\end{array}
\right]=\left[
\begin{array}{c}
 u_2(t)\\
 \left[1- 0,1 u_3(t)\right]u_2(t)- u_1(t)\\
 \left[u_1(t)^2-u_3(t)\right]/10
\end{array}
\right]
\end{eqnarray*}
sujeito às condições iniciais dadas por:
\begin{eqnarray*}
u'(0)&=&\left[
\begin{array}{c}
 x'(0)\\y'(0)\\z'(0)
\end{array}
\right]=\left[
\begin{array}{c}
3\\0\\10
\end{array}
\right]
\end{eqnarray*}

\end{resol}



\subsection*{Exercícios}

\begin{exer}Resolva o problema de valor inicial dado por
\begin{eqnarray*}
x'&=& -2x + \sqrt{y}\\
y'&=& x - y\\
x(0)&=&0\\
y(0)&=&2\\
\end{eqnarray*}
com passo $h=2\cdot 10^{-1}$ $h=2\cdot 10^{-2}$, $h=2\cdot 10^{-3}$ e $h=2\cdot 10^{-4}$ para obter aproximações para $x(2)$ e $y(2)$.  
\end{exer}
\begin{resp}
 
 \begin{center}
 \begin{tabular}{|l|l|l|l|l|l|}%\label{pvi:tab_euler}
\hline
   h                   & &$2\cdot 10^{-2}$&$2\cdot 10^{-2}$&$2\cdot 10^{-3}$&$2\cdot 10^{-4}$\\
   \hline
 \multirow{2}{*}{Euler}&x&0,4302019&0,4355057&0,4358046&0,4358324\\
                       &y& 0,6172935&0,6457760&0,6486383&0,6489245\\
  \hline
 
 \multirow{2}{*}{Euler mod,}&x&  0,4343130&0,4358269&0,4358354&0,4358355\\
                       &y& 0,6515479&0,6489764&0,6489566&0,6489564 \\

   \hline
   
 
 \end{tabular}
\end{center}
\end{resp}

\begin{exer} Considere o problema de segunda order dado por:
\begin{eqnarray*}
 x''(t)+x'(t)+\sin(x(t))=1,
\end{eqnarray*}
sujeito às condições iniciais dadas por:
\begin{eqnarray*}
x(0)&=&2,\\
x'(0)&=&0.
\end{eqnarray*}
Resolva numericamente para obter o valor de $x(0,5)$, $x(1)$, $x(1,5)$ e $x(2)$ com passo $h=10^{-2}$ e $h=10^{-3}$ via método de Euler modificado.
\end{exer}

\begin{resp} \begin{center}
 \begin{tabular}{|l|l|l|l|l|l|}%\label{pvi:tab_euler}
\hline
   h&&$t=0,5$&$t=1,0$&$t=1,5$&$t=2,0$\\
   \hline
   \multirow{2}{*}{$10^{-3}$} &x & 1,9023516 &1,6564208&1,3124281&0,9168299\\
		              &y'& -0,3635613&-0,6044859&-0,7564252& -0,8072298\\			
   \hline

   \multirow{2}{*}{$10^{-4}$} &x & 1,9023552 & 1,6564243 & 1,3124309 & 0,9168319\\
		              &y'& -0,3635670&-0,6044930&-0,7564334& -0,8072397 \\			
   \hline
   
   \end{tabular}
\end{center}
 
\end{resp}



\section{Erro de truncamento}\index{Ordem! de precisão}\index{Erro! de truncamento}\label{sec:erro_truncamento}
Nas seções \ref{sec:euler} e \ref{sec:sec_euler_mod}, construimos dois métodos numéricos para resolver problemas de valor inicial. No Exercício Resolvido~\ref{exeresol:exeresol1_euler_melhorado}, vimos que o erro do método de Euler de do método de Euler melhorado caem quando se reduz o passo $h$, ademais, o erro do método de Euler melhorado cai conforme o quadrado de $h$, enquando o do método de Euler cai conforme $h^2$. Este fenômeno motiva a definção de \emph{ordem de precisão}.

\begin{defn} O \emph{erro de truncamento local} é definido como o erro introduzido em cada passo pelo truncamento da equação diferencial supondo conhecida a solução exata no início do intervalo. Um método numérico é dito ter \emph{ordem de precisão} $p$ se o erro de truncamento local for da ordem de $h^{p+1}$.
\end{defn}
\begin{ex} O método de Euler tem erro de truncamento local de ordem $1$. Para obter este resultado, observamos via expansão de Taylor que:
$$u(t+h)=u(t)+hu'(t)+\frac{h^2}{2}u''(t)+O(h^3).$$
%Subtraindo $u(t)$ desta expressão e dividindo por $h$, obtemos:
%$$\frac{u(t+h)-u(t)}{h}=u'(t)+\frac{h}{2}u''(t)+O(h^2)$$
Se escolhermos nesta expressão $t=t^{(n)}$ e, portanto, $t+h=t^{(n)}+h=t^{(n+1)}$, temos:
$$t^{(n+1)}=t^{(n)}+hu'(t^{(n)})+\frac{h^2}{2}u''(t^{(n)})+O(h^3)$$
Agora notamos que o termo principal do erro é dado por $\frac{h^2}{2}u''(t^{(n)})$, como a derivada segunda da solução não depende de $h$, o erro local de truncamento decresce conforme $h^2$ e assim a ordem de precisão do método é 1.
\end{ex}

\begin{defn} O \emph{erro de truncamento global} é definido como erro acumulado ao longo de todos os passo de resolução, supondo a condição inicial exata.
\end{defn}

A relação entre o erro de truncamento global e o erro de truncamento local depende da função $f(t,u)$ envolvida. Diante de suficiente regularidade, o erro acumulado é da mesma ordem de grande do erro de truncamento local acumulado ao longo do processo, isto é, pode ser estimado multiplicando o erro local pelo número de passos. Como o número de passos $N$ necessários para calcular a solução de um problema de valor inicial no ponto $t=t_{f}$ é dado por $N=\frac{t_f}{h}$, temos que a erro de truncamento global é uma ordem inferior ao erro de truncamento local e equivale à ordem de precisão do método.

Usamos também a notação $ETL$ para o erro de truncamento local e $ETG$ para o erro de truncamento global. De forma que, para o método de Euler, temos:
$$ETL_{Euler} = O(h^2) ~~~ \hbox{e} ~~~ ETG_{Euler}=O(h).$$
\begin{ex} Vamos obter o erro de truncamento local do método de Euler melhorado.
Partimos da construção do esquema iterativo de Euler melhorado:
\begin{eqnarray}
  \int_{t^{(1)}}^{t^{(2)}} u'(t) \;dt &=& \int_{t^{(1)}}^{t^{(2)}} f(t,u(t)) \; dt\\
  u(t^{(2)})-u(t^{(1)})               &=& \int_{t^{(1)}}^{t^{(2)}} f(t,u(t)) \; dt\\
  u(t^{(2)})                      &=& u(t^{(1)}) +  \int _{t^{(1)}}^{t^{(2)}} f(t,u(t)) \; dt
\end{eqnarray}
Neste ponto, usamos o erro de truncamento do método de trapézios para aproximar a integral envolvida:
\begin{eqnarray*}
\int _{t^{(1)}}^{t^{(2)}} f(t,u(t)) \; dt = \frac{h}{2}\left[f\left(t^{(1)},u(t^{(1)})\right) + f\left(t^{(2)},u(t^{(2)})\right)\right] + O(h^3)
\end{eqnarray*}
Assim, temos que o erro de truncamento local do método de Euler melhorado é $O(h^3)$ e, portanto, um método de ordem 2.
\end{ex}

\begin{exer}Aplique o método de Euler e o método de Euler melhorado para resolver o problema de valor inicial dado por
\begin{eqnarray*}
u'&=& -2u + \sqrt{u}\\
u(0)&=&1
\end{eqnarray*}
com passo $h=10^{-1}$, $h=10^{-2}$, $h=10^{-3}$, $h=10^{-4}$ e $h=10^{-5}$ para obter aproximações para $u(1)$. Compare com a solução exata dada do problema dada por $u(t) =  \left({1+2 e^{-t}+e^{-2 t}}\right)/{4}$ através do erro relativo e observe a ordem de precisão do método.
\end{exer}
\begin{resp}

\begin{center}
 \begin{tabular}{|l|l|l|l|l|l|l|l|}%\label{pvi:tab_euler}
\hline
   h&$10^{-1}$&$10^{-2}$&$10^{-3}$&$10^{-4}$&$10^{-5}$\\
   \hline
   Euler & 0,4495791 & 0,4660297 & 0,4675999 & 0,4677562 & 0,4677718\\
   \hline
   $\varepsilon_{rel}$ &9,1e-03 &  8,9e-04  & 8,9e-05&   8,9e-06 &  8,9e-07\\
   \hline
  Euler mod, & 0,4686037 & 0,4677811 & 0,4677736 & 0,4677735 & 0,4677735\\
   \hline
   $\varepsilon_{rel}$ & 1,8e-03 & 1,6e-05 & 1,6e-07 & 1,6e-09 & 1,6e-11\\
   \hline
   \end{tabular}
\end{center}
A solução exata vale $u(1)=\frac{1+2e^{-1}+e^{-2}}{4}= \left(\frac{1+e^{-1}}{2}\right)^2\approx 0.467773541395$.
\end{resp}


\begin{exer}Resolva o problema de valor inicial dado por
\begin{eqnarray*}
u'&=& \cos(tu(t))\\
u(0)&=&1\\
\end{eqnarray*}
com passo $h=10^{-1}$, $h=10^{-2}$, $h=10^{-3}$, $h=10^{-4}$ e $h=10^{-5}$ para obter aproximações para $u(2)$
\end{exer}
\begin{resp}
\begin{center}
 \begin{tabular}{|l|l|l|l|l|l|l|l|}%\label{pvi:tab_euler}
\hline
   h&$10^{-1}$&$10^{-2}$&$10^{-3}$&$10^{-4}$&$10^{-5}$\\
   \hline
   Euler & 1,1617930 & 1,1395726 & 1,1374484 & 1,1372369 & 1,1372157\\
   \hline
  Euler mod & 1,1365230 & 1,1372075 & 1,1372133 & 1,1372134 & 1,1372134\\
   \hline
   \end{tabular}
\end{center}
\end{resp}




\section{Métodos de Runge-Kutta explícitos}\label{sec:sec_RK}\index{método!de Runge-Kutta exlícito}
Nas seções anteriores, exploramos os métodos de Euler e Euler modificado para resolver problemas de valor inicial. Neste momento, deve estar claro ao leitor que o método de Euler melhorado produz soluções de melhor qualidade que o método de Euler para a maior parte dos problemas estudados. Isso se deve, conforme Seção~\ref{sec:erro_truncamento}, à ordem de precisão do método. 

Os métodos de Runge-Kutta generalizam o esquema do método de Euler melhorado, inserindo mais estágios de cálculo e buscando ordens de precisão mais altas. Nesta seção, trataremos dos métodos de \emph{Runge-Kutta explícitos}\footnote{Existem também os métodos implícitos que serão abordados na Seção~\ref{sec:sec_IRK}. Ver Observação~\ref{obs:obs_imp}.}

Para tal, considere o  problema de valor inicial:
\begin{eqnarray}
  u'(t) &=&f(t,u(t)), \\
  u(t_0) &=&a.
\end{eqnarray}

Integrando a EDO em $[t^{(n)},t^{(n+1)}]$ obtemos
\begin{eqnarray}
  u^{(n+1)}  &=&u^{(n)}  + \int _{t^{(n)}}^{t^{(n+1)}} f(t,u(t)) \; dt
\end{eqnarray}
O método de Euler aproxima a integral no lado direito da expressão acima utilizando apenas o valor de $f(t,u)$ em $t=t^{(n)}$, o método de Euler melhorado aproxima esta integral utilizando os valores de $f(t,u)$ $t=t^{(n)}$ e $t=t^{(n+1)}$. Os métodos de Runge-Kutta explícitos admitem estágios intermediários, utilizando outros valores de $f(t,u)$ nos pontos $\{\tau_1,\tau_2,\ldots,\tau_\nu\}$ dentro do intervalo $[t_n,t^{(n+1)}]$. Veja esquema abaixo:
\begin{center}
\begin{tabular}{c c c c c}
  $u_n$ &      &       &      & $u_{n+1}$ \\
  $|$   &      &       &      &  $|$  \\ \hline
%  $-+-$ & $--$ &  $--$ & $--$ & $-+-$ \\
  $t^{(n)}$ &      &       &      & $t^{(n+1)}$ \\
  $\tau _1$  & $\tau _2$ & $\cdots $ &      & $\tau_\nu $\\
  $t^{(n)}$~~  &~~ $t^{(n)}+c_2h$~~ & $\cdots $ &      & ~~$t^{(n)}+c_\nu h $~~
  \end{tabular}
\end{center}
Observe que $\tau_j=t^{(n)}+c_jh$ com $0=c_1 \leq c_2 \leq \cdots \leq c_\nu \leq 1$, isto é, o primeira ponto sempre coincide com o extremo esquerdo do intervalo, mas o último ponto não precisa ser o extremo direito. Ademais, um mesmo ponto pode ser usado mais de uma vez com aproximações diferentes para $u(\tau_j)$.

Desta forma, aproximamos a integral por um esquema de quadratura com $\nu$ pontos: 
\begin{eqnarray}
   \int_{t^{(n)}}^{t^{(n+1)}} f(t,u(t)) \; dt 
           &\approx&u_n  + h\sum_{j=1}^\nu   b_j f\left(\tau_j,u(\tau_j)\right) %%revisar itentidades
\end{eqnarray}
onde $b_j$ são os pesos da quadratura numérica que aproxima a integral. Assim como na construção do método de Euler melhorado, não dispomos dos valores $u(\tau_j)$ antes de calculá-los e, por isso, precisamos estimá-los com base nos estágios anteriores:
\begin{equation}\label{eq:RKa}
  \begin{split}
    \tilde{u}_1 &= u^{(n)} \\
    \tilde{u}_2 &= u_n  + h a_{21}k_1 \\
    \tilde{u}_3 &= u_n  + h \left[a_{31}k_1 + a_{32}k_2\right] \\
    \tilde{u}_4 &= u_n  + h \left[a_{41}k_1 + a_{42}k_2 + a_{43}k_3\right] \\
    &\vdots \\
    \tilde{u}_\nu &= u_n  + h \left[a_{\nu 1}k_1 + a_{\nu 2}k_2 + \cdots +a_{\nu \nu}k_\nu\right] \\
    u^{(n+1)}&= u_n + h [ b_1k_1+b_2k_2+\cdots + b_\nu k_\nu],
  \end{split}
\end{equation}
onde $k_j=f\left(\tau_j,\tilde{u}_{j}\right)$, $A:=\left(a_{ij}\right)$ é a matriz Runge-Kutta (triangular inferior com diagonal zero), $b_j$ são os pesos Runge-Kutta e $c_j$ são os nós Runge-Kutta. Estes coeficientes podem ser escritos de forma compacta em uma tabela conforme a seguir:
\begin{center}
\begin{tabular}{c|c}
  \begin{tabular}{c}\\$c$\\~\\ \end{tabular} &\begin{tabular}{ccc}& $A$&\end{tabular}     \\  \hline
      & $b^T$
\end{tabular}
~~~$=$~~~
\begin{tabular}{c|ccc}
  $c_1$  & $0$      & $0$      & $0$ \\
  $c_2$  & $a_{21}$ & $0$      & $0$ \\
  $c_3$ & $a_{31}$ & $a_{22}$ & $0$\\  \hline
        & $b_1$     & $b_2$     & $b_3$.
\end{tabular}
~~~$=$~~~
\begin{tabular}{c|ccc}
  $c_1$  &          &         &       \\
  $c_2$  & $a_{21}$ &          &      \\
  $c_3$  & $a_{31}$ & $a_{22}$ &      \\  \hline
         & $b_1$    & $b_2$    & $b_3$.
\end{tabular}

Na tabela mais à direita, omitimos os termos obrigatoriamente nulos.
\end{center}
 
\begin{ex} O método de Euler modificado pode ser escrito conforme:
\begin{eqnarray*}
 \tilde{u}_1 &=& u^{(n)}\\
 \tilde{u}_2 &=& u^{(n)} + h k_1\\
 u^{(n+1)} &=& u^{(n)} + h \left[\frac{1}{2}k_1+\frac{1}{2}k_2\right]\\
  \end{eqnarray*}
Identificando os coeficientes, obtemos $\nu=2$, $c_1=0$, $c_2=1$, $a_{21}=1$, $b_1=\frac{1}{2}$ e $b_2=\frac{1}{2}$. Escrevendo na forma tabular, temos:
\begin{center}
\begin{tabular}{c|cc}
  \multirow{2}{*}{$c$}    &  \multicolumn{2}{c}{\multirow{2}{*}{$A$}} 	     \\
                          &  \multicolumn{2}{c}{}                            \\ 
  \hline
                          & \multicolumn{2}{c}{$~~b~~$}   
\end{tabular}
~~~=~~~
\begin{tabular}{c|cc}
  $0$ &   &   \\
  $1$ & $1$ &   \\  \hline
      & $\frac{1}{2}$ &$\frac{1}{2}$
\end{tabular}
\end{center}
\end{ex}
 
 \begin{obs}\label{obs:obs_imp} Nos métodos chamados explícitos, os elementos da diagonal principal (e acima dela) da matriz $A$ devem ser nulos, pois a aproximação em cada estágio é calculada com base nos valores dos estágios anteriores. Nos métodos implícitos, essa restrição é removida, neste caso, o cálculo da aproximação em um estágio envolve a solução de uma equação algébrica.
  \end{obs}

\begin{obs}
Além da condição fixa $c_1=0$, para que um método seja de ordem pelo menos unitária, isto é, $p\geq 1$. 
\end{obs}

 
\subsection{Métodos de Runge-Kutta com dois estágios}\label{eq:RK_2_subsec}
Os métodos de Runge-Kutta com dois estágios $\left(\nu=2\right)$ são da seguinte forma:
\begin{eqnarray}
  \tilde{u}_1 &=&u^{(n)} \\
  \tilde{u}_2 &=&u^{(n)}  + h a_{21}k_1 \\
  u^{(n+1)}&=&u^{(n)}  + h [ b_1k_1+b_2k_2],\label{eq:RK2a}
\end{eqnarray}
onde $k_1=f(t^{(n)},u^{(n)})$ e $k_2=f(t^{(n)}+c_2h,\tilde{u}_2)$

Assumindo suavidade suficiente em $f$, usamos o polinônio de Taylor:
\begin{eqnarray}
k_2 &=&f(t^{(n)}+c_2h, \tilde{u}_2)\\
   &=&f(t^{(n)}+c_2h, u^{(n)}+ a_{21}h k_1)\\
   &=&f(t^{(n)}, u^{(n)}) + h\left[c_2 \frac{\partial f}{\partial t}+ a_{21} k_1\frac{\partial f}{\partial u}\right]+O(h^2)\\
   &=&k_1 + h\left(c_2 \frac{\partial f}{\partial t}+ a_{21} k_1\frac{\partial f}{\partial u}\right)+O(h^2)
   \end{eqnarray}
fazendo com que \eqref{eq:RK2a} se torne
\begin{eqnarray}
  u^{(n+1)}&=&u^{(n)}   + h b_1 k_1 + hb_2k_2 \\
         &=&u_n  + h b_1 k1 + hb_2\left[ k_1 + h\left(c_2 \frac{\partial f}{\partial t}+ a_{21} k_1\frac{\partial f}{\partial u}\right)+O(h^2)\right]\\
         &=&u_n  + h (b_1+b_2) k_1 + h^2 b_2\left(c_2 \frac{\partial f}{\partial t}+ a_{21} k_1\frac{\partial f}{\partial u}\right)+O(h^3)\label{eq:rk2_1}
\end{eqnarray}
Usando a equação diferencial ordinária que desejamos resolver e derivando-a em $t$, obtemos:
\begin{eqnarray}
  u'(t)    &=&f(t,u(t)),\\
  u''(t) &=&\frac{\partial f}{\partial t}+\frac{\partial f}{\partial u} u'(t) = \frac{\partial f}{\partial t} +\frac{\partial f}{\partial u} f(t,u(t)).
\end{eqnarray}
Agora,  expandimos em série de Taylor a solução exata $u(t)$ em $t=t^{(n)}$,
\begin{eqnarray}
  u(t^{(n+1)})&=&u(t^{(n)}+h)=u^{(n)}  + hu'(t) +\frac{h^2}{2} u''(t) + O(h^3)\\
            &=&u^{(n)}  + hf(t,u^{(n)}) +\frac{h^2}{2}\frac{d }{d t}f(t,u(t))+O(h^3)\\
	    &=&u^{(n)}  + hf(t,u^{(n)}) +\frac{h^2}{2}\left(\frac{\partial f}{\partial t}+\frac{\partial f}{\partial u}u'(t^{(n)})\right) +O(h^3)\\
           &=&u^{(n)}  + hf(t,u^{(n)}) +\frac{h^2}{2}\left(\frac{\partial f}{\partial t}+\frac{\partial f}{\partial u}f(t,u^{(n)})\right) +O(h^3)\\
           &=&u^{(n)}  + hf(t,u^{(n)}) +\frac{h^2}{2}\left(\frac{\partial f}{\partial t}+\frac{\partial f}{\partial u}k_1\right) +O(h^3)\label{eq:rk2_2}
           \end{eqnarray}
Finalmente comparamos os termos em \eqref{eq:rk2_1} e \eqref{eq:rk2_2} de forma a haver concordância na expansão de Taylor até segunda ordem, isto é, restanto apenas o erro de ordem 3 e produzindo um método de ordem de precisão $p=2$:
\begin{equation}\label{eq:cond_euler_2est}
  b_1+b_2=1, \quad b_2c_2 = \frac{1}{2} \quad \hbox{e} \quad a_{21}=c_2
\end{equation}
Este sistema é formada por três equações e quatro incógnitas, pelo que admite infinitas soluções. Para construir toda a família de soluções, escolha um parâmetro $\alpha\in (0,1] $ e defina a partir de \eqref{eq:cond_euler_2est}:
$$b_1=1-\frac{1}{2\alpha},~~b_2=\frac{1}{2\alpha},~~c_2=\alpha,~~a_{21}=\alpha $$
Portanto, obtemos o seguinte esquema genérico:
\begin{center}
\begin{tabular}{c|cc}
  \multirow{2}{*}{$c$}    &  \multicolumn{2}{c}{\multirow{2}{*}{$A$}} 	     \\
                          &  \multicolumn{2}{c}{}                            \\ 
  \hline
                          & \multicolumn{2}{c}{$~~b~~$}   
\end{tabular}
~~~=~~~
\begin{tabular}{c|cc}
   &   &   \\
  $c_2$ & $a_{21}$ &   \\  \hline
      & $b_1$ &$b_2$
\end{tabular}
~~~=~~~
\begin{tabular}{c|cc}
   &   &   \\
  $\alpha$ & $\alpha$ &   \\  \hline
      & $\left(1-\frac{1}{2\alpha}\right)$ &$\frac{1}{2\alpha}$
\end{tabular},~~~$0<\alpha\leq 1$.
\end{center}
 Algumas escolhas comuns são $\alpha=\frac{1}{2}$, $\alpha=\frac{2}{3}$ e $\alpha=1$:
\begin{center}
\begin{tabular}{c|cc}
  $0$ &     &   \\
  $\frac{1}{2}$ & $\frac{1}{2}$ &   \\  \hline
      & $0$ & $1$
\end{tabular},\hphantom{~~~e~~~}
\begin{tabular}{c|cc}
  $0$ &   &   \\
  $\frac{2}{3}$ & $\frac{2}{3}$ &   \\  \hline
    & $\frac{1}{4}$ & $\frac{3}{4}$
\end{tabular} ~~~e~~~
\begin{tabular}{c|cc}
  $0$ &   &   \\
  $1$ & $1$ &   \\  \hline
      & $\frac{1}{2}$ &$\frac{1}{2}$
\end{tabular}.
\end{center}
Note que a tabela da direita fornece o método Euler modificado $\left(\alpha=1\right)$. O esquema iterativo assume a seguinte forma:
\begin{eqnarray}
  \tilde{u}_1 &=&u^{(n)} \\
  \tilde{u}_2 &=&u^{(n)}  +  h\alpha k_1 \\
  u^{(n+1)}&=&u^{(n)}  + h \left[ \left(1-\frac{1}{2\alpha}\right)k_1+\frac{1}{2\alpha} k_2\right],
\end{eqnarray}
onde $k_1=f(t^{(n)},u^{(n)})$ e $k_2=f(t^{(n)}+h\alpha,\tilde{u}_2)$. Ou, equivalentemente:
\begin{eqnarray}
k_1&=&f(t^{(n)},u^{(n)})\\
k_2&=&f(t^{(n)}+h\alpha,u^{(n)}  +  h\alpha k_1)\\
  u^{(n+1)}&=&u^{(n)}  + h \left[\left (1-\frac{1}{2\alpha}\right)k_1+\frac{1}{2\alpha} k_2\right],
\end{eqnarray}


\subsection{Métodos de Runge-Kutta com três estágios}\label{sec:RK_3_subsec}
Os métodos de Runge-Kutta com 3 estágios podem ser descritos na forma tabular como:
\begin{center}
\begin{tabular}{c|ccc}
  $0$              &          &               & \\
  $c_2$            & $a_{21}$ &               & \\
  $c_3$            & $a_{31}$ & $a_{32}$      & \\  \hline
                   & $b_{1}$  & $b_{2}$       & $b_{3}$
\end{tabular}
\end{center}


Seguindo um procedimento similar ao da Seção~\ref{sec:RK_2_subsec}, podemos obter as condições equivalentes às condições \eqref{eq:cond_euler_2est} para um método com $\nu =3$ e ordem $p=3$, as quais são:
\begin{eqnarray}\label{eq:cond_euler_3est}
  b_1+b_2+b_3		&=& 1,            		\label{eq:cond_euler_3est:1} \\
  b_2c_2+b_3c_3 	&=& \frac{1}{2},	 	\label{eq:cond_euler_3est:2} \\
  b_2c_2^2+b_3c_3^2	&=& \frac{1}{3}, 		\label{eq:cond_euler_3est:3} \\
  b_3a_{32}c_2		&=& \frac{1}{6},		\label{eq:cond_euler_3est:4} \\
  a_{21}		&=& c_2,			\label{eq:cond_euler_3est:5} \\
  a_{31}+a_{32}		&=& c_3.			\label{eq:cond_euler_3est:6} 
\end{eqnarray}
Assim, temos 6 condições para determinar 8 incógnitas, o que implica a existência de uma enorme família de métodos de Runge-Kutta com três estágios e ordem $p=3$. Se fixarmos os coeficientes $c_2$ e $c_3$, podemos os outros de forma única:
\begin{eqnarray*}
 b_1&=&1-\frac{1}{2c_2}-\frac{1}{2c_3}+\frac{1}{3c_2c_3}\\
 b_2&=&\frac{3c_3-2}{6c_2(c_3-c_2)}\\
 b_3&=&\frac{2-3c_2}{6c_3(c_3-c_2)}\\
 a_{21}&=&c_2\\
 a_{31}&=&c_3-\frac{1}{6b_3c_2}\\
 a_{32}&=&\frac{1}{6b_3c_2}
\end{eqnarray*}


 Alguns exemplos de métodos de Runge-Kutta de 3 estágios são o método clássico de Runge-Kutta $\left(c_2=\frac{1}{2}\hbox{ e } c_3=1\right)$ e o método de Nyström $\left(c_2=c_3=\frac{2}{3}\right)$:
\begin{center}
\begin{tabular}{c|ccc}
  $0$           &               &               & \\
  $\frac{1}{2}$ & $\frac{1}{2}$ &               & \\
  $1$           & $-1$          & $2$           & \\  \hline
                & $\frac{1}{6}$ & $\frac{4}{6}$ & $\frac{1}{6}$
\end{tabular}
~~~e~~~~~~\begin{tabular}{c|ccc} 
  $0$ &     &   & \\
  $\frac{2}{3}$ & $\frac{2}{3}$ &   & \\
  $\frac{2}{3}$ & $0$ &$\frac{2}{3}$& \\  \hline
      & $\frac{2}{8}$ &$\frac{3}{8}$& $\frac{3}{8}$
\end{tabular}.
\end{center}

\subsection{Métodos de Runge-Kutta com quatro estágios}\label{ssec:RK_4_subsec}
As técnicas utilizadas nas seções \ref{sec:RK_2_subsec}  e \ref{sec:RK_3_subsec} podem ser usadas para obter métodos de quarta ordem ($p=4$) e quatro estágios ($\nu=4$). As seguintes tabelas descrevem os dois esquemas mais conhecidos de Runge-Kutta quarta ordem com quatro estágios. O primeiro é denominado \emph{método de Runge-Kutta 3/8} e o segundo é chamado de \emph{método de Runge-Kutta clássico}.

\begin{center}
\begin{tabular}{c|cccc}
  $0$ 		&     			&   		&   		&    \\
  $\frac{1}{3}$ & $\frac{1}{3}$ 	&   		&   		&    \\
  $\frac{2}{3}$ & $-\frac{1}{3}$ 	&$1$		&   		&    \\
  $1$ 		& $1$ 			&$-1$		&$1$		&    \\  \hline
		& $\frac{1}{8}$ 	&$\frac{3}{8}$	& $\frac{3}{8}$	& $\frac{1}{8}$
\end{tabular}
~~~\hbox{e}~~~~~~~
\begin{tabular}{c|cccc}
  $0$ 		&     			&   		&   		&    \\
  $\frac{1}{2}$ & $\frac{1}{2}$ 	&   		&   		&    \\
  $\frac{1}{2}$ & $0$ 			&$\frac{1}{2}$	&   		&    \\
  $1$ 		& $0$ 			&$0$		&$1$		&    \\  \hline
		& $\frac{1}{6}$ 	&$\frac{2}{6}$	& $\frac{2}{6}$	& $\frac{1}{6}$
\end{tabular}
\end{center}

O método de Runge-Kutta clássico é certamente o mais notório dos métodos de Runge-Kutta e seu esquema iterativo pode ser escrito como a seguir:
\begin{eqnarray*}
k_1&=&f\left(t^{(n)},u^{(n)}\right)\\
k_2&=&f\left(t^{(n)}+h/2,u^{(n)}+k_1/2\right)\\
k_3&=&f\left(t^{(n)}+h/2,u^{(n)}+k_2/2\right)\\
k_4&=&f\left(t^{(n)}+h,u^{(n)}+k_3\right)\\
u^{(n+1)}&=&u^{(n)}+h\frac{k_1+2k_2+2k_3+k_4}{6}
\end{eqnarray*}
A seguinte heurística, usando o método de Simpson para quadratura numérica, pode ajudar a compreender os estranhos coeficientes:
\begin{eqnarray*}
u({t^{(n+1)}})-u({t^{(n)}})&=&\int_{t^{(n)}}^{t^{(n+1)}}f(t,u(s))ds \\
&\approx& \frac{h}{6}\left[ f\left(t^{(n)},u(t^{(n)})\right)+4f\left(t^{(n)}+h/2,u(t^{(n)}+h/2)\right)\right.\\
&+&\left.f\left(t^{(n)}+hu(t^{(n)}+h)\right)\right]\\
&\approx& h\frac{k_1+4(\frac{k_2+k_3}{2})+k_4}{6}
\end{eqnarray*}
onde $k_1$ e $k_4$ representam os valores de $f(t,u)$ nos extremos; $k_2$ e $k_3$ são duas aproximações diferentes para a inclinação no meio do intervalo.

\subsection*{Exercícios resolvidos}
\begin{exeresol}\label{exeresol:problema_resolvido_RK3} Construa o esquema iterativo o método clássico de Runge-Kutta três estágios cuja tabela é dada a seguir:

\begin{tabular}{c|ccc}
  $0$           &               &               & \\
  $\frac{1}{2}$ & $\frac{1}{2}$ &               & \\
  $1$           & $-1$          & $2$           & \\  \hline
                & $\frac{1}{6}$ & $\frac{4}{6}$ & $\frac{1}{6}$
\end{tabular}

\end{exeresol}
\begin{resol}
 \begin{eqnarray*}
    k_1&=&f\left(t^{(n)},u^{(n)}\right)\\
    k_2&=&f\left(t^{(n)}+h/2,u^{(n)}+k_1/2\right)\\
    k_3&=&f\left(t^{(n)}+h,u^{(n)}-k_1+2k_2\right)\\
  u^{(n+1)}&=&u^{(n)}+h\frac{k_1+4k_2+k_4}{6}
 \end{eqnarray*}
\end{resol}

\begin{exeresol} Utilize o método clássico de Runge-Kutta três estágios para calcular o valor de $u(2)$ com passos $h=10^{-1}$ e $h=10^{-2}$ para o seguinte problema de valor inicial:
\begin{eqnarray*}
 u'(t)&=& -u(t)^2 + t,\\
 u(0) &=&0.
\end{eqnarray*}
\ifisscilab
Aplicando o processo iterativo obtido no Problema Resolvido~\ref{exeresol:problema_resolvido_RK3}, obtemos a seguinte rotina:
\begin{verbatim}
function [y]= f(t,u)
    y= t-u**2
endfunction


function [u] = RK3_classico(h,Tmax,u1)
    itmax = Tmax/h;
    u=zeros(itmax+1)
    u(1)=u1

    for i = 1:itmax
        t=(i-1)*h
        k1 = f(t, u(i))
        k2 = f(t+h/2, u(i) + h*k1/2)
        k3 = f(t+h,   u(i) + h*(2*k2-k1))

        u(i+1) = u(i) + h*(k1+4*k2+k3)/6
    end
endfunction
 \end{verbatim}
A qual pode ser invocada com:
\begin{verbatim}
->sol=RK3_classico(1e-2,2,0);sol(201)
 ans  =
 
    1.1935760016451  
 
-->sol=RK3_classico(1e-3,2,0);sol(2001)
 ans  =
 
    1.1935759753635 
\end{verbatim}

 
\fi
\ifisoctave
Aplicando o processo iterativo obtido no Problema Resolvido~\ref{exeresol:problema_resolvido_RK3}, obtemos a seguinte rotina:
\begin{verbatim}
function [y]= f(t,u)
  y= t-u**2;
endfunction

function [u] = RK3_classico(h,Tmax,u1)
  itmax = Tmax/h;
  u=zeros(itmax+1);
  u(1)=u1;

  for i = 1:itmax
    t=(i-1)*h;
    k1 = f(t, u(i));
    k2 = f(t+h/2, u(i) + h*k1/2);
    k3 = f(t+h,   u(i) + h*(2*k2-k1));

    u(i+1) = u(i) + h*(k1+4*k2+k3)/6;
  end
endfunction
 \end{verbatim}
A qual pode ser invocada como:
\begin{verbatim}
>> sol=RK3_classico(1e-2,2,0);sol(201)
ans =  1.19357600164514
>> sol=RK3_classico(1e-3,2,0);sol(2001)
ans =  1.19357597536351 
\end{verbatim}
\fi

\ifispython
Aplicando o processo iterativo obtido no Problema Resolvido~\ref{exeresol:problema_resolvido_RK3}, obtemos a seguinte rotina:
\begin{verbatim}
 def f(t,u):
	return t-u**2


def RK3_classico(h,Tmax,u1):
  	itmax = Tmax/h;
	u=np.empty(itmax+1)
	u[0]=u1
	
	for i in np.arange(0,itmax):
		t=i*h

		k1 = f(t,     u[i])
		k2 = f(t+h/2, u[i] + h*k1/2)
		k3 = f(t+h,   u[i] + h*(2*k2-k1))

		u[i+1] = u[i] + h*(k1+4*k2+k3)/6
	return u


	
Tmax=2			#tempo maximo de simulacao
u1=0			#condicoes iniciais na forma vetorial
h=1e-2			#passo

sol=RK3_classico(h,Tmax,u1);
itmax=Tmax/h
print(sol[itmax])
\end{verbatim}

\fi
 
\end{exeresol}

\subsection*{Exercícios}
\begin{exer} Aplique o esquema de Runge-Kutta segunda ordem com dois estágios cujos coeficientes são dados na tabela a seguir
 \begin{tabular}{c|cc}
  $0$ &   &   \\
  $\frac{2}{3}$ & $\frac{2}{3}$ &   \\  \hline
    & $\frac{1}{4}$ & $\frac{3}{4}$
\end{tabular}
para resolver o problema de valor inicial dado por:
\begin{eqnarray*}
x'(t)&=&\sin(x(t)),\\
x(0)&=&2.
\end{eqnarray*}
para $t=2$ com $h=1e-1$, $h=1e-2$ e $h=1e-3$. Expresse sua resposta com oito dígitos significativos corretos.
\end{exer}
\begin{resp}
 $2,9677921$, $2,9682284$ e $2,9682325$.
\end{resp}


\begin{exer}Resolva pelo método de Euler, Euler melhorado, Runge-Kutta clássico três estágios e Runge-Kutta clássico quatro estágios o problema de valor inicial tratados nos  exercícios resolvidos \ref{exeresol:exeresol1} e \ref{exeresol:exeresol1_euler_melhorado} dado por:
\begin{eqnarray}
     u'(t)&=& -0,5u(t)+2+t\\
            u(0) &=&  8
\end{eqnarray}

Usando os seguinte passos: $h=1$, $h=10^{-1}$, $h=10^{-2}$ e $h=10^{-3}$ e compare a solução aproximada em $t=1$ com as soluções obtidas com a solução exata dada por:
\begin{equation}
     u(t) = 2t+8e^{-t/2} \Longrightarrow u(1)=2+8e^{-1/2} \approx 6,85224527770107
\end{equation}
\end{exer}

\begin{resp} 
\begin{center}
 \begin{tabular}{|l|l|l|l|l|}%\label{pvi:tab_euler}
\hline
Euler & 6,0000000 & 6,7898955 & 6,8461635 & 6,8516386\\
\hline
$\varepsilon_{rel}$ & 1,2e-01 & 9,1e-03 & 8,9e-04 & 8,9e-05\\
\hline
Euler mod, & 7,0000000 & 6,8532949 & 6,8522554 & 6,8522454\\
\hline
$\varepsilon_{rel}$ & 2,2e-02 & 1,5e-04 & 1,5e-06 & 1,5e-08\\
\hline
RK$_3$ & 6,8333333 & 6,8522321 & 6,8522453 & 6,8522453\\
\hline
$\varepsilon_{rel}$ & 2,8e-03 & 1,9e-06 & 1,9e-09 & 1,8e-12\\
\hline
RK$_4$ & 6,8541667 & 6,8522454 & 6,8522453 & 6,8522453\\
\hline
$\varepsilon_{rel}$ & 2,8e-04 & 1,9e-08 & 1,9e-12 & 1,3e-15\\
\hline
\end{tabular}
\end{center}

% Veja o gráfico da solução para $h=1, 0.5, 0.1, 0.05$:
% \begin{figure}
% \includegraphics[width=\textwidth]{euler.eps}
% \end{figure}
\end{resp}



\begin{exer}Aplique o método de Euler, o método de Euler melhorado, o método clássico de Runge-Kutta três estágios e o método clássico de Runge-Kutta quatro estágios para resolver o problema de valor inicial dado por
\begin{eqnarray*}
u'&=& u + t\\
u(0)&=&1
\end{eqnarray*}
com passo $h=1$, $h=10^{-1}$, $h=10^{-2}$ e $h=10^{-3}$  para obter aproximações para $u(1)$. Compare com a solução exata dada do problema dada por $u(t) =  2e^t-t-1$ através do erro relativo e observe a ordem de precisão do método. Expresse a sua resposta com oito dígitos significativos para a solução e 2 dígitos significativos para o erro relativo.
\end{exer}
\begin{resp}

\begin{center}
 \begin{tabular}{|l|l|l|l|l|}%\label{pvi:tab_euler}
\hline
Euler & 2,0000000 & 3,1874849 & 3,4096277 & 3,4338479\\
\hline
$\varepsilon_{rel}$ & 4,2e-01 & 7,2e-02 & 7,8e-03 & 7,9e-04\\
\hline
Euler mod & 3,0000000 & 3,4281617 & 3,4364737 & 3,4365628\\
\hline
$\varepsilon_{rel}$ & 1,3e-01 & 2,4e-03 & 2,6e-05 & 2,6e-07\\
\hline
RK$_3$ & 3,3333333 & 3,4363545 & 3,4365634 & 3,4365637\\
\hline
$\varepsilon_{rel}$ & 3,0e-02 & 6,1e-05 & 6,5e-08 & 6,6e-11\\
\hline
RK$_4$ & 3,4166667 & 3,4365595 & 3,4365637 & 3,4365637\\
\hline
$\varepsilon_{rel}$ & 5,8e-03 & 1,2e-06 & 1,3e-10 & 1,2e-14\\
\hline
\end{tabular}
\end{center}
\end{resp}



\section{Métodos de Runge-Kutta implícitos}\label{sec:sec_IRK}\index{método!de Runge-Kutta exlícito}
Nas seções anteriores contruímos os métodos de Runge-Kutta implícito, nesta seção veremos uma nova família de métodos chamados implícitos. Nos métodos implícitos o processo recursivo produz uma equação implícita para $y^{(n+1)}$ em termos de $y^{(n)}$, como por exemplo:
\begin{equation}\label{eq:IRK:exemplo1}
  \begin{split}
    y^{(n+1)}&=y^{(n)}+h y^{(n+1)}\\
    y^{(1)}&=1
  \end{split}
\end{equation}
para resolver o problema de valor inicial dado por:
\begin{eqnarray}
 y'(t)&=&y(t)\\
 y(0)&=&1
 \end{eqnarray}
Note que este método é \emph{implícito} pois a expressão que define a iteração depende de $u^{(n+1)}$ dos dois lados da Equação~\eqref{eq:IRK:exemplo1}, exigindo que o termo seja isolado para a aplicação do método. 
 

\subsection{Método de Euler implícito}\index{método!de Euler}\label{sec:euler_imp}
Contruiremos, agora, o mais simples dos métodos para resolver problemas de valor inicial: o método de Euler implícito, uma variante do método de Euler (explícito) que vimos na Seção~\ref{sec:euler}. Seguinte o mesmo raciocício daquela seção, integramos o problema de valor inicial dado por
\begin{eqnarray}
  u'(t)  &=& f(t,u(t)) \\
  u(t^{(1)}) &=& a
\end{eqnarray}
de $t^{(1)}$ até $t^{(2)}$ obtemos (como feito anteriormente) para obter
\begin{eqnarray}
  u(t^{(2)})      &=& u(t^{(1)}) +  \int_{t^{(1)}}^{t^{(2)}} f(t,u(t)) \; dt
\end{eqnarray}

Diferentemente do método de Euler estudado, o método de Euler implícito aproxima a função $f(t,u)$ pela uma função constante $f(t,u(t)) \approx  f(t^{(2)},u^{(2)})$ e, assim, obtemos o seguinte esquema:
\begin{eqnarray}
   u^{(2)} &=&  u^{(1)} + h f(t^{(2)},u^{(2)})
\end{eqnarray}


Generalizando este procedimento para $t_n$ obtemos o \emph{método de Euler implícito}
\begin{eqnarray}
u^{(n+1)}=u^{(n)} + h\;f(t^{(n+1)},u^{(n+1)}).
\end{eqnarray}


Note que este método é \emph{implícito} (a equação é implícita) pois depende de $u_{n+1}$ dos dois lados da equação. Se a função $f$ for simples o suficiente, podemos resolver a equação isolando o termo $u_{n+1}$. Se isso não for possível, devemos usar um dos métodos vistos anteriormente para calcular as raízes da equação (por exemplo, método da bissecção e método de Newton).

\begin{ex} Considere o problema de valor inicial dado por
\begin{eqnarray}
u'(t)&=& \lambda u(t) \\
u(0)&=&1
\end{eqnarray}
A relação de recorrência do método de Euler implícito é dado por:
\begin{eqnarray*}
 y^{(n+1)}&=&y^{(n)} h\lambda y^{(n+1)}\\
 y^{(1)}&=&1
\end{eqnarray*}
Isolando a $y^{(n+1)}$ na primeira equação, obtemos o processo iterativo dado por:
\begin{eqnarray*}
 y^{(n+1)}&=&\frac{y^{(n)}}{1-\lambda h}\\
 y^{(1)}&=&1
\end{eqnarray*}
\end{ex}


\subsection{O método trapezoidal}\index{método!trapezoidal}\label{sec:trapezoidal}
O método de Euler aproxima a função $f(t,u)$ como uma constante no intervalo $[t^{(1)},t^{(2)}]$. O método trapezoidal é muito semelhante ao método de Euler melhorado estudado na Seção~\ref{sec:sec_euler_mod}, integramos de $t^{(1)}$ até $t^{(2)}$ a equação diferencial envolvida no problema de valor inicial
\begin{equation}\label{eq:EDO}
  \begin{split}
    u'(t)  &= f(t,u(t)),~~t>t^{(1)} \\
    u(t^{(1)}) &= a.
  \end{split}
\end{equation}
para obter:
\begin{eqnarray}
  \int_{t^{(1)}}^{t^{(2)}} u'(t) \;dt &=& \int_{t^{(1)}}^{t^{(2)}} f(t,u(t)) \; dt\\
  u(t^{(2)})-u(t^{(1)})               &=& \int_{t^{(1)}}^{t^{(2)}} f(t,u(t)) \; dt\\
  u(t^{(2)})                      &=& u(t^{(1)}) +  \int _{t^{(1)}}^{t^{(2)}} f(t,u(t)) \; dt
\end{eqnarray}
Exatamente como no método de Euler melhorado, aplicamos a regra do trapézio (ver \ref{sec:trapezio}) à integral envolvida no lado direito da expressão, isto é:
\begin{eqnarray*}
\int _{t^{(1)}}^{t^{(2)}} f(t,u(t)) \; dt = \left[\frac{f\left(t^{(1)},u(t^{(1)})\right)+f\left(t^{(2)},u(t^{(2)})\right)}{2}\right]h + O(h^3)
\end{eqnarray*}
onde $h=t^{(2)}-t^{(1)}$.

Repetindo este procedimento para cada $n$, obtemos o esquema iterativo do método trapezoidal:
\begin{eqnarray}
  u^{(n+1)} &=& u^{(n)} +  \frac{h}{2} \left(f(t^{(n)},u^{(n)})+f(t^{(n+1)},u^{(n+1)})\right)\\
  u^{(1)} &=&a
\end{eqnarray}


\begin{ex} Considere o problema de valor inicial dado por
\begin{eqnarray}
u'(t)&=& \lambda u(t) \\
u(0)&=&1
\end{eqnarray}
onde $\lambda$ é uma constante. A relação de recorrência do método de Euler trapezoidal é dado por:
\begin{eqnarray*}
 y^{(n+1)}&=&y^{(n)}-\frac{\lambda h}{2} \left[y^{(n+1)}+y^{(n)}\right]\\
 y^{(1)}&=&1
\end{eqnarray*}
Isolando a $y^{(n+1)}$ na primeira equação, obtemos o processo iterativo dado por:
\begin{eqnarray*}
 y^{(n+1)}&=&\frac{1+ \lambda h/2}{1-\lambda h/2}y^{(n)}\\
 y^{(1)}&=&1
\end{eqnarray*}
\end{ex}

\subsection{O método theta}
O método theta é uma generalização dos métodos de Euler e trapezoidal. A relação de recorrência do método theta é dada por:
\begin{eqnarray}
  u^{(n+1)} &= u^{(n)} +  h (\theta f(t^{(n)},u^{(n)})+(1-\theta )f(t^{(n+1)},u^{(n+1)}))
\end{eqnarray}
Observe que quando $\theta =1$, a relação recai no método de Euler, quando $\theta =\frac{1}{2}$, no método trapezoidal e quando $\theta=0$, no método de Euler implícito.


\subsection*{Exercícios resolvidos}

\begin{exeresol}
 Considere o problema de valor inicial dado por:
 \begin{eqnarray*}
  y'(t)&=&y(t)\left(1-y(t)\right),\\
  y(0)&=&\frac{1}{2}.
 \end{eqnarray*}
Construa a recursão via método de Euler implícito e explicite o termo $y^{(n+1)}$.
\end{exeresol}
\begin{resol}
 O método de Euler implícito produz a seguinte recursão:
 $$y^{(n+1)}=y^{(n)}+h y^{(n+1)}\left(1-y^{(n+1)}\right)$$
 a qual pode ser escrita como:
 $$h\left[y^{(n+1)}\right]^2+ (1-h)y^{(n+1)}-y^{(n)}=0$$
 Usando a fórmula da equação quadrática temos:
 $$y^{(n+1)}= \frac{-(1-h)\pm \sqrt{(1-h)^2+4hy^{(n)}}}{2h}$$
Como a condição inicial é positiva, é fácil ver que $y(t)>0$ para todo $t$ e, portanto:
 \begin{eqnarray*}
y^{(n+1)}&=& \frac{-(1-h)+ \sqrt{(1-h)^2+4hy^{(n)}}}{2h}\\
&=&  (1-h) \frac{-1+ \sqrt{1+ \frac{4hy^{(n)}}{(1-h)^2} }}{2h}\\
&=&  \frac{(1-h)}{2h} \left[\sqrt{1+ \frac{4hy^{(n)}}{(1-h)^2} }-1\right]\\
&=&  \frac{(1-h)}{2h} \frac{4hy^{(n)}}{(1-h)^2} \frac{1}{\left[\sqrt{1+ \frac{4hy^{(n)}}{(1-h)^2} }+1\right]}\\
&=&  \frac{2}{(1-h)} \ \frac{1}{\left[1+\sqrt{1+ \frac{4hy^{(n)}}{(1-h)^2} }\right]}y^{(n)}\\
 \end{eqnarray*}
\end{resol}


%\section{Métodos de passo múltiplo}
\section{Método de Adams-Bashforth}\index{Método!de Adams-Bashforth}\index{Método!de passo múltiplo}\label{sec:Adams_Bashforth}
Seja o problema de valor inicial
\begin{eqnarray}
  u'(t) &= f(t,u(t)) \\
  u(t_0) &= a
\end{eqnarray}

Nos métodos de passo simples, os valores calculados para $f\left(t^{(n)},u(t^{(n)})\right)$ nos passos anteriores são desprezados ao calcular o próximo passo. Nos métodos de passo múltiplo, os valores de $f\left(t,u)\right)$, nos passos $n$, $n+1$, ..., $n+s-1$ são utilizados ao calcular $f$ em $t^{(n+s)}$.

Integrando a equação diferencial no intervalo $[t^{(n+s-1)},t^{(n+s)}]$, obtemos:
\begin{equation}\label{eq:mpm1}
  u^{(n+s)}  = u^{(n+s-1)}  + \int_{t^{(n+s-1)}}^{t^{(n+s)}} f(t,u(t)) dt
\end{equation}
No método de Adams-Bashforth, o integrando em \eqref{eq:mpm1} é aproximado pelo polinômio que interpola $f(t^{(k)},u^{(k)})$ para $k = n, n+1, n+2, \ldots, n+s-1$, isto é:
\begin{eqnarray}%\label{pvi:mpm1}
  u^{(n+s)}  &=& u^{(n+s-1)}  + \int_{t^{(n+s-1)}}^{t^{(n+s)}} p(t) dt
\end{eqnarray}
onde $p(t)$ é polinômio de grau $s-1$ dado na forma de Lagrange por:
$$p(t)=\sum_{j=0}^{s-1}\left[f(t^{(n)},u^{(n)}) \prod_{k=0,k\neq j}^{s-1} \frac{t-t^{(n+k)}}{t^{(n+j)}-t^{(n+k)}}\right]$$
Agora observamos que
$$\int _{t^{(n+s-1)}}^{t^{(n+s)}} p(t) dt=h\sum_{j=0}^{s-1}\beta_j f(t^{(n+j)},u^{(n+j)})$$
onde 
\begin{equation}\label{eq:betaj}
\beta_j= \frac{1}{h}\int_{t^{(n+s-1)}}^{t^{(n+s)}} \prod_{k=0,k\neq j}^{s-1} \frac{t-t^{(n+k)}}{t^{(n+j)}-t^{(n+k)}}dt
\end{equation}

e obtemos a relação de recorrência:
\begin{eqnarray}\label{eq:mpm2}
  u^{(n+s)}  &=& u^{(n+s-1)}  + h\sum_{j=0}^{s-1}\beta_j f(t^{(n+j)},u^{(n+j)})
\end{eqnarray}
Observe que a integral envolvida no cálculo dos coeficientes $\beta_j$ em \eqref{eq:betaj} pode ser simplificada via a mudança de variáveis $t=t^{(n+s-1)}+h\tau$:
\begin{eqnarray}\label{eq:betaj_s}
\beta_j&=& \int_{0}^{1} \prod_{k=0,k\neq j}^{s-1} \frac{\tau+s-k-1}{j-k}d\tau\\
&=& \frac{(-1)^{s-j-1}}{j!(s-j-1)!}\int_{0}^{1} \prod_{k=0,k\neq j}^{s-1}(\tau+s-k-1)d\tau\\
&=& \frac{(-1)^{s-j-1}}{j!(s-j-1)!}\int_{0}^{1} \prod_{k=0,k\neq s-j-1}^{s-1}(\tau+k)d\tau\\
\end{eqnarray}


\begin{obs}[Ordem do método de Adasm-Bashforth] Da teoria de interpolação (ver capítulos \ref{cap:cap_interpola} e \ref{cap:cap_integra}), temos que o erro de aproximação da integral de uma função suficientemente suave por um polinômio interpolador em $s$ pontos é de ordem $s+1$. Assim, o erro local de truncamento do método de Adams-Bashforth com $s$ passos é $s+1$ e, portanto, o erro global de truncamente é de ordem $s$.   
\end{obs}


\begin{ex} Calcule os coeficientes de Adams-Bashforth para $s=2$ e, depois, construa seu processo iterativo.
 \begin{eqnarray*}
  \beta_0&=&-\int_0^1 (\tau+2-1-1)d\tau=-\frac{1}{2}\\
  \beta_1&=&\int_0^1 (\tau+2-0-1)d\tau=\frac{3}{2}
 \end{eqnarray*}
 O processo iterativo é dado por:
 \begin{equation*}
  y^{(n+2)}=y^{(n)}+\frac{h}{2}\left[3f\left(t^{(n+1)},u(t^{(n+1)})\right)-f\left(t^{(n)},u(t^{(n)})\right)\right]
 \end{equation*}
\end{ex}


\begin{ex} Calcule os coeficientes e de Adams-Bashforth para $s=3$ e, depois, construa o processo iterativo.
 \begin{eqnarray*}
  \beta_0&=&\frac{1}{2}\int_0^1 {(\tau+3-1-1)}\cdot {(\tau+3-2-1)} d\tau=\frac{5}{12}\\
  \beta_1&=&-\int_0^1 {(\tau+3-0-1)}\cdot {(\tau+3-2-1)} d\tau=-\frac{4}{3}\\
  \beta_2&=&\frac{1}{2}\int_0^1 {(\tau+3-0-1)}\cdot {(\tau+3-1-1)} d\tau=\frac{23}{12}
 \end{eqnarray*}
 O processo iterativo é dado por:
 \begin{equation*}
  y^{(n+3)}=y^{(n)}+\frac{h}{12}\left[23f\left(t^{(n+2)},u(t^{(n+2)})\right)-16f\left(t^{(n+1)},u(t^{(n+1)})\right)+5f\left(t^{(n)},u(t^{(n)})\right)\right]
 \end{equation*}
\end{ex}

\begin{obs} Os coeficientes do método de Adams-Bashforth de ordem $s$ podem, alternativamente, ser obtidos exigindo que o sistema seja exato para $f(t,u)=t^0$, $f(t,u)=t^1$, $f(t,u)=t^2$, \ldots, $f(t,u)=t^{s-1}$.
 
\end{obs}

\begin{ex} Obtenha o método de Adams-Bashforth para $s=4$ como
\begin{eqnarray}
  u^{(n+4)}  &=& u^{(n+3)}  + \int _{t^{(n+3)}}^{t^{(n+4)}} f(t,u(t)) dt \\
  u^{(n+4)}  &=& u^{(n+3)}  + h \sum_{m=0}^{3} b_m f^{(n+m)} \\
  u^{(n+4)}  &=& u^{(n+3)}  + h \left[b_3f^{(n+3)} +b_2f^{(n+2)} +b_1f^{(n+1)} +b_0f^{(n)}\right]
\end{eqnarray}
Para isso devemos obter $[b_3,b_2,b_1,b_0]$ tal que o método seja exato para polinômios até ordem $3$. Podemos obter esses coeficientes de maneira análoga a obter os coeficientes de um método para integração.

Supondo que os nós $t^{(k)}$ estejam igualmente espaçados, e para facilidade dos cálculos, como o intervalo de integração é $[t^{(n+3)},t^{(n+4)}]$, translade $t^{(n+3)}$ para a origem tal que $[t^{(n)},t^{(n+1)},\ldots ,t^{(n+4)}]=[-3h,-2h,-h,0,h]$.

Considere a base $[1, t, t^2, t^3]$ e substitua $f(t)$ por cada um dos elementos desta basa, obtendo:
\begin{eqnarray*}
  \int _0^{h} 1  \;dt = h             &=& h\left[ b_0(1)  +b_1(1)    + b_2(1)   + b_3(1)    \right]\\
  \int _0^{h} t  \;dt = \frac{h^2}{2}  &=& h\left[ b_0(0)  +b_1(-h)   + b_2(-2h) + b_3(-3h)  \right]\\
  \int _0^{h} t^2 \;dt = \frac{h^3}{3}  &=& h\left[b_0(0)^2 +b_1(-h)^2  + b_2(-2h)^2+ b_3(-3h)^2 \right]\\
  \int _0^{h} t^3 \;dt = \frac{h^4}{4} &=& h\left[ b_0(0)^3 +b_1(-h)^3  + b_2(-2h)^3+ b_3(-3h)^3 \right]
\end{eqnarray*}
que pode ser escrito na forma matricial
\begin{eqnarray}
\left(
  \begin{array}{cccc}
    1  &  1    & 1   & 1\\
    0  &  -1   & -2  & -3\\
    0  &  1    & 4   &  9\\
    0  &  -1   & -8  & -27
  \end{array}
\right)
\left(\begin{array}{c}  b_0 \\ b_1\\ b_2\\b_3   \end{array}\right)
=
\left(\begin{array}{c}  1  \\ 1/2 \\ 1/3 \\ 1/4  \end{array}\right)
\end{eqnarray}
Resolvendo o sistema obtemos
$$[b_0,b_1,b_2,b_3]=\left[-\frac{9}{24},\frac{37}{24},-\frac{59}{24},\frac{55}{24}\right]$$
fornecendo o \emph{método de Adams-Bashforth de $4$ estágios}
\begin{eqnarray}\label{AB4}
  u^{(n+4)}  &= u^{(n+3)}  + \frac{h}{24} [55 f^{(n+3)} -59f^{(n+2)} +37f^{(n+1)} -9f^{(n)}]
\end{eqnarray}
\end{ex}


A tabela abaixo mostra as coeficientes do método de Adams-Bashforth para até 8 passos. 
 \begin{center}
\begin{tabular}{|c|cccccccc|}
\hline
1 & 1 &&&&&&&\\
2 & $-\frac{1}{2}$ & $\frac{3}{2}$&&&&&&\\
3 & $\frac{5}{12}$ & $-\frac{4}{3}$ & $\frac{23}{12}$&&&&&\\
4 & $-\frac{3}{8}$ & $\frac{37}{24}$ & -$\frac{59}{24}$ & $\frac{55}{24}$&&&&\\
5 & $\frac{251}{720}$ & -$\frac{637}{360}$ & $\frac{109}{30}$& -$\frac{1387}{360}$& $\frac{1901}{720}$&&&\\
6 & $-{\frac {95}{288}}$ & ${\frac {959}{480}}$ & $-{\frac {3649}{720}}$ & ${\frac {4991}{720}}$ & $-{\frac {2641}{480}}$ & $\frac {4277}{1440}$&&\\
7 & $\frac {19087}{60480}$ & $-{\frac {5603}{2520}}$ & ${\frac {135713}{20160}}$ & $-{\frac {10754}{945}}$ & ${\frac {235183}{20160}}$ & $-{\frac {18637}{2520}}$ & ${\frac {198721}{60480}}$&\\
8 & $-{\frac {5257}{17280}}$&${\frac {32863}{13440}}$&$-{\frac {115747}{13440}}$&${\frac {2102243}{120960}}$&$-{\frac {296053}{13440}}$&${\frac {242653}{13440}}$&$-{\frac {1152169}{120960}}$&${\frac {16083}{4480}}$\\
\hline 
\end{tabular}
 \end{center}
\begin{obs} Note que os métodos de múltiplo passo requerem o conhecimento dos $s$ valores previamente computados para calcular $y^{(n+s)}$. Assim, para inicializar um algoritmo com mais de um passo, não é suficiente conhecer a condição inicial. Usualmente, calcula-se os primeiros $s$ passos usando um algoritmo de passo simples da mesma ordem do método múltiplo passo a ser aplicado.
\end{obs}




\subsection*{Exercícios resolvidos}
\begin{exeresol}\label{exeresol:Adams_Bashforth:prob1} Resolva numericamente o problema de valor inicial dado por:
\begin{eqnarray*}
 y'(t)&=& \sqrt{1+y(t)}\\
 y(0)&=& 0
\end{eqnarray*}
aplicando o método de Adams-Bashforth de dois passos e inicializando o método através do método de Euler modificado. Calcule o valor de $y(1)$ com passo de tamanho $h=0,1$.
\end{exeresol}
\begin{resol} 
Primeiro observamos que o processo resursivo do método de Adams é dado por: 
\begin{equation*}
  y^{(n+2)}=y^{(n+1)}+\frac{h}{2}\left[3f\left(t^{(n+1)},u(t^{(n+1)})\right)-f\left(t^{(n)},u(t^{(n)})\right)\right],~n=1,2,\ldots
 \end{equation*}
O valor inicial é dado por $y^{(1)}=0$. No entanto, para inicializar o método, precisamos calcular $y^{(2)}$, para tal, aplicamos o método de Euler modificado:
\begin{eqnarray*}
 k_1&=&\sqrt{1+0}=1\\
 k_2&=&\sqrt{1+0,1}=\sqrt{1,1}\approx 1,0488088 \\
 y^{(2)}&=& \frac{0,1}{2}\left(1+1,0488088 \right)=0,10244044 
  \end{eqnarray*}
Aplicando o método de Adams-Bashforth, obtemos:  
  \begin{eqnarray*}
 y^{(1)}&=&0\\
 y^{(2)}&=&0,10244044\\
 y^{(3)}&=&0,20993619\\
 y^{(4)}&=&0,32243326\\
 y^{(5)}&=&0,43993035\\
 y^{(6)}&=&0,56242745\\
 y^{(7)}&=&0,68992455\\
 y^{(8)}&=&0,82242165\\
 y^{(9)}&=&0,95991874\\
 y^{(10)}&=&1,10241584\\
 y^{(11)}&=&1,24991294
  \end{eqnarray*}
\end{resol}

\ifispython
A seguinte rotina implementa o método:
\begin{verbatim}
def f(t,u):
	return np.sqrt(1+u)

def adams_bash_2(h,Tmax,u1):
	dim=np.size(u1)
	itmax=np.int(Tmax/h)
	u=np.empty((itmax+1,dim))
	u[0,:]=u1

	#inicaliza com RK2
	k1 = f(0,   u[0,:])
	k2 = f(h, u[0,:] + k1* h)
	u[1,:] = u[0,:] + (k1+k2)* h/2

	fn_0=k1
	for i in np.arange(0,itmax-1):
		t=(i+1)*h
		fn_1 = f(t,   u[i+1,:])
		u[i+2,:] = u[i+1,:] + h*(-.5*fn_0 + 1.5*fn_1)
		fn_0=fn_1
	return u





u0=0
h=1e-1
Tmax=1
u=adams_bash_2(h,Tmax,u0)

print u
\end{verbatim}
\fi

\emconstrucao

\subsection*{Exercícios}

\emconstrucao

\section{Método de Adams-Moulton}\index{Método!de Adams-Moulton}\label{sec:Adams_Moulton}
O método de Adams-Moulton, assim como o método de Adams-Bashforth, é um método de passo múltiplo. A diferença entre estes dois métodos é que Adams-Bashforth é explícito, enquanto Adams-Moulton é implícito, isto é, os valores de $f\left(t,u)\right)$, nos passos $n$, $n+1$, ..., $n+s-1$ e, inclusive, $n+s$ são utilizados ao calcular $f$ em $t^{(n+s)}$.

Considere o problema de valor inicial
\begin{eqnarray*}
  u'(t) &= f(t,u(t)) \\
  u(t_0) &= a
\end{eqnarray*}

Integrando a equação diferencial no intervalo $[t^{(n+s-1)},t^{(n+s)}]$, obtemos:
\begin{equation}\label{eq:mpm2a}
  u^{(n+s)}  = u^{(n+s-1)}  + \int_{t^{(n+s-1)}}^{t^{(n+s)}} f(t,u(t)) dt
\end{equation}
 Agora o integrando em \eqref{eq:mpm2a} é aproximado pelo polinômio que interpola $f(t^{(k)},u^{(k)})$ para $k = n, n+1, n+2, \ldots, n+s$, isto é:
\begin{equation}\label{eq:mpm2b}
  u^{(n+s)} = u^{(n+s-1)}  + \int_{t^{(n+s-1)}}^{t^{(n+s)}} p(t) dt
\end{equation}
onde $p(t)$ é polinômio de grau $s$ dado na forma de Lagrange por:
$$p(t)=\sum_{j=0}^{s}\left[f(t^{(n)},u^{(n)}) \prod_{k=0,k\neq j}^{s} \frac{t-t^{(n+k)}}{t^{(n+j)}-t^{(n+k)}}\right]$$
Agora observamos que
$$\int _{t^{(n+s-1)}}^{t^{(n+s)}} p(t) dt=h\sum_{j=0}^{s}\beta_j f(t^{(n+j)},u^{(n+j)})$$
onde 
\begin{equation}
\beta_j= \frac{1}{h}\int_{t^{(n+s-1)}}^{t^{(n+s)}} \prod_{k=0,k\neq j}^{s} \frac{t-t^{(n+k)}}{t^{(n+j)}-t^{(n+k)}}dt
\end{equation}
Aplicando a mudança de variáveis $t=t^{(n+s-1)}+h\tau$, temos:
\begin{eqnarray*}\label{eq:betaj_m}
\beta_j&=& \int_0^1 \prod_{k=0,k\neq j}^{s} \frac{\tau+s-k-1}{j-k}dt\\
&=& \frac{(-1)^{s-j}}{j!(s-j)!}\int_{0}^{1} \prod_{k=0,k\neq j}^{s}(\tau+s-k-1)d\tau\\
&=& \frac{(-1)^{s-j}}{j!(s-j)!}\int_{0}^{1} \prod_{k=0,k\neq s-j-1}^{s}(\tau+k-1)d\tau
\end{eqnarray*}

Assim, obtemos a relação de recorrência:
\begin{eqnarray}\label{eq:mam}
  u^{(n+s)}  &=& u^{(n+s-1)}  + h\sum_{j=0}^{s}\beta_j f(t^{(n+j)},u^{(n+j)})
\end{eqnarray}


\begin{obs} Os coeficientes do método de Adams-Moulton de $s$ passos podem, alternativamente, ser obtidos exigindo que o sistema seja exato para $f(t,u)=t^0$, $f(t,u)=t^1$, $f(t,u)=t^2$, \ldots, $f(t,u)=t^{s}$.

\end{obs}

\begin{ex} Obtenha o método de Adams-Moulton para $s=3$ como
\begin{eqnarray}\label{AM4}
  u^{(n+3)}  &= u^{(n+2)}  + \int _{t^{(n+3)}}^{t^{(n+4)}} f(t,u(t)) \; dt \\
  u^{(n+3)}  &= u^{(n+2)}  + h \sum_{m=0}^{3} b_m f^{(n+m)} \\
  u^{(n+3)}  &= u^{(n+2)}  + h \left[b_3f^{(n+3)} +b_2f^{(n+2)} +b_1f^{(n+1)} +b_0f^{(n)}\right]
\end{eqnarray}
Para isso devemos obter $[b_3,b_2,b_1,b_0]$ tal que o método seja exato para polinômios até ordem $3$. Podemos obter esses coeficientes de maneira análoga a obter os coeficientes de um método para integração.

Supondo que os nós $t_k$ estejam igualmente espaçados, e para facilidade dos cálculos, como o intervalo de integração é $[t^{(n+2)},t^{(n+3)}]$, translade $t^{(n+2)}$ para a origem tal que $[t^{(n)},t^{(n+1)},\ldots ,t^{(n+3)}]=[-2h,-h,0,h]$.

Considere a base $[1, t, t^2, t^3]$ e substitua $f(t)$ por cada um dos elementos da base,obtendo:
\begin{eqnarray*}
      \int _0^{h} 1  \;dt = h             &= h( b_0(1)  +b_1(1)   + b_2(1)   + b_3(1)    )\\
      \int _0^{h} t  \;dt = \frac{h^2}{2}  &= h( b_0(h)  +b_1(0)   + b_2(-h) + b_3(-2h)  )\\
      \int _0^{h} t^2 \;dt = \frac{h^3}{3}  &= h( b_0(h)^2 +b_1(0)^2  + b_2(-h)^2+ b_3(-2h)^2 )\\
      \int _0^{h} t^3 \;dt = \frac{h^4}{4} &= h( b_0(h)^3 +b_1(0)^3  + b_2(-h)^3+ b_3(-2h)^3 )
\end{eqnarray*}
que pode ser escrito na forma matricial
\begin{eqnarray}
\left(
  \begin{array}{cccc}
    1  & 0 & 1    & 1   \\
    1  & 0 & -1   & -2  \\
    1  & 0 & 1    & 4   \\
    1  & 0 & -1   & -8
  \end{array}
\right)
\left(\begin{array}{c}  b_0 \\ b_1\\ b_2\\b_3   \end{array}\right)
=
\left(\begin{array}{c}  1  \\ 1/2 \\ 1/3 \\ 1/4  \end{array}\right)
\end{eqnarray}
Resolvendo o sistema obtemos
$$[b_0,b_1,b_2,b_3]=\left[\frac{1}{24},-\frac{5}{24},\frac{19}{24},\frac{9}{24}\right]$$
fornecendo a regra
\begin{eqnarray}
  u_{n+3}  &= u_{n+2}  + \frac{h}{24} [9 f_{n+3} +19f_{n+2} -5f_{n+1} +f^{(n)}]
\end{eqnarray}
\end{ex}




A tabela abaixo mostra as coeficientes do método de Adams-Moulton para até oito passos. 
 \begin{center}
\begin{tabular}{|c|cccccccc|}
\hline
1 & 1 &&&&&&&\\
2 & $\frac{1}{2}$ & $\frac{1}{2}$&&&&&&\\
3 & $-\frac{1}{12}$&$\frac{2}{3}$&$\frac{5}{12}$&&&&&\\
4 & $\frac{1}{24}$ & $-{\frac {5}{24}}$ & ${\frac {19}{24}}$ & $\frac{3}{8}$ &&&&\\
5 & $-{\frac {19}{720}}$ & ${\frac {53}{360}}$ & $-{\frac {11}{30}}$ & ${\frac {323}{360}}$ & ${\frac {251}{720}}$&&&\\
6 & ${\frac {3}{160}}$ & $-{\frac {173}{1440}}$ & ${\frac {241}{720}}$ & $-{\frac {133}{240}}$ & ${\frac {1427}{1440}}$ & ${\frac {95}{288}}$&&\\
7 & $-{\frac {863}{60480}}$ & ${\frac {263}{2520}}$ & $-{\frac {6737}{20160}}$ & ${\frac {586}{945}}$ & $-{\frac {15487}{20160}}$ & ${\frac {2713}{2520}}$ &${\frac {19087}{60480}}$&\\
8 &$ {\frac {275}{24192}}$ & $-{\frac {11351}{120960}}$&${\frac {1537}{4480}}$&$-{\frac {88547}{120960}}$&${\frac {123133}{120960}}$&$-{\frac {4511}{4480}}$&${\frac {139849}{120960}}$&${\frac {5257}{17280}}$\\
\hline 
\end{tabular}
 \end{center}

\begin{ex} O esquema iterativo de Adams-Moulton com três passos, isto é, $s=2$ é dado na forma:
 \begin{equation*}
  u^{(n+2)}=u^{(n+1)}+\frac{h}{12}\left[5f\left(t^{(n+2)},u(t^{(n+2)})\right)+8f\left(t^{(n+1)},u(t^{(n+1)})\right)-f\left(t^{(n)},u(t^{(n)})\right)\right]
 \end{equation*}
 \end{ex}

 
\subsection*{Exercícios resolvidos}
\begin{exeresol} Resolva o problema de valor inicial dado por:
\begin{eqnarray*}
 u'(t)&=& -2u(t) + te^{-t}\\
 u(0)&=&-1
\end{eqnarray*}
via Adams-Moulton com $s=2$ (três passos) com $h=0,1$ e $h=0,01$ e compare com a solução exata dada por $u(t)=(t-1)e^{-t}$ nos instantes $t=1$ e $t=2$. Inicialize com Euler modificado.

 \end{exeresol}
\begin{resol}
 Primeiro observamos que $f(u,t)=-2u + te^{-t}$ e que o esquema de Adams-Moulton pode ser escrito como:
 \begin{equation*}
  u^{(n+2)}=u^{(n+1)}+\frac{h}{12}\left[5f\left(t^{(n+2)},u(t^{(n+2)})\right)+8f\left(t^{(n+1)},u(t^{(n+1)})\right)-f\left(t^{(n)},u(t^{(n)})\right)\right]
 \end{equation*}
 de forma que:
 \begin{eqnarray*}
  u^{(n+2)}&=&u^{(n+1)}+\frac{h}{12}\left[8f\left(t^{(n+1)},u(t^{(n+1)})\right)-f\left(t^{(n)},u(t^{(n)})\right)\right]+\frac{5h}{12}f\left(t^{(n+2)},u(t^{(n+2)})\right)\\
  &=&u^{(n)}+\frac{h}{12}\left[8f^{(n+1)}-f^{(n)}\right]+\frac{5h}{12}\left(t^{(n+2)}e^{t^{(n+2)}} -2 u(t^{(n+2)})\right)
   \end{eqnarray*}
 Assim:
 \begin{eqnarray*}
  \left(1+\frac{5h}{6}\right)u^{(n+2)}
  &=&u^{(n+1)}+\frac{h}{12}\left[8f^{(n+1)}-f^{(n)}\right]+\frac{5h}{12}t^{(n+2)}e^{-t^{(n+2)}} 
  \end{eqnarray*}
 
 Os valores obtidos são:
 
 \begin{tabular}{|c|c|c|}
 \hline
  &t=1&t=2\\
 \hline
  h=0,1  & -0,000223212480142 & 0,135292280956\\
  h=0,01 & -2,02891229566e-07& 0,135335243537\\
  Exato  & 0 & 0,135335283237\\
  \hline
 \end{tabular}
\ifispython
A seguinte rotina implementa a recursão:
\begin{verbatim}
## resolve u'(t)=l*u(t) + g

def g(t):
	return t*np.exp(-t)




u0=-1
h=1e-2
Tmax=2
itmax=np.int(Tmax/h)

u=np.empty(itmax+1)
fn=np.empty(itmax+1)

u[0]=u0
l=-2
#Iniciliza com Euler modificado
k1= l*u[0] + g(0)
k2= l*(u[0]+h*k1) + g(h)
u[1]= u[0]+ h *(k1+k2)/2

fn[0]= k1
fn[1]= l*u[1] + g(h)



for n in np.arange(0,itmax-1):
	gn2=g((n+2)*h)
	
	u[n+2]= (u[n+1] + h/12*(8*fn[n+1]-fn[n]) + 5*h/12*gn2 ) / (1+5*h/6)
	fn[n+2]=l*u[n+2]+gn2

for n in np.arange(0,itmax+1):
	print h*n,u[n], (h*n-1)*np.exp(-h*n)
\end{verbatim}

\fi
\ifisoctave
A seguinte rotina implementa a recursão:
\begin{verbatim}
function y = g(x)
  y=x*exp(-x);
endfunction

function u=moulton(u0,h,Tmax)
  itmax=Tmax/h;

  u=[0:h:Tmax];
  fn=[0:h:Tmax];

  u(1)=u0;
  l=-2;
  %Iniciliza com Euler modificado
  k1= l*u(1) + g(0);
  k2= l*(u(1)+h*k1) + g(h);
  u(2)= u(1)+ h *(k1+k2)/2;

  fn(1)= k1;
  fn(2)= l*u(2) + g(h);



  for n=1:itmax-1
    gn2=g((n+1)*h);
    
    u(n+2)= (u(n+1) + h/12*(8*fn(n+1)-fn(n)) + 5*h/12*gn2 ) / (1+5*h/6);
    fn(n+2)=l*u(n+2)+gn2;
  endfor
  
endfunction
\end{verbatim}

\fi
\ifisoctave
A seguinte rotina implementa a recursão:
\begin{verbatim}
function y = g(x)
  y=x*exp(-x);
endfunction

function u=moulton(u0,h,Tmax)
  itmax=Tmax/h;

  u=[0:h:Tmax];
  fn=[0:h:Tmax];

  u(1)=u0;
  l=-2;
  Iniciliza com Euler modificado
  k1= l*u(1) + g(0);
  k2= l*(u(1)+h*k1) + g(h);
  u(2)= u(1)+ h *(k1+k2)/2;

  fn(1)= k1;
  fn(2)= l*u(2) + g(h);



  for n=1:itmax-1
    gn2=g((n+1)*h);
    
    u(n+2)= (u(n+1) + h/12*(8*fn(n+1)-fn(n)) + 5*h/12*gn2 ) / (1+5*h/6);
    fn(n+2)=l*u(n+2)+gn2;
  end
  
endfunction

\end{verbatim}


\fi

\end{resol}




\begin{exeresol}\label{exeresol:Adams_Moulton:prob1} Repita o Problema~\ref{exeresol:Adams_Bashforth:prob1} pelo método de Adams-Moulton, isto, é  resolva numericamente o problema de valor inicial dado por:
\begin{eqnarray*}
 y'(t)&=& \sqrt{1+y(t)},\\
 y(0)&=& 0,
\end{eqnarray*}
aplicando o método de Adams-Moulton de dois passos. Calcule o valor de $y(1)$ com passo de tamanho $h=0,1$.
\end{exeresol}
\begin{resol} 
Primeiro observamos que o processo resursivo do método de Adams é dado por: 
\begin{equation}\label{eq:Adams_Moulton:prob1:rec}
  y^{(n+1)}=y^{(n)}+\frac{h}{2}\left[f\left(t^{(n+1)},u(t^{(n+1)})\right)+f\left(t^{(n)},u(t^{(n)})\right)\right],~n=1,2,\ldots
 \end{equation}
O valor inicial é dado por $y^{(1)}=0$. Primeiramente, precisamos isolar $y^{(x+1)}$ na Equação~\ref{eq:Adams_Moulton:prob1:rec}:
\begin{equation*}
  y^{(n+1)}=y^{(n)}+\frac{h}{8} \sqrt {{h}^{2}+16+16y^{(n)}+8h\sqrt {1+y^{(n)}}} +\frac{h}{2}\sqrt {1+y^{(n)}}+\frac{h^2}{8},~n=1,2,\ldots
 \end{equation*}

  \begin{eqnarray*}
 y^{(1)}&=&0\\
 y^{(2)}&=&0,1025\\
y^{(3)}&=&0,21\\
y^{(4)}&=&0,3225\\
y^{(5)}&=&0,44\\
y^{(6)}&=&0,5625\\
y^{(7)}&=&0,69\\
y^{(8)}&=&0,8225\\
y^{(9)}&=&0,96\\
y^{(10)}&=&1,1025\\
y^{(11)}&=&1,25
\end{eqnarray*}

\end{resol}


\begin{exeresol}\label{exeresol:exersol_moulton_nl} Resolva o problema de valor inicial dado por
\begin{eqnarray*}
 y'(t)&=& y^3-y+t,\\
 y(0)&=& 0,
\end{eqnarray*}
aplicando o método de Adams-Moulton de dois passos. Calcule o valor de $y(1)$ com passo de tamanho $h=0,1$ e $h=0,01$.
Primeiro observamos que o processo resursivo do método de Adams é dado por: 
\begin{eqnarray*}
  y^{(n+1)}&=&y^{(n)}+\frac{h}{2}\left[f\left(t^{(n+1)},u(t^{(n+1)})\right)+f\left(t^{(n)},u(t^{(n)})\right)\right],~n=1,2,\ldots,\\
  y^{(1)}&=&0
 \end{eqnarray*}
Observamos que o problema de isolar $y^{(n+1)}$ pode ser escrito como
\begin{eqnarray*}
  y^{(n+1)}-\frac{h}{2}f\left(t^{(n+1)},u(t^{(n+1)})\right) &=&y^{(n)}+\frac{h}{2}f\left(t^{(n)},u(t^{(n)})\right)
 \end{eqnarray*}
O termo da esquerda é uma expressão não linear em $y^{(n+1)}$ e o termo da direita é conhecido, isto é, pode ser calculado com base nos valores anteriormente calculados. Devemos, então, escolher um método numérico de solução de equações algébricas não lineares (veja capítulo~\ref{cap:cap_equacao1d}) como o método de Newton visto na Seção~\ref{sec:metodo_newton_1d} para resolver uma equação do tipo:
$$u-\frac{h}{2}f(t^{(n+1)},u)=a,$$
isto é:
$$u-\frac{h}{2}\left(u^3-u+t^{(n+1)}\right)=a,$$
com $a=y^{(n)}+\frac{h}{2}f\left(t^{(n)},u(t^{(n)})\right)$
Os valores obtidos são: $0,37496894$ e $0,37512382$ quando o método é inicializadom com Euler melhorado.
\ifispython
A seguinte rotina implementa a recursão:
\begin{verbatim}
def f(t,u):
	return u**3-u+t

def resolve(u,t,a,h,beta): #resolve equacao nao-linear por metodo de Newton
	ctrl=2
	cont=0
	while (ctrl>0):
		cont=cont+1
		residuo=u-beta*h*(u**3-u+t)-a
		derivada=1-beta*h*(3*u**2-1)
		u1=u-residuo/derivada
		if np.abs(u1-u)<derivada*1e-10:
			ctrl=ctrl-1
		u=u1
#	print cont
	return u


def adams_moulton_2(h,Tmax,u1):
	itmax=np.int(Tmax/h)
	u=np.empty((itmax+1,1))
	u[0]=u1

	for i in np.arange(0,itmax):
		t=i*h
		fn=f(t,u[i])
		a=u[i] + h*fn/2
		u[i+1]=resolve(u[i]+ h*fn,t+h,a,h,1/2)
	return u



u0=0
h=1e-2
Tmax=1
itmax=np.int(Tmax/h)

u=adams_moulton_2(h,Tmax,u0)
print u[itmax]

\end{verbatim}

\fi
\end{exeresol}


\subsection*{Exercícios}

\begin{exer}
Encontre o método de Adams-Moulton para $s=0$.
\end{exer}
\begin{resp}
 \begin{equation*}
  y^{(n)}=y^{(n)}+hf\left(t^{(n)},u(t^{(n)}\right)
 \end{equation*}
Este esquema é equivalente ao método de Euler Implícito.
 \end{resp}



\begin{exer}
Encontre o método de Adams-Moulton para $s=1$.
\end{exer}
\begin{resp}
 \begin{equation*}
  y^{(n+1)}=y^{(n)}+\frac{h}{2}\left[f\left(t^{(n+1)},u(t^{(n+1)})\right)+f\left(t^{(n)},u(t^{(n)})\right)\right]
 \end{equation*}
 Este esquema é equivalente ao método trapezoidal.
 \end{resp}

 
 \begin{exer} Repita o Problema~\ref{exeresol:exersol_moulton_nl} usando Adams-Moulton com 3 passos e inicilizando com Runge-Kutta quarta ordem clássico.
  \end{exer}
\begin{resp}
 $0,37517345$ e  $0,37512543$.
\end{resp}

  
%  
% 
% \section{Método BDF}
% Um método de ordem $s$ com $s$ estágios é chamado de \emph{método BDF-Backward Differentiation Formula} se $\sigma (w)=b_sw^s$, onde $b_s \in \mathbb{R}$, ou seja,
% \begin{eqnarray}\label{BDF}
%   a_s u_{n+s}+ ...+ a_1 u_{n+1} + a_0u_{n} &=  h b_sf_{n+s}
% \end{eqnarray}
% 
% \begin{ex}
% Mostre que o método BDF com $s=3$ é
% \begin{eqnarray}
%   u_{n+3} -\frac{18}{11} u_{n+2}+\frac{9}{11}u_{n+1}-\frac{2}{11}u^{(n)} &= \frac{6}{11}h f_{n+3}
% \end{eqnarray}
% \end{ex}
% 
% \subsection*{Exercícios}
% 
% \begin{exer}
% Mostre que o método BDF com $s=1$ é o método de Euler implícito.
% \end{exer}
% 
% \begin{exer}
% Mostre que o método BDF com $s=2$ é
% \begin{eqnarray}
%   u_{n+2} -\frac{4}{3} u_{n+1} + \frac{1}{3}u^{(n)} &= \frac{2}{3}h f_{n+2}
% \end{eqnarray}
% \end{exer}
% 
% 
% \section{O método de Taylor}
% Uma maneira simples de aumentar a ordem dos métodos de Euler anteriormente descritos consiste em truncar a série de Taylor de $u(t+h)$:
% \begin{eqnarray}
%  u(t+h)=u(t) +h u'(t)+ \frac{h^2}{2!}u''(t)+\frac{h^3}{3!}u'''(t)+\ldots
% \end{eqnarray}
% 
% Utilizando dois termos temos o método de Euler. Utilizando os três primeiros termos da série e substituindo $u'(t)=f(t,x)$ e $u''(t)=\frac{\partial f}{\partial t}(t,x)$ temos o \emph{método de Taylor de ordem $2$}
% \begin{eqnarray}
%    u_{n+1}=u^{(n)} +h f(t^{(n)},u^{(n)})+ \frac{h^2}{2!} \frac{\partial f}{\partial t}(t^{(n)},u^{(n)})
% \end{eqnarray}
% 
% 
% O método de Taylor de ordem $3$ é
% \begin{eqnarray*}
%    u_{n+1}=u^{(n)} +h f(t^{(n)},u^{(n)})+ \frac{h^2}{2!}\frac{\partial f}{\partial t}(t^{(n)},u^{(n)})+\frac{h^3}{3!}\frac{\partial^2 f}{\partial t^2}(t^{(n)},u^{(n)})
% \end{eqnarray*}
% 
% \subsection*{Exercícios resolvidos}
% 
% \emconstrucao
% 
% \subsection*{Exercícios}
% 
% \emconstrucao
% 
% \section{Estabilidade dos métodos de Taylor}
% \begin{ex}
% Prove que para um método de Taylor de ordem $p$ para a EDO \eqref{EDO4.7} temos
% \begin{eqnarray}
%   p(z)= 1 + z+ \frac{z^2}{2!} +\frac{z^3}{3!}+\ldots +\frac{z^p}{p!}
% \end{eqnarray}
% onde  $u^{(n)} = (p(z))^nu_0$ e a região de estabilidade é dada por
% \begin{eqnarray}
%  \mathcal D_{T} = \{z \in  \field{C}:  \left|p(z)\right|<1\}
% \end{eqnarray}
% 
% Trace as regiões de estabilidade para o método de Taylor para $p=1,\ldots ,6$ no mesmo gráfico.
% \end{ex}
% 
% % \begin{figure}
% % \begin{center}
% %   \includegraphics[width=8cm]{RegiaoTaylor.eps}\\
% %   \caption{Região de estabilidade paras os métodos de Taylor de ordem $1,\ldots ,4$ (interior as curvas). A curva mais interna é para $p=1$}\label{RegiaoTaylor}
% % \end{center}
% % \end{figure}
% 
% 
% \begin{ex}
% Aproxime a solução do problema de valor inicial
% \begin{eqnarray}
%    \frac{du}{dt} &=\sin{t}\\
%             u(0) &= 1
% \end{eqnarray}
% para  $t\in [0,10]$.
% 
% \begin{enumerate}
% \item [a.] Trace a solução para $h=0,16$, $0,08$, $0,04$, $0,02$ e $0,01$ para o método de Taylor de ordem $1$, $2$ e $3$. (Trace todos de ordem $1$ no mesmo gráfico, ordem $2$ em outro gráfico e ordem $3$ outro gráfico separado.)
% 
% \item [b.] Utilizando a solução exata, trace um gráfico do erro em escala logar\'itmica.
% Comente os resultados (novamente, em cada gráfico separado para cada método repita os valores acima)
% 
% \item [c.] Fixe agora o valor $h=0,02$ e trace no mesmo gráfico uma curva para cada método.
% 
% \item [d.] Trace em um gráfico o erro em $t=10$ para cada um dos métodos (uma curva para cada ordem) a medida que $h$ diminui. (Use escala \verb#loglog#)
% \end{enumerate}
% \end{ex}

\section{Método de Adams-Moulton para sistemas lineares}
Esquemas implícitos como o de Adams-Moulton apresentam a dificuldade adicional de necessitar do valor de $f(t^{(n+1)},u^{(n+1)})$ para calcular o valor de $u^{(n+1)}$. Pelo menos para sistemas lineares, o método pode ser explicitado. Seja o seguinte problema de valor inicial linear:
\begin{eqnarray*}
 u'(t)&=& Au(t) + g(t),\\
 u(t^{(1)}) &=& a.
\end{eqnarray*}
Onde $u(t)$ é um vetor de n entradas e $A$ é uma matriz $n\times n$.

Considere agora o esquema de Adams-Moulton dado na Equação~\eqref{eq:mam} com $f(t,u)=Au+g(t)$:
\begin{eqnarray*}
  u^{(n+s)}  &=& u^{(n+s-1)}  + h\sum_{j=0}^{s}\beta_j \left[Au^{(n+j)} + g(t^{(n+j)})\right]
\end{eqnarray*}
o que pode ser escrito como:
\begin{eqnarray}
  \left(I_d-h\beta_s A\right)u^{(n+s)}  &=& u^{(n+s-1)}  + h\sum_{j=0}^{s-1}\beta_j \left[Au^{(n+j)}+ g(t^{(n+j)})\right]\nonumber\\
  &+& h\beta_s g(t^{(n+s)}) \label{eq:adams:lineares}
\end{eqnarray}
onde $I_d$ é matriz identidade $n\times n$. O sistema linear envolvido em \eqref{eq:adams:lineares} pode ser resolvido sempre que $I_d-h\beta_s A$ for inversível, o que sempre acontece quando $h$ é suficientemente pequeno.



\section{Estratégia preditor-corretor}\index{preditor-corretor}
Esquemas implícitos como o de Adams-Moulton (Seção~\ref{sec:Adams_Moulton}) e o de Runge-Kutta (Seção~\ref{sec:sec_IRK}), embora úteis para resolver problemas rígidos (ver Seção~\ref{sec:stiff}), apresentam a dificuldade de necessitar do valor de $f(t^{(n+1)},u^{(n+1)})$ para calcular o valor de $u^{(n+1)}$, exigindo a solução de uma equação algébrica a cada passo. Uma forma de aproximar o comportamento de um método ímplicito através de um esquema implícito consiste em aplicar a, assim chamada, \emph{estratégia preditor-corretor}.

Os métodos do tipo preditor-corretor empregam um esquema explícito para \emph{predizer} o valor de $u^{(n+1)}$ e, depois, um método implícito para recalcular, isto é, \emph{corrigir} $u^{(n+1)}$.

\begin{ex} Considere o método de Euler implícito (ver \ref{sec:euler_imp}) aplicado para resolver  o problema de valor inicial
\begin{eqnarray*}
  u'(t)  &=& f(t,u(t)) \\
  u(t^{(1)}) &=& a
\end{eqnarray*}
cujo processo iterativo é dado por
\begin{eqnarray*}
u^{(n+1)}=u^{(n)} + h f(t^{(n+1)},u^{(n+1)}).
\end{eqnarray*}
Agora aplicamos o método de Euler (ver \ref{sec:euler}) para predizer $u^{(n+1)}$:
\begin{eqnarray*}
u^{(n+1)}=u^{(n)} + h f(t^{(n)},u^{(n)}).
\end{eqnarray*}
E agora, retornamos ao método de Euler implícito:
\begin{eqnarray*}
u^{(n+1)}=u^{(n)} + h f(t^{(n+1)},\tilde{u}^{(n+1)}).
\end{eqnarray*}
Desta forma, a estratégia preditor-corretor aplicada ao método de Euler implícito com predição via método de Euler produz o método de Euler melhorado, ver \ref{sec:sec_euler_mod}, isto é:
\begin{eqnarray*}
\tilde{u}^{(n+1)}&=&u^{(n)} + h f(t^{(n)},u^{(n)}),\\
u^{(n+1)}&=&u^{(n)} + h f(t^{(n+1)},\tilde{u}^{(n+1)}).
\end{eqnarray*}
\end{ex}


\begin{ex} Considere o método de trapezoidal (ver \ref{sec:trapezoidal}) aplicado para resolver o problema de valor inicial
\begin{eqnarray*}
  u'(t)  &=& f(t,u(t)) \\
  u(t^{(1)}) &=& a
\end{eqnarray*}
cujo processo iterativo é dado por
\begin{eqnarray*}
u^{(n+1)}=u^{(n)} + \frac{h}{2} \left[f(t^{(n)},u^{(n)})+f(t^{(n+1)},u^{(n+1)})\right].
\end{eqnarray*}
Agora aplicamos o método de Euler (ver \ref{sec:euler}) para predizer $u^{(n+1)}$:
\begin{eqnarray*}
\tilde{u}^{(n+1)}=u^{(n)} + h f(t^{(n)},u^{(n)}).
\end{eqnarray*}
E agora, retornamos ao método trapezoidal para obter:
\begin{eqnarray*}
\tilde{u}^{(n+1)}&=&u^{(n)} + h f(t^{(n)},u^{(n)}),\\
u^{(n+1)}&=&u^{(n)} + \frac{h}{2} \left[f(t^{(n)},u^{(n)})+f(t^{(n+1)},\tilde{u}^{(n+1)})\right].
\end{eqnarray*}
\end{ex}

\begin{ex} Considere o método de Adams-Moulton de segunda ordem (ver \ref{sec:Adams_Moulton}) aplicado para resolver  o problema de valor inicial
\begin{eqnarray*}
  u'(t)  &=& f(t,u(t)) \\
  u(t^{(1)}) &=& a
\end{eqnarray*}
cujo processo iterativo é dado por
\begin{eqnarray*}
u^{(n+1)}=u^{(n)} + \frac{h}{2} \left[f(t^{(n)},u^{(n)})+f(t^{(n+1)},u^{(n+1)})\right].
\end{eqnarray*}
Agora aplicamos o método de Adams-Bashforth de segunda ordem (ver \ref{sec:Adams_Bashforth}) para predizer $u^{(n+1)}$:
\begin{eqnarray*}
\tilde{u}^{(n+1)}=u^{(n)} + \frac{h}{2} \left[-f(t^{(n-1)},u^{(n-1)})+3f(t^{(n)},u^{(n)})\right].
\end{eqnarray*}
Assim, obtemos o seguinte método:
\begin{eqnarray*}
\tilde{u}^{(n+1)}&=&u^{(n)} + \frac{h}{2} \left[-f(t^{(n-1)},u^{(n-1)})+3f(t^{(n)},u^{(n)})\right],\\
u^{(n+1)}&=&u^{(n)} + \frac{h}{2} \left[f(t^{(n)},u^{(n)})+f(t^{(n+1)},\tilde{u}^{(n+1)})\right].
\end{eqnarray*}
\end{ex}

\subsection*{Exercícios}
\begin{exer} Construa o esquema preditor corretor combinando Adams-Moulton de quarta ordem e Adams-Bashforth de quarta ordem. 
\end{exer}
\begin{resp}
\begin{eqnarray*}
\tilde{u}^{(n+1)}&=&u^{(n)} + \frac{h}{24} \left[-9f(t^{(n-3)},u^{(n-3)}) + 37f(t^{(n-2)},u^{(n-2)}) - 59 f(t^{(n-1)},u^{(n-1)})  + 55f(t^{(n)},u^{(n)})\right],\\
u^{(n+1)}&=&u^{(n)} + \frac{h}{24} \left[f(t^{(n-2)},u^{(n-2)})-5f(t^{(n-1)},u^{(n-1)})+19f(t^{(n)},u^{(n)})+9 f(t^{(n+1)},\tilde{u}^{(n+1)})\right].
\end{eqnarray*}
 \end{resp}

 \begin{exer} Seja o problema de valor inicial dado por:
\begin{eqnarray*}
  u'(t)  &=& \sqrt{u(t)+1} \\
  u(0) &=& 0
\end{eqnarray*} Resolva numericamente esse problema pelo método de Adams-Bashforth de segunda ordem e pelo método preditor corretor combinando Adams-Bashforth de segunda order com Adams-Moulton de segunda ordem. Compare a solução obtida para $t=10$ com a solução exata dada por:
$$u(t)=\frac{t^2}{4}+t.$$
 Inicialize os métodos empregando Runge-Kuta de segunda ordem. 
\end{exer}
\begin{resp}
 Adams-Bashforth: 34,99965176,~~ Preditor-corretor: 34,99965949,~~  Exato: 35
\end{resp}

\section{Problemas rígidos}\index{problema!rígido}\label{sec:stiff}

\construirSec

%Considere o problema
%\begin{eqnarray*}
% \left[\begin{array}{c}x_1'\\y_1'\end{array}\right]=\left[\begin{array}{cc}0 & 1\\100 & -101\end{array}\right]=\left[\begin{array}{c}x_1\\y_1\end{array}\right]
%\end{eqnarray*}
%com $x_1(0)=101$ e $x_2(0)=-10001$.

\section{Validação e {\it Benchmarking}}\index{validação}\index{Benchmark} %Talvez um dia mover para um capítulo aparte falando de métodos numéricos em geral.
Toda metodologia numérica deve ser validada ao ser aplicada para resolver um problema. A validação aumenta a confiabilidade na qualidade dos resultados obtidos. A validação procura detectar erros de implementação, características numéricas espúrias não prevista em projeto, como propagação catastrófica erros de arrendodamento, inadequação do método para o problema proposto etc.  A principal técnica de validação consiste em comparar a solução produzida com soluções de alta qualidade e confiabilidade, os chamados {\it benchmarks}. Quando um {\it benchmark} não estiver disponível, ainda se dispõe de algumas técnicas para avaliar a qualidade do método. Existe uma hierarquia das técnicas de  validação conforme listados a seguir:
\begin{enumerate}
\item Expressão analítica: Testar o código com problemas que admitem soluções analíticas constitui a forma mais confiável para validar o esquema usado. Expressões analíticas são expressões matemáticas das seguintes formas:
\subitem Expressão aritmética: Expressões envolvendo apenas um número finito de operações aritméticas elementares (soma, subtração, multiplicação e divisão) e números inteiros. Ex: $u(t)=\frac{t^2+1}{3t-4}$ ou $u(t)=t^3+\frac{3}{4}.$
\subitem Expressão algébrica:  Expressões envolvendo apenas um número finito de operações aritméticas elementares e expoentes fracionários. Ex: $u(t)=t^2+ \sqrt{t} $ ou $u(t)=t^{3/2}+\sqrt{2}$. 
\subitem Expressão forma-fechada: Expressões envolvendo apenas um número finito de operações aritméticas elementares, expoentes reais, logaritmos, exponencias, funções trigonométricas e funções trigonométricas inversas. Ex: $u(t)=\ln(1+t^{\pi})$, $u(t)=e^{-t}\sin(t)$ ou $u(t)=\tan^{-1}(t+1)$.
\subitem Expressão envolvendo funções especiais: Além das operações e funções acima, são permitidas funções especiais, como a função gama, funções de Bessel, séries de taylor, series de Fourier e outras séries envolvendo funções elementares e especiais. 
\item Expressão matemática semi-analítica: Expressão matemática envolvendo, além das operações e funções acima, outros processos de limite, como derivação e integração. Ex: $u(t)=\int_{0}^1 \log|t-x|x^t dx$.  
\item Solução numérica com reformulação analítica prévia: Neste caso, não se dispõe de uma expressão matemática para a solução, mas pode-se comparar resultado produzido pelo método numérico com outro problema numérico cuja solução é a mesma e pode ser obtida por outra metodologia numérica mais confiável. 
\item Benchmark puramente numérico: Um {\it benchmark} puramente numérico é uma aproximação numérica para a solução de um problema muito bem estabelecida e de alta confiabilidade. Os benchmarks numéricos normalmente são produzidos comparando diversos métodos numéricos diferentes e independentes e com grande refinamento.
\item Validação por comparação: Quando não se dispõe de {\it benchmarks}, ainda se pode comparar o resultado obtido com outros métodos numéricos. Em caso de divergência, pode ser bastante difícil dissernir qual método produz melhores resultados.
\item Convergência numérica: Este é o teste mais rudimentar que se aplica a métodos e numéricos e consiste em comparar o resultados produzidos com diferentes malhas de cálculo diferentes. Espera-se que o refino da malha produza soluções que convergem para a soluçao exata. Resultados muito próximos entre refinos sugere qualidade nos resultados.
\end{enumerate}


\begin{ex}[Expressão analítica] A solução do problema de valor inicial estudado no Exercício Resolvido~\ref{sec:exeresol1} dado por:
\begin{eqnarray*}
     u'(t)&=& -0,5u(t)+2+t,\\
            u(0) &=&  8,
\end{eqnarray*}
admite uma solução em forma de expressão analítica dada por:
\begin{equation*}
     u(t) = 2t+8e^{-t/2}. 
\end{equation*} 
\end{ex}

\begin{ex}[Expressão envolvendo funções especiais] A solução do problema de valor inicial dado por:
\begin{eqnarray*}
u'(t)&=&-u^3(t)+u^2(t)\\
u(0)&=&\frac{1}{2}
\end{eqnarray*}
é dada na forma:
\begin{eqnarray*}
u(t)=\frac{1}{1+W(e^{1-t})}
\end{eqnarray*}
onde $W$ é a função de Lambert é a função inversa de $f(y)=ye^y$, onde $y=W(x)$.
\end{ex}


\begin{ex}[Expressão matemática semi-analítica] A solução do problema de valor inicial dado por:
\begin{eqnarray*}
u^{(5)}&+&au^{(4)}+bu'''(t)+cu''(t)+du'(t)+u(t)=1\\
u^{(4)}(0)&=&u'''(0)=u''(0)=u'(0)=u(0)
\end{eqnarray*}
é dada na forma:
\begin{eqnarray*}
u(t)=1+Ae^{r_1 t} + Be^{r_2 t} + Ce^{r_3 t} + De^{r_4 t} + Ee^{r_5 t} 
\end{eqnarray*}
onde $r_1$, $r_2$, $r_3$, $r_4$ e $r_5$ são as raízes do polinômio característico $$p(x)=x^5+ax^4+bx^3+cx^2+dx+1,$$
cujas raízes, salvo casos particulares, só pode ser obtida por aproximações numéricas.
 \end{ex}


 \begin{ex}[Solução numérica com reformulação analítica prévia] A solução do problema de valor inicial dado por:
\begin{eqnarray*}
u'(t)&=&u^3(t)+u^2(t)+u(t)+1\\
u(0)&=&0
\end{eqnarray*}
é dada na forma:
\begin{eqnarray*}
\ln  \left(\frac{\left(u(t) +1\right)^2}{u(t)^2+1}\right) +2\arctan \left( u \left( t \right)  \right) =4t
\end{eqnarray*}
Esta analítica estabelece uma relação funcional implícita entre $t$ e $u(t)$, no entanto, é necessário resolver uma equação algébrica não-linear para cada $t$. Ainda assim, pode ser um excelente benchmark, pois o valor de $u(t)$ é dado explicitamente em função de $t$, isto é, podemos ver $t$ como uma função de $u$. Por exemplo, é fácil descobrir que $u=1$ quando $t=\frac{\ln(2)}{4}+\frac{\pi}{8}\approx 0.5659858768387104$. Além disso tomando o limite $u\to +\infty$, descobrimos que $u(t)$ tende a infinito quando $t\to \frac{\pi}{4}-$.
\end{ex}

 \begin{ex}[Solução numérica com reformulação analítica prévia] A solução do problema de valor inicial dado por:
\begin{eqnarray*}
u'(t)&=&\left[\cos\left(u(t)\right)+u(t)\right](1+\cos(t))\\
u(0)&=&0
\end{eqnarray*}
é dada na forma:
\begin{eqnarray*}
\int_0^{y(t)}\frac{d\tau}{\cos(\tau)+\tau}=t+\sin(t)
\end{eqnarray*}
Esta expressão reformula o problema como uma equação integral em $y(t)$. Esta nova reformulação pode ser bastante útil para produzir resultados de benchmark se fixamos o forma de $y(t)$, usamos uma técnica de quadratura númerica de boa qualidade para aproximar a integral do lado esquerdo da equação. Por exemplo, escolhendo $t(y)=100$, temos:
\begin{eqnarray*}
\int_0^{100}\frac{d\tau}{\cos(\tau)+\tau}=5,574304717298400
\end{eqnarray*}
Resolvendo a equação algébrica 
$$t+\sin(t)=5,574304717298400,$$
obtemos:
$$t=5,924938036503083.$$
A tabela a seguir mostra os valores de $y(t)$ para $t=5,924938036503083$ obtidos por quatro métodos de quarta-ordem: Runge-Kutta clássico, Adams-Bashforth, preditor-corretor com Adams-Bashforth quarta ordem e Adams-Moulton quarta ordem e Adams-Moulton. Os últimos três métodos foram inicializados com Runge-Kutta clássico de quarta ordem.

\begin{tabular}{|c|c|c|c|c|}
 \hline
 &$h=10^{-1}$&$h=10^{-2}$&$h=10^{-3}$&$h=10^{-4}$\\
 \hline
  Runge-Kutta 4 &       95,02737096& 99,04376734& 99,81705606 & 99,9925711 \\
  \hline
  Adams-Bashforth 4 &   94,68537569& 99,04349066& 99,81705572& 99,9925711\\
  \hline
  Pred. Corr. 4 &       94,68537569& 99,04349066& 99,81705572& 99,9925711\\
  \hline
  Adams-Moulton 4 &     94,71724913& 99,04324261& 99,81705570& 99,9925711\\
\hline
 \end{tabular}
Para $h=10^{-6}$, todos os quatro métodos produzem o resultado $99,99999287$.
\end{ex}

\section{Convergência, consistência e estabilidade}

\emconstrucao

\section{Exercícios finais}


\begin{exer} Considere o problema de valor inicial dado por
\begin{eqnarray*}
\frac{d u(t)}{dt} &=& -u(t) + e^{-t} \\
u(0)&=&0
\end{eqnarray*}
Resolva analiticamente este problema usando as técnicas elementares de equações diferenciais ordinárias. A seguir encontre aproximações numéricas usando os métodos de Euler, Euler modificado, Runge-Kutta clássico e Adams-Bashforth de ordem 4 conforme pedido nos itens.
\begin{itemize}
\item[a)]  Construa uma tabela apresentando valores com 7 algarismos significativos para comparar a solução analítica com as aproximações numéricas produzidas pelos métodos sugeridos. Construa também uma tabela para o erro absoluto obtido por cada método numérico em relação à solução analítica. Nesta última tabela, expresse o erro com 2 algarismos significativos em formato científico. Dica: $format('e',8)$ para a segunda tabela.
\begin{center}
\begin{tabular}{|c|c|c|c|c|c|}
\hline
&0,5&1,0&1,5&2,0&2,5\\
\hline
Analítico&&&&&\\
\hline
Euler&&&&&\\
\hline
Euler modificado&&&&&\\
\hline
Runge-Kutta clássico&&&&&\\
\hline
Adams-Bashforth ordem 4&&&&&\\
\hline
\end{tabular} 
\end{center}

\begin{center}
\begin{tabular}{|c|c|c|c|c|c|}
\hline
&0,5&1,0&1,5&2,0&2,5\\
\hline
Euler&&&&&\\
\hline
Euler modificado&&&&&\\
\hline
Runge-Kutta clássico&&&&&\\
\hline
Adams-Bashforth ordem 4&&&&&\\
\hline
\end{tabular}
\end{center}

\item[b)] Calcule o valor produzido por cada um desses método para $u(1)$ com passo $h=0,1$, $h=0,05$, $h=0,01$, $h=0,005$ e $h=0,001$. Complete a tabela com os valores para o erro absoluto encontrado.
\begin{center}
\begin{tabular}{|c|c|c|c|c|c|}
\hline
&0,1&0,05&0,01&0,005&0,001\\
\hline
Euler&&&&&\\
\hline
Euler modificado&&&&&   \\
\hline
Runge-Kutta clássico&&&&&\\
\hline
Adams-Bashforth ordem 4&&&&&\\
\hline
\end{tabular}
\end{center}

\end{itemize}

\end{exer}


\begin{resp}

\begin{center}
\begin{tabular}{|c|c|c|c|c|c|}
\hline
&0,5&1,0&1,5&2,0&2,5\\
\hline
Analítico&  0,3032653 &   0,3678794  &  0,3346952  &  0,2706706 &   0,2052125  \\
\hline
Euler& 0,3315955 &   0,3969266 &   0,3563684 &   0,2844209  &  0,2128243\\
\hline
Euler modificado &0,3025634 &   0,3671929 &   0,3342207 &   0,2704083  &  0,2051058 \\
\hline
Runge-Kutta clássico& 0,3032649  &  0,3678790  &  0,3346949  &  0,2706703  &  0,2052124\\
\hline
Adams-Bashforth ordem 4& 0,3032421  &  0,3678319 &   0,3346486  &  0,2706329  &  0,2051848  \\
\hline
\end{tabular}
\end{center}


\begin{center}
\begin{tabular}{|c|c|c|c|c|c|}
\hline
&0,5&1,0&1,5&2,0&2,5\\
\hline
Euler& 2,8e-2  &  2,9e-2  &  2,2e-2  &  1,4e-2 &   7,6e-3\\
\hline
Euler modificado& 7,0e-4  &  6,9e-4   & 4,7e-4 &   2,6e-4 &   1,1e-4\\
\hline
Runge-Kutta clássico& 4,6e-7 &   4,7e-7    &3,5e-7  &  2,2e-7 &   1,2e-7\\
\hline
Adams-Bashforth ordem 4&  2,3e-5 &   4,8e-5  &  4,7e-5  &  3,8e-5  &  2,8e-5 \\
\hline
\end{tabular}
\end{center}

\begin{center}
\begin{tabular}{|c|c|c|c|c|c|}
\hline
&0,1&0,05&0,01&0,005&0,001\\
\hline
Euler&2,9e-2  &  5,6e-3 &   2,8e-3 &   5,5e-4 &   2,8e-4\\
\hline
Euler modificado&6,9e-4 &   2,5e-5  &  6,2e-6 &   2,5e-7 &   6,1e-8   \\
\hline
Runge-Kutta clássico& 4,7e-7 &   6,9e-10 &   4,3e-11   & 6,8e-14  &  4,4e-15\\
\hline
Adams-Bashforth ordem 4&4,8e-5 &   9,0e-8 &   5,7e-9 &   9,2e-12 &   5,8e-13  \\
\hline
\end{tabular}
\end{center}

\end{resp}

\begin{exer} Considere o seguinte modelo para o crescimento de uma colônia de bactérias, baseado na equação logística (ver \eqref{eq:logistica})
$$u'(t)=\alpha u(t) \left(A-u(t)\right)$$
onde $u(t)$ indica a densidade de bactérias em unidades arbitrárias na colônia e $\alpha$ e $A$ são constantes positivas.
Pergunta-se: % Solução $ u(t)=\frac{Au_0}{(A-u_0)e^{-A\alpha at}+u_0}$
\begin{itemize}
\item[a)] Se $A=10$ e $\alpha=1$ e $u(0)=1$, use métodos numéricos para obter aproximação para $u(t)$ em $t=5\cdot 10^{-2}$, $t=\cdot 10^{-1}$, $t=5\cdot 10^{-1}$ e $t=1$.
\item[b)] Se $A=10$ e $\alpha=1$ e $u(0)=1$, use métodos numéricos para obter tempo necessário para que a população dobre?
\item[c)] Se $A=10$ e $\alpha=1$ e $u(0)=4$, use métodos numéricos para obter tempo necessário para que a população dobre?
\end{itemize}
\end{exer}
\begin{resp}
\begin{itemize}
 \item [a)] $1,548280989603$,   $2,319693166841$,    $9,42825618574$ e  $9,995915675174$.
 \item [b)] $0,081093021622$.
 \item [c)] $0,179175946923$.
 \end{itemize}

Obs: A solução analitica do problema de valor inicial é dada por:
 $$u(t)=\frac{Au_0}{(A-u_0)e^{-A\alpha at}+u_0}$$
 Os valores exatos para os itens b e c são:$\frac{1}{10}\ln\left(\frac{9}{4}\right)$ e $\frac{1}{10}\ln\left(6\right)$. 

\end{resp}

\begin{exer} Considere o seguinte modelo para a evolução da velocidade de um objeto em queda:
$$v'=g-\alpha v^2$$
Sabendo que $g=9,8$ e $\alpha=10^{-2}$ e $v(0)=0$. Pede-se a velocidade ao tocar o solo e o instante quando isto acontece, dado que a altura inicial era 100.
\end{exer}
\begin{resp}
O valor exato é $\sqrt{\frac{g}{\alpha}\left[1-e^{{-200\alpha}}\right]}\approx 29,109644835142$ em $t=\frac{1}{\sqrt{g\alpha}}\tanh^{-1}\left(\sqrt{1-e^{{-200\alpha}}}\right)\approx 2,3928380185497$
\end{resp}



\begin{exer} Considere o seguinte modelo para o oscilador não linear de Van der Pol:
$$u''(t) - \alpha (A-u(t)^2)u'(t) + w_0^2u(t)=0$$
onde $A$, $\alpha$ e $w_0$ são constantes positivas.
\begin{itemize}
\item[a)] Encontre a frequência e a amplitude de oscilações quando $w_0=1$, $\alpha=.1$ e $A=10$. (Teste diversas condições iniciais)
\item[b)] Estude a dependência da frequência e da amplitude com os parâmetros  $A$, $\alpha$ e $w_0$. (Teste diversas condições iniciais)
\item[c)] Que diferenças existem entre esse oscilador não linear e o oscilador linear?
\end{itemize}
\end{exer}

\begin{exer} Considere o seguinte modelo para um oscilador não linear:
\begin{eqnarray*}
u''(t)-\alpha(A-z(t))u'(t)+w_0^2 u(t)&=&0\\
Cz'(t)+z(t)&=&u(t)^2
\end{eqnarray*}
onde $A$, $\alpha$, $w_0$ e $C$ são constantes positivas.
\begin{itemize}
\item[a)] Encontre a frequência e a amplitude de oscilações quando $w_0=1$, $\alpha=.1$, $A=10$ e $C=10$. (Teste diversas condições iniciais)
\item[b)] Estude a dependência da frequência e da amplitude com os parâmetros  $A$, $\alpha$, $w_0$ e $C$. (Teste diversas condições iniciais)
\end{itemize}
\end{exer}

\begin{exer} Considere o seguinte modelo para o controle de temperatura em um processo químico:
\begin{eqnarray*}
CT'(t)+T(t)&=&\kappa P(t)+T_{ext}\\
P'(t)&=&\alpha(T_{set}-T(t))
\end{eqnarray*}
onde $C$, $\alpha$ e $\kappa$ são constantes positivas e $P(t)$ indica o potência do aquecedor. Sabendo que $T_{set}$ é a temperatura desejada, interprete o funcionamento esse sistema de controle. Faça o que se pede:
\begin{itemize}
\item[a)] Calcule a solução quando a temperatura externa $T_{ext}=0$, $T_{set}=1000$, $C=10$, $\kappa=.1$ e $\alpha=.1$. Considere condições iniciais nulas.
\item[b)] Quanto tempo demora o sistema para atingir a temperatura 900K?
\item[c)] Refaça os dois primeiros itens com $\alpha=0,2$ e $\alpha=1$
\item[b)] Faça testes para verificar a influência de $T_{ext}$, $\alpha$ e $\kappa$ na temperatura final.
\end{itemize}
\end{exer}

\begin{exer} Considere a equação do pêndulo dada por:
$$\frac{d^2\theta(t)}{dt^2}+\frac{g}{l}\sin(\theta(t))=0$$
onde $g$ é o módulo da aceleração da gravidade e $l$ é o comprimento da haste.
\begin{itemize}
\item[a)] Mostre analiticamente que a energia total do sistema dada por
$$\frac{1}{2}\left(\frac{d\theta(t)}{dt}\right)^2-\frac{g}{l}\cos(\theta(t))$$
é mantida constante.
\item[b)] Resolva numericamente esta equação para $g=9,8m/s^2$ e $l=1m$ e as seguintes condições iniciais:
\subitem i. $\theta(0)=0,5$ e $\theta'(0)=0$.
\subitem ii. $\theta(0)=1,0$ e $\theta'(0)=0$.
\subitem iii. $\theta(0)=1,5$ e $\theta'(0)=0$.
\subitem iv. $\theta(0)=2,0$ e $\theta'(0)=0$.
\subitem v. $\theta(0)=2,5$ e $\theta'(0)=0$.
\subitem vi. $\theta(0)=3,0$ e $\theta'(0)=0$.
\end{itemize}
Em todos os casos, verifique se o método numérico reproduz a lei de conservação de energia e calcule período e amplitude.
\end{exer}

\begin{exer} Considere o modelo simplificado de FitzHugh-Nagumo para o potencial elétrico sobre a membrana de um neurônio:
\begin{eqnarray*}
\frac{d V}{dt}& = &  V-V^3/3 - W +  I  \\
\frac{d W}{dt} & = & 0,08(V+0,7 - 0,8W)
\end{eqnarray*}
onde $I$ é a corrente de excitação.
\begin{itemize}
\item Encontre o único estado estacionário $\left(V_0,W_0\right)$ com $I=0$.
\item Resolva numericamente o sistema com condições iniciais dadas por $\left(V_0,W_0\right)$ e
\subitem $I=0$
\subitem $I=0,2$
\subitem $I=0,4$
\subitem $I=0,8$
\subitem $I=e^{-t/200}$
\end{itemize}
\end{exer}



%\end{document}



