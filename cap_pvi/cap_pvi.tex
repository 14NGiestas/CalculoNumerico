%Este está licenciado sob a Licença Creative Commons Atribuição-CompartilhaIgual 3.0 Não Adaptada. Para ver uma cópia desta licença, visite http://creativecommons.org/licenses/bu-sa/3.0/ ou envie uma carta para Creative Commons, PO Box 1866, Mountain View, CA 94042, USA.

%\documentclass[main.tex]{subfiles}
%\begin{document}

\chapter{Problemas de valor inicial}\index{problema de valor inicial}
Neste capítulo, vamos estudar metodologias numéricas para aproximar a solução de problema de valor inicial (problema de valor inicial) para equações diferenciais ordinárias. Primeiramente, daremos atenção aos problemas de primeira ordem e, depois, mostraremos que estas técnicas podem ser estendidas para problemas e sistemas de ordem superior. Considere um problema de valor inicial de primeira ordem dado por:
\begin{subequations}\label{PVI}
\begin{eqnarray}
  u'(t) &=& f(t, u(t)),~~t>t^{(1)}\label{PVI_EDO}\\
  u(t^{(1)}) &=& a ~~ \text{(condição inicial)}.\label{PVI_CI}
\end{eqnarray}
\end{subequations}

A incógnita de um problema de valor inicial é uma função que satisfaz a equação diferencial \eqref{PVI_EDO} e a condição inicial \eqref{PVI_CI}.

Considere os próximos três exemplos:
\begin{ex}
\begin{eqnarray}
   u'(t) &=t\\
   u(0) &= 2
\end{eqnarray}
\end{ex}

\begin{ex}
\begin{eqnarray}
   u'(t) &=u(t)\\
            u(0) &= 1
\end{eqnarray}
\end{ex}

\begin{ex}
\begin{eqnarray}
   u'(t) &=&\sin(u(t)^2+\sin(t))\\
            u(0) &=& a
\end{eqnarray}
\end{ex}

A solução do primeiro exemplo é $u(t)=t^2/2+2$ pois satisfaz a equação diferencial e a condição inicial. A solução do segundo também é facilmente obtida: $u(t)=e^t$. Porém como podemos resolver o terceiro problema?

Para muitos problemas de valor inicial da forma \eqref{PVI}, não é possível encontrar uma expressão analítica fechada, ou seja, sabe-se que a solução existe e é única, porém não podemos expressá-la em termos de funções elementares. Por isso é necessário calcular aproximações numéricas para a solução. 

Existem uma enorme família de metodologias para construir soluções numéricas para problemas de valor inicial. Aqui, vamos nos limitar a estudar métodos que aproximam $u(t)$ em um conjunto finito de valores de $t$. Este conjunto de valores será chamado de \emph{malha} e será denotado por  $\{t^{(i)}\}_{i=1}^N=\{t^{(1)}, t^{(2)}, t^{(3)},\ldots, t^{(N)}\}$. Desta forma, aproximamos a solução $u(t^{(i)})$ por $u^{(i)}$ em cada ponto da malha usando diferentes esquemas numéricos.

%%%%%%%%%%%%%%%%%%%%
% python
%%%%%%%%%%%%%%%%%%%%
\ifispython
Nos códgos em \verb+Python+ apresentados neste capítulo, assumiremos que as seguintes bibliotecas e módulos estão importados:
\begin{verbatim}
from __future__ import division
import numpy as np
from numpy import linalg
import matplotlib.pyplot as plt
\end{verbatim}
\fi
%%%%%%%%%%%%%%%%%%%%


\section{Método de Euler}\index{método!de Euler}
Nesta seção, contruiremos o mais simples dos métodos para resolver problemas de valor inicial: o método de Euler com passo constante.\index{Passo} Por passo constante, queremos dizer que os pontos da malha estão todos igualmente espaçados, isto é:
$$t^{(i)}=(i-1)h,~~~i=1,2,\ldots,N.$$
onde $h$ é passo, ou seja, a distância entre dois pontos da malha.

Considere, agora, o problema de valor inicial dado por:
\begin{eqnarray}\label{EDO1}
  u'(t)  &=& f(t,u(t)),~~t>t^{(1)} \\
  u(t^{(1)}) &=& a.
\end{eqnarray}
Ao invés de tentar solucionar o problema para qualquer $t>t^{(1)}$, iremos aproximar $u(t)$ em $t=t^{(2)}$.

Integrando \eqref{EDO1} de $t^{(1)}$ até $t^{(2)}$, obtemos:
\begin{eqnarray}
  \int_{t^{(1)}}^{t^{(2)}} u'(t) \;dt &=& \int_{t^{(1)}}^{t^{(2)}} f(t,u(t)) \; dt\\
  u(t^{(2)})-u(t^{(1)})               &=& \int_{t^{(1)}}^{t^{(2)}} f(t,u(t)) \; dt\\
  u(t^{(1)})                      &=& u(t^{(1)}) +  \int _{t^{(1)}}^{t^{(2)}} f(t,u(t)) \; dt
\end{eqnarray}

Seja $u_n$ a aproximação de $u(t_n)$. Para obter o método numérico mais simples aproximamos $f$ em $[t^{(1)},t^{(2)}]$ pela função constante $f(t,u(t)) \approx  f(t^{(1)},u^{(1)})$,
\begin{eqnarray}
  u^{(2)} &=&  u^{(1)} +   f(t^{(1)},u^{(1)}) \int _{t^{(1)}}^{t^{(2)}}  \; dt \\
  u^{(2)} &=&  u^{(1)} +   f(t^{(1)},u^{(1)}) (t^{(2)}-t^{(1)}) \\
  u^{(2)} &=&  u^{(1)} + h f(t^{(1)},u^{(1)})
\end{eqnarray}

Este procedimento pode ser repetido para  $t^{(3)}$, $t^{(4)}$, $\ldots$, obtendo, assim, o chamado \emph{método de Euler}:
\begin{eqnarray}\label{euler}
u^{(n+1)}&=&u^{(n)} + h\;f(t^{(n)},u^{(n)}),\\
u^{(1)}&=&u^{(1)}=u(t^{(1)})~ \hbox{(condição inicial)}.
\end{eqnarray}



% Note que o método numérico pode ser escrito como $R_h(u)_n=f(t_n,u_n)$ onde, para o método de Euler,
% $$R_h(u)_n=\frac{u_{n+1}-u_n}{h}$$


\begin{ex}
Considere o problema de valor inicial
\begin{eqnarray*}
  u'(t)&=&2u(t)\\
  u(0)&=&1
\end{eqnarray*}
cuja solução é $u(t)=e^{2t}$. O método de Euler aplicado a este problema produz o  esquema:
\begin{eqnarray}\label{exemplo_y_2y_euler}
  u^{(k+1)}&=&u^{(k)}+2hu^{(k)}=(1+2h)u^{(k)}\\
  u^{(1)}&=&1,
\end{eqnarray}
Suponha que queremos calcular o valor aproximado de $u(1)$ com $h=0,2$. Então os pontos $t^{(1)}=0$, $t^{(2)}=0,2$, $t^{(3)}=0,4$, $t^{(4)}=0,6$, $t^{(5)}=0,8$ e $t^{(6)}=1,0$ formam os seis pontos da malha. As aproximações para a solução nos pontos da malha usando o método de Euler são:
\begin{eqnarray*}
  u(0)  &\approx &u^{(1)}=1\\
  u(0,2)&\approx &u^{(2)}=(1+2h) u^{(1)}=1,4 u^{(1)}=1,4\\
  u(0,4)&\approx &u^{(3)}=1,4 u^{(2)}=1,96\\
  u(0,6)&\approx &u^{(4)}=1,4 u^{(3)}=2,744\\
  u(0,8)&\approx &u^{(5)}=1,4 u^{(4)}=3,8416\\
  u(1,0)&\approx &u^{(6)}=1,4 u^{(5)}=5,37824
\end{eqnarray*}
Essa aproximação é bem grosseira quando comparamos com a solução do problema em $t=1$: $u(1)=e^{2}\approx 7,38906$. Não obstante, se tivéssemos escolhido um passo menor, teríamos obtido uma aproximação melhor. Veja tabela abaixo com valores obtidos com diferentes valores de passo $h$.

\begin{tabular}{|l|l|l|l|l|l|l|l|}%\label{pvi:tab_euler}
\hline
   h&$10^{-1}$&$10^{-2}$&$10^{-3}$&$10^{-4}$&$10^{-5}$&$10^{-6}$&$10^{-7}$\\
   \hline
   $u^{(N)}$& 6,1917 &  6,7275 &  7,0400 &  7,2096  & 7,2980 &  7,3432  & 7,3660\\
   \hline
  \end{tabular}
 
De fato, podemos mostrar que quando $h$ se aproxima de $0$, a solução aproximada via método de Euler converge para a solução exata $e^2$. Para isto, basta observar que a solução da relação de recorrência \eqref{exemplo_y_2y_euler} é dada por
 $$u^{(k)}=(1+2h)^{k-1}.$$
 Como $t^{(k)}=(k-1)h$ e queremos a solução em $t=2$, a solução aproximada pelo método de Euler com passo $h$ em é dada por:
 $$u^{(k)}= (1+2h)^{k-1}= (1+2h)^{\frac{2}{h}}.$$
Aplicando o limite $h\to 0+$, temos:
  $$\lim_{h\to 0+} (1+2h)^{\frac{2}{h}}= e^{2}.$$

%%%%%%%%%%%%%%%%%%%%
% python
%%%%%%%%%%%%%%%%%%%%
\ifispython
Em \verb+Python+, podemos computar a solução numérica deste problema de valor inicial via o método de Euler com o seguite código:
\begin{verbatim}
#define f(t,u)
def f(t,u):
    return 2*u

#tamanho e num. de passos
h = 0.2
N = 6

#cria vetor t e u
t = np.empty(N)
u = np.copy(t)

#C.I.
t[0] = 0
u[0] = 1

#iteracoes
for i in np.arange(N-1):
    t[i+1] = t[i] + h
    u[i+1] = u[i] + h*f(t[i],u[i])

#imprime
for i,tt in enumerate(t):
    print("%1.1f %1.4f" % (t[i],u[i]))
\end{verbatim}
\fi
%%%%%%%%%%%%%%%%%%%%
\end{ex}

\begin{exeresol} Aproxime a solução do problema de valor inicial
\begin{eqnarray}
     u'(t)&=& -0,5u(t)+2+t\\
            u(0) &=&  8
\end{eqnarray}
Usando os seguinte passos: $h=10^{-1}$, $h=10^{-2}$, $h=10^{-3}$, $h=10^{-4}$ e $h=10^{-5}$ e compare a solução aproximada em $t=1$ com a solução exata dada por:
\begin{equation}
     u(t) = 2t+8e^{-t/2} \Longrightarrow u(1)=2+8e^{-1/2} \approx 6,85224527770107
\end{equation}
\end{exeresol}
\begin{resol} Primeramente itentificamos $f(t,u)=-0,5u+2+t$ e construímos o processo iterativo do método de Euler: 
\begin{eqnarray}
  u^{(n+1)}&=&u^{(n)} + h( -0,5u^{(n)}+2+t^{(n)}),~~n=1,2,3,\ldots\\
  u^{(1)}&=&8
\end{eqnarray}

\ifisscilab
O seguinte código pode ser usado para implementar no \verb+Scilab+ a recursão acima:
\begin{verbatim}
function u=euler(h,Tmax)
  u= 8;
  t= 0;
  itmax = Tmax/h;
  for n=1:itmax
    t= t + h;
    u= u + h*(-0.5*u+2+t);
   end
endfunction
\end{verbatim}
o qual pode ser invocado da seguinte forma:
\begin{verbatim}
-->euler(1e-1,1)
 ans  =
 
    6.8701481 
\end{verbatim}
Podemos construir um vetor com as cinco soluções da seguinte forma:
\begin{verbatim}
 -->S=[euler(1e-1,1) euler(1e-2,1) euler(1e-3,1) euler(1e-4,1) euler(1e-5,1)]
 S  =
 
    6.8701481    6.8540481    6.8524257    6.8522633    6.8522471  
\end{verbatim}

\fi

\ifisoctave
O seguinte código pode ser usado para implementar no \verb+Octave+ a recursão acima:
\begin{verbatim}
function u=euler(h,Tmax)
  u= 8;
  t= 0;
  itmax = Tmax/h;
  for n=1:itmax
    t= t + h;
    u= u + h*(-0.5*u+2+t);
   end
endfunction
\end{verbatim}
o qual pode ser invocado da seguinte forma:
\begin{verbatim}
>> euler(1e-1,1)
ans =  6.8701
\end{verbatim}
Podemos construir um vetor com as cinco soluções da seguinte forma:
\begin{verbatim}
 >> S=[euler(1e-1,1) euler(1e-2,1) euler(1e-3,1) euler(1e-4,1) euler(1e-5,1)]
S =

   6.8701   6.8540   6.8524   6.8523   6.8522
\end{verbatim}
\fi


\ifispython
O seguinte código pode ser usado para implementar no \verb+Python+ a recursão acima:

\begin{verbatim}
 def euler(h,Tmax):
	u=8
	t=0
  	itmax = Tmax/h;
	for i in np.arange(itmax):
		t = t + h
		u = u + h*(-0.5*u+2+t)
	return u

	
sol_exata = 2+8*np.exp(-.5)
h=1e-1
for i in np.arange(1,5):
	sol_euler=euler(h,1);
	erro_relativo = np.fabs((sol_euler-sol_exata)/sol_exata)
	print("h=%1.0e - u(1) =~ %1.7f - erro_relativo = %1.1e" % (h, sol_euler, erro_relativo) )
	h=h/10
\end{verbatim}
\fi
A seguinte tabela resume os resultados obtidos:
\begin{center}
 \begin{tabular}{|l|l|l|l|l|l|l|l|}%\label{pvi:tab_euler}
\hline
   h&$10^{-1}$&$10^{-2}$&$10^{-3}$&$10^{-4}$&$10^{-5}$\\
   \hline
   $u^{(N)}$ & 6,8701481  &  6,8540481  &  6,8524257  &  6,8522633  &  6,8522471  \\
   \hline
   $\varepsilon_{rel}$ &2,6e-03 &  2,6e-04  & 2,6e-05&   2,6e-06 &  2,6e-07\\
   \hline
  \end{tabular}
\end{center}


% Veja o gráfico da solução para $h=1, 0.5, 0.1, 0.05$:
% \begin{figure}
% \includegraphics[width=\textwidth]{euler.eps}
% \end{figure}

\end{resol}


Vamos agora, analisar o desempenho do método de Euler usando um exemplo mais complicado, porém ainda simples suficiente para que possamos obter a solução exata:
\begin{ex}\label{ex_euler_1}
Considere o problema de valor inicial relacionado à equação logística\index{equação!logística}:
\begin{eqnarray*}
u'(t)&=&u(t)(1-u(t))\\
u(0)&=&1/2
\end{eqnarray*}
\end{ex}
Podemos obter a solução exata desta equação usando o método de separação de variáveis\index{método!de separação de variáveis} e o método das frações parciais\index{método das frações parciais}. Para tal escrevemos:
\begin{equation*}
\frac{du(t)}{u(t)(1-u(t))}=dt
\end{equation*}
O termo $\frac{1}{u(t)(1-u(t))}$ pode ser decomposto em frações parciais como $\frac{1}{u}+\frac{1}{1-u}$ e chegamos na seguinte equação diferencial:
\begin{equation*}
\left(\frac{1}{u(t)}+\frac{1}{1-u(t)}\right)du=dt.
\end{equation*}
Integrando termo-a-termo, temos a seguinte equação algébrica relacionando $u(t)$ e $t$:
\begin{equation*}
\ln(u(t))-\ln\left(1-u(t)\right)=t+C
\end{equation*}
Onde $C$ é a constante de integração, que é definida pela condição inicial, isto é, $u=1/2$ em $t=0$. Substituindo, temos $C=0$. O que resulta em:
\begin{equation*}
\ln\left(\frac{u(t)}{1-u(t)}\right)=t
\end{equation*}
Equivalente a
\begin{equation*}
\frac{u(t)}{1-u(t)}=e^{t} \Longrightarrow u(t)=(1-u(t))e^{t} \Longrightarrow (1+e^t)u(t)=e^{t}
\end{equation*}
E, finalmente, encontramos a solução exata dada por $u(t)=\frac{e^t}{1+e^{t}}$.

Vejamos, agora, o esquema iterativo produzido pelo método de Euler:
\begin{eqnarray*}
u^{(k+1)}&=& u^{(k)}+h u^{(k)}(1-u^{(k)}), \\
u^{(1)}&=& 1/2.
\end{eqnarray*}


\ifisscilab
O seguinte código pode ser usado para implementar no \verb+Scilab+ a recursão acima:
\begin{verbatim}
function u=euler(h,Tmax)
  u= .5;
  itmax = Tmax/h;
  for n=1:itmax
    u= u + h*u*(1-u);
   end
endfunction
\end{verbatim}
o qual pode ser invocado da seguinte forma $\left(h=1e-1, t=2\right)$:
\begin{verbatim}
-->euler(1e-1,2)
 ans  =
 
    0.8854273  
\end{verbatim}

\fi

\ifisoctave
O seguinte código pode ser usado para implementar no \verb+Octave+ a recursão acima:
\begin{verbatim}
function u=euler(h,Tmax)
  u= .5;
  itmax = Tmax/h;
  for n=1:itmax
    u= u + h*u*(1-u);
   end
endfunction
\end{verbatim}
o qual pode ser invocado da seguinte forma $\left(h=1e-1, t=2\right)$:
\begin{verbatim}
>> euler(1e-1,1)
ans =  0.88543
\end{verbatim}
\fi


\ifispython
O seguinte código pode ser usado para implementar no \verb+Python+ a recursão acima:

\begin{verbatim}
def euler(h,Tmax):
	u=.5
  	itmax = Tmax/h;
	for i in np.arange(itmax):
		u = u + h*u*(1-u)
	return u

	
h=1e-1
for t in [.5, 1, 2, 3]:
	sol_euler=euler(h,t);
	sol_exata=1/(1+np.exp(-t))
	erro_relativo = np.fabs((sol_euler-sol_exata)/sol_exata)
	print("h=%1.0e - u(%1.1f) =~ %1.7f - erro_relativo = %1.1e" % (h, t, sol_euler, erro_relativo) )

h=1e-2
print;
for t in [.5, 1, 2, 3]:
	sol_euler=euler(h,t);
	sol_exata=1/(1+np.exp(-t))
	erro_relativo = np.fabs((sol_euler-sol_exata)/sol_exata)
	print("h=%1.0e - u(%1.1f) =~ %1.7f - erro_relativo = %1.1e" % (h, t, sol_euler, erro_relativo) )

\end{verbatim}
\fi
Para fins de comparação, calculamos a solução exata e aproximada para alguns valores de $t$ e de passo $h$ e resumimos na tabela abaixo:

%\begin{table}[h]
 % \caption{Tabela comparativa entre método de Euler e solução exata para Problema~\ref{ex_euler_1}.}
 % \label{tab:log}
  \begin{tabular}{|c|c|c|c|}\hline
    $t$ & $\text{Exato}$ & $\text{Euler}~~ h=0,1$ & $\text{Euler}~~ h=0,01$\\\hline
    $0$ & $1/2$ & $0,5$ & $0,5$\\\hline
    $1/2$ & $\frac{e^{1/2}}{1+e^{1/2}}\approx 0,6224593$ & $0,6231476$ & $0,6225316$\\\hline
    $1$ & $\frac{e}{1+e}\approx 0,7310586$ & $0,7334030$ & $0,7312946$\\\hline
    $2$ & $\frac{e^2}{1+e^2}\approx  0,8807971$ & $0,8854273$  & $0,8812533$ \\\hline
    $3$ & $\frac{e^3}{1+e^3}\approx   0,9525741$  & $0,9564754$ & $0,9529609$ \\\hline
  \end{tabular}
%\end{table}

\subsection*{Exercícios}



\begin{exer} Resolva o problema de valor inicial a seguir envolvendo uma equação não autônoma\index{equação diferencial!não autônoma}, isto é, quando a função $f(t,u)$ depende explicitamente do tempo. Use passo $h=0,1$ e $h=0,01$. Depois compare com a solução exata dada por $u(t)=2e^{-t}+t-1$ nos intantes $t=0$, $t=1$, $t=2$ e $t=3$.
  \begin{eqnarray*}
    u'(t)&=&-u(t)+t\\
    u(0)&=&1,
  \end{eqnarray*}
\end{exer}

\begin{resp}
O esquema recursivo de Euler fica:
\begin{eqnarray*}
  u^{(k+1)}&=&u^{(k)}+h(-u^{(k)}+t^{(k)})\\
  u^{(1)}&=&1
\end{eqnarray*}

Comparação:
\begin{center}
\begin{tabular}{|c|c|c|c|}\hline
$t$ &  Exato & Euler~~ $h=0,1$ & Euler~~ $h=0,01$\\\hline
$0$ &  $1$ & $1$ & $1$\\\hline
$1$ &   $2e^{-1}\approx 0,7357589$ & $0,6973569$   &   $0,7320647$  \\\hline
$2$ &   $2e^{-2}+1\approx  1,2706706$ & $ 1,2431533 $   &  $ 1,2679593$     \\\hline
$3$ &   $2e^{-3}+2\approx 2,0995741$  & $ 2,0847823$ & $2,0980818$   \\\hline
\end{tabular}
\end{center}
\end{resp}

\begin{exer} Resolva o prolema de valor inicial envolvendo uma equação não linear\index{Problema de valor inicial!não linear} usando passo $h=0,1$ e $h=0,01$. 
 \begin{eqnarray*}
    u'(t)&=&\cos(u(t))\\
    u(0)&=&0,
  \end{eqnarray*}

Depois compare com a solução exata dada por 
$$u(t)=\tan^{-1} \left( \frac {e^{2t}-1}{{2 e^t}}
 \right).
$$
nos intantes $t=0$, $t=1$, $t=2$ e $t=3$.
\end{exer}
\begin{resp}
 \begin{eqnarray*}
  u^{(k+1)}&=&u^{(k)}+h\cos(u^{(k)})\\
  u^{(1)}&=&0
\end{eqnarray*}
Comparação:
\begin{center}
\begin{tabular}{|c|c|c|c|}\hline
$t$ &  Exato & Euler~~ $h=0,1$ & Euler~~ $h=0,01$\\\hline
$0$ &  $0$ & $0$ & $0$\\\hline
$1$ &   $0,8657695 $ & $ 0.8799602$   &   $0.8671764 $  \\\hline
$2$ &   $1,3017603 $ & $ 1.3196842 $   &  $  1.3035243$     \\\hline
$3$ &   $1,4713043 $  & $ 1.4827638 $ & $1.4724512 $   \\\hline
\end{tabular}
\end{center}
\end{resp}

\begin{exer} Resolva a seguinte problema de valor inicial linear com passo $h=10^{-4}$ via método de Euler e compare a solução obtida com o valor exato $y(t)=e^{\sin(t)}$ em $t=2$:
 \begin{eqnarray*}
    y'(t)&=&\cos(t)y(t)\\
    y(0)&=&1.
  \end{eqnarray*}
\end{exer}  
\begin{resp}
 Aproximação via Euler: $2,4826529 $, exata: $e^{\sin(2)}\approx 2,4825777 $. Erro relativo aproximado: $ 3\times 10^{-5}$.
\end{resp}

 




\subsection{Ordem de precisão}

A \emph{precisão} de um método numérico que aproxima a solução de um problema de valor inicial é dada pela ordem do erro acumulado ao calcular a aproximação em um ponto $t^{(n+1)}$ em função do espaçamento da malha $h$.

Se $u(t^{(n+1)})$ for aproximado por $u_{n+1}$ com erro da ordem $O(h^{p+1})$ dizemos que o método tem \textbf{ordem de precisão $p$}\index{método!de Euler!ordem de precisão}.


Queremos obter a ordem de precisão do método de Euler. Para isso, substituímos a EDO $u'=f(t,u)$ na expansão em série de Taylor
\begin{eqnarray}\label{taylor}
   u(t^{(n+1)})=u(t_n)+hu'(t_n)+h^2u''(t_n)/2+ \mathcal O(h^3)
\end{eqnarray}
e obtemos
\begin{eqnarray}\label{tayloreuler}
 u(t^{(n+1)})=u(t_n)+hf(t_n,u(t_n))+h^2u''(t_n)/2+ \mathcal O(h^3)
\end{eqnarray}
Subtraindo \eqref{tayloreuler} do método de Euler
\begin{eqnarray}
    u_{n+1}=u_n + h\;f(t_n,u_n)
\end{eqnarray}
obtemos
\begin{eqnarray}
   e_{n+1}   &=& u_{n+1}-u(t^{(n+1)}) \\
             &=&u_n - u(t_n)  +h(f(t_n,u(t_n)+e_n)- f(t_n,u(t_n))) +\\
             &+&\frac{h^2}{2}u''_n+\mathcal O(h^3)
\end{eqnarray}
Defina o \emph{erro numérico} como $e_n=u_n-u(t_n)$ onde $u(t_n)$ é a solução exata e $u_n$ é a solução aproximada. Assim
\begin{eqnarray}
   e_{n+1}    =&e_n + h(f(t_n,u(t_n)+e_n)- f(t_n,u(t_n))) +\frac{h^2}{2}u''_n+\mathcal O(h^3)
\end{eqnarray}
Usando a condição de Lipschitz em $f$  temos
\begin{eqnarray}
   |e_{n+1}|      &\le &  |e_n| + h|f(t_n,u(t_n)+e_n)- f(t_n,u(t_n))|+\frac{h^2}{2}|u''_n|+\mathcal O(h^3)\\
                  &\le &  |e_n| + hL |u(t_n)+e_n- u(t_n)|+\frac{h^2}{2}|u''_n|+\mathcal O(h^3)\\
                  &\le &  |e_n| + hL |e_n|+\frac{h^2}{2}|u''_n|+\mathcal O(h^3)\\
                  &\le &  (1+ hL) |e_n|+\frac{h^2}{2}|u''_n|+\mathcal O(h^3)
\end{eqnarray}

\subsection{Erro de truncamento local}

O \emph{erro de truncamento local} é o erro cometido em \textbf{uma} iteração do método numérico supondo que a solução exata é conhecida no passo anterior.

Assim, supondo que a solução é exata em $t_n$ ($|e_n|=0$), obtemos que o ETL é
$$ETL_{Euler}^{n+1}= h^2/2|u''|+ \mathcal O(h^3) = \mathcal O(h^2)$$

Como o $ETL=\mathcal O(h^2)$ temos que o método de Euler possui ordem $1$.


\subsection{Erro de truncamento global}
O \emph{erro de truncamento global} é o erro cometido durante \textbf{várias} iterações do método numérico.

Supondo que a solução exata é conhecida em $t^{(1)}$ ($\|e_1\|=0$), então realizando $n=\frac{T}{h}$ iterações obtemos
\begin{eqnarray}
   ETG &=& nETL \\
       &=& n[h^2/2|u''|+ \mathcal O(h^3)] \\
       &=& Th/2|u''|+ \mathcal O(h^2)
\end{eqnarray}
ou seja
$$ETG_{Euler}^{n+1} = \mathcal \mathcal{O}(h)$$

\subsection*{Exercícios resolvidos}

\emconstrucao

\subsection*{Exercícios}
\begin{exer}Resolva o problema de valor inicial dado por
\begin{eqnarray*}
u'&=& -2u + \sqrt{u}\\
u(0)&=&1
\end{eqnarray*}
com passo $h=0,1$ e $h=0,01$ para obter aproximações para $u(1)$. Compare com a solução exata dada por $u(t) =  \left({1+2 e^{-t}+e^{-2 t}}\right)/{4}$
\end{exer}
\begin{resp}

 $0,4496$ com $h=0,1$ e $0,4660$ com $h=0,01$. A solução exata vale $u(1)=\frac{1+2e^{-1}+e^{-2}}{4}= \left(\frac{1+e^{-1}}{2}\right)^2\approx 0,4678$

\end{resp}


\begin{exer}Resolva o problema de valor inicial dado por
\begin{eqnarray*}
u'&=& -2u + \sqrt{z}\\
z'&=& -z + u\\
u(0)&=&0\\
z(0)&=&2\\
\end{eqnarray*}
com passo $h=0,2$, $h=0,02$, $h=0,002$ e $h=0,0002$ para obter aproximações para $u(2)$ e $z(2)$.
\end{exer}
\begin{resp}

$u(2)\approx 0,430202$ e $z(2)=0,617294$ com $h=0,2$,
$u(2)\approx 0,435506$ e $z(2)=0,645776$ com $h=0,02$,
$u(2)\approx 0,435805$ e $z(2)=0,648638$ com $h=0,002$ e
$u(2)\approx 0,435832$ e $z(2)=0,648925$ com $h=0,0002$.

\end{resp}

\begin{exer}Resolva o problema de valor inicial dado por
\begin{eqnarray*}
u'&=& \cos(tu(t))\\
u(0)&=&1\\
\end{eqnarray*}
com passo $h=0,1$, $h=0,01$, $h=0,001$, $h=0,0001$ e $0,00001$ para obter aproximações para $u(2)$.
\end{exer}
\begin{resp}

$u(2)\approx 1,161793$ com $h=0,1$,
$u(2)\approx 1,139573$ com $h=0,01$,
$u(2)\approx 1,137448$ com $h=0,001$,
$u(2)\approx 1,137237$ com $h=0,0001$,
$u(2)\approx 1,137216$ com $h=0,00001$
\end{resp}


\section{Convergência, consistência e estabilidade}
Nesta seção veremos três conceitos fundamentais em análise numérica: convergência, consistência e estabilidade.

\subsection{Convergência}
Um método é dito \emph{convergente} se para toda EDO com $f$ Lipschitz e todo $t>0$ temos que
$$ \lim_{h \rightarrow 0} |u_n - u(t_n)| =0, \quad \quad \forall n$$
Convergência significa que a solução numérica tende a solução do problema de valor inicial.


\begin{teo}
O método de Euler é convergente.
\end{teo}

De fato, se $f$ é Lipschitz-contínua e $|e_0|=0$, temos que
\begin{eqnarray}
 \lim_{h\rightarrow 0} |e_{n+1}|  &= \lim_{h\rightarrow 0} \mathcal \mathcal{O}(h) = 0
\end{eqnarray}


\subsection{Consistência}
\begin{defn}
Dizemos que um método numérico $R_h(u_n)=f$ é consistente com o problema de valor inicial $u'(t)=f$ se para qualquer $u(t)$
\begin{eqnarray}
  \lim_{h \rightarrow 0} |u'(t_n)-R_h(u_n)| = 0, \quad  \forall n
\end{eqnarray}
\end{defn}

Isto é equivalente a
\begin{eqnarray}
  \lim_{h \rightarrow 0} \frac{ETL}{h} = 0
\end{eqnarray}



%\novapagina
%\section{Estabilidade}
%\begin{defn}\label{def:estnum}
% Um método numérico $P_h(u_n)$ para um problema de valor inicial é \emph{estável} em uma região $\Lambda $ se  $\exists  J$ inteiro tal que $\forall T>0$, $\exists  C_T$ tal que
% \begin{eqnarray}
% |u_n|  \leq   C_T |u_0|
% \end{eqnarray}
%para $0\leq nh\leq T$, com $h\in \Lambda $.
%\end{defn}
%
%Isto significa que para ser estável a solução em $t\in [0,T]$ deve permanecer limitada por $C_T$ vezes a norma dos $J+1$ dados iniciais ($J=0$ para métodos de passo simples e $J>0$ para passo múltiplo).
%

\subsection{Estabilidade}
\begin{defn}
Um método numérico é \emph{estável} se
$$ |u_n-v_n| \leq  C_1|u^{(1)}-v_1|, \quad  \forall n$$
\end{defn}
Isto significa que dadas duas condições iniciais $u^{(1)}$ e $v_1$, teremos que as soluções $u_n$ e $v_n$ estarão a uma distância limitada  por uma constante $C_1$ vezes $|u^{(1)}-v_1|$. Se $u^{(1)}$ e $v_1$ estiverem próximas então $u_n$ e $v_n$ estão também próximas dependendo da constante $C_1$ (obviamente $C_1$ depende da função $f$).


Considere o problema de valor inicial linear bem-posto
\begin{eqnarray}\label{EDO4.7}
  u'(t)= \lambda u(t), \quad  u(0)=1,
\end{eqnarray}
onde $\lambda  \in  \mathbb{C}$. Note que:
\begin{itemize}
\item Possui solução exata $u(t)=e^{\lambda t}.$
\item O problema de valor inicial é \emph{assintoticamente estável}, isto é, $\lim_{t\rightarrow \infty }u(t)=0$, se e somente se $\Re{\lambda }<0$.
\
\end{itemize}
%Não estamos interessados no momento em soluções que crescem rapidamente ($\Re{\lambda }>0$).

%\begin{defn}
%O problema de valor inicial linear é \emph{assintoticamente estável} se e somente se $\Re{\lambda }<0$.
%\end{defn}



\begin{defn}
O \emph{domínio de estabilidade linear} $\mathcal D$ do método numérico é o conjunto de todos $h\lambda  \in  \mathbb{C}$ tal que $\lim_{n\rightarrow \infty }u_n=0$.
\end{defn}

Ou seja, $\mathcal D$ é o conjunto de todos $h\lambda $ para o qual o correto comportamento assintótico de \eqref{EDO4.7} seja recuperado.

%tal que essa equação seja estável.


\begin{ex}
Utilizando o \textbf{Método de Euler} para solucionar \eqref{EDO4.7} obtemos ($u^{(1)}=1$)
\begin{eqnarray}
 u_{n+1}   & =& u_n+h\lambda u_n, \\
 u_{n+1}   & =& (1+h\lambda )u_n, \\
 u_{n+1}   & =& (1+h\lambda )^2u_{n-1}, \\
 u_{n+1}   & =& (1+h\lambda )^{n+1}u^{(1)} \\
 u_{n+1}   & =& (1+h\lambda )^{n+1}  , \quad  n=0,1,\ldots
\end{eqnarray}
Para que o método de Euler seja estável, é necessário que $h$ seja escolhido tal que $|1+h\lambda |<1$. Ou seja, $h\lambda $ deve estar em $\mathcal D_{Euler}$ onde
\begin{eqnarray}
 \mathcal D_{Euler} = \{z \in  \mathbb{C}: |1+z|<1\}
\end{eqnarray}
é o interior de um disco no plano complexo de raio $1$ e centro em $z=-1$.% como na Fig.\ref{RegiaoEuler}.
% \begin{figure}[htp]
% \begin{center}
%   \includegraphics[width=8cm]{RegiaoEuler.eps}\\
%   \caption{Região de estabilidade para o método de Euler, $|1+z|<1$ }\label{RegiaoEuler}
% \end{center}
% \end{figure}
\end{ex}


Tal análise pode ser facilmente estendida para $u'=\lambda u+b$ (veja exercícios).

Para o caso EDO não linear, seja
\begin{eqnarray}
u'= f(t,u), \quad  t\geq t_0, \quad  u(t_0)=u_0
\end{eqnarray}
é comum requerer que $h\lambda _{n,k} \in \mathcal D$ onde $\lambda _{n,k}$ são os autovalores da matriz jacobiana $J_n := \frac{\partial f}{\partial u}|_{(t_n,u_n)},$ baseado na hipótese que o comportamento local da EDO é modelado por
\begin{eqnarray}
  u'= u_n + J_n(u-u_n)
\end{eqnarray}
Esta prática não é exata e fornece apenas uma ideia local do comportamento da EDO (podendo levar a conclusões errôneas).


Um dos teoremas mais importantes em análise numérica é o seguinte:

\begin{teo}
Um método numérico \emph{consistente} para um problema de valor inicial bem-posto é \emph{convergente} se e somente se ele é \emph{estável}.
\end{teo}


Ele também é usado da seguinte forma:

\begin{teo}
Se um método numérico é \emph{consistente} e \emph{estável} em $[a,b]$ então ele é \emph{convergente}.
\end{teo}



\subsection*{Exercícios resolvidos}

\emconstrucao

\subsection*{Exercícios}

\emconstrucao


\section{O método de Euler implícito}
Integrando o problema de valor inicial
\begin{eqnarray}
  u'(t)  &=& f(t,u(t)) \\
  u(t^{(1)}) &=& a
\end{eqnarray}
de $t^{(1)}$ até $t^{(2)}$ obtemos (como feito anteriormente)
\begin{eqnarray}
  u(t^{(2)})      &=& u(t^{(1)}) +  \int_{t^{(1)}}^{t^{(2)}} f(t,u(t)) \; dt
\end{eqnarray}

Entretanto se aproximarmos a função $f$ por uma função constante $f(t,u(t)) \approx  f(t^{(2)},u^{(2)})$, obteremos um novo método
\begin{eqnarray}
  u^{(2)} &=&  u^{(1)} + f(t^{(2)},u^{(2)}) \int _{t^{(1)}}^{t^{(2)}}  \; dt \\
  u^{(2)} &=&  u^{(1)} + h f(t^{(2)},u^{(2)})
\end{eqnarray}


Generalizando este procedimento para $t_n$ obtemos o \emph{método de Euler implícito}
\begin{eqnarray}
u_{n+1}=u_n + h\;f(t^{(n+1)},u_{n+1}).
\end{eqnarray}

Note que este método é \emph{implícito} (a equação é implícita) pois depende de $u_{n+1}$ dos dois lados da equação. Se a função $f$ for simples o suficiente, podemos resolver a equação isolando o termo $u_{n+1}$. Se isso não for possível, devemos usar um dos métodos vistos anteriormente para calcular as raízes da equação (por exemplo, método da bissecção e método de Newton).



Pode ser mostrado que o erro de truncamento local é
$$ETL_{EulImp}^{n+1}= \mathcal{O}(h^2).$$
portanto o método é de ordem $1$. E o erro de truncamento global é
$$ETG_{EulImp}^{n+1}= \mathcal{O}(h).$$



\begin{ex}
Utilizando o \textbf{método de Euler implícito} para solucionar \eqref{EDO4.7} obtemos
\begin{eqnarray}
 u_{n+1}      &=& u_n+h\lambda u_{n+1}, \\
 (1-h\lambda )u_{n+1} & =& u_n, \\
       u_{n+1} & =& \left(\frac{1}{1- h\lambda }\right)u_n, \\
       u_{n+1} & =& \left(\frac{1}{1- h\lambda }\right)^2u_{n-1}, \\
       u_{n+1} & =& \left(\frac{1}{1- h\lambda }\right)^{n+1}, \quad  n=0,1,\ldots
\end{eqnarray}
onde $u^{(1)}=1$.
Concluímos então que
\begin{eqnarray}
 \mathcal D_{EulImp} = \{z \in  \field{C}:  \left|\frac{1}{1- z}\right|<1\}
\end{eqnarray}
ou ainda,
\begin{eqnarray}
 \mathcal D_{EulImp} = \{z \in  \field{C}:  |1- z|>1\}
\end{eqnarray}

Para que o método de Euler implícito seja estável, é necessário que $h$ seja escolhido tal que $\left|\frac{1}{1- h\lambda }\right|<1$, ou ainda, $|1-h\lambda |>1$. Ou seja, $h\lambda $ deve estar em $\mathcal D_{EulImp}$ onde
\begin{eqnarray}
 \mathcal D_{EulImp} = \{z \in  \field{C}: |1-z|>1\}
\end{eqnarray}
é o exterior de um disco no plano complexo de raio $1$ e centro em $z=1$.

Note que $\mathcal D_{EulImp}$ inclui todo o semiplano negativo. Portanto o método de Euler implícito imita a estabilidade assintótica da EDO linear sem restrição no passo $h$.

\end{ex}

\begin{defn}
Um método numérico é chamado \emph{A-estável} ou \emph{incondicionalmente estável}\index{incondicionalmente estável}\index{estabilidade!incondicional} se seu domínio de estabilidade linear incluir todo o semiplano complexo com parte real negativa,
$$    \{z \in \field{C}: \Re{z}<0\} \subseteq \mathcal D$$
\end{defn}


Portanto o método de Euler implícito é $A$-estável (incondicionalmente estável).


\section{Método trapezoidal}\index{método!trapezoidal}
O método de Euler aproxima $f$ como uma constante no intervalo $[t^{(1)},t^{(2)}]$. Podemos melhorar isso usando a regra do trapézio,

\begin{eqnarray}
  u(t^{(2)}) &=& u(t^{(1)}) +  \int _{t^{(1)}}^{t^{(2)}}  f(t,u(t)) \; dt \\
  u^{(2)}    &=&   u^{(1)}  +  (t^{(2)}-t^{(1)})\left(\frac{1}{2}f(t^{(1)},u^{(1)})+\frac{1}{2}f(t^{(2)},u^{(2)})\right)
\end{eqnarray}
motivando o \emph{método trapezoidal}
\begin{eqnarray}
  u_{n+1} &=& u_n +  \frac{h}{2} \left(f(t_n,u_n)+f(t^{(n+1)},u_{n+1})\right)
\end{eqnarray}
O método trapezoidal é dito \textbf{implícito}, pois para obter $u_{n+1}$ é necessário calcular $f(t^{(n+1)},u_{n+1})$.

Entretanto, pode ser mostrado que o erro de truncamento local é
$$ETL_{Trap}^{n+1}= O(h^3)$$
portanto o método é de ordem $2$. E o erro de truncamento global é
$$ETG_{Trap}^{n+1}= O(h^2)$$


\begin{ex}
Utilizando o \textbf{método trapezoidal} para solucionar \eqref{EDO4.7} obtemos
\begin{eqnarray}
 u_{n+1} = \left(\frac{1+ h\lambda /2}{1- h\lambda /2}\right)^{n+1}, \quad  n=0,1,\ldots
\end{eqnarray}
Concluímos então que
\begin{eqnarray}
 \mathcal D_{Tr} = \{z \in  \field{C}:  \left|\frac{1+ z/2}{1- z/2}\right|<1\}
\end{eqnarray}
Note que $\mathcal D_{Tr}=\field{C}^-$, o semiplano negativo. Portanto o método do trapézio imita a estabilidade assintótica da EDO linear sem restrição no passo $h$.
\end{ex}



\subsection*{Exercícios resolvidos}

\emconstrucao

\subsection*{Exercícios}

\emconstrucao


\section{O método de Heun}
Também chamado de método de \emph{Euler modificado}. A ideia é calcular primeiramente um valor intermediário $\tilde{u}$ usando o método de Euler expl\'icito e usar esse valor na equação para o método do Trapézio. Ou seja, o \emph{método de Heun} é
\begin{eqnarray}
  \tilde{u} &=& u_n +   h f(t_n,u_n) \\
  u_{n+1}   &=& u_n +  \frac{h}{2} \left(f(t_n,u_n)+f(t^{(n+1)},\tilde{u})\right)
\end{eqnarray}

Este é um exemplo de um método preditor-corretor.

Felizmente o erro de truncamento local continua sendo
$$ETL_{Heun}^{n+1}= O(h^3)$$
e o erro de truncamento global é
$$ETG_{Heun}^{n+1}= O(h^2)$$


\subsection*{Exercícios resolvidos}

\emconstrucao

\subsection*{Exercícios}

\begin{exer} Use o método de Euler melhorado para obter uma aproximação numérica do valor de $u(1)$ quando $u(t)$ satisfaz o seguinte problema de valor inicial
\begin{eqnarray*}
 u'(t)&=&-u(t)+ e^{u(t)},\\
 u(0)&=&0,
\end{eqnarray*}
usando passos $h=0,1$ e $h=0,01$.
\end{exer}
\begin{resp}
 $u(1)\approx 1,317078$ quando $h=0,1$ e $u(1)\approx 1,317045$.
\end{resp}


\begin{exer}
Use o método de Euler e o método de Euler melhorado para obter aproximações numéricas para a solução do seguinte problema de valor inicial para $t\in[0,1]$:
\begin{eqnarray*}
 u'(t)&=&-u(t)- u(t)^2,\\
 u(0)&=&1,
\end{eqnarray*}
usando passo $h=0,1$. Compare os valores da solução exata dada por $u(t)=\frac{1}{2e^t-1}$ com os numéricos nos pontos $t=0$, $t=0,1$, $t=0,2$, $t=0,3$, $t=0,4$, $t=0,5$, $t=0,6$, $t=0,7$, $t=0,8$, $t=0,9$, $t=1,0$.
\end{exer}
\begin{resp}
 $$\begin{array}{|c|c|c|c|c|c|}
\hline
t &  \text{Exato} & \text{Euler} & \text{Euler melhorado} & \text{Erro Euler} & \text{Erro Euler melhorado}\\
\hline
0,0&    1,          &1,          &1,          &0,          &0,       \\
0,1&    0,826213    &0,8         &0,828       &0,026213    &0,001787\\
0,2&    0,693094    &0,656       &0,695597    &0,037094    &0,002502  \\
0,3&    0,588333    &0,547366    &0,591057    &0,040967    &0,002724  \\
0,4&    0,504121    &0,462669    &0,506835    &0,041453    &0,002714  \\
0,5&    0,435267    &0,394996    &0,437861    &0,040271    &0,002594  \\
0,6&    0,378181    &0,339894    &0,380609    &0,038287    &0,002428  \\
0,7&    0,330305    &0,294352    &0,332551    &0,035953    &0,002246  \\
0,8&    0,289764    &0,256252    &0,291828    &0,033512    &0,002064  \\
0,9&    0,255154    &0,224061    &0,257043    &0,031093    &0,001889  \\
1,0&    0,225400    &0,196634    &0,227126    &0,028766    &0,001726\\

\hline
\end{array}
$$

\ifisscilab
      No Scilab, esta tabela pode ser produzida com o código:
      \begin{verbatim}
       deff('du=f(u)','du=-u-u^2')
       sol_Euler=Euler(f,0,1,10,1)'
       sol_Euler_mod=Euler_mod(f,0,1,10,1)'
       deff('u=u_exata(t)','u=1/(2*exp(t)-1)')
       t=[0:.1:1]'
       sol_exata=feval(t,u_exata)
       tabela=[t sol_exata sol_Euler sol_Euler_mod abs(sol_exata-sol_Euler) abs(sol_exata-sol_Euler_mod)]
      \end{verbatim}

    \fi
\end{resp}


\section{O método theta}
Tanto o método de Euler quanto o método trapezoidal se encaixam no método
\begin{eqnarray}
  u_{n+1} &= u_n +  h (\theta f(t_n,u_n)+(1-\theta )f(t^{(n+1)},u_{n+1}))
\end{eqnarray}
com $\theta =1$ e $\theta =\frac{1}{2}$ respectivamente. O método é explícito somente para $\theta =1$. Para $\theta =0$, obtemos o método implícito de Euler.

\subsection*{Exercícios resolvidos}

\emconstrucao

\subsection*{Exercícios}

\emconstrucao


\section{O método de Taylor}
Uma maneira simples de aumentar a ordem dos métodos de Euler anteriormente descritos consiste em truncar a série de Taylor de $u(t+h)$:
\begin{eqnarray}
 u(t+h)=u(t) +h u'(t)+ \frac{h^2}{2!}u''(t)+\frac{h^3}{3!}u'''(t)+\ldots
\end{eqnarray}

Utilizando dois termos temos o método de Euler. Utilizando os três primeiros termos da série e substituindo $u'(t)=f(t,x)$ e $u''(t)=\frac{\partial f}{\partial t}(t,x)$ temos o \emph{método de Taylor de ordem $2$}
\begin{eqnarray}
   u_{n+1}=u_n +h f(t_n,u_n)+ \frac{h^2}{2!} \frac{\partial f}{\partial t}(t_n,u_n)
\end{eqnarray}


O método de Taylor de ordem $3$ é
\begin{eqnarray*}
   u_{n+1}=u_n +h f(t_n,u_n)+ \frac{h^2}{2!}\frac{\partial f}{\partial t}(t_n,u_n)+\frac{h^3}{3!}\frac{\partial^2 f}{\partial t^2}(t_n,u_n)
\end{eqnarray*}

\subsection*{Exercícios resolvidos}

\emconstrucao

\subsection*{Exercícios}

\emconstrucao

\section{Estabilidade dos métodos de Taylor}
\begin{ex}
Prove que para um método de Taylor de ordem $p$ para a EDO \eqref{EDO4.7} temos
\begin{eqnarray}
  p(z)= 1 + z+ \frac{z^2}{2!} +\frac{z^3}{3!}+\ldots +\frac{z^p}{p!}
\end{eqnarray}
onde  $u_n = (p(z))^nu_0$ e a região de estabilidade é dada por
\begin{eqnarray}
 \mathcal D_{T} = \{z \in  \field{C}:  \left|p(z)\right|<1\}
\end{eqnarray}

Trace as regiões de estabilidade para o método de Taylor para $p=1,\ldots ,6$ no mesmo gráfico.
\end{ex}

% \begin{figure}
% \begin{center}
%   \includegraphics[width=8cm]{RegiaoTaylor.eps}\\
%   \caption{Região de estabilidade paras os métodos de Taylor de ordem $1,\ldots ,4$ (interior as curvas). A curva mais interna é para $p=1$}\label{RegiaoTaylor}
% \end{center}
% \end{figure}


\begin{ex}
Aproxime a solução do problema de valor inicial
\begin{eqnarray}
   \frac{du}{dt} &=\sin{t}\\
            u(0) &= 1
\end{eqnarray}
para  $t\in [0,10]$.

\begin{enumerate}
\item [a.] Trace a solução para $h=0,16$, $0,08$, $0,04$, $0,02$ e $0,01$ para o método de Taylor de ordem $1$, $2$ e $3$. (Trace todos de ordem $1$ no mesmo gráfico, ordem $2$ em outro gráfico e ordem $3$ outro gráfico separado.)

\item [b.] Utilizando a solução exata, trace um gráfico do erro em escala logar\'itmica.
Comente os resultados (novamente, em cada gráfico separado para cada método repita os valores acima)

\item [c.] Fixe agora o valor $h=0,02$ e trace no mesmo gráfico uma curva para cada método.

\item [d.] Trace em um gráfico o erro em $t=10$ para cada um dos métodos (uma curva para cada ordem) a medida que $h$ diminui. (Use escala \verb#loglog#)
\end{enumerate}
\end{ex}











% \section{Ordem de precisão}\index{ordem de precisão}
% Considere o problema de valor inicial dado por
% \begin{eqnarray*}
% u'(t)&=&f(t,u(t)),\\
% u(0)&=&u_0.
% \end{eqnarray*}
% Nessa seção vamos definir a precisão de um método numérico pela ordem do erro acumulado ao calcular o valor da função em um ponto $t_N$ em função do espaçamento da malha $h$. Se $u(t_n)$ pode ser aproximado por uma expressão que depende de $f$, $h$, $u(t_0)$, $u(t^{(1)})$, $\cdots$, $u(t_n)$, com erro da ordem de $O(h^{p+1})$, ou seja,
% \begin{equation}{\label{erro_local_1}}
% u(t^{(n+1)})=\mathcal{F}(f, h, u(t_n), u(t_{n-1}), \cdots, u_0) + O(h^{p+1})
% \end{equation}
% para cada função analítica $f$, dizemos que o método tem erro de truncamento da ordem de $O(h^{p})$ ou {\bf ordem de precisão $p$}\index{método!de Euler!ordem de precisão}. Essa afirmação faz sentido quando fazemos a seguinte análise informal: para aproximar $u^{(1)}$, acumulamos erros da ordem $O(h^{p+1})$, para calcular $u^{(2)}$ acumulamos os erros de $u^{(1)}$ e novos erros $O(h^{p+1})$. Para calcular $u_N$, acumulamos todos os erros até $t_N$, ou seja, $N$ vezes $O(h^{p+1})$. Como $N=O(1/h)$, temos que os erros ao calcular $u_N$ são da ordem $O(h^p)$. É verdade que essa análise só vale quando impomos condições de suavidade para $f$ e condições adequada para a expressão $\mathcal{F}(f, h, u(t_n), u(t_{n-1}), \cdots, u_0)$. Para explicar melhor esse pequeno texto, fazemos em detalhes essa operação para o método de Euler na Seção~\ref{sec_pre_euler}.
%
% \subsection{Ordem de precisão do método de Euler}{\label{sec_pre_euler}}
% Primeiro lembramos da expressão (\ref{erro_local}) que origina a seguinte relação de recorrência:
% \begin{eqnarray}{\label{es_Euler}}
% u(t^{(n+1)})&=&u(t_n)+hf(t,u(t_n),t_n)+O(h^2).
% \end{eqnarray}
% Para entender melhor o motivo de na expressão (\ref{es_Euler}) aparecer $O(h^2)$ e o método ser de precisão 1, vamos a seguinte análise informal: observemos que
% \begin{eqnarray*}
%  u(t^{(1)})&=&u(t_0)+hf(t,u(t_0),t_0)+O(h^2)\\
%  &=&u_0+hf(u_0,t_0)+O(h^2)=u^{(1)}+O(h^2)
% \end{eqnarray*}
% onde $u^{(i)}$ é a aproximação pelo método de Euler para o valor exato $u(t^{(i)})$. Subsequentemente, temos
% \begin{eqnarray*}
%  u(t^{(2)})&=&u(t^{(1)})+hf(t,u(t^{(1)}),t^{(1)})+O(h^2)\\
%  &=&u(t^{(1)})+hf(u^{(1)}+O(h^2),t^{(1)})+O(h^2)\\
%  &=&u(t^{(1)})+hf(u^{(1)},t^{(1)}) +O(h^2)\\
%  &=&u^{(1)}+O(h^2)+hf(u^{(1)},t^{(1)}) +O(h^2)= u^{(2)}+O(h^2)+O(h^2).
% \end{eqnarray*}
% onde usamos o primeiro termo da série de Taylor $hf(u^{(1)}+O(h^2),t^{(1)})=hf(u^{(1)},t^{(1)})+O(h^3)$ na passagem da segunda para terceira linha. Repetindo sucessivamente o passo anterior, obtemos uma expressão geral para o valor exato $u(t_N)$ em termos do valor aproximado $u_N$:
% \begin{eqnarray*}
%  u(t_N)=u_N+N O(h^2) .
% \end{eqnarray*}
% Como $N=(t_f-t_0)/h$, temos
% \begin{eqnarray}{\label{euler_precisao}}
%  u(t_N)&=&=u_{N}+\frac{t-t_0}{h}O(h^2)=u_{N}+\mathcal{O}(h),
% \end{eqnarray}
% ou seja, o erro entre o valor exato e o aproximado é de ordem $h$. Uma demonstração mais formal que garante que o erro é limitado por uma expressão que é proporcional a $h$ está discutido na Seção~\ref{sec_conv_Euler}.
%
% \subsection{Ordem de precisão do método de Euler melhorado}
% Para obter o erro de precisão do método de Euler melhorado vamos calcular o erro de truncamento do método, ou seja, precisamos demonstrar que:
% \begin{equation}\label{es_Euler_melhorado}
% u(t+h)=u(t)+\frac{h}{2} f(t,u(t))+\frac{h}{2} f(t,u(t)+hf(t,u(t))),t+h)+O(h^3)
% \end{equation}
% De fato, tomando a diferença do termo da esquerda o os termos da direita, temos:
% \begin{eqnarray*}
% &&u(t+h)-\left(u(t)+\frac{h}{2} f(t,u(t))+\frac{h}{2} f(t,u(t)+hf(t,u(t))),t+h)\right)\\
% &&=u(t)+hu'(t)+\frac{h^2}{2}u''(t)+O(h^3)\\
% &&-\left(u(t)+\frac{h}{2} u'(t)+\frac{h}{2} f(t,u(t)+hf(t,u(t))),t+h)\right),
% \end{eqnarray*}
% onde usamos uma expansão em série de Taylor para $u(t+h)$ e a equação diferencial $u'(t)=f(t,u(t))$. Portanto,
% \begin{eqnarray*}
% &&u(t+h)-\left(u(t)+\frac{h}{2} f(t,u(t))+\frac{h}{2} f(t,u(t)+hf(t,u(t))),t+h)\right)\\
% &&=\frac{h}{2}u'(t)+\frac{h^2}{2}u''(t)-\frac{h}{2} f(t,u(t)+hf(t,u(t))),t+h)+O(h^3).
% \end{eqnarray*}
% Agora, usamos a série de Taylor de $f(t,u(t)+hf(t,u(t)),t+h)$ e, torno de $(t,u)$:
% \begin{eqnarray*}
% &&u(t+h)-\left(u(t)+\frac{h}{2} f(t,u(t))+\frac{h}{2} f(t,u(t)+hf(t,u(t))),t+h)\right)\\
% &&=\frac{h}{2}u'(t)+\frac{h^2}{2}u''(t)+O(h^3)\\
% &&-\frac{h}{2}\left(f(t,u(t))+\frac{\partial f(t,u(t)) }{\partial t}h +\frac{\partial f(t,u(t))}{\partial u} hf(t,u(t))+O(h^2)\right).
% \end{eqnarray*}
% Usando a equação diferencial $u'(t)=f(t,u(t))$ obtemos
% $$
% u''(t)=\frac{f(t,u(t))}{\partial t}+\frac{f(t,u(t))}{\partial u}u'(t)=\frac{f(t,u(t))}{\partial t}+\frac{f(t,u(t))}{\partial u}f(t,u(t)).
% $$
% Logo,
% \begin{eqnarray*}
% &&u(t+h)-\left(u(t)+\frac{h}{2} f(t,u(t))+\frac{h}{2} f(t,u(t)+hf(t,u(t))),t+h)\right)\\
% &&=\frac{h}{2}u'(t)+\frac{h^2}{2}u''(t)+O(h^3)\\
% &&-\frac{h}{2}\left(f(t,u(t))+hu''(t)+O(h^2)\right)\\
% &&=\frac{h}{2}u'(t)+\frac{h^2}{2}u''(t)\\
% &&-\frac{h}{2}\left(u'(t)+hu''(t)\right)+O(h^3)=O(h^3)
% \end{eqnarray*}
% Portanto, a expressão (\ref{es_Euler_melhorado}) é válida. Logo, usando uma discussão análoga aquela feita na Seção~\ref{sec_pre_euler} para o método de Euler, concluímos que o método de Euler melhorado possui ordem de precisão 2.

% \section{Convergência}
%
% \emconstrucao
%
% \subsection{Convergência do método de Euler}\label{sec_conv_Euler}
%
% \emconstrucao
%
% \subsection{Convergência do método de Euler melhorado}
%
% \emconstrucao




\subsection*{Exercícios resolvidos}

\emconstrucao

\subsection*{Exercícios}

\emconstrucao

\section{Métodos de passo múltiplo}
Seja o problema de valor inicial
\begin{eqnarray}
  u'(t) &= f(t,u(t)) \\
  u(t_0) &= a
\end{eqnarray}


Integrando a EDO em $[t^{(n+1)},t^{(n)}]$ obtemos
\begin{eqnarray}
  u_{n+1}  &= u_n  + \int _{t_n}^{t^{(n+1)}} f(t,u(t)) \; dt
\end{eqnarray}
Denote por $f_n\equiv f(t_n,u_n)$. Um método de passo simples utiliza $f_{n+1}$ e $f_{n}$. Um método de passo múltiplo utiliza também $s$ valores anteriores já calculados como $f_{n-1},f_{n-2},\ldots ,f_{n-s}$, onde $s\geq 1$ inteiro.

\begin{eqnarray}
  u_{n+1}  &= u_{n}  + h[b_s f_{n+1}+b_{s-1}f_{n}+\ldots +b_1f_{n-s+2}+b_0f_{n-s+1}]
%\\  u_{n+1}  &= u_{n}  + h \sum_{m=0}^s b_m f_{n-s+1+m}
\end{eqnarray}

Para conformidade com \cite{iserles2009first}, translade $s-1$ índices,
\begin{eqnarray}
  u_{n+s}  &= u_{n+s-1}  + h[b_s f_{n+s}+b_{s-1}f_{n+s-1}+\ldots +b_1f_{n+1}+b_0f_n] \label{multiad}
\end{eqnarray}
e teremos
\begin{eqnarray}
  u_{n+s}  &= u_{n+s-1}  + h \sum_{m=0}^s b_m f_{n+m}
\end{eqnarray}

De forma geral um \emph{método de passo múltiplo} será
\begin{eqnarray}
  \sum_{m=0}^s a_m u_{n+m}  &=  h \sum_{m=0}^s b_m f_{n+m}
\end{eqnarray}

\subsection*{Exercícios resolvidos}

\emconstrucao

\subsection*{Exercícios}

\emconstrucao


\section{O método de Adams-Bashforth}
Quando $a_s=1$, $a_{s-1}=-1$, $a_m=0$ para $m=s-2,\ldots ,0$, $b_s=0$ temos um método de Adams-Bashforth do tipo
\begin{eqnarray}\label{AB}
  u_{n+s}  &= u_{n+s-1}  + h \sum_{m=0}^{s-1} b_m f_{n+m}
\end{eqnarray}
Note que os métodos de Adams-Bashforth são \emph{explícitos} pois $b_s=0$.



\begin{ex}
Vamos obter o método de Adams-Bashforth para $s=4$ como
\begin{eqnarray}
  u_{n+4}  &= u_{n+3}  + \int _{t_{n+3}}^{t_{n+4}} f(t,u(t)) \; dt \\
  u_{n+4}  &= u_{n+3}  + h \sum_{m=0}^{3} b_m f_{n+m} \\
  u_{n+4}  &= u_{n+3}  + h [b_3f_{n+3} +b_2f_{n+2} +b_1f_{n+1} +b_0f_n]
\end{eqnarray}
Para isso devemos obter $[b_3,b_2,b_1,b_0]$ tal que o método seja exato para polinômios até ordem $3$. Podemos obter esses coeficientes de maneira análoga a obter os coeficientes de um método para integração.

Supondo que os nós $t_k$ estejam igualmente espaçados, e para facilidade dos cálculos, como o intervalo de integração é $[t_{n+3},t_{n+4}]$, translade $t_{n+3}$ para a origem tal que $[t_n,t^{(n+1)},\ldots ,t_{n+4}]=[-3h,-2h,-h,0,h]$.

Considere a base $[\phi _0(t),\ldots ,\phi _3(t)]=[1, t, t^2, t^3]$ e substitua $f(t)$ por $\phi _k(t)$ obtendo
\begin{eqnarray*}
  \int _0^{h} 1  \;dt = h             &= h( b_0(1)  +b_1(1)    + b_2(1)   + b_3(1)    )\\
  \int _0^{h} t  \;dt = \frac{h^2}{2}  &= h( b_0(0)  +b_1(-h)   + b_2(-2h) + b_3(-3h)  )\\
  \int _0^{h} t^2 \;dt = \frac{h^3}{3}  &= h( b_0(0)^2 +b_1(-h)^2  + b_2(-2h)^2+ b_3(-3h)^2 )\\
  \int _0^{h} t^3 \;dt = \frac{h^4}{4} &= h( b_0(0)^3 +b_1(-h)^3  + b_2(-2h)^3+ b_3(-3h)^3 )
\end{eqnarray*}
que pode ser escrito na forma matricial
\begin{eqnarray}
\left(
  \begin{array}{cccc}
    1  &  1    & 1   & 1\\
    0  &  -1   & -2  & -3\\
    0  &  1    & 4   &  9\\
    0  &  -1   & -8  & -27
  \end{array}
\right)
\left(\begin{array}{c}  b_0 \\ b_1\\ b_2\\b_3   \end{array}\right)
=
\left(\begin{array}{c}  1  \\ 1/2 \\ 1/3 \\ 1/4  \end{array}\right)
\end{eqnarray}
Resolvendo o sistema obtemos
$$[b_0,b_1,b_2,b_3]=[-\frac{9}{24},\frac{37}{24},-\frac{59}{24},\frac{55}{24}]$$
fornecendo o \emph{método de Adams-Bashforth de $4$ estágios}
\begin{eqnarray}\label{AB4}
  u_{n+4}  &= u_{n+3}  + \frac{h}{24} [55 f_{n+3} -59f_{n+2} +37f_{n+1} -9f_n]
\end{eqnarray}
\end{ex}

\subsection*{Exercícios resolvidos}

\emconstrucao

\subsection*{Exercícios}

\begin{exer}
Mostre que o método de Adams-Bashforth para $s=2$ é dado por
\begin{eqnarray}\label{AB2}
  u_{n+2}  &= u_{n+1}  + \frac{h}{2} [3 f_{n+1} -f_{n}]
\end{eqnarray}
\end{exer}

\begin{exer}
Mostre que o método de Adams-Bashforth para $s=3$ é dado por
\begin{eqnarray}\label{AB3}
  u_{n+3}  &= u_{n+2}  + \frac{h}{12} [23f_{n+2}-16 f_{n+1} +5f_{n}]
\end{eqnarray}
\end{exer}




\section{O método de Adams-Moulton}
Quando $a_s=1$, $a_{s-1}=-1$, $a_m=0$ para $m=s-2,\ldots ,0$, $b_s\neq 0$ temos um método de Adams-Moulton do tipo
\begin{eqnarray}\label{AM}
  u_{n+s}  &= u_{n+s-1}  + h \sum_{m=0}^{s} b_m f_{n+m}
\end{eqnarray}
Note que os métodos de Adams-Moulton são implícitos pois $b_s\neq 0$.



\begin{ex}
Vamos obter o método de Adams-Moulton para $s=3$ como
\begin{eqnarray}\label{AM4}
  u_{n+3}  &= u_{n+2}  + \int _{t_{n+3}}^{t_{n+4}} f(t,u(t)) \; dt \\
  u_{n+3}  &= u_{n+2}  + h \sum_{m=0}^{3} b_m f_{n+m} \\
  u_{n+3}  &= u_{n+2}  + h [b_3f_{n+3} +b_2f_{n+2} +b_1f_{n+1} +b_0f_n]
\end{eqnarray}
Para isso devemos obter $[b_3,b_2,b_1,b_0]$ tal que o método seja exato para polinômios até ordem $3$. Podemos obter esses coeficientes de maneira análoga a obter os coeficientes de um método para integração.

Supondo que os nós $t_k$ estejam igualmente espaçados, e para facilidade dos cálculos, como o intervalo de integração é $[t_{n+2},t_{n+3}]$, translade $t_{n+2}$ para a origem tal que $[t_n,t^{(n+1)},\ldots ,t_{n+3}]=[-2h,-h,0,h]$.

Considere a base $[\phi _0(t),\ldots ,\phi _3(t)]=[1, t, t^2, t^3]$ e substitua $f(t)$ por $\phi _k(t)$ obtendo
\begin{eqnarray*}
      \int _0^{h} 1  \;dt = h             &= h( b_0(1)  +b_1(1)   + b_2(1)   + b_3(1)    )\\
      \int _0^{h} t  \;dt = \frac{h^2}{2}  &= h( b_0(h)  +b_1(0)   + b_2(-h) + b_3(-2h)  )\\
      \int _0^{h} t^2 \;dt = \frac{h^3}{3}  &= h( b_0(h)^2 +b_1(0)^2  + b_2(-h)^2+ b_3(-2h)^2 )\\
      \int _0^{h} t^3 \;dt = \frac{h^4}{4} &= h( b_0(h)^3 +b_1(0)^3  + b_2(-h)^3+ b_3(-2h)^3 )
\end{eqnarray*}
que pode ser escrito na forma matricial
\begin{eqnarray}
\left(
  \begin{array}{cccc}
    1  & 0 & 1    & 1   \\
    1  & 0 & -1   & -2  \\
    1  & 0 & 1    & 4   \\
    1  & 0 & -1   & -8
  \end{array}
\right)
\left(\begin{array}{c}  b_0 \\ b_1\\ b_2\\b_3   \end{array}\right)
=
\left(\begin{array}{c}  1  \\ 1/2 \\ 1/3 \\ 1/4  \end{array}\right)
\end{eqnarray}
Resolvendo o sistema obtemos
$$[b_0,b_1,b_2,b_3]=[\frac{1}{24},-\frac{5}{24},\frac{19}{24},\frac{9}{24},]$$
fornecendo a regra
\begin{eqnarray}
  u_{n+3}  &= u_{n+2}  + \frac{h}{24} [9 f_{n+3} +19f_{n+2} -5f_{n+1} +f_n]
\end{eqnarray}
\end{ex}

\subsection*{Exercícios resolvidos}

\emconstrucao

\subsection*{Exercícios}

\begin{exer}
Encontre o método de Adams-Moulton para $s=2$.
%\begin{eqnarray}\label{AM2}
%  u_{n+2}  &= u_{n+1}  + \frac{h}{2} [3 f_{n+1} -f_{n}]
%\end{eqnarray}
\end{exer}

\begin{exer}
Encontre o método de Adams-Moulton para $s=3$.
%\begin{eqnarray}
%  u_{n+3}  &= u_{n+2}  + \frac{h}{12} [23f_{n+2}-16 f_{n+1} +5f_{n}]
%\end{eqnarray}
\end{exer}

\section{Método BDF}
Um método de ordem $s$ com $s$ estágios é chamado de \emph{método BDF-Backward Differentiation Formula} se $\sigma (w)=b_sw^s$, onde $b_s \in \mathbb{R}$, ou seja,
\begin{eqnarray}\label{BDF}
  a_s u_{n+s}+ ...+ a_1 u_{n+1} + a_0u_{n} &=  h b_sf_{n+s}
\end{eqnarray}

\begin{ex}
Mostre que o método BDF com $s=3$ é
\begin{eqnarray}
  u_{n+3} -\frac{18}{11} u_{n+2}+\frac{9}{11}u_{n+1}-\frac{2}{11}u_n &= \frac{6}{11}h f_{n+3}
\end{eqnarray}
\end{ex}

\subsection*{Exercícios}

\begin{exer}
Mostre que o método BDF com $s=1$ é o método de Euler implícito.
\end{exer}

\begin{exer}
Mostre que o método BDF com $s=2$ é
\begin{eqnarray}
  u_{n+2} -\frac{4}{3} u_{n+1} + \frac{1}{3}u_n &= \frac{2}{3}h f_{n+2}
\end{eqnarray}
\end{exer}


\section{Ordem e convergência de métodos de passo múltiplo}
Mais geralmente, um método de passo múltiplo será da forma
\begin{eqnarray}\label{multistep}
  a_s u_{n+s}+ ...+ a_1 u_{n+1} + a_0u_{n} &=  h [b_sf_{n+s} +... +b_1f_{n+1} +b_0f_n]
\end{eqnarray}
Por convenção normalizamos a equação acima tomando $a_s=1$. Quando $b_s=0$ temos um método explícito e quando $b_s \neq 0$ temos um método implícito.

O método será de ordem $p$ se o $ETL=\mathcal O(h^{p+1})$.


Dois polinômios são usados para estudar o método \eqref{multistep}:
\begin{eqnarray}
\rho (w)= a_s w^s + ...+a_1w+a_0, \quad \quad \sigma (w)=b_s w^s + ...+b_1w+b_0,
\end{eqnarray}



%\begin{teo}\label{teo:multiordem}
%Um método de passo múltiplo é de ordem $p$ se e somente se
%$$ \rho (w)=\sigma (\frac{dw}{dt}),  \quad  w=t^k, k=0,...,p-1 $$
%\end{teo}
%\begin{proof}
%Para o método ser de ordem $p$ é necessário e suficiente que $\rho (w)=\sigma (\frac{dw}{dt})$ para todos os polinômios até ordem $p$, ou seja,
%\begin{eqnarray} \label{condconv}
%  a_s +...+a_1+a_0      &=0 \\
%  sa_s+...+1a_1+0a_0    &=b_s +...+b_1+b_0         \\
%  s^2a_s+...+1^2a_1+0^2a_0 &=2(s b_s+...+1 b_1+0 b_0) \\
%  s^3a_s+...+1^3a_1+0^3a_0 &=3(s^2b_s+...+1^2b_1+0^2b_0) \\
%         \vdots    &=&\vdots  \\
%  s^ka_s+...+1^ka_1+0^ka_0 &=(k-1)(s^{k-1}b_s+...+1^{k-1}b_1+0b_0)
%\end{eqnarray}
%se e somente se o método é de ordem $p$.
%\end{proof}
%
%
%
%
%\begin{ex}
%Mostre que o método de Adams-Bashforth para $s=2$ dado por
%\begin{eqnarray}\label{AB2}
%  u_{n+2}  &=&u_{n+1}  + \frac{h}{2} [3 f_{n+1} -f_{n}]
%\end{eqnarray}
%é de ordem $2$.
%
%Temos que $\rho (w)=w^2-w$ e $\sigma (w)=\frac{3}{2}w-\frac{1}{2}$.
%
%Assim para $\phi (t)=1,t,t^2$ obtemos
%\begin{eqnarray} \label{condconv}
%  \rho (1)-\sigma (0) = (1^2-1) -0 =0 \\
%  \rho (t)-\sigma (1) = (t^2-t) -(\frac{3}{2}-\frac{1}{2} )=0 \\
%  \rho (t^2)-\sigma (2t) = (t^4-t^2) -(\frac{3}{2}-\frac{1}{2} )=0 \\
%  2 a_2+1 a_1      &=b_2+b_1+b_0         \\
%  2^2a_2+1^2a_1      &=2(2 b_2+1 b_1+0 b_0) \\
%  2^3a_2+1^3a_1+0a_0  &=3(2^2b_2+1^2b_1+0^2b_0)
%\end{eqnarray}
%
%
%
%\begin{eqnarray} \label{condconv}
%  a_2  +  a_1+a_0   &=0 \\
%  2 a_2+1 a_1      &=b_2+b_1+b_0         \\
%  2^2a_2+1^2a_1      &=2(2 b_2+1 b_1+0 b_0) \\
%  2^3a_2+1^3a_1+0a_0  &=3(2^2b_2+1^2b_1+0^2b_0)
%\end{eqnarray}
%
%
%
%
%
%
%\end{ex}
%
%


%
%\subsection{Convergência de métodos de passo múltiplo}
%O erro de truncamento local para qualquer $u(t)$ suave
%\begin{eqnarray}
%   ETL(u(t)) &=&a_s u(t+sh)+...+a_1u(t+h)+a_0u(t) -  h [b_s u'(t+sh)+...+b_1u'(t+h)+b_0u'(t)] \\
%             &=&\mathcal O(h^{p+1})
%\end{eqnarray}
%se e somente se $ETL(q(t))=0$ $\forall q(t) \in \mathcal P^p$.
%
%Isto é equivalente a $ETL( \phi _k(t) )=0$ para a base polinomial $\phi _k(t)=t^k$, $k=0,\ldots ,p$.
%
%Para $\phi _0(t)=1$ obtemos
%\begin{eqnarray}
%   ETL(\phi _0) &=& a_s u(t+sh)+...+a_1u(t+h)+a_0u(t) -  h [b_s u'(t+sh)+...+b_1u'(t+h)+b_0u'(t)] \\
%           &=& a_s +...+a_1+a_0-  h [b_s 0+...+b_10+b_00] \\
%           &=& a_s +...+a_1+a_0 =0 \\
%\end{eqnarray}
%e para $\psi _1(t)=t$, obtemos
%\begin{eqnarray}
%   ETL(\phi _1) &=& a_s u(t+sh)+...+a_1u(t+h)+a_0u(t) -  h [b_s u'(t+sh)+...+b_1u'(t+h)+b_0u'(t)] \\
%           &=& a_s sh+...+a_1h+a_00 -  h [b_s +...+b_1+b_0] =0
%\end{eqnarray}
%e para $\psi _k(t)=t^k$, obtemos
%\begin{eqnarray}
%   ETL(\phi _k) &=& a_s u(t+sh)+...+a_1u(t+h)+a_0u(t) -  h [b_s u'(t+sh)+...+b_1u'(t+h)+b_0u'(t)] \\
%           &=& a_s (sh)^k+...+a_2(2h)^k+a_1(h)^k+a_00 -  h k[b_s (sh)^{k-1}+...+b_1h^{k-1}+b_00] =0
%\end{eqnarray}
%que é igual a zero para $k=1,\ldots ,p$ sob a condição \eqref{condconv}.
%
%

\begin{ex}
O método \eqref{AB3} de Adams-Bashforth para $s=3$ estágios é de ordem $3$ de convergência, ou seja, $ETL = \mathcal O(h^4)$. Ele é construído de tal maneira que seja exato para os polinômios $1, t, t^2, t^3$.
\end{ex}


\subsection{Consistência, estabilidade e convergência}
\begin{teo}
Um método de passo múltiplo é \emph{consistente} se $\rho (1)=0$ e $\rho '(1)=\sigma (1)$.
\end{teo}


\begin{teo}
Um método de passo múltiplo é \emph{estável} se todas as raízes de  $\rho (z)$ estão em $|z|\leq 1$ e as raízes com $|z|=1$ são simples.
\end{teo}


\begin{teo}
Se um método numérico é \emph{consistente} e \emph{estável} em $[a,b]$ então ele é \emph{convergente}.
\end{teo}



\begin{ex}
Prove que o método de passo $3$
\begin{eqnarray}\label{multis3}
  u_{n+3} +\frac{27}{11}u_{n+2} -\frac{27}{11}u_{n+1} -u_{n}  =\\
   =\frac{h}{11} [3 f_{n+3}+27f_{n+2}+27f_{n+1} +3f_{n}]
\end{eqnarray}
não é estável.
\end{ex}
\begin{sol}
O polinômio
\begin{eqnarray}
   \rho (w) &=&w^3+\frac{27}{11}w^2 -\frac{27}{11}w -1 \\
        &=&(w-1)\left(w+\frac{19+4\sqrt{15}}{11}\right)
                \left(w+\frac{19-4\sqrt{15}}{11}\right)
\end{eqnarray}
falha na condição da raiz.
\end{sol}


\subsection*{Exercícios resolvidos}

\emconstrucao

\subsection*{Exercícios}

\begin{exer}
Prove que todos os métodos de Adams-Bashforth satisfazem a condição da raiz.
%\begin{solution}
%Para o método de Adams-Bashforth temos que $\rho (w)=w^{s-1}(w-1)$ satisfazendo a condição da raiz.
%\end{solution}
\end{exer}

\begin{teo}
O polinômio $\rho (w)$ em \eqref{BDF} satisfaz a condição da raiz e o método BDF é convergente se e somente se $1\leq s\leq 6$.
\end{teo}

\begin{exer}
Mostre que os métodos BDF com $s=2$ e $s=3$ são convergentes.
\end{exer}



\subsection{As barreiras de Dahlquist}
Um método de passo múltiplo possui $2s+1$ coeficientes $a_m$, $b_m$. Poderíamos definir tais coeficientes de tal forma a obter ordem máxima.

Conclusão? Poderíamos obter métodos com $s$ estágios e ordem $2s$.

Entretanto tal método (implícito de passo $s$ e ordem $2s$) não é convergente para $s\geq 3$ .

É possível provar que a ordem máxima de convergência para um método de passo múltiplo $s$ é no máximo  $2\lfloor(s+2)/2\rfloor$ para métodos implícitos e $s$ para métodos explícitos. Esta é a \emph{primeira barreira de Dahlquist}.

\section{Estabilidade dos métodos de passo múltiplo}

%Suponha que um método de passo múltiplo \eqref{multistep} seja aplicado a EDO linear \eqref{EDO4.7} obtendo
%\begin{eqnarray}
%  \sum_{m=0}^s a_m  u_{n+m}  &=& h  \sum_{m=0}^s b_m f_{n+m} \\
%  \sum_{m=0}^s a_m  u_{n+m}  &=& h\lambda  \sum_{m=0}^s b_m u_{n+m}
%\end{eqnarray}
%que pode ser escrita como
%\begin{eqnarray}
%  \sum_{m=0}^s (a_m  - h\lambda  b_m) u_{n+m}&=0, \\
%  \sum_{m=0}^s \alpha _m u_{n+m}&=0, \quad  n=0,1,\ldots
%\end{eqnarray}
%Similarmente a equações diferenciais, a equação \`a diferenças possui solução como o polinômio característico
%\begin{eqnarray}
%  \eta (w) :=   \sum_{m=0}^s \alpha _m w^n.
%\end{eqnarray}
%Sejam $w_1,\ldots ,w_q$ as raízes de $\eta (w)$ com multiplicidade $k_1,\ldots ,k_q$, onde $\sum k_i=s$. A solução da equação a diferenças é
%\begin{eqnarray}
%  u_n = \sum_{i=1}^q (\sum_{j=0}^{k_i-1}c_{ij} n^j)w_i^n, \quad  n=0,1,\ldots
%\end{eqnarray}
%onde as constantes $c_{ij}$ são unicamente determinadas pelos valores iniciais $u_0,\ldots ,u_{s-1}$.

\begin{teo}
O método BDF de 2 estágios é A-estável.
\end{teo}

\begin{teo}
[A segunda barreira de Dahlquist] A ordem máxima de um método de passo múltiplo A-estável é dois.
\end{teo}

%%\begin{ex}
%%O método BDF de 2 estágios
%%\begin{eqnarray}
%%  u_{n+2} - \frac{4}{3} u_{n+1} + \frac{1}{3} u_n = \frac{2}{3}h f_{n+2}
%%\end{eqnarray}
%%para o problema de valor inicial $u'=\lambda u$, que possui polinômio característico
%%\begin{eqnarray}
%%  w^2 - \frac{4}{3} w + \frac{1}{3} = \frac{2}{3}h \lambda (w^2)
%%\end{eqnarray}
%%ou
%%\begin{eqnarray}
%%  \eta (w) := (1-z\frac{2}{3}) w^2 - \frac{4}{3} w + \frac{1}{3}.
%%\end{eqnarray}
%%
%%O polinômio $\eta (w)$ possui duas raízes (encontradas com o Maple)
%%\begin{eqnarray}
%%  w_1 &=&\frac{-4+2(1+2z)^{1/2}}{-6+4z},\\
%%  w_2 &=&\frac{-4-2(1+2z)^{1/2}}{-6+4z}
%%\end{eqnarray}
%%
%%Para que o método seja estável, é necessário que $|w_1|<1$ e $|w_2|<1$.
%%
%%Tal condição é satisfeita para a região dentro da curva na Figura~\ref{RegiaoBDF}.
%%
%%\begin{figure}
%%\begin{center}
%%  \includegraphics[width=8cm]{RegiaoBDF.eps}\\
%%  \caption{Região de estabilidade (exterior a curva) para o método BDF $s=2$ }\label{RegiaoBDF}
%%\end{center}
%%\end{figure}
%%\end{ex}

%\begin{ex}
%Plote a região de estabilidade do método de BDF de ordem $2$ e $3$.
%\end{ex}
%
%\begin{ex}
%Plote a região de estabilidade do método de Adams-Bashforth de ordem $2$ e $3$.
%\end{ex}
%
%\begin{ex}
%Plote a região de estabilidade do método de Adams-Moulton de ordem $2$ e $3$.
%\end{ex}
%


%
%
% \section{Métodos de passo múltiplo - Adams-Bashforth}\index{método!de passo múltiplo!Adams-Bashforth}
%
% O método de Adams-Bashforth consiste de um esquema recursivo do tipo:
% $$u^{(n+1)}=u^{(n)}+\sum_{j=0}^k w_jf(u^{(n-j)},t^{(n-j)})$$
%
% \begin{ex} Adams-Bashforth de segunda ordem
% $$u^{(n+1)}=u^{(n)}+\frac{h}{2}\left[3f\left(u^{(n)},t^{(n)}\right)-f\left(u^{(n-1)},t^{(n-1)}\right)\right]$$
% \end{ex}
%
% \begin{ex} Adams-Bashforth de terceira ordem
% $$u^{(n+1)}=u^{(n)}+\frac{h}{12}\left[23f\left(u^{(n)},t^{(n)}\right)-16f\left(u^{(n-1)},t^{(n-1)}\right)+5f\left(u^{(n-2)},t^{(n-2)}\right)\right]$$
% \end{ex}
%
% \begin{ex} Adams-Bashforth de quarta ordem
%   \begin{equation*}
%     \begin{split}
%       u^{(n+1)} &=&u^{(n)} + \frac{h}{24}\left[55f\left(u^{(n)},t^{(n)}\right)-59f\left(u^{(n-1)},t^{(n-1)}\right)\right.\\
%         &+\left. 37f\left(u^{(n-2)},t^{(n-2)}\right)-9f\left(u^{(n-3)},t^{(n-3)}\right)\right]
%     \end{split}
%   \end{equation*}
% \end{ex}
% Os métodos de passo múltiplo evitam os múltiplos estágios do métodos de Runge-Kutta, mas exigem ser "iniciados" com suas condições iniciais.
%
% \section{Métodos de passo múltiplo - Adams-Moulton}\index{método de passo múltiplo!Adams-Moulton}
%
% O método de Adams-Moulton consiste de um esquema recursivo do tipo:
% $$u^{(n+1)}=u^{(n)}+\sum_{j=-1}^k w_jf(u^{(n-j)},t^{(n-j)})$$
%
% \begin{ex} Adams-Moulton de quarta ordem
%   \begin{equation*}
%     \begin{split}
%       u^{(n+1)} &=&u^{(n)} + \frac{h}{24}\left[9f\left(u^{(n+1)},t^{(n+1)}\right) + 19f\left(u^{(n)},t^{(n)}\right) \right.\\
%       &-\left. 5f\left(u^{(n-1)},t^{(n-1)}\right) + f\left(u^{(n-2)},t^{(n-2)}\right)\right]
%     \end{split}
%   \end{equation*}
% \end{ex}
% O método de Adams-Moulton é implícito, ou seja, exige que a cada passo, uma equação em $u^{(n+1)}$ seja resolvida.
%
% \section{Estabilidade}\index{estabilidade}
%
% Consideremos o seguinte problema de teste:
% $$\left\{\begin{array}{rcl}u'&=&-\alpha u\\u(0)&=&1\end{array}\right.$$
% cuja solução exata é dada por $u(t)=e^{-\alpha t}$.
%
% Considere agora o método de Euler aplicado a este problema com passa $h$:
% $$\left\{\begin{array}{rcl}u_{k+1}&=&u_k-\alpha h u_k\\u^{(1)}&=&1\end{array}\right.$$
% A solução exata do esquema de Euler é dada por
% $$u_{k+1}=(1-\alpha h)^{k}$$
% e, portanto,
% $$\tilde{u}(t)=u_{k+1}=(1-\alpha h)^{t/h}$$
%
% Fixamos um $\alpha>0$, de forma que $u(t)\to 0$. Mas observamos que $\tilde{u}(t)\to 0$ somente quando $|1-\alpha h|<1$ e solução positivas somente quando $\alpha h<1$.
%
% {\bf Conclusão:} Se o passo $h$ for muito grande, o método pode se tornar instável, produzindo solução espúrias.
%
%




\section{Métodos de Runge-Kutta}

\subsection{Método de Runge-Kutta explícito}
Seja a EDO
\begin{eqnarray}
  u'(t) &=&f(t,u(t)) \\
  u(t_0) &=&a
\end{eqnarray}

Integrando a EDO em $[t_n,t^{(n+1)}]$ obtemos
\begin{eqnarray}
  u_{n+1}  &=&u_n  + \int _{t_n}^{t^{(n+1)}} f(t,u(t)) \; dt
\end{eqnarray}
Um método de passo simples utiliza $f_n$ e $f_{n+1}$. Queremos aumentar a ordem do método, porém utilizando somente valores de $f$ entre $[t_n,t^{(n+1)}]$.

\begin{center}
\begin{tabular}{ccccc}
  $u_n$ &      &       &      & $u_{n+1}$ \\
  $|$   &      &       &      &  $|$  \\ \hline
%  $-+-$ & $--$ &  $--$ & $--$ & $-+-$ \\
  $t_n$ &      &       &      & $t^{(n+1)}$ \\
  $\tau _1$  & $\tau _2$ & $\cdots $ &      & $\tau _\nu $
\end{tabular}
\end{center}

\begin{eqnarray}
  u_{n+1}  &=&u_n  + \int _{t_n}^{t^{(n+1)}} f(t,u(t)) \; dt \\
           &=&u_n  + h\int _0^1f(t_n+h\tau ,u(t_n+h\tau )) \; d\tau  \\
           &=&u_n  + h\sum_{j=1}^\nu  b_jf(t_n+c_jh,u(t_n+c_jh))
\end{eqnarray}


Por exemplo, se $\nu = 3$ estágios, teremos $[\tau _0, \tau _1, \tau _2]=[t_n+c_0h, t_n+c_1h, t_n+c_2h]$, $U_j\equiv u(\tau _j)$ e $F_j\equiv f(\tau _j, U_j)$, $j=1, 2, 3$. Inicie com  $U_1=u_n$ ($c_1=0$) como a solução no passo anterior e aproxime $U_2, U_3$, com uma combinação linear dos valores de $F_j$ anteriores, ou seja:
\begin{eqnarray}\label{RKa}
  U_1 &=&u_n \\
  U_2 &=&u_n  + h a_{21}F_1 \\
  U_3 &=&u_n  + h a_{31}F_1 +h a_{32}F_2 \\ \label{RK}
  u_{n+1}&=&u_n  + h [ b_1F_1+b_2F_2+b_3F_3],
\end{eqnarray}
 onde $A=(a_{ij})$ é a matriz de RK (triangular inferior com diagonal zero), $b_j$ são os pesos RK e $c_j$ são os nós RK.

Os coeficientes podem ser resumidos em uma tabela na forma

\begin{center}
\begin{tabular}{c|c}
  \begin{tabular}{c}\\$c$\\~\\ \end{tabular} &\begin{tabular}{ccc}& $A$&\end{tabular}     \\  \hline
      & $b^T$
\end{tabular}
$=$
\begin{tabular}{c|ccc}
  $c_1$  & $0$      & $0$      & $0$ \\
  $c_2$  & $a_{21}$ & $0$      & $0$ \\
  $c_3$ & $a_{31}$ & $a_{22}$ & $0$\\  \hline
        & $b_1$     & $b_2$     & $b_3$
\end{tabular}
\end{center}

\subsection{Método de RK $\nu =2$}
Assumindo suavidade suficiente em $f$, usamos o polinônio de Taylor:
\begin{eqnarray}
F_2 &=&f(t_n+c_2h,U_2)\\
   &=&f(t_n+c_2h,u_n+ a_{21}h f_n)\\
   &=&f_n +h[c_2 \frac{\partial f_n}{\partial t}+ a_{21} \frac{\partial f_n}{\partial u} f_n]+O(h^2)
\end{eqnarray}
fazendo com que \eqref{RK} se torne
\begin{eqnarray}
  u_{n+1}&=&u_n  + h [ b_1 F_1+b_2F_2] \\
         &=&u_n  + h(b_1+b_2)f_n +h^2b_2[c_2 \frac{\partial f_n}{\partial t}+ a_{21} \frac{\partial f_n}{\partial u} f_n]+O(h^3) \label{3_6}
\end{eqnarray}
Usando a EDO e derivando-a obtemos
\begin{eqnarray}
  u_t    &=&f(t,u)\\
  u_{tt} &=&f_t+f_uu_t = f_t+f_uf
\end{eqnarray}
e expandindo em série de Taylor a solução exata em $t^{(n+1)}$,
\begin{eqnarray}
  u(t^{(n+1)})&=&u_n  + hu_t +\frac{h^2}{2}u_{tt} + O(h^3)\\
            &=&u_n  + hf_n +\frac{h^2}{2}[f_t+f_uf]+O(h^3)
\end{eqnarray}
e comparando com \eqref{3_6} obtemos as condições para ordem $p\geq 2$,
\begin{eqnarray}
  b_1+b_2=1, \quad b_2c_2 = \frac{1}{2} \quad a_{21}=c_2
\end{eqnarray}

O sistema possui mais de uma solução. Algumas escolhas comuns são
\begin{center}
\begin{tabular}{c|cc}
  $0$ &     &   \\
  $\frac{1}{2}$ & $\frac{1}{2}$ &   \\  \hline
      & $0$ & $1$
\end{tabular},~~~~
\begin{tabular}{c|cc}
  $0$ &   &   \\
  $\frac{2}{3}$ & $\frac{2}{3}$ &   \\  \hline
    & $\frac{1}{4}$ & $\frac{3}{4}$
\end{tabular} ~~e~~
\begin{tabular}{c|cc}
  $0$ &   &   \\
  $1$ & $1$ &   \\  \hline
      & $\frac{1}{2}$ &$\frac{1}{2}$
\end{tabular}
\end{center}
onde a última tabela fornece o método de Heun (ou Euler modificado):
\begin{eqnarray}
  U_1 &=&u_n \\
  U_2 &=&u_n  + h F_1 \\
  u_{n+1}&=&u_n  + h [ \frac{1}{2} F_1+\frac{1}{2}F_2]
\end{eqnarray}
Note que o método é de ordem $p=2$ pois os termos que sobraram são de $O(h^3)$.


Seguindo um procedimento similar, podemos obter as condições para um método com $\nu =3$ e ordem $p=3$, que são
\begin{eqnarray}
  b_1+b_2+b_3=1,               & b_2c_2+b_3c_3 = \frac{1}{2} \\
  b_2c_2^2+b_3c_3^2=\frac{1}{3}, & b_3a_{32}c_2=\frac{1}{6}
\end{eqnarray}

Alguns exemplos de métodos de RK de 3 estágios são o método clássico de Runge-Kutta:
\begin{center}
\begin{tabular}{c|ccc}
  $0$ &     &   & \\
  $\frac{1}{2}$ & $\frac{1}{2}$ &   & \\
  $1$ & $-1$ &$2$& \\  \hline
      & $\frac{1}{6}$ &$\frac{4}{6}$& $\frac{1}{6}$
\end{tabular}
\end{center}
e o método de Nyström:
\begin{center}
\begin{tabular}{c|ccc}
  $0$ &     &   & \\
  $\frac{2}{3}$ & $\frac{2}{3}$ &   & \\
  $\frac{2}{3}$ & $0$ &$\frac{2}{3}$& \\  \hline
      & $\frac{2}{8}$ &$\frac{3}{8}$& $\frac{3}{8}$
\end{tabular}
\end{center}

Com paciência e a ajuda de um software de manipulação algébrica simbólica,  é possível encontrar um método de quarta ordem e $\nu =4$ estágios como:
\begin{center}
\begin{tabular}{c|cccc}
  $0$ &     &   &   &    \\
  $\frac{1}{2}$ & $\frac{1}{2}$ &   &   &    \\
  $\frac{1}{2}$ & $0$ &$\frac{1}{2}$&   &    \\
  $1$ & $0$ &$0$&$1$&    \\  \hline
      & $\frac{1}{6}$ &$\frac{2}{6}$& $\frac{2}{6}$& $\frac{1}{6}$
\end{tabular}
\end{center}


\subsection{Método de Runge-Kutta implícito (IRK)}
No conjunto de equações \eqref{RKa}-\eqref{RK}, $U_k$ depende em valores conhecidos $F_1,\ldots ,F_{k-1}$ tornando o método explícito.

Entretanto se $U_k$ depender de $F_1,\ldots ,F_\nu $ temos um método implícito como
\begin{eqnarray}\label{IRK}
  U_j &=&u_n  + h \sum_{i=1}^\nu  a_{ji} F_i, \quad  j=1,\ldots ,\nu \\
  u_{n+1}&=&u_n  + h \sum_{i=1}^{\nu } b_i F_i
\end{eqnarray}
onde $A=(a_{ij})$ é a matriz de RK. É necessário que
\begin{eqnarray}
 \sum_{i=1}^{\nu } a_{ji} = c_j, \quad \quad  j=1,\ldots ,\nu
\end{eqnarray}
para que o método possua ordem $p\geq 1$.


\begin{ex}
Um método de Runge-Kutta Implícito (IRK) de dois estágios é dado por
\begin{eqnarray}
  U_1 &=&u_n  + h/4  [ f(t_n,U_1) - f(t_n+\frac{2}{3}h,U_2)]\\
  U_2 &=&u_n  + h/12 [3f(t_n,U_1) +5f(t_n+\frac{2}{3}h,U_2)]\\
  u_{n+1}&=&u_n  + h/4 [f(t_n,U_1) +3f(t_n+\frac{2}{3}h,U_2)]
\end{eqnarray}
que possui uma tabela como
\begin{center}
\begin{tabular}{c|cc}
  $0$ & $\frac{1}{4}$ &$-\frac{1}{4}$  \\
  $\frac{2}{3}$ & $\frac{1}{4}$ &$\frac{5}{12}$  \\  \hline
      & $\frac{1}{4}$ &$\frac{3}{4}$
\end{tabular}
\end{center}
\end{ex}


% \section{Métodos de Runge-Kutta}\label{sec_RK}\index{método!de Runge-Kutta}
%
% Os métodos de Runge-Kutta consistem em iterações do tipo:
% $$u_{k+1}=u_k+w_1 k_1 + \ldots + w_n k_n$$
% onde
% \begin{eqnarray*}
% k_1&=&hf(u_k,t_k)\\
% k_2&=&hf(u_k+\alpha_{2,1}k_1,t_k+\beta_{2}h)\\
% k_3&=&hf(u_k+\alpha_{3,1}k_1+\alpha_{3,2}k_2,t_k+\beta_{3}h)\\
% &\vdots&\\
% k_n&=&hf(u_k+\alpha_{n,1}k_1+\alpha_{n,2}k_2+\ldots \alpha_{n,n-1}k_{n-1},t_k+\beta_{n}h)\\
% \end{eqnarray*}
%
% Os coeficientes são escolhidos de forma que a expansão em Taylor de $u_{k+1}$ e $u_k+w_1 k_1 + \ldots + w_n k_n$ coincidam até ordem $n+1$.
%
% \begin{ex} O método de Euler melhorado é um exemplo de Runge-Kutta de segunda ordem
% $$u^{(n+1)}=u^{(n)}+\frac{k_1+k_2}{2}$$
% onde $k_1=hf(u^{(n)},t^{(n)})$ e $k_2=hf(u^{(n)}+k_1,t^{(n)}+h)$.
% \end{ex}
%
% \subsection{Métodos de Runge-Kutta - quarta ordem}\index{método!de Runge-Kutta!de quarta ordem}
%
% $$u^{(n+1)}=u^{(n)}+\frac{k_1+2k_2+2k_3+k_4}{6}$$
% onde
% \begin{eqnarray*}
% k_1&=&hf(u^{(n)},t^{(n)})\\
% k_2&=&hf(u^{(n)}+k_1/2,t^{(n)}+h/2)\\
% k_3&=&hf(u^{(n)}+k_2/2,t^{(n)}+h/2)\\
% k_4&=&hf(u^{(n)}+k_3,t^{(n)}+h)\\
% \end{eqnarray*}
% Este método tem ordem de precisão 4. Uma discussão heurística usando método de Simpson pode ajudar a compreender os estranhos coeficientes:
% \begin{eqnarray*}
% u({t^{(n+1)}})-u({t^{(n)}})&=&\int_{t^{(n)}}^{t^{(n+1)}}f(t,u(s),s)ds \\
% &\approx& \frac{h}{6}\left[ f\left(u(t^{(n)}),t^{(n)}\right)+4f\left(u(t^{(n)}+h/2),t^{(n)}+h/2\right)\right.\\
% &+&\left.f\left(u(t^{(n)}+h),t^{(n)}+h\right)\right]\\
% &\approx& \frac{k_1+4(\frac{k_2+k_3}{2})+k_4}{6}
% \end{eqnarray*}
% onde $k_1$ e $k_4$ representam as inclinações nos extremos e $k_2$ e $k_3$ são duas aproximações diferentes para a inclinação no meio do intervalo.


\section{Estimativa da ordem de convergência}

Raramente temos a solução exata $u(t)$ para calcular o erro obtido na solução numérica. Entretanto, se a solução é suave o suficiente e o espaçamento $h$ é pequeno suficientemente, podemos usar o seguinte procedimento para estimar a ordem do método (ou ainda, o erro na solução).

Como visto nos exemplo numéricos anteriores, em gráficos na escala \verb#loglog#, se $h$ é grande não obtemos a ordem de convergência utilizada (por exemplo, encontramos que o método de Euler possui ordem $p\approx 0,7$ onde deveria ser $1$). A medida que $h$ decresce se aproximando de $0$, a ordem de convergência tende a se aproximar de $p\approx 1$. (Entretanto $h$ não pode ficar muito pequeno a ponto que as operações de ponto flutuante atrapalhem na convergência).

Portanto existe uma faixa $h_{min} < h < h_{max}$ onde o método apresenta a ordem desejada. Essa região depende do método e do problema de valor inicial estudado.

Mas se estivermos nessa região podemos aproximar a ordem do método da seguinte forma: Considere a solução para um determinado $t=T^*$ fixo, $u(T^*)$. Considere também as aproximações das soluções obtidas com espaçamento $h$, denotada por $u^{h}$; a aproximação obtida com espaçamento dividido por $2$, $h/2$, denotada por $u^{h/2}$; a aproximação obtida com espaçamento $h/4$, denotada por $u^{h/4}$, $\ldots $ e assim por diante, todas calculadas em $t=T^*$.

\subsection{Método 1}
Podemos utilizar uma solução bem refinada, por exemplo, $u^{h/16}$ como sendo uma boa aproximação da solução exata e supormos que $u^*=u^{h/16}$. Desta forma podemos aproximar o erro por $e^{h}=\|u^{(h)}-u^*\|$ e a ordem do método é estimada como
\begin{eqnarray}
  p  & \approx  \frac{ \log(e^{h})-\log(e^{h/2})}{\log(h)-\log(h/2)} \\
     & \approx  \frac{ \log \left(   \frac{e^{h}}{e^{h/2}} \right)  }{\log(h /(h/2))} \\
     & \approx  \frac{ \log \left(   \frac{e^{h}}{e^{h/2}} \right)  }{\log(2)} \\
     & \approx  \frac{ \log \left(   \frac{\|u^{h}-u^*\|}{\|u^{h/2}-u^*\|} \right)  }{\log(2)} \\
\end{eqnarray}

\subsection{Método 2}
Segundo Ferziger/Peric/Roache, podemos também estimar $p$ diretamente de
\begin{eqnarray}
  p  & \approx  \frac{ \log \left(   \frac{\|u^{h/2}-u^{h}\|}{\|u^{h/4}-u^{h/2}\|} \right)  }{\log(2)} \\
\end{eqnarray}




%\chapter{Método de Colocação}
%Páginas 42-47 de \cite{iserles2009first}. %Todo método de colocação é um método de RK, mas a recíproca não é verdadeira.









\subsection*{Exercícios}

\begin{exer} Resolva o problema 1 pelos diversos métodos e verifique heuristicamente a estabilidade para diversos valores de $h$.
\end{exer}











\section{Sistemas de equações diferenciais e equações de ordem superior}
O Problema~\eqref{PVI} pode ser um sistema de equações de primeira ordem, isto é, a incógnita $y(t)$ pode ser um vetor de funções, como mostra o Exemplo~\ref{eq_exemplo_sistema}
\begin{ex}\label{exemplo_sistema_PVI}O problema de valor inicial
\begin{subequations}\label{eq_exemplo_sistema}
\begin{eqnarray}
u'(t)&=&v(t),\\
v'(t)&=&u(t),\\
u(0)&=&1.\\
v(0)&=&0.
\end{eqnarray}
\end{subequations}
pode ser escrito na forma \eqref{PVI} com $y(t)$
\end{ex}

No Exemplo~\ref{sys_edo}, mostramos como o método de Euler pode ser facilmente estendido para problemas envolvendo sistemas de equações diferenciais.\index{sistemas!de equações diferenciais}.
\begin{ex}\label{sys_edo} Escreva o processo iterativo de Euler para resolver numericamente o seguinte sistema de equações diferenciais
\begin{eqnarray*}
x'&=&-y\\
y'&=&x\\
x(0)&=&1\\
y(0)&=&0,\\
\end{eqnarray*}
cuja solução exata é $x(t)=\cos(t)$ e $y(t)=\sin(t)$.
\end{ex}
Para aplicar o método de Euler a um sistema, devemos encarar as diversas incógnitas do sistema como formando um vetor, neste caso, escrevemos:
 $$z(t)=\left[\begin{array}{c}x(t)\\y(t)\end{array}\right].$$
 O sistema é igualmente escrito na forma vetorial:
\begin{eqnarray*}
\left[\begin{array}{c}x^{(k+1)}\\y^{(k+1)}\end{array}\right]=\left[\begin{array}{c}x^{(k)}\\y^{(k)}\end{array}\right]+h\left[\begin{array}{c}-y^{(k)}\\x^{(k)}\end{array}\right].
\end{eqnarray*}
Observe que este processo iterativo é equivalente a:
\begin{eqnarray*}
x^{(k+1)}&=&x^{(k)}-hy^{(k)}\\
y^{(k+1)}&=&y^{(k)}+hx^{(k)}.
\end{eqnarray*}


\begin{ex} Escreva o problema de valor inicial de segunda ordem dado por
\begin{eqnarray*}
y''+y'+y&=&\cos(t),\\
y(0)&=&1,\\
y'(0)&=&0,
\end{eqnarray*}
como um problema envolvendo um sistema de primeira ordem.
\end{ex}
A fim de transformar a equação diferencial dada em um sistema de equações de primeira ordem, introduzimos a substituição $w=y'$, de forma que obteremos o sistema:
\begin{eqnarray*}
y'&=&w\\
w'&=&-w-y+\cos(t)\\
y(0)&=&1\\
w(0)&=&0
\end{eqnarray*}
Portanto, o método de Euler produz o seguinte processo iterativo:
\begin{eqnarray*}
y^{(k+1)}&=&y^{(k)}+hw^{(k)},\\
w^{(k+1)}&=&w^{(k)}-hw^{(k)}-hy^{(k)}+h\cos(t^{(k)}),\\
y^{(1)}&=&1,\\
w^{(1)}&=&0.
\end{eqnarray*}


\subsection*{Exercícios}

\begin{exer}Resolva o problema de valor inicial dado por
\begin{eqnarray*}
y'&=& -2y + \sqrt{z}\\
z'&=& -z + y\\
y(0)&=&0\\
z(0)&=&2\\
\end{eqnarray*}
com passo $h=0,2$, $h=0,02$, $h=0,002$ e $h=0,0002$ para obter aproximações para $y(2)$ e $z(2)$.
\end{exer}


\section{Exercícios finais}

\begin{exer} Considere o seguinte modelo para o crescimento de uma colônia de bactérias:
$$\frac{du}{dt}=\alpha u (A-u)$$
onde $u$ indica a densidade de bactérias em unidades arbitrárias na colônia e $\alpha$ e $A$ são constantes positivas.
Pergunta-se:
\begin{itemize}
\item[a)] Qual a solução quando a condição inicial $u(0)$ é igual a $0$ ou $A$?
\item[b)] O que acontece quando a condição inicial $u(0)$ é um número entre $0$ e $A$?
\item[c)] O que acontece quando a condição inicial $u(0)$ é um número negativo?
\item[d)] O que acontece quando a condição inicial $u(0)$ é um número positivo maior que A?
\item[e)] Se $A=10$ e $\alpha=1$ e $u(0)=1$, use métodos numéricos para obter tempo necessário para que a população dobre?
\item[f)] Se $A=10$ e $\alpha=1$ e $u(0)=4$, use métodos numéricos para obter tempo necessário para que a população dobre?
\end{itemize}
\end{exer}
\begin{resp}

Os valores exatos para os itens e e f são:$\frac{1}{10}\ln\left(\frac{9}{4}\right)$ e $\frac{1}{10}\ln\left(6\right)$

\end{resp}

\begin{exer} Considere o seguinte modelo para a evolução da velocidade de um objeto em queda (unidades no SI):
$$v'=g-\alpha v^2$$
Sabendo que $g=9,8$ e $\alpha=10^{-2}$ e $v(0)=0$. Pede-se a velocidade ao tocar o solo, sabendo que a altura inicial era 100.

\end{exer}
\begin{resp}

O valor exato é $\sqrt{\frac{g}{\alpha}\left[1-e^{{-200\alpha}}\right]}$ em $t=\frac{1}{\sqrt{g\alpha}}\tanh^{-1}\left(\sqrt{1-e^{{-200\alpha}}}\right)$

\end{resp}


\begin{exer} Considere o seguinte modelo para o oscilador não linear de Van der Pol:
$$u''(t) - \alpha (A-u(t)^2)u'(t) + w_0^2u(t)=0$$
onde $A$, $\alpha$ e $w_0$ são constantes positivas.
\begin{itemize}
\item Encontre a frequência e a amplitude de oscilações quando $w_0=1$, $\alpha=.1$ e $A=10$. (Teste diversas condições iniciais)
\item Estude a dependência da frequência e da amplitude com os parâmetros  $A$, $\alpha$ e $w_0$. (Teste diversas condições iniciais)
\item Que diferenças existem entre esse oscilador não linear e o oscilador linear?
\end{itemize}
\end{exer}

\begin{exer} Considere o seguinte modelo para um oscilador não linear:
\begin{eqnarray*}
u''(t)-\alpha(A-z(t))u'(t)+w_0^2 u(t)&=&0\\
Cz'(t)+z(t)&=&u(t)^2
\end{eqnarray*}
onde $A$, $\alpha$, $w_0$ e $C$ são constantes positivas.
\begin{itemize}
\item Encontre a frequência e a amplitude de oscilações quando $w_0=1$, $\alpha=.1$, $A=10$ e $C=10$. (Teste diversas condições iniciais)
\item Estude a dependência da frequência e da amplitude com os parâmetros  $A$, $\alpha$, $w_0$ e $C$. (Teste diversas condições iniciais)
\end{itemize}
\end{exer}

\begin{exer} Considere o seguinte modelo para o controle de temperatura em um processo químico:
\begin{eqnarray*}
CT'(t)+T(t)&=&\kappa P(t)+T_{ext}\\
P'(t)&=&\alpha(T_{set}-T(t))
\end{eqnarray*}
onde $C$, $\alpha$ e $\kappa$ são constantes positivas e $P(t)$ indica o potência do aquecedor. Sabendo que $T_{set}$ é a temperatura desejada, interprete o funcionamento esse sistema de controle.
\begin{itemize}
\item Calcule a solução quando a temperatura externa $T_{ext}=0$, $T_{set}=1000$, $C=10$, $\kappa=.1$ e $\alpha=.1$. Considere condições iniciais nulas.
\item Quanto tempo demora o sistema para atingir a temperatura 900K?
\item Refaça os dois primeiros itens com $\alpha=0,2$ e $\alpha=1$
\item Faça testes para verificar a influência de $T_{ext}$, $\alpha$ e $\kappa$ na temperatura final.
\end{itemize}
\end{exer}

\begin{exer} Considere a equação do pêndulo dada por:
$$\frac{d^2\theta(t)}{dt^2}+\frac{g}{l}\sin(\theta(t))=0$$
onde $g$ é o módulo da aceleração da gravidade e $l$ é o comprimento da haste.
\begin{itemize}
\item Mostre analiticamente que a energia total do sistema dada por
$$\frac{1}{2}\left(\frac{d\theta(t)}{dt}\right)^2-\frac{g}{l}\cos(\theta(t))$$
é mantida constante.
\item Resolva numericamente esta equação para $g=9,8m/s^2$ e $l=1m$ e as seguintes condições iniciais:
\subitem $\theta(0)=0,5$ e $\theta'(0)=0$.
\subitem $\theta(0)=1,0$ e $\theta'(0)=0$.
\subitem $\theta(0)=1,5$ e $\theta'(0)=0$.
\subitem $\theta(0)=2,0$ e $\theta'(0)=0$.
\subitem $\theta(0)=2,5$ e $\theta'(0)=0$.
\subitem $\theta(0)=3,0$ e $\theta'(0)=0$.
\end{itemize}
Em todos os casos, verifique se o método numérico reproduz a lei de conservação de energia e calcule período e amplitude.
\end{exer}

\begin{exer} Considere o modelo simplificado de FitzHugh-Nagumo para o potencial elétrico sobre a membrana de um neurônio:
\begin{eqnarray*}
\frac{d V}{dt}& = &  V-V^3/3 - W +  I  \\
\frac{d W}{dt} & = & 0,08(V+0,7 - 0,8W)
\end{eqnarray*}
onde $I$ é a corrente de excitação.
\begin{itemize}
\item Encontre o único estado estacionário $\left(V_0,W_0\right)$ com $I=0$.
\item Resolva numericamente o sistema com condições iniciais dadas por $\left(V_0,W_0\right)$ e
\subitem $I=0$
\subitem $I=0,2$
\subitem $I=0,4$
\subitem $I=0,8$
\subitem $I=e^{-t/200}$
\end{itemize}
\end{exer}


\begin{exer} Considere o problema de valor inicial dado por
\begin{eqnarray*}
\frac{d u(t)}{dt} &=& -u(t) + e^{-t} \\
u(0)&=&0
\end{eqnarray*}
Resolva analiticamente este problema usando as técnicas elementares de equações diferenciais ordinárias. A seguir encontre aproximações numéricas usando os métodos de Euler, Euler modificado, Runge-Kutta clássico e Adams-Bashforth de ordem 4 conforme pedido nos itens.
\begin{itemize}
\item[a)]  Construa uma tabela apresentando valores com 7 algarismos significativos para comparar a solução analítica com as aproximações numéricas produzidas pelos métodos sugeridos. Construa também uma tabela para o erro absoluto obtido por cada método numérico em relação à solução analítica. Nesta última tabela, expresse o erro com 2 algarismos significativos em formato científico. Dica: $format('e',8)$ para a segunda tabela.
\begin{center}
\begin{tabular}{|c|c|c|c|c|c|}
\hline
&0,5&1,0&1,5&2,0&2,5\\
\hline
Analítico&&&&&\\
\hline
Euler&&&&&\\
\hline
Euler modificado&&&&&\\
\hline
Runge-Kutta clássico&&&&&\\
\hline
Adams-Bashforth ordem 4&&&&&\\
\hline
\end{tabular}
\end{center}

\begin{center}
\begin{tabular}{|c|c|c|c|c|c|}
\hline
&0,5&1,0&1,5&2,0&2,5\\
\hline
Euler&&&&&\\
\hline
Euler modificado&&&&&\\
\hline
Runge-Kutta clássico&&&&&\\
\hline
Adams-Bashforth ordem 4&&&&&\\
\hline
\end{tabular}
\end{center}

\item[b)] Calcule o valor produzido por cada um desses método para $u(1)$ com passo $h=0,1$, $h=0,05$, $h=0,01$, $h=0,005$ e $h=0,001$. Complete a tabela com os valores para o erro absoluto encontrado.
\begin{center}
\begin{tabular}{|c|c|c|c|c|c|}
\hline
&0,1&0,05&0,01&0,005&0,001\\
\hline
Euler&&&&&\\
\hline
Euler modificado&&&&&   \\
\hline
Runge-Kutta clássico&&&&&\\
\hline
Adams-Bashforth ordem 4&&&&&\\
\hline
\end{tabular}
\end{center}

\end{itemize}

\end{exer}


\begin{resp}

\begin{center}
\begin{tabular}{|c|c|c|c|c|c|}
\hline
&0,5&1,0&1,5&2,0&2,5\\
\hline
Analítico&  0,3032653 &   0,3678794  &  0,3346952  &  0,2706706 &   0,2052125  \\
\hline
Euler& 0,3315955 &   0,3969266 &   0,3563684 &   0,2844209  &  0,2128243\\
\hline
Euler modificado &0,3025634 &   0,3671929 &   0,3342207 &   0,2704083  &  0,2051058 \\
\hline
Runge-Kutta clássico& 0,3032649  &  0,3678790  &  0,3346949  &  0,2706703  &  0,2052124\\
\hline
Adams-Bashforth ordem 4& 0,3032421  &  0,3678319 &   0,3346486  &  0,2706329  &  0,2051848  \\
\hline
\end{tabular}
\end{center}


\begin{center}
\begin{tabular}{|c|c|c|c|c|c|}
\hline
&0,5&1,0&1,5&2,0&2,5\\
\hline
Euler& 2,8e-2  &  2,9e-2  &  2,2e-2  &  1,4e-2 &   7,6e-3\\
\hline
Euler modificado& 7,0e-4  &  6,9e-4   & 4,7e-4 &   2,6e-4 &   1,1e-4\\
\hline
Runge-Kutta clássico& 4,6e-7 &   4,7e-7    &3,5e-7  &  2,2e-7 &   1,2e-7\\
\hline
Adams-Bashforth ordem 4&  2,3e-5 &   4,8e-5  &  4,7e-5  &  3,8e-5  &  2,8e-5 \\
\hline
\end{tabular}
\end{center}

\begin{center}
\begin{tabular}{|c|c|c|c|c|c|}
\hline
&0,1&0,05&0,01&0,005&0,001\\
\hline
Euler&2,9e-2  &  5,6e-3 &   2,8e-3 &   5,5e-4 &   2,8e-4\\
\hline
Euler modificado&6,9e-4 &   2,5e-5  &  6,2e-6 &   2,5e-7 &   6,1e-8   \\
\hline
Runge-Kutta clássico& 4,7e-7 &   6,9e-10 &   4,3e-11   & 6,8e-14  &  4,4e-15\\
\hline
Adams-Bashforth ordem 4&4,8e-5 &   9,0e-8 &   5,7e-9 &   9,2e-12 &   5,8e-13  \\
\hline
\end{tabular}
\end{center}

\end{resp}

%\end{document}


\chapter{Limbo}
\section{Teoria de equações diferenciais}
Uma questão fundamental é analisar se um dado problema de valor inicial é um problema \emph{bem posto}. Ou seja,
\begin{itemize}
 \item Existe uma solução para o problema de valor inicial?
 \item A solução é única?
 \item A solução do problema de valor inicial é pouco sensível a pequenas perturbações nas condições iniciais?
\end{itemize}


\begin{defn}
A função $f(t, u)$ é Lipschitz em $u$ se existe uma constante $L$, tal que $\forall t \in [a, b]$ e $u, v \in \mathbb R$,
$$ |f(t, u)-f(t, v)| \leq L|u(t)-v(t)|. $$
\end{defn}


\begin{teo}
Seja $f(t, u)$ contínua em $t$ e Lipschitz em $u$. Então existe uma única solução para o problema de valor inicial
\begin{eqnarray}
  u'(t)  &=& f(t, u(t)) \\
  u(t^{(1)}) &=& a.
\end{eqnarray}
\end{teo}

\begin{defn}
  \emph{Estabilidade dinâmica} refere-se a propriedade de pequenas perturbações sobre o estado inicial de um sistema gerarem pequenas variações no estado final deste sistema (haverá decaimento nas variações, ou pelo menos não crescimento, quanto $t$ cresce).
\end{defn}

\begin{teo}[Dependência na condição inicial]
Se $u(t)$ e $v(t)$ são soluções do problema de valor inicial com $f$ Lipschitz com $u(t^{(1)})=u^{(1)}$, $v(t^{(1)})=v_1$, então
$$ |u(t)-v(t)| \leq  e^{L(t-t^{(1)})}|u^{(1)}-v_1| . $$
\end{teo}

\subsection*{Exercícios resolvidos}

\begin{exeresol} A função $f(t,u)=\sqrt{u},~~u\geq 0$ não é uma função Lipschitz em $u$, pois
$$\lim_{u\to 0+} \frac{|f(t,u)-f(t,0)|}{|u-0|}=\lim_{u\to 0+} \frac{\sqrt{u}}{u}=\lim_{u\to 0+} \frac{1}{\sqrt{u}}=\infty$$
 Mostre que o seguinte problema de valor inicial não admite solução única:
\begin{eqnarray}
   \frac{du}{dt} &=&\sqrt{u},~~u>0,\\
            u(0) &=& 0.
\end{eqnarray}
\end{exeresol}
\begin{resol}
 A função identicamente nula, $u(t)=0$, satisfaz a equação diferencial e a condição de contorno, logo é uma solução do problema de valor inicial. No entanto, a função\footnote{Esta solução pode ser obtida por separação de variáveis.} $u(t)=\frac{t^2}{4}$ satisfaz a condição inicial, pois $u(0)=0$ e a equação diferencial pois $\frac{du}{dt}=\frac{t}{2}=\sqrt{\frac{t^2}{4}}$.
 
 De fato, qualquer função do tipo
 $$u(t)=\left\{
 \begin{array}{ll}
  0,&0\leq t \leq t_0\\
  \frac{(t-t_0)^2}{4},& t >t_0\\
 \end{array}
 \right.$$
 é solução do problema de valor inicial dado.
\end{resol}


\emconstrucao

\subsection*{Exercícios}

\emconstrucao
