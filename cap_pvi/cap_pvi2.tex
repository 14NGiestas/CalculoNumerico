%Este está licenciado sob a Licença Creative Commons Atribuição-CompartilhaIgual 3.0 Não Adaptada. Para ver uma cópia desta licença, visite https://creativecommons.org/licenses/by-sa/3.0/ ou envie uma carta para Creative Commons, PO Box 1866, Mountain View, CA 94042, USA.

%\documentclass[main.tex]{subfiles}
%\begin{document}



\emconstrucao

\subsection*{Exercícios}

\emconstrucao




\chapter{Em reestruturação}

Pode ser mostrado que o erro de truncamento local é
\begin{equation} ETL_{EulImp}^{n+1}= \mathcal{O}(h^2). \end{equation}
portanto o método é de ordem $1$. E o erro de truncamento global é
\begin{equation} ETG_{EulImp}^{n+1}= \mathcal{O}(h). \end{equation}



\begin{ex}
Utilizando o \textbf{método de Euler implícito} para solucionar \eqref{EDO4.7} obtemos
\begin{eqnarray}
 u_{n+1}      &=& u^{(n)}+h\lambda u_{n+1}, \\
 (1-h\lambda )u_{n+1} & =& u^{(n)}, \\
       u_{n+1} & =& \left(\frac{1}{1- h\lambda }\right)u^{(n)}, \\
       u_{n+1} & =& \left(\frac{1}{1- h\lambda }\right)^2u_{n-1}, \\
       u_{n+1} & =& \left(\frac{1}{1- h\lambda }\right)^{n+1}, \quad  n=0,1,\ldots
\end{eqnarray}
onde $u^{(1)}=1$.
Concluímos então que
\begin{eqnarray}
 \mathcal D_{EulImp} = \{z \in  \field{C}:  \left|\frac{1}{1- z}\right|<1\}
\end{eqnarray}
ou ainda,
\begin{eqnarray}
 \mathcal D_{EulImp} = \{z \in  \field{C}:  |1- z|>1\}
\end{eqnarray}


\end{ex}




































\subsection{Ordem de precisão}




A \emph{precisão} de um método numérico que aproxima a solução de um problema de valor inicial é dada pela ordem do erro acumulado ao calcular a aproximação em um ponto $t^{(n+1)}$ em função do espaçamento da malha $h$.

Se $u(t^{(n+1)})$ for aproximado por $u_{n+1}$ com erro da ordem $O(h^{p+1})$ dizemos que o método tem \textbf{ordem de precisão $p$}\index{método!de Euler!ordem de precisão}.


Queremos obter a ordem de precisão do método de Euler. Para isso, substituímos a EDO $u'=f(t,u)$ na expansão em série de Taylor
\begin{eqnarray}\label{taylor}
   u(t^{(n+1)})=u(t^{(n)})+hu'(t^{(n)})+h^2u''(t^{(n)})/2+ \mathcal O(h^3)
\end{eqnarray}
e obtemos
\begin{eqnarray}\label{tayloreuler}
 u(t^{(n+1)})=u(t^{(n)})+hf(t^{(n)},u(t^{(n)}))+h^2u''(t^{(n)})/2+ \mathcal O(h^3)
\end{eqnarray}
Subtraindo \eqref{tayloreuler} do método de Euler
\begin{eqnarray}
    u_{n+1}=u^{(n)} + h\;f(t^{(n)},u^{(n)})
\end{eqnarray}
obtemos
\begin{eqnarray}
   e_{n+1}   &=& u_{n+1}-u(t^{(n+1)}) \\
             &=&u^{(n)} - u(t^{(n)})  +h(f(t^{(n)},u(t^{(n)})+e_n)- f(t^{(n)},u(t^{(n)}))) +\\
             &+&\frac{h^2}{2}u''_n+\mathcal O(h^3)
\end{eqnarray}
Defina o \emph{erro numérico} como $e_n=u^{(n)}-u(t^{(n)})$ onde $u(t^{(n)})$ é a solução exata e $u^{(n)}$ é a solução aproximada. Assim
\begin{eqnarray}
   e_{n+1}    =&e_n + h(f(t^{(n)},u(t^{(n)})+e_n)- f(t^{(n)},u(t^{(n)}))) +\frac{h^2}{2}u''_n+\mathcal O(h^3)
\end{eqnarray}
Usando a condição de Lipschitz em $f$  temos
\begin{eqnarray}
   |e_{n+1}|      &\le &  |e_n| + h|f(t^{(n)},u(t^{(n)})+e_n)- f(t^{(n)},u(t^{(n)}))|+\frac{h^2}{2}|u''_n|+\mathcal O(h^3)\\
                  &\le &  |e_n| + hL |u(t^{(n)})+e_n- u(t^{(n)})|+\frac{h^2}{2}|u''_n|+\mathcal O(h^3)\\
                  &\le &  |e_n| + hL |e_n|+\frac{h^2}{2}|u''_n|+\mathcal O(h^3)\\
                  &\le &  (1+ hL) |e_n|+\frac{h^2}{2}|u''_n|+\mathcal O(h^3)
\end{eqnarray}

\subsection{Erro de truncamento local}

O \emph{erro de truncamento local} é o erro cometido em \textbf{uma} iteração do método numérico supondo que a solução exata é conhecida no passo anterior.

Assim, supondo que a solução é exata em $t^{(n)}$ ($|e_n|=0$), obtemos que o ETL é
\begin{equation} ETL_{Euler}^{n+1}= h^2/2|u''|+ \mathcal O(h^3) = \mathcal O(h^2) \end{equation}

Como o $ETL=\mathcal O(h^2)$ temos que o método de Euler possui ordem $1$.


\subsection{Erro de truncamento global}
O \emph{erro de truncamento global} é o erro cometido durante \textbf{várias} iterações do método numérico.

Supondo que a solução exata é conhecida em $t^{(1)}$ ($\|e_1\|=0$), então realizando $n=\frac{T}{h}$ iterações obtemos
\begin{eqnarray}
   ETG &=& nETL \\
       &=& n[h^2/2|u''|+ \mathcal O(h^3)] \\
       &=& Th/2|u''|+ \mathcal O(h^2)
\end{eqnarray}
ou seja
\begin{equation} ETG_{Euler}^{n+1} = \mathcal \mathcal{O}(h) \end{equation}

\subsection*{Exercícios resolvidos}

\emconstrucao

\subsection*{Exercícios}
\

\section{Convergência, consistência e estabilidade}
Nesta seção veremos três conceitos fundamentais em análise numérica: convergência, consistência e estabilidade.

\subsection{Convergência}
Um método é dito \emph{convergente} se para toda EDO com $f$ Lipschitz e todo $t>0$ temos que
\begin{equation}  \lim_{h \rightarrow 0} |u^{(n)} - u(t^{(n)})| =0, \quad \quad \forall n \end{equation}
Convergência significa que a solução numérica tende a solução do problema de valor inicial.


\begin{teo}
O método de Euler é convergente.
\end{teo}

De fato, se $f$ é Lipschitz-contínua e $|e_0|=0$, temos que
\begin{eqnarray}
 \lim_{h\rightarrow 0} |e_{n+1}|  &= \lim_{h\rightarrow 0} \mathcal \mathcal{O}(h) = 0
\end{eqnarray}


\subsection{Consistência}
\begin{defn}
Dizemos que um método numérico $R_h(u^{(n)})=f$ é consistente com o problema de valor inicial $u'(t)=f$ se para qualquer $u(t)$
\begin{eqnarray}
  \lim_{h \rightarrow 0} |u'(t^{(n)})-R_h(u^{(n)})| = 0, \quad  \forall n
\end{eqnarray}
\end{defn}

Isto é equivalente a
\begin{eqnarray}
  \lim_{h \rightarrow 0} \frac{ETL}{h} = 0
\end{eqnarray}



%\novapagina
%\section{Estabilidade}
%\begin{defn}\label{def:estnum}
% Um método numérico $P_h(u^{(n)})$ para um problema de valor inicial é \emph{estável} em uma região $\Lambda $ se  $\exists  J$ inteiro tal que $\forall T>0$, $\exists  C_T$ tal que
% \begin{eqnarray}
% |u^{(n)}|  \leq   C_T |u_0|
% \end{eqnarray}
%para $0\leq nh\leq T$, com $h\in \Lambda $.
%\end{defn}
%
%Isto significa que para ser estável a solução em $t\in [0,T]$ deve permanecer limitada por $C_T$ vezes a norma dos $J+1$ dados iniciais ($J=0$ para métodos de passo simples e $J>0$ para passo múltiplo).
%

\subsection{Estabilidade}
\begin{defn}
Um método numérico é \emph{estável} se
\begin{equation}  |u^{(n)}-v_n| \leq  C_1|u^{(1)}-v_1|, \quad  \forall n \end{equation}
\end{defn}
Isto significa que dadas duas condições iniciais $u^{(1)}$ e $v_1$, teremos que as soluções $u^{(n)}$ e $v_n$ estarão a uma distância limitada  por uma constante $C_1$ vezes $|u^{(1)}-v_1|$. Se $u^{(1)}$ e $v_1$ estiverem próximas então $u^{(n)}$ e $v_n$ estão também próximas dependendo da constante $C_1$ (obviamente $C_1$ depende da função $f$).


Considere o problema de valor inicial linear bem-posto
\begin{eqnarray}\label{EDO4.7}
  u'(t)= \lambda u(t), \quad  u(0)=1,
\end{eqnarray}
onde $\lambda  \in  \mathbb{C}$. Note que:
\begin{itemize}
\item Possui solução exata $u(t)=e^{\lambda t}.$
\item O problema de valor inicial é \emph{assintoticamente estável}, isto é, $\lim_{t\rightarrow \infty }u(t)=0$, se e somente se $\Re{\lambda }<0$.
\
\end{itemize}
%Não estamos interessados no momento em soluções que crescem rapidamente ($\Re{\lambda }>0$).

%\begin{defn}
%O problema de valor inicial linear é \emph{assintoticamente estável} se e somente se $\Re{\lambda }<0$.
%\end{defn}



\begin{defn}
O \emph{domínio de estabilidade linear} $\mathcal D$ do método numérico é o conjunto de todos $h\lambda  \in  \mathbb{C}$ tal que $\lim_{n\rightarrow \infty }u^{(n)}=0$.
\end{defn}

Ou seja, $\mathcal D$ é o conjunto de todos $h\lambda $ para o qual o correto comportamento assintótico de \eqref{EDO4.7} seja recuperado.

%tal que essa equação seja estável.


\begin{ex}
Utilizando o \textbf{Método de Euler} para solucionar \eqref{EDO4.7} obtemos ($u^{(1)}=1$)
\begin{eqnarray}
 u_{n+1}   & =& u^{(n)}+h\lambda u^{(n)}, \\
 u_{n+1}   & =& (1+h\lambda )u^{(n)}, \\
 u_{n+1}   & =& (1+h\lambda )^2u_{n-1}, \\
 u_{n+1}   & =& (1+h\lambda )^{n+1}u^{(1)} \\
 u_{n+1}   & =& (1+h\lambda )^{n+1}  , \quad  n=0,1,\ldots
\end{eqnarray}
Para que o método de Euler seja estável, é necessário que $h$ seja escolhido tal que $|1+h\lambda |<1$. Ou seja, $h\lambda $ deve estar em $\mathcal D_{Euler}$ onde
\begin{eqnarray}
 \mathcal D_{Euler} = \{z \in  \mathbb{C}: |1+z|<1\}
\end{eqnarray}
é o interior de um disco no plano complexo de raio $1$ e centro em $z=-1$.% como na Fig.\ref{RegiaoEuler}.
% \begin{figure}[htp]
% \begin{center}
%   \includegraphics[width=8cm]{RegiaoEuler.eps}\\
%   \caption{Região de estabilidade para o método de Euler, $|1+z|<1$ }\label{RegiaoEuler}
% \end{center}
% \end{figure}
\end{ex}


Tal análise pode ser facilmente estendida para $u'=\lambda u+b$ (veja exercícios).

Para o caso EDO não linear, seja
\begin{eqnarray}
u'= f(t,u), \quad  t\geq t_0, \quad  u(t_0)=u_0
\end{eqnarray}
é comum requerer que $h\lambda _{n,k} \in \mathcal D$ onde $\lambda _{n,k}$ são os autovalores da matriz jacobiana $J_n := \frac{\partial f}{\partial u}|_{(t^{(n)},u^{(n)})},$ baseado na hipótese que o comportamento local da EDO é modelado por
\begin{eqnarray}
  u'= u^{(n)} + J_n(u-u^{(n)})
\end{eqnarray}
Esta prática não é exata e fornece apenas uma ideia local do comportamento da EDO (podendo levar a conclusões errôneas).


Um dos teoremas mais importantes em análise numérica é o seguinte:

\begin{teo}
Um método numérico \emph{consistente} para um problema de valor inicial bem-posto é \emph{convergente} se e somente se ele é \emph{estável}.
\end{teo}


Ele também é usado da seguinte forma:

\begin{teo}
Se um método numérico é \emph{consistente} e \emph{estável} em $[a,b]$ então ele é \emph{convergente}.
\end{teo}



\subsection*{Exercícios resolvidos}

\emconstrucao

\subsection*{Exercícios}

\emconstrucao


\section{O método de Euler implícito}
Integrando o problema de valor inicial
\begin{eqnarray}
  u'(t)  &=& f(t,u(t)) \\
  u(t^{(1)}) &=& a
\end{eqnarray}
de $t^{(1)}$ até $t^{(2)}$ obtemos (como feito anteriormente)
\begin{eqnarray}
  u(t^{(2)})      &=& u(t^{(1)}) +  \int_{t^{(1)}}^{t^{(2)}} f(t,u(t)) \; dt
\end{eqnarray}

Entretanto se aproximarmos a função $f$ por uma função constante $f(t,u(t)) \approx  f(t^{(2)},u^{(2)})$, obteremos um novo método
\begin{eqnarray}
  u^{(2)} &=&  u^{(1)} + f(t^{(2)},u^{(2)}) \int _{t^{(1)}}^{t^{(2)}}  \; dt \\
  u^{(2)} &=&  u^{(1)} + h f(t^{(2)},u^{(2)})
\end{eqnarray}


Generalizando este procedimento para $t^{(n)}$ obtemos o \emph{método de Euler implícito}
\begin{eqnarray}
u_{n+1}=u^{(n)} + h\;f(t^{(n+1)},u_{n+1}).
\end{eqnarray}

Note que este método é \emph{implícito} (a equação é implícita) pois depende de $u_{n+1}$ dos dois lados da equação. Se a função $f$ for simples o suficiente, podemos resolver a equação isolando o termo $u_{n+1}$. Se isso não for possível, devemos usar um dos métodos vistos anteriormente para calcular as raízes da equação (por exemplo, método da bissecção e método de Newton).



Pode ser mostrado que o erro de truncamento local é
\begin{equation} ETL_{EulImp}^{n+1}= \mathcal{O}(h^2). \end{equation}
portanto o método é de ordem $1$. E o erro de truncamento global é
\begin{equation} ETG_{EulImp}^{n+1}= \mathcal{O}(h). \end{equation}



\begin{ex}
Utilizando o \textbf{método de Euler implícito} para solucionar \eqref{EDO4.7} obtemos
\begin{eqnarray}
 u_{n+1}      &=& u^{(n)}+h\lambda u_{n+1}, \\
 (1-h\lambda )u_{n+1} & =& u^{(n)}, \\
       u_{n+1} & =& \left(\frac{1}{1- h\lambda }\right)u^{(n)}, \\
       u_{n+1} & =& \left(\frac{1}{1- h\lambda }\right)^2u_{n-1}, \\
       u_{n+1} & =& \left(\frac{1}{1- h\lambda }\right)^{n+1}, \quad  n=0,1,\ldots
\end{eqnarray}
onde $u^{(1)}=1$.
Concluímos então que
\begin{eqnarray}
 \mathcal D_{EulImp} = \{z \in  \field{C}:  \left|\frac{1}{1- z}\right|<1\}
\end{eqnarray}
ou ainda,
\begin{eqnarray}
 \mathcal D_{EulImp} = \{z \in  \field{C}:  |1- z|>1\}
\end{eqnarray}

Para que o método de Euler implícito seja estável, é necessário que $h$ seja escolhido tal que $\left|\frac{1}{1- h\lambda }\right|<1$, ou ainda, $|1-h\lambda |>1$. Ou seja, $h\lambda $ deve estar em $\mathcal D_{EulImp}$ onde
\begin{eqnarray}
 \mathcal D_{EulImp} = \{z \in  \field{C}: |1-z|>1\}
\end{eqnarray}
é o exterior de um disco no plano complexo de raio $1$ e centro em $z=1$.

Note que $\mathcal D_{EulImp}$ inclui todo o semiplano negativo. Portanto o método de Euler implícito imita a estabilidade assintótica da EDO linear sem restrição no passo $h$.

\end{ex}

\begin{defn}
Um método numérico é chamado \emph{A-estável} ou \emph{incondicionalmente estável}\index{incondicionalmente estável}\index{estabilidade!incondicional} se seu domínio de estabilidade linear incluir todo o semiplano complexo com parte real negativa,
\begin{equation}     \{z \in \field{C}: \Re{z}<0\} \subseteq \mathcal D \end{equation}
\end{defn}


Portanto o método de Euler implícito é $A$-estável (incondicionalmente estável).





\section{Método trapezoidal}\index{método!trapezoidal}
O método de Euler aproxima $f$ como uma constante no intervalo $[t^{(1)},t^{(2)}]$. Podemos melhorar isso usando a regra do trapézio,

\begin{eqnarray}
  u(t^{(2)}) &=& u(t^{(1)}) +  \int _{t^{(1)}}^{t^{(2)}}  f(t,u(t)) \; dt \\
  u^{(2)}    &=&   u^{(1)}  +  (t^{(2)}-t^{(1)})\left(\frac{1}{2}f(t^{(1)},u^{(1)})+\frac{1}{2}f(t^{(2)},u^{(2)})\right)
\end{eqnarray}
motivando o \emph{método trapezoidal}
\begin{eqnarray}
  u_{n+1} &=& u^{(n)} +  \frac{h}{2} \left(f(t^{(n)},u^{(n)})+f(t^{(n+1)},u_{n+1})\right)
\end{eqnarray}
O método trapezoidal é dito \textbf{implícito}, pois para obter $u_{n+1}$ é necessário calcular $f(t^{(n+1)},u_{n+1})$.

Entretanto, pode ser mostrado que o erro de truncamento local é
\begin{equation} ETL_{Trap}^{n+1}= O(h^3) \end{equation}
portanto o método é de ordem $2$. E o erro de truncamento global é
\begin{equation} ETG_{Trap}^{n+1}= O(h^2) \end{equation}


\begin{ex}
Utilizando o \textbf{método trapezoidal} para solucionar \eqref{EDO4.7} obtemos
\begin{eqnarray}
 u_{n+1} = \left(\frac{1+ h\lambda /2}{1- h\lambda /2}\right)^{n+1}, \quad  n=0,1,\ldots
\end{eqnarray}
Concluímos então que
\begin{eqnarray}
 \mathcal D_{Tr} = \{z \in  \field{C}:  \left|\frac{1+ z/2}{1- z/2}\right|<1\}
\end{eqnarray}
Note que $\mathcal D_{Tr}=\field{C}^-$, o semiplano negativo. Portanto o método do trapézio imita a estabilidade assintótica da EDO linear sem restrição no passo $h$.
\end{ex}



\subsection*{Exercícios resolvidos}

\emconstrucao

\subsection*{Exercícios}

\emconstrucao


\section{O método de Heun}
Também chamado de método de \emph{Euler modificado}. A ideia é calcular primeiramente um valor intermediário $\tilde{u}$ usando o método de Euler expl\'icito e usar esse valor na equação para o método do Trapézio. Ou seja, o \emph{método de Heun} é
\begin{eqnarray}
  \tilde{u} &=& u^{(n)} +   h f(t^{(n)},u^{(n)}) \\
  u_{n+1}   &=& u^{(n)} +  \frac{h}{2} \left(f(t^{(n)},u^{(n)})+f(t^{(n+1)},\tilde{u})\right)
\end{eqnarray}

Este é um exemplo de um método preditor-corretor.

Felizmente o erro de truncamento local continua sendo
\begin{equation} ETL_{Heun}^{n+1}= O(h^3) \end{equation}
e o erro de truncamento global é
\begin{equation} ETG_{Heun}^{n+1}= O(h^2) \end{equation}


\subsection*{Exercícios resolvidos}

\emconstrucao

\subsection*{Exercícios}

\begin{exer} Use o método de Euler melhorado para obter uma aproximação numérica do valor de $u(1)$ quando $u(t)$ satisfaz o seguinte problema de valor inicial
\begin{eqnarray}
 u'(t)&=&-u(t)+ e^{u(t)},\\
 u(0)&=&0,
\end{eqnarray}
usando passos $h=0,1$ e $h=0,01$.
\end{exer}
\begin{resp}
 $u(1)\approx 1,317078$ quando $h=0,1$ e $u(1)\approx 1,317045$.
\end{resp}


\begin{exer}
Use o método de Euler e o método de Euler melhorado para obter aproximações numéricas para a solução do seguinte problema de valor inicial para $t\in[0,1]$:
\begin{eqnarray}
 u'(t)&=&-u(t)- u(t)^2,\\
 u(0)&=&1,
\end{eqnarray}
usando passo $h=0,1$. Compare os valores da solução exata dada por $u(t)=\frac{1}{2e^t-1}$ com os numéricos nos pontos $t=0$, $t=0,1$, $t=0,2$, $t=0,3$, $t=0,4$, $t=0,5$, $t=0,6$, $t=0,7$, $t=0,8$, $t=0,9$, $t=1,0$.
\end{exer}
\begin{resp}
 \begin{equation}\begin{array}{|c|c|c|c|c|c|}
\hline
t &  \text{Exato} & \text{Euler} & \text{Euler melhorado} & \text{Erro Euler} & \text{Erro Euler melhorado}\\
\hline
0,0&    1,          &1,          &1,          &0,          &0,       \\
0,1&    0,826213    &0,8         &0,828       &0,026213    &0,001787\\
0,2&    0,693094    &0,656       &0,695597    &0,037094    &0,002502  \\
0,3&    0,588333    &0,547366    &0,591057    &0,040967    &0,002724  \\
0,4&    0,504121    &0,462669    &0,506835    &0,041453    &0,002714  \\
0,5&    0,435267    &0,394996    &0,437861    &0,040271    &0,002594  \\
0,6&    0,378181    &0,339894    &0,380609    &0,038287    &0,002428  \\
0,7&    0,330305    &0,294352    &0,332551    &0,035953    &0,002246  \\
0,8&    0,289764    &0,256252    &0,291828    &0,033512    &0,002064  \\
0,9&    0,255154    &0,224061    &0,257043    &0,031093    &0,001889  \\
1,0&    0,225400    &0,196634    &0,227126    &0,028766    &0,001726\\

\hline
\end{array}
\end{equation}

\ifisscilab
      No Scilab, esta tabela pode ser produzida com o código:
      \begin{verbatim}
       deff('du=f(u)','du=-u-u^2')
       sol_Euler=Euler(f,0,1,10,1)'
       sol_Euler_mod=Euler_mod(f,0,1,10,1)'
       deff('u=u_exata(t)','u=1/(2*exp(t)-1)')
       t=[0:.1:1]'
       sol_exata=feval(t,u_exata)
       tabela=[t sol_exata sol_Euler sol_Euler_mod abs(sol_exata-sol_Euler) abs(sol_exata-sol_Euler_mod)]
      \end{verbatim}

    \fi
\end{resp}



\subsection*{Exercícios resolvidos}

\emconstrucao

\subsection*{Exercícios}

\emconstrucao






\subsection*{Exercícios resolvidos}

\emconstrucao

\subsection*{Exercícios}

\emconstrucao

\section{Estabilidade dos métodos de Taylor}
\begin{ex}
Prove que para um método de Taylor de ordem $p$ para a EDO \eqref{EDO4.7} temos
\begin{eqnarray}
  p(z)= 1 + z+ \frac{z^2}{2!} +\frac{z^3}{3!}+\ldots +\frac{z^p}{p!}
\end{eqnarray}
onde  $u^{(n)} = (p(z))^nu_0$ e a região de estabilidade é dada por
\begin{eqnarray}
 \mathcal D_{T} = \{z \in  \field{C}:  \left|p(z)\right|<1\}
\end{eqnarray}

Trace as regiões de estabilidade para o método de Taylor para $p=1,\ldots ,6$ no mesmo gráfico.
\end{ex}

% \begin{figure}
% \begin{center}
%   \includegraphics[width=8cm]{RegiaoTaylor.eps}\\
%   \caption{Região de estabilidade paras os métodos de Taylor de ordem $1,\ldots ,4$ (interior as curvas). A curva mais interna é para $p=1$}\label{RegiaoTaylor}
% \end{center}
% \end{figure}


\begin{ex}
Aproxime a solução do problema de valor inicial
\begin{eqnarray}
   \frac{du}{dt} &=\sin{t}\\
            u(0) &= 1
\end{eqnarray}
para  $t\in [0,10]$.

\begin{enumerate}
\item [a.] Trace a solução para $h=0,16$, $0,08$, $0,04$, $0,02$ e $0,01$ para o método de Taylor de ordem $1$, $2$ e $3$. (Trace todos de ordem $1$ no mesmo gráfico, ordem $2$ em outro gráfico e ordem $3$ outro gráfico separado.)

\item [b.] Utilizando a solução exata, trace um gráfico do erro em escala logar\'itmica.
Comente os resultados (novamente, em cada gráfico separado para cada método repita os valores acima)

\item [c.] Fixe agora o valor $h=0,02$ e trace no mesmo gráfico uma curva para cada método.

\item [d.] Trace em um gráfico o erro em $t=10$ para cada um dos métodos (uma curva para cada ordem) a medida que $h$ diminui. (Use escala \verb#loglog#)
\end{enumerate}
\end{ex}











% \section{Ordem de precisão}\index{ordem de precisão}
% Considere o problema de valor inicial dado por
% \begin{eqnarray}
% u'(t)&=&f(t,u(t)),\\
% u(0)&=&u_0.
% \end{eqnarray}
% Nessa seção vamos definir a precisão de um método numérico pela ordem do erro acumulado ao calcular o valor da função em um ponto $t_N$ em função do espaçamento da malha $h$. Se $u(t^{(n)})$ pode ser aproximado por uma expressão que depende de $f$, $h$, $u(t_0)$, $u(t^{(1)})$, $\cdots$, $u(t^{(n)})$, com erro da ordem de $O(h^{p+1})$, ou seja,
% \begin{equation}{\label{erro_local_1}}
% u(t^{(n+1)})=\mathcal{F}(f, h, u(t^{(n)}), u(t_{n-1}), \cdots, u_0) + O(h^{p+1})
% \end{equation}
% para cada função analítica $f$, dizemos que o método tem erro de truncamento da ordem de $O(h^{p})$ ou {\bf ordem de precisão $p$}\index{método!de Euler!ordem de precisão}. Essa afirmação faz sentido quando fazemos a seguinte análise informal: para aproximar $u^{(1)}$, acumulamos erros da ordem $O(h^{p+1})$, para calcular $u^{(2)}$ acumulamos os erros de $u^{(1)}$ e novos erros $O(h^{p+1})$. Para calcular $u_N$, acumulamos todos os erros até $t_N$, ou seja, $N$ vezes $O(h^{p+1})$. Como $N=O(1/h)$, temos que os erros ao calcular $u_N$ são da ordem $O(h^p)$. É verdade que essa análise só vale quando impomos condições de suavidade para $f$ e condições adequada para a expressão $\mathcal{F}(f, h, u(t^{(n)}), u(t_{n-1}), \cdots, u_0)$. Para explicar melhor esse pequeno texto, fazemos em detalhes essa operação para o método de Euler na Seção~\ref{sec_pre_euler}.
%
% \subsection{Ordem de precisão do método de Euler}{\label{sec_pre_euler}}
% Primeiro lembramos da expressão (\ref{erro_local}) que origina a seguinte relação de recorrência:
% \begin{eqnarray}{\label{es_Euler}}
% u(t^{(n+1)})&=&u(t^{(n)})+hf(t,u(t^{(n)}),t^{(n)})+O(h^2).
% \end{eqnarray}
% Para entender melhor o motivo de na expressão (\ref{es_Euler}) aparecer $O(h^2)$ e o método ser de precisão 1, vamos a seguinte análise informal: observemos que
% \begin{eqnarray}
%  u(t^{(1)})&=&u(t_0)+hf(t,u(t_0),t_0)+O(h^2)\\
%  &=&u_0+hf(u_0,t_0)+O(h^2)=u^{(1)}+O(h^2)
% \end{eqnarray}
% onde $u^{(i)}$ é a aproximação pelo método de Euler para o valor exato $u(t^{(i)})$. Subsequentemente, temos
% \begin{eqnarray}
%  u(t^{(2)})&=&u(t^{(1)})+hf(t,u(t^{(1)}),t^{(1)})+O(h^2)\\
%  &=&u(t^{(1)})+hf(u^{(1)}+O(h^2),t^{(1)})+O(h^2)\\
%  &=&u(t^{(1)})+hf(u^{(1)},t^{(1)}) +O(h^2)\\
%  &=&u^{(1)}+O(h^2)+hf(u^{(1)},t^{(1)}) +O(h^2)= u^{(2)}+O(h^2)+O(h^2).
% \end{eqnarray}
% onde usamos o primeiro termo da série de Taylor $hf(u^{(1)}+O(h^2),t^{(1)})=hf(u^{(1)},t^{(1)})+O(h^3)$ na passagem da segunda para terceira linha. Repetindo sucessivamente o passo anterior, obtemos uma expressão geral para o valor exato $u(t_N)$ em termos do valor aproximado $u_N$:
% \begin{eqnarray}
%  u(t_N)=u_N+N O(h^2) .
% \end{eqnarray}
% Como $N=(t_f-t_0)/h$, temos
% \begin{eqnarray}{\label{euler_precisao}}
%  u(t_N)&=&=u_{N}+\frac{t-t_0}{h}O(h^2)=u_{N}+\mathcal{O}(h),
% \end{eqnarray}
% ou seja, o erro entre o valor exato e o aproximado é de ordem $h$. Uma demonstração mais formal que garante que o erro é limitado por uma expressão que é proporcional a $h$ está discutido na Seção~\ref{sec_conv_Euler}.
%
% \subsection{Ordem de precisão do método de Euler melhorado}
% Para obter o erro de precisão do método de Euler melhorado vamos calcular o erro de truncamento do método, ou seja, precisamos demonstrar que:
% \begin{equation}\label{es_Euler_melhorado}
% u(t+h)=u(t)+\frac{h}{2} f(t,u(t))+\frac{h}{2} f(t,u(t)+hf(t,u(t))),t+h)+O(h^3)
% \end{equation}
% De fato, tomando a diferença do termo da esquerda o os termos da direita, temos:
% \begin{eqnarray}
% &&u(t+h)-\left(u(t)+\frac{h}{2} f(t,u(t))+\frac{h}{2} f(t,u(t)+hf(t,u(t))),t+h)\right)\\
% &&=u(t)+hu'(t)+\frac{h^2}{2}u''(t)+O(h^3)\\
% &&-\left(u(t)+\frac{h}{2} u'(t)+\frac{h}{2} f(t,u(t)+hf(t,u(t))),t+h)\right),
% \end{eqnarray}
% onde usamos uma expansão em série de Taylor para $u(t+h)$ e a equação diferencial $u'(t)=f(t,u(t))$. Portanto,
% \begin{eqnarray}
% &&u(t+h)-\left(u(t)+\frac{h}{2} f(t,u(t))+\frac{h}{2} f(t,u(t)+hf(t,u(t))),t+h)\right)\\
% &&=\frac{h}{2}u'(t)+\frac{h^2}{2}u''(t)-\frac{h}{2} f(t,u(t)+hf(t,u(t))),t+h)+O(h^3).
% \end{eqnarray}
% Agora, usamos a série de Taylor de $f(t,u(t)+hf(t,u(t)),t+h)$ e, torno de $(t,u)$:
% \begin{eqnarray}
% &&u(t+h)-\left(u(t)+\frac{h}{2} f(t,u(t))+\frac{h}{2} f(t,u(t)+hf(t,u(t))),t+h)\right)\\
% &&=\frac{h}{2}u'(t)+\frac{h^2}{2}u''(t)+O(h^3)\\
% &&-\frac{h}{2}\left(f(t,u(t))+\frac{\partial f(t,u(t)) }{\partial t}h +\frac{\partial f(t,u(t))}{\partial u} hf(t,u(t))+O(h^2)\right).
% \end{eqnarray}
% Usando a equação diferencial $u'(t)=f(t,u(t))$ obtemos
% \begin{equation}
% u''(t)=\frac{f(t,u(t))}{\partial t}+\frac{f(t,u(t))}{\partial u}u'(t)=\frac{f(t,u(t))}{\partial t}+\frac{f(t,u(t))}{\partial u}f(t,u(t)).
% \end{equation}
% Logo,
% \begin{eqnarray}
% &&u(t+h)-\left(u(t)+\frac{h}{2} f(t,u(t))+\frac{h}{2} f(t,u(t)+hf(t,u(t))),t+h)\right)\\
% &&=\frac{h}{2}u'(t)+\frac{h^2}{2}u''(t)+O(h^3)\\
% &&-\frac{h}{2}\left(f(t,u(t))+hu''(t)+O(h^2)\right)\\
% &&=\frac{h}{2}u'(t)+\frac{h^2}{2}u''(t)\\
% &&-\frac{h}{2}\left(u'(t)+hu''(t)\right)+O(h^3)=O(h^3)
% \end{eqnarray}
% Portanto, a expressão (\ref{es_Euler_melhorado}) é válida. Logo, usando uma discussão análoga aquela feita na Seção~\ref{sec_pre_euler} para o método de Euler, concluímos que o método de Euler melhorado possui ordem de precisão 2.

% \section{Convergência}
%
% \emconstrucao
%
% \subsection{Convergência do método de Euler}\label{sec_conv_Euler}
%
% \emconstrucao
%
% \subsection{Convergência do método de Euler melhorado}
%
% \emconstrucao




\subsection*{Exercícios resolvidos}

\emconstrucao

\subsection*{Exercícios}

\emconstrucao

\section{Métodos de passo múltiplo}
Seja o problema de valor inicial
\begin{eqnarray}
  u'(t) &= f(t,u(t)) \\
  u(t_0) &= a
\end{eqnarray}


Integrando a EDO em $[t^{(n+1)},t^{(n)}]$ obtemos
\begin{eqnarray}
  u_{n+1}  &= u^{(n)}  + \int _{t^{(n)}}^{t^{(n+1)}} f(t,u(t)) \; dt
\end{eqnarray}
Denote por $f^{(n)}\equiv f(t^{(n)},u^{(n)})$. Um método de passo simples utiliza $f_{n+1}$ e $f_{n}$. Um método de passo múltiplo utiliza também $s$ valores anteriores já calculados como $f^{(n-1)},f_{n-2},\ldots ,f_{n-s}$, onde $s\geq 1$ inteiro.

\begin{eqnarray}
  u_{n+1}  &= u_{n}  + h[b_s f_{n+1}+b_{s-1}f_{n}+\ldots +b_1f_{n-s+2}+b_0f_{n-s+1}]
%\\  u_{n+1}  &= u_{n}  + h \sum_{m=0}^s b_m f_{n-s+1+m}
\end{eqnarray}

Para conformidade com \cite{iserles2009first}, translade $s-1$ índices,
\begin{eqnarray}
  u_{n+s}  &= u_{n+s-1}  + h[b_s f_{n+s}+b_{s-1}f_{n+s-1}+\ldots +b_1f_{n+1}+b_0f^{(n)}] \label{multiad}
\end{eqnarray}
e teremos
\begin{eqnarray}
  u_{n+s}  &= u_{n+s-1}  + h \sum_{m=0}^s b_m f_{n+m}
\end{eqnarray}

De forma geral um \emph{método de passo múltiplo} será
\begin{eqnarray}
  \sum_{m=0}^s a_m u_{n+m}  &=  h \sum_{m=0}^s b_m f_{n+m}
\end{eqnarray}

\subsection*{Exercícios resolvidos}

\emconstrucao

\subsection*{Exercícios}

\emconstrucao


\section{O método de Adams-Bashforth}
Quando $a_s=1$, $a_{s-1}=-1$, $a_m=0$ para $m=s-2,\ldots ,0$, $b_s=0$ temos um método de Adams-Bashforth do tipo
\begin{eqnarray}\label{AB}
  u_{n+s}  &= u_{n+s-1}  + h \sum_{m=0}^{s-1} b_m f_{n+m}
\end{eqnarray}
Note que os métodos de Adams-Bashforth são \emph{explícitos} pois $b_s=0$.



\begin{ex}
Vamos obter o método de Adams-Bashforth para $s=4$ como
\begin{eqnarray}
  u_{n+4}  &= u_{n+3}  + \int _{t_{n+3}}^{t_{n+4}} f(t,u(t)) \; dt \\
  u_{n+4}  &= u_{n+3}  + h \sum_{m=0}^{3} b_m f_{n+m} \\
  u_{n+4}  &= u_{n+3}  + h [b_3f_{n+3} +b_2f_{n+2} +b_1f_{n+1} +b_0f^{(n)}]
\end{eqnarray}
Para isso devemos obter $[b_3,b_2,b_1,b_0]$ tal que o método seja exato para polinômios até ordem $3$. Podemos obter esses coeficientes de maneira análoga a obter os coeficientes de um método para integração.

Supondo que os nós $t_k$ estejam igualmente espaçados, e para facilidade dos cálculos, como o intervalo de integração é $[t_{n+3},t_{n+4}]$, translade $t_{n+3}$ para a origem tal que $[t^{(n)},t^{(n+1)},\ldots ,t_{n+4}]=[-3h,-2h,-h,0,h]$.

Considere a base $[\phi _0(t),\ldots ,\phi _3(t)]=[1, t, t^2, t^3]$ e substitua $f(t)$ por $\phi _k(t)$ obtendo
\begin{eqnarray}
  \int _0^{h} 1  \;dt = h             &= h( b_0(1)  +b_1(1)    + b_2(1)   + b_3(1)    )\\
  \int _0^{h} t  \;dt = \frac{h^2}{2}  &= h( b_0(0)  +b_1(-h)   + b_2(-2h) + b_3(-3h)  )\\
  \int _0^{h} t^2 \;dt = \frac{h^3}{3}  &= h( b_0(0)^2 +b_1(-h)^2  + b_2(-2h)^2+ b_3(-3h)^2 )\\
  \int _0^{h} t^3 \;dt = \frac{h^4}{4} &= h( b_0(0)^3 +b_1(-h)^3  + b_2(-2h)^3+ b_3(-3h)^3 )
\end{eqnarray}
que pode ser escrito na forma matricial
\begin{eqnarray}
\left(
  \begin{array}{cccc}
    1  &  1    & 1   & 1\\
    0  &  -1   & -2  & -3\\
    0  &  1    & 4   &  9\\
    0  &  -1   & -8  & -27
  \end{array}
\right)
\left(\begin{array}{c}  b_0 \\ b_1\\ b_2\\b_3   \end{array}\right)
=
\left(\begin{array}{c}  1  \\ 1/2 \\ 1/3 \\ 1/4  \end{array}\right)
\end{eqnarray}
Resolvendo o sistema obtemos
\begin{equation} [b_0,b_1,b_2,b_3]=[-\frac{9}{24},\frac{37}{24},-\frac{59}{24},\frac{55}{24}] \end{equation}
fornecendo o \emph{método de Adams-Bashforth de $4$ estágios}
\begin{eqnarray}\label{AB4}
  u_{n+4}  &= u_{n+3}  + \frac{h}{24} [55 f_{n+3} -59f_{n+2} +37f_{n+1} -9f^{(n)}]
\end{eqnarray}
\end{ex}

\subsection*{Exercícios resolvidos}

\emconstrucao

\subsection*{Exercícios}

\begin{exer}
Mostre que o método de Adams-Bashforth para $s=2$ é dado por
\begin{eqnarray}\label{AB2}
  u_{n+2}  &= u_{n+1}  + \frac{h}{2} [3 f_{n+1} -f_{n}]
\end{eqnarray}
\end{exer}

\begin{exer}
Mostre que o método de Adams-Bashforth para $s=3$ é dado por
\begin{eqnarray}\label{AB3}
  u_{n+3}  &= u_{n+2}  + \frac{h}{12} [23f_{n+2}-16 f_{n+1} +5f_{n}]
\end{eqnarray}
\end{exer}




\section{O método de Adams-Moulton}
Quando $a_s=1$, $a_{s-1}=-1$, $a_m=0$ para $m=s-2,\ldots ,0$, $b_s\neq 0$ temos um método de Adams-Moulton do tipo
\begin{eqnarray}\label{AM}
  u_{n+s}  &= u_{n+s-1}  + h \sum_{m=0}^{s} b_m f_{n+m}
\end{eqnarray}
Note que os métodos de Adams-Moulton são implícitos pois $b_s\neq 0$.



\begin{ex}
Vamos obter o método de Adams-Moulton para $s=3$ como
\begin{eqnarray}\label{AM4}
  u_{n+3}  &= u_{n+2}  + \int _{t_{n+3}}^{t_{n+4}} f(t,u(t)) \; dt \\
  u_{n+3}  &= u_{n+2}  + h \sum_{m=0}^{3} b_m f_{n+m} \\
  u_{n+3}  &= u_{n+2}  + h [b_3f_{n+3} +b_2f_{n+2} +b_1f_{n+1} +b_0f^{(n)}]
\end{eqnarray}
Para isso devemos obter $[b_3,b_2,b_1,b_0]$ tal que o método seja exato para polinômios até ordem $3$. Podemos obter esses coeficientes de maneira análoga a obter os coeficientes de um método para integração.

Supondo que os nós $t_k$ estejam igualmente espaçados, e para facilidade dos cálculos, como o intervalo de integração é $[t_{n+2},t_{n+3}]$, translade $t_{n+2}$ para a origem tal que $[t^{(n)},t^{(n+1)},\ldots ,t_{n+3}]=[-2h,-h,0,h]$.

Considere a base $[\phi _0(t),\ldots ,\phi _3(t)]=[1, t, t^2, t^3]$ e substitua $f(t)$ por $\phi _k(t)$ obtendo
\begin{eqnarray}
      \int _0^{h} 1  \;dt = h             &= h( b_0(1)  +b_1(1)   + b_2(1)   + b_3(1)    )\\
      \int _0^{h} t  \;dt = \frac{h^2}{2}  &= h( b_0(h)  +b_1(0)   + b_2(-h) + b_3(-2h)  )\\
      \int _0^{h} t^2 \;dt = \frac{h^3}{3}  &= h( b_0(h)^2 +b_1(0)^2  + b_2(-h)^2+ b_3(-2h)^2 )\\
      \int _0^{h} t^3 \;dt = \frac{h^4}{4} &= h( b_0(h)^3 +b_1(0)^3  + b_2(-h)^3+ b_3(-2h)^3 )
\end{eqnarray}
que pode ser escrito na forma matricial
\begin{eqnarray}
\left(
  \begin{array}{cccc}
    1  & 0 & 1    & 1   \\
    1  & 0 & -1   & -2  \\
    1  & 0 & 1    & 4   \\
    1  & 0 & -1   & -8
  \end{array}
\right)
\left(\begin{array}{c}  b_0 \\ b_1\\ b_2\\b_3   \end{array}\right)
=
\left(\begin{array}{c}  1  \\ 1/2 \\ 1/3 \\ 1/4  \end{array}\right)
\end{eqnarray}
Resolvendo o sistema obtemos
\begin{equation} [b_0,b_1,b_2,b_3]=[\frac{1}{24},-\frac{5}{24},\frac{19}{24},\frac{9}{24},] \end{equation}
fornecendo a regra
\begin{eqnarray}
  u_{n+3}  &= u_{n+2}  + \frac{h}{24} [9 f_{n+3} +19f_{n+2} -5f_{n+1} +f^{(n)}]
\end{eqnarray}
\end{ex}

\subsection*{Exercícios resolvidos}

\emconstrucao

\subsection*{Exercícios}

\begin{exer}
Encontre o método de Adams-Moulton para $s=2$.
%\begin{eqnarray}\label{AM2}
%  u_{n+2}  &= u_{n+1}  + \frac{h}{2} [3 f_{n+1} -f_{n}]
%\end{eqnarray}
\end{exer}

\begin{exer}
Encontre o método de Adams-Moulton para $s=3$.
%\begin{eqnarray}
%  u_{n+3}  &= u_{n+2}  + \frac{h}{12} [23f_{n+2}-16 f_{n+1} +5f_{n}]
%\end{eqnarray}
\end{exer}

\section{Método BDF}
Um método de ordem $s$ com $s$ estágios é chamado de \emph{método BDF-Backward Differentiation Formula} se $\sigma (w)=b_sw^s$, onde $b_s \in \mathbb{R}$, ou seja,
\begin{eqnarray}\label{BDF}
  a_s u_{n+s}+ ...+ a_1 u_{n+1} + a_0u_{n} &=  h b_sf_{n+s}
\end{eqnarray}

\begin{ex}
Mostre que o método BDF com $s=3$ é
\begin{eqnarray}
  u_{n+3} -\frac{18}{11} u_{n+2}+\frac{9}{11}u_{n+1}-\frac{2}{11}u^{(n)} &= \frac{6}{11}h f_{n+3}
\end{eqnarray}
\end{ex}

\subsection*{Exercícios}

\begin{exer}
Mostre que o método BDF com $s=1$ é o método de Euler implícito.
\end{exer}

\begin{exer}
Mostre que o método BDF com $s=2$ é
\begin{eqnarray}
  u_{n+2} -\frac{4}{3} u_{n+1} + \frac{1}{3}u^{(n)} &= \frac{2}{3}h f_{n+2}
\end{eqnarray}
\end{exer}


\section{Ordem e convergência de métodos de passo múltiplo}
Mais geralmente, um método de passo múltiplo será da forma
\begin{eqnarray}\label{multistep}
  a_s u_{n+s}+ ...+ a_1 u_{n+1} + a_0u_{n} &=  h [b_sf_{n+s} +... +b_1f_{n+1} +b_0f^{(n)}]
\end{eqnarray}
Por convenção normalizamos a equação acima tomando $a_s=1$. Quando $b_s=0$ temos um método explícito e quando $b_s \neq 0$ temos um método implícito.

O método será de ordem $p$ se o $ETL=\mathcal O(h^{p+1})$.


Dois polinômios são usados para estudar o método \eqref{multistep}:
\begin{eqnarray}
\rho (w)= a_s w^s + ...+a_1w+a_0, \quad \quad \sigma (w)=b_s w^s + ...+b_1w+b_0,
\end{eqnarray}



%\begin{teo}\label{teo:multiordem}
%Um método de passo múltiplo é de ordem $p$ se e somente se
%\begin{equation}  \rho (w)=\sigma (\frac{dw}{dt}),  \quad  w=t^k, k=0,...,p-1  \end{equation}
%\end{teo}
%\begin{proof}
%Para o método ser de ordem $p$ é necessário e suficiente que $\rho (w)=\sigma (\frac{dw}{dt})$ para todos os polinômios até ordem $p$, ou seja,
%\begin{eqnarray} \label{condconv}
%  a_s +...+a_1+a_0      &=0 \\
%  sa_s+...+1a_1+0a_0    &=b_s +...+b_1+b_0         \\
%  s^2a_s+...+1^2a_1+0^2a_0 &=2(s b_s+...+1 b_1+0 b_0) \\
%  s^3a_s+...+1^3a_1+0^3a_0 &=3(s^2b_s+...+1^2b_1+0^2b_0) \\
%         \vdots    &=&\vdots  \\
%  s^ka_s+...+1^ka_1+0^ka_0 &=(k-1)(s^{k-1}b_s+...+1^{k-1}b_1+0b_0)
%\end{eqnarray}
%se e somente se o método é de ordem $p$.
%\end{proof}
%
%
%
%
%\begin{ex}
%Mostre que o método de Adams-Bashforth para $s=2$ dado por
%\begin{eqnarray}\label{AB2}
%  u_{n+2}  &=&u_{n+1}  + \frac{h}{2} [3 f_{n+1} -f_{n}]
%\end{eqnarray}
%é de ordem $2$.
%
%Temos que $\rho (w)=w^2-w$ e $\sigma (w)=\frac{3}{2}w-\frac{1}{2}$.
%
%Assim para $\phi (t)=1,t,t^2$ obtemos
%\begin{eqnarray} \label{condconv}
%  \rho (1)-\sigma (0) = (1^2-1) -0 =0 \\
%  \rho (t)-\sigma (1) = (t^2-t) -(\frac{3}{2}-\frac{1}{2} )=0 \\
%  \rho (t^2)-\sigma (2t) = (t^4-t^2) -(\frac{3}{2}-\frac{1}{2} )=0 \\
%  2 a_2+1 a_1      &=b_2+b_1+b_0         \\
%  2^2a_2+1^2a_1      &=2(2 b_2+1 b_1+0 b_0) \\
%  2^3a_2+1^3a_1+0a_0  &=3(2^2b_2+1^2b_1+0^2b_0)
%\end{eqnarray}
%
%
%
%\begin{eqnarray} \label{condconv}
%  a_2  +  a_1+a_0   &=0 \\
%  2 a_2+1 a_1      &=b_2+b_1+b_0         \\
%  2^2a_2+1^2a_1      &=2(2 b_2+1 b_1+0 b_0) \\
%  2^3a_2+1^3a_1+0a_0  &=3(2^2b_2+1^2b_1+0^2b_0)
%\end{eqnarray}
%
%
%
%
%
%
%\end{ex}
%
%


%
%\subsection{Convergência de métodos de passo múltiplo}
%O erro de truncamento local para qualquer $u(t)$ suave
%\begin{eqnarray}
%   ETL(u(t)) &=&a_s u(t+sh)+...+a_1u(t+h)+a_0u(t) -  h [b_s u'(t+sh)+...+b_1u'(t+h)+b_0u'(t)] \\
%             &=&\mathcal O(h^{p+1})
%\end{eqnarray}
%se e somente se $ETL(q(t))=0$ $\forall q(t) \in \mathcal P^p$.
%
%Isto é equivalente a $ETL( \phi _k(t) )=0$ para a base polinomial $\phi _k(t)=t^k$, $k=0,\ldots ,p$.
%
%Para $\phi _0(t)=1$ obtemos
%\begin{eqnarray}
%   ETL(\phi _0) &=& a_s u(t+sh)+...+a_1u(t+h)+a_0u(t) -  h [b_s u'(t+sh)+...+b_1u'(t+h)+b_0u'(t)] \\
%           &=& a_s +...+a_1+a_0-  h [b_s 0+...+b_10+b_00] \\
%           &=& a_s +...+a_1+a_0 =0 \\
%\end{eqnarray}
%e para $\psi _1(t)=t$, obtemos
%\begin{eqnarray}
%   ETL(\phi _1) &=& a_s u(t+sh)+...+a_1u(t+h)+a_0u(t) -  h [b_s u'(t+sh)+...+b_1u'(t+h)+b_0u'(t)] \\
%           &=& a_s sh+...+a_1h+a_00 -  h [b_s +...+b_1+b_0] =0
%\end{eqnarray}
%e para $\psi _k(t)=t^k$, obtemos
%\begin{eqnarray}
%   ETL(\phi _k) &=& a_s u(t+sh)+...+a_1u(t+h)+a_0u(t) -  h [b_s u'(t+sh)+...+b_1u'(t+h)+b_0u'(t)] \\
%           &=& a_s (sh)^k+...+a_2(2h)^k+a_1(h)^k+a_00 -  h k[b_s (sh)^{k-1}+...+b_1h^{k-1}+b_00] =0
%\end{eqnarray}
%que é igual a zero para $k=1,\ldots ,p$ sob a condição \eqref{condconv}.
%
%

\begin{ex}
O método \eqref{AB3} de Adams-Bashforth para $s=3$ estágios é de ordem $3$ de convergência, ou seja, $ETL = \mathcal O(h^4)$. Ele é construído de tal maneira que seja exato para os polinômios $1, t, t^2, t^3$.
\end{ex}


\subsection{Consistência, estabilidade e convergência}
\begin{teo}
Um método de passo múltiplo é \emph{consistente} se $\rho (1)=0$ e $\rho '(1)=\sigma (1)$.
\end{teo}


\begin{teo}
Um método de passo múltiplo é \emph{estável} se todas as raízes de  $\rho (z)$ estão em $|z|\leq 1$ e as raízes com $|z|=1$ são simples.
\end{teo}


\begin{teo}
Se um método numérico é \emph{consistente} e \emph{estável} em $[a,b]$ então ele é \emph{convergente}.
\end{teo}



\begin{ex}
Prove que o método de passo $3$
\begin{eqnarray}\label{multis3}
  u_{n+3} +\frac{27}{11}u_{n+2} -\frac{27}{11}u_{n+1} -u_{n}  =\\
   =\frac{h}{11} [3 f_{n+3}+27f_{n+2}+27f_{n+1} +3f_{n}]
\end{eqnarray}
não é estável.
\end{ex}
\begin{sol}
O polinômio
\begin{eqnarray}
   \rho (w) &=&w^3+\frac{27}{11}w^2 -\frac{27}{11}w -1 \\
        &=&(w-1)\left(w+\frac{19+4\sqrt{15}}{11}\right)
                \left(w+\frac{19-4\sqrt{15}}{11}\right)
\end{eqnarray}
falha na condição da raiz.
\end{sol}


\subsection*{Exercícios resolvidos}

\emconstrucao

\subsection*{Exercícios}

\begin{exer}
Prove que todos os métodos de Adams-Bashforth satisfazem a condição da raiz.
%\begin{solution}
%Para o método de Adams-Bashforth temos que $\rho (w)=w^{s-1}(w-1)$ satisfazendo a condição da raiz.
%\end{solution}
\end{exer}

\begin{teo}
O polinômio $\rho (w)$ em \eqref{BDF} satisfaz a condição da raiz e o método BDF é convergente se e somente se $1\leq s\leq 6$.
\end{teo}

\begin{exer}
Mostre que os métodos BDF com $s=2$ e $s=3$ são convergentes.
\end{exer}



\subsection{As barreiras de Dahlquist}
Um método de passo múltiplo possui $2s+1$ coeficientes $a_m$, $b_m$. Poderíamos definir tais coeficientes de tal forma a obter ordem máxima.

Conclusão? Poderíamos obter métodos com $s$ estágios e ordem $2s$.

Entretanto tal método (implícito de passo $s$ e ordem $2s$) não é convergente para $s\geq 3$ .

É possível provar que a ordem máxima de convergência para um método de passo múltiplo $s$ é no máximo  $2\lfloor(s+2)/2\rfloor$ para métodos implícitos e $s$ para métodos explícitos. Esta é a \emph{primeira barreira de Dahlquist}.

\section{Estabilidade dos métodos de passo múltiplo}

%Suponha que um método de passo múltiplo \eqref{multistep} seja aplicado a EDO linear \eqref{EDO4.7} obtendo
%\begin{eqnarray}
%  \sum_{m=0}^s a_m  u_{n+m}  &=& h  \sum_{m=0}^s b_m f_{n+m} \\
%  \sum_{m=0}^s a_m  u_{n+m}  &=& h\lambda  \sum_{m=0}^s b_m u_{n+m}
%\end{eqnarray}
%que pode ser escrita como
%\begin{eqnarray}
%  \sum_{m=0}^s (a_m  - h\lambda  b_m) u_{n+m}&=0, \\
%  \sum_{m=0}^s \alpha _m u_{n+m}&=0, \quad  n=0,1,\ldots
%\end{eqnarray}
%Similarmente a equações diferenciais, a equação \`a diferenças possui solução como o polinômio característico
%\begin{eqnarray}
%  \eta (w) :=   \sum_{m=0}^s \alpha _m w^n.
%\end{eqnarray}
%Sejam $w_1,\ldots ,w_q$ as raízes de $\eta (w)$ com multiplicidade $k_1,\ldots ,k_q$, onde $\sum k_i=s$. A solução da equação a diferenças é
%\begin{eqnarray}
%  u^{(n)} = \sum_{i=1}^q (\sum_{j=0}^{k_i-1}c_{ij} n^j)w_i^n, \quad  n=0,1,\ldots
%\end{eqnarray}
%onde as constantes $c_{ij}$ são unicamente determinadas pelos valores iniciais $u_0,\ldots ,u_{s-1}$.

\begin{teo}
O método BDF de 2 estágios é A-estável.
\end{teo}

\begin{teo}
[A segunda barreira de Dahlquist] A ordem máxima de um método de passo múltiplo A-estável é dois.
\end{teo}

%%\begin{ex}
%%O método BDF de 2 estágios
%%\begin{eqnarray}
%%  u_{n+2} - \frac{4}{3} u_{n+1} + \frac{1}{3} u^{(n)} = \frac{2}{3}h f_{n+2}
%%\end{eqnarray}
%%para o problema de valor inicial $u'=\lambda u$, que possui polinômio característico
%%\begin{eqnarray}
%%  w^2 - \frac{4}{3} w + \frac{1}{3} = \frac{2}{3}h \lambda (w^2)
%%\end{eqnarray}
%%ou
%%\begin{eqnarray}
%%  \eta (w) := (1-z\frac{2}{3}) w^2 - \frac{4}{3} w + \frac{1}{3}.
%%\end{eqnarray}
%%
%%O polinômio $\eta (w)$ possui duas raízes (encontradas com o Maple)
%%\begin{eqnarray}
%%  w_1 &=&\frac{-4+2(1+2z)^{1/2}}{-6+4z},\\
%%  w_2 &=&\frac{-4-2(1+2z)^{1/2}}{-6+4z}
%%\end{eqnarray}
%%
%%Para que o método seja estável, é necessário que $|w_1|<1$ e $|w_2|<1$.
%%
%%Tal condição é satisfeita para a região dentro da curva na Figura~\ref{RegiaoBDF}.
%%
%%\begin{figure}
%%\begin{center}
%%  \includegraphics[width=8cm]{RegiaoBDF.eps}\\
%%  \caption{Região de estabilidade (exterior a curva) para o método BDF $s=2$ }\label{RegiaoBDF}
%%\end{center}
%%\end{figure}
%%\end{ex}

%\begin{ex}
%Plote a região de estabilidade do método de BDF de ordem $2$ e $3$.
%\end{ex}
%
%\begin{ex}
%Plote a região de estabilidade do método de Adams-Bashforth de ordem $2$ e $3$.
%\end{ex}
%
%\begin{ex}
%Plote a região de estabilidade do método de Adams-Moulton de ordem $2$ e $3$.
%\end{ex}
%


%
%
% \section{Métodos de passo múltiplo - Adams-Bashforth}\index{método!de passo múltiplo!Adams-Bashforth}
%
% O método de Adams-Bashforth consiste de um esquema recursivo do tipo:
% \begin{equation} u^{(n+1)}=u^{(n)}+\sum_{j=0}^k w_jf(u^{(n-j)},t^{(n-j)}) \end{equation}
%
% \begin{ex} Adams-Bashforth de segunda ordem
% \begin{equation} u^{(n+1)}=u^{(n)}+\frac{h}{2}\left[3f\left(u^{(n)},t^{(n)}\right)-f\left(u^{(n-1)},t^{(n-1)}\right)\right] \end{equation}
% \end{ex}
%
% \begin{ex} Adams-Bashforth de terceira ordem
% \begin{equation} u^{(n+1)}=u^{(n)}+\frac{h}{12}\left[23f\left(u^{(n)},t^{(n)}\right)-16f\left(u^{(n-1)},t^{(n-1)}\right)+5f\left(u^{(n-2)},t^{(n-2)}\right)\right] \end{equation}
% \end{ex}
%
% \begin{ex} Adams-Bashforth de quarta ordem
%   \begin{equation}
%     \begin{split}
%       u^{(n+1)} &=&u^{(n)} + \frac{h}{24}\left[55f\left(u^{(n)},t^{(n)}\right)-59f\left(u^{(n-1)},t^{(n-1)}\right)\right.\\
%         &+\left. 37f\left(u^{(n-2)},t^{(n-2)}\right)-9f\left(u^{(n-3)},t^{(n-3)}\right)\right]
%     \end{split}
%   \end{equation}
% \end{ex}
% Os métodos de passo múltiplo evitam os múltiplos estágios do métodos de Runge-Kutta, mas exigem ser "iniciados" com suas condições iniciais.
%
% \section{Métodos de passo múltiplo - Adams-Moulton}\index{método de passo múltiplo!Adams-Moulton}
%
% O método de Adams-Moulton consiste de um esquema recursivo do tipo:
% \begin{equation} u^{(n+1)}=u^{(n)}+\sum_{j=-1}^k w_jf(u^{(n-j)},t^{(n-j)}) \end{equation}
%
% \begin{ex} Adams-Moulton de quarta ordem
%   \begin{equation}
%     \begin{split}
%       u^{(n+1)} &=&u^{(n)} + \frac{h}{24}\left[9f\left(u^{(n+1)},t^{(n+1)}\right) + 19f\left(u^{(n)},t^{(n)}\right) \right.\\
%       &-\left. 5f\left(u^{(n-1)},t^{(n-1)}\right) + f\left(u^{(n-2)},t^{(n-2)}\right)\right]
%     \end{split}
%   \end{equation}
% \end{ex}
% O método de Adams-Moulton é implícito, ou seja, exige que a cada passo, uma equação em $u^{(n+1)}$ seja resolvida.
%
% \section{Estabilidade}\index{estabilidade}
%
% Consideremos o seguinte problema de teste:
% \begin{equation} \left\{\begin{array}{rcl}u'&=&-\alpha u\\u(0)&=&1\end{array}\right. \end{equation}
% cuja solução exata é dada por $u(t)=e^{-\alpha t}$.
%
% Considere agora o método de Euler aplicado a este problema com passa $h$:
% \begin{equation} \left\{\begin{array}{rcl}u_{k+1}&=&u_k-\alpha h u_k\\u^{(1)}&=&1\end{array}\right. \end{equation}
% A solução exata do esquema de Euler é dada por
% \begin{equation} u_{k+1}=(1-\alpha h)^{k} \end{equation}
% e, portanto,
% \begin{equation} \tilde{u}(t)=u_{k+1}=(1-\alpha h)^{t/h} \end{equation}
%
% Fixamos um $\alpha>0$, de forma que $u(t)\to 0$. Mas observamos que $\tilde{u}(t)\to 0$ somente quando $|1-\alpha h|<1$ e solução positivas somente quando $\alpha h<1$.
%
% {\bf Conclusão:} Se o passo $h$ for muito grande, o método pode se tornar instável, produzindo solução espúrias.
%
%




\section{Métodos de Runge-Kutta}%\label{pvi:sec_RK}


\subsection{Método de Runge-Kutta implícito (IRK)}
No conjunto de equações \eqref{RKa}-\eqref{RK}, $U_k$ depende em valores conhecidos $F_1,\ldots ,F_{k-1}$ tornando o método explícito.

Entretanto se $U_k$ depender de $F_1,\ldots ,F_\nu $ temos um método implícito como
\begin{eqnarray}\label{IRK}
  U_j &=&u^{(n)}  + h \sum_{i=1}^\nu  a_{ji} F_i, \quad  j=1,\ldots ,\nu \\
  u_{n+1}&=&u^{(n)}  + h \sum_{i=1}^{\nu } b_i F_i
\end{eqnarray}
onde $A=(a_{ij})$ é a matriz de RK. É necessário que
\begin{eqnarray}
 \sum_{i=1}^{\nu } a_{ji} = c_j, \quad \quad  j=1,\ldots ,\nu
\end{eqnarray}
para que o método possua ordem $p\geq 1$.


\begin{ex}
Um método de Runge-Kutta Implícito (IRK) de dois estágios é dado por
\begin{eqnarray}
  U_1 &=&u^{(n)}  + h/4  [ f(t^{(n)},U_1) - f(t^{(n)}+\frac{2}{3}h,U_2)]\\
  U_2 &=&u^{(n)}  + h/12 [3f(t^{(n)},U_1) +5f(t^{(n)}+\frac{2}{3}h,U_2)]\\
  u_{n+1}&=&u^{(n)}  + h/4 [f(t^{(n)},U_1) +3f(t^{(n)}+\frac{2}{3}h,U_2)]
\end{eqnarray}
que possui uma tabela como
\begin{center}
\begin{tabular}{c|cc}
  $0$ & $\frac{1}{4}$ &$-\frac{1}{4}$  \\
  $\frac{2}{3}$ & $\frac{1}{4}$ &$\frac{5}{12}$  \\  \hline
      & $\frac{1}{4}$ &$\frac{3}{4}$
\end{tabular}
\end{center}
\end{ex}




\section{Estimativa da ordem de convergência}

Raramente temos a solução exata $u(t)$ para calcular o erro obtido na solução numérica. Entretanto, se a solução é suave o suficiente e o espaçamento $h$ é pequeno suficientemente, podemos usar o seguinte procedimento para estimar a ordem do método (ou ainda, o erro na solução).

Como visto nos exemplo numéricos anteriores, em gráficos na escala \verb#loglog#, se $h$ é grande não obtemos a ordem de convergência utilizada (por exemplo, encontramos que o método de Euler possui ordem $p\approx 0,7$ onde deveria ser $1$). A medida que $h$ decresce se aproximando de $0$, a ordem de convergência tende a se aproximar de $p\approx 1$. (Entretanto $h$ não pode ficar muito pequeno a ponto que as operações de ponto flutuante atrapalhem na convergência).

Portanto existe uma faixa $h_{min} < h < h_{max}$ onde o método apresenta a ordem desejada. Essa região depende do método e do problema de valor inicial estudado.

Mas se estivermos nessa região podemos aproximar a ordem do método da seguinte forma: Considere a solução para um determinado $t=T^*$ fixo, $u(T^*)$. Considere também as aproximações das soluções obtidas com espaçamento $h$, denotada por $u^{h}$; a aproximação obtida com espaçamento dividido por $2$, $h/2$, denotada por $u^{h/2}$; a aproximação obtida com espaçamento $h/4$, denotada por $u^{h/4}$, $\ldots $ e assim por diante, todas calculadas em $t=T^*$.

\subsection{Método 1}
Podemos utilizar uma solução bem refinada, por exemplo, $u^{h/16}$ como sendo uma boa aproximação da solução exata e supormos que $u^*=u^{h/16}$. Desta forma podemos aproximar o erro por $e^{h}=\|u^{(h)}-u^*\|$ e a ordem do método é estimada como
\begin{eqnarray}
  p  & \approx  \frac{ \log(e^{h})-\log(e^{h/2})}{\log(h)-\log(h/2)} \\
     & \approx  \frac{ \log \left(   \frac{e^{h}}{e^{h/2}} \right)  }{\log(h /(h/2))} \\
     & \approx  \frac{ \log \left(   \frac{e^{h}}{e^{h/2}} \right)  }{\log(2)} \\
     & \approx  \frac{ \log \left(   \frac{\|u^{h}-u^*\|}{\|u^{h/2}-u^*\|} \right)  }{\log(2)} \\
\end{eqnarray}

\subsection{Método 2}
Segundo Ferziger/Peric/Roache, podemos também estimar $p$ diretamente de
\begin{eqnarray}
  p  & \approx  \frac{ \log \left(   \frac{\|u^{h/2}-u^{h}\|}{\|u^{h/4}-u^{h/2}\|} \right)  }{\log(2)} \\
\end{eqnarray}




%\chapter{Método de Colocação}
%Páginas 42-47 de \cite{iserles2009first}. %Todo método de colocação é um método de RK, mas a recíproca não é verdadeira.









\subsection*{Exercícios}

\begin{exer} Resolva o problema 1 pelos diversos métodos e verifique heuristicamente a estabilidade para diversos valores de $h$.
\end{exer}












\section{Exercícios finais}

\begin{exer} Considere o seguinte modelo para o crescimento de uma colônia de bactérias:
\begin{equation} \frac{du}{dt}=\alpha u (A-u) \end{equation}
onde $u$ indica a densidade de bactérias em unidades arbitrárias na colônia e $\alpha$ e $A$ são constantes positivas.
Pergunta-se:
\begin{itemize}
\item[a)] Qual a solução quando a condição inicial $u(0)$ é igual a $0$ ou $A$?
\item[b)] O que acontece quando a condição inicial $u(0)$ é um número entre $0$ e $A$?
\item[c)] O que acontece quando a condição inicial $u(0)$ é um número negativo?
\item[d)] O que acontece quando a condição inicial $u(0)$ é um número positivo maior que A?
\item[e)] Se $A=10$ e $\alpha=1$ e $u(0)=1$, use métodos numéricos para obter tempo necessário para que a população dobre?
\item[f)] Se $A=10$ e $\alpha=1$ e $u(0)=4$, use métodos numéricos para obter tempo necessário para que a população dobre?
\end{itemize}
\end{exer}
\begin{resp}

Os valores exatos para os itens e e f são:$\frac{1}{10}\ln\left(\frac{9}{4}\right)$ e $\frac{1}{10}\ln\left(6\right)$

\end{resp}

\begin{exer} Considere o seguinte modelo para a evolução da velocidade de um objeto em queda (unidades no SI):
\begin{equation} v'=g-\alpha v^2 \end{equation}
Sabendo que $g=9,8$ e $\alpha=10^{-2}$ e $v(0)=0$. Pede-se a velocidade ao tocar o solo, sabendo que a altura inicial era 100.

\end{exer}
\begin{resp}

O valor exato é $\sqrt{\frac{g}{\alpha}\left[1-e^{{-200\alpha}}\right]}$ em $t=\frac{1}{\sqrt{g\alpha}}\tanh^{-1}\left(\sqrt{1-e^{{-200\alpha}}}\right)$

\end{resp}


\begin{exer} Considere o seguinte modelo para o oscilador não linear de Van der Pol:
\begin{equation} u''(t) - \alpha (A-u(t)^2)u'(t) + w_0^2u(t)=0 \end{equation}
onde $A$, $\alpha$ e $w_0$ são constantes positivas.
\begin{itemize}
\item Encontre a frequência e a amplitude de oscilações quando $w_0=1$, $\alpha=.1$ e $A=10$. (Teste diversas condições iniciais)
\item Estude a dependência da frequência e da amplitude com os parâmetros  $A$, $\alpha$ e $w_0$. (Teste diversas condições iniciais)
\item Que diferenças existem entre esse oscilador não linear e o oscilador linear?
\end{itemize}
\end{exer}

\begin{exer} Considere o seguinte modelo para um oscilador não linear:
\begin{eqnarray}
u''(t)-\alpha(A-z(t))u'(t)+w_0^2 u(t)&=&0\\
Cz'(t)+z(t)&=&u(t)^2
\end{eqnarray}
onde $A$, $\alpha$, $w_0$ e $C$ são constantes positivas.
\begin{itemize}
\item Encontre a frequência e a amplitude de oscilações quando $w_0=1$, $\alpha=.1$, $A=10$ e $C=10$. (Teste diversas condições iniciais)
\item Estude a dependência da frequência e da amplitude com os parâmetros  $A$, $\alpha$, $w_0$ e $C$. (Teste diversas condições iniciais)
\end{itemize}
\end{exer}

\begin{exer} Considere o seguinte modelo para o controle de temperatura em um processo químico:
\begin{eqnarray}
CT'(t)+T(t)&=&\kappa P(t)+T_{ext}\\
P'(t)&=&\alpha(T_{set}-T(t))
\end{eqnarray}
onde $C$, $\alpha$ e $\kappa$ são constantes positivas e $P(t)$ indica o potência do aquecedor. Sabendo que $T_{set}$ é a temperatura desejada, interprete o funcionamento esse sistema de controle.
\begin{itemize}
\item Calcule a solução quando a temperatura externa $T_{ext}=0$, $T_{set}=1000$, $C=10$, $\kappa=.1$ e $\alpha=.1$. Considere condições iniciais nulas.
\item Quanto tempo demora o sistema para atingir a temperatura 900K?
\item Refaça os dois primeiros itens com $\alpha=0,2$ e $\alpha=1$
\item Faça testes para verificar a influência de $T_{ext}$, $\alpha$ e $\kappa$ na temperatura final.
\end{itemize}
\end{exer}

\begin{exer} Considere a equação do pêndulo dada por:
\begin{equation} \frac{d^2\theta(t)}{dt^2}+\frac{g}{l}\sin(\theta(t))=0 \end{equation}
onde $g$ é o módulo da aceleração da gravidade e $l$ é o comprimento da haste.
\begin{itemize}
\item Mostre analiticamente que a energia total do sistema dada por
\begin{equation} \frac{1}{2}\left(\frac{d\theta(t)}{dt}\right)^2-\frac{g}{l}\cos(\theta(t)) \end{equation}
é mantida constante.
\item Resolva numericamente esta equação para $g=9,8m/s^2$ e $l=1m$ e as seguintes condições iniciais:
\subitem $\theta(0)=0,5$ e $\theta'(0)=0$.
\subitem $\theta(0)=1,0$ e $\theta'(0)=0$.
\subitem $\theta(0)=1,5$ e $\theta'(0)=0$.
\subitem $\theta(0)=2,0$ e $\theta'(0)=0$.
\subitem $\theta(0)=2,5$ e $\theta'(0)=0$.
\subitem $\theta(0)=3,0$ e $\theta'(0)=0$.
\end{itemize}
Em todos os casos, verifique se o método numérico reproduz a lei de conservação de energia e calcule período e amplitude.
\end{exer}

\begin{exer} Considere o modelo simplificado de FitzHugh-Nagumo para o potencial elétrico sobre a membrana de um neurônio:
\begin{eqnarray}
\frac{d V}{dt}& = &  V-V^3/3 - W +  I  \\
\frac{d W}{dt} & = & 0,08(V+0,7 - 0,8W)
\end{eqnarray}
onde $I$ é a corrente de excitação.
\begin{itemize}
\item Encontre o único estado estacionário $\left(V_0,W_0\right)$ com $I=0$.
\item Resolva numericamente o sistema com condições iniciais dadas por $\left(V_0,W_0\right)$ e
\subitem $I=0$
\subitem $I=0,2$
\subitem $I=0,4$
\subitem $I=0,8$
\subitem $I=e^{-t/200}$
\end{itemize}
\end{exer}


\begin{exer} Considere o problema de valor inicial dado por
\begin{eqnarray}
\frac{d u(t)}{dt} &=& -u(t) + e^{-t} \\
u(0)&=&0
\end{eqnarray}
Resolva analiticamente este problema usando as técnicas elementares de equações diferenciais ordinárias. A seguir encontre aproximações numéricas usando os métodos de Euler, Euler modificado, Runge-Kutta clássico e Adams-Bashforth de ordem 4 conforme pedido nos itens.
\begin{itemize}
\item[a)]  Construa uma tabela apresentando valores com 7 algarismos significativos para comparar a solução analítica com as aproximações numéricas produzidas pelos métodos sugeridos. Construa também uma tabela para o erro absoluto obtido por cada método numérico em relação à solução analítica. Nesta última tabela, expresse o erro com 2 algarismos significativos em formato científico. Dica: $format('e',8)$ para a segunda tabela.
\begin{center}
\begin{tabular}{|c|c|c|c|c|c|}
\hline
&0,5&1,0&1,5&2,0&2,5\\
\hline
Analítico&&&&&\\
\hline
Euler&&&&&\\
\hline
Euler modificado&&&&&\\
\hline
Runge-Kutta clássico&&&&&\\
\hline
Adams-Bashforth ordem 4&&&&&\\
\hline
\end{tabular}
\end{center}

\begin{center}
\begin{tabular}{|c|c|c|c|c|c|}
\hline
&0,5&1,0&1,5&2,0&2,5\\
\hline
Euler&&&&&\\
\hline
Euler modificado&&&&&\\
\hline
Runge-Kutta clássico&&&&&\\
\hline
Adams-Bashforth ordem 4&&&&&\\
\hline
\end{tabular}
\end{center}

\item[b)] Calcule o valor produzido por cada um desses método para $u(1)$ com passo $h=0,1$, $h=0,05$, $h=0,01$, $h=0,005$ e $h=0,001$. Complete a tabela com os valores para o erro absoluto encontrado.
\begin{center}
\begin{tabular}{|c|c|c|c|c|c|}
\hline
&0,1&0,05&0,01&0,005&0,001\\
\hline
Euler&&&&&\\
\hline
Euler modificado&&&&&   \\
\hline
Runge-Kutta clássico&&&&&\\
\hline
Adams-Bashforth ordem 4&&&&&\\
\hline
\end{tabular}
\end{center}

\end{itemize}

\end{exer}


\begin{resp}

\begin{center}
\begin{tabular}{|c|c|c|c|c|c|}
\hline
&0,5&1,0&1,5&2,0&2,5\\
\hline
Analítico&  0,3032653 &   0,3678794  &  0,3346952  &  0,2706706 &   0,2052125  \\
\hline
Euler& 0,3315955 &   0,3969266 &   0,3563684 &   0,2844209  &  0,2128243\\
\hline
Euler modificado &0,3025634 &   0,3671929 &   0,3342207 &   0,2704083  &  0,2051058 \\
\hline
Runge-Kutta clássico& 0,3032649  &  0,3678790  &  0,3346949  &  0,2706703  &  0,2052124\\
\hline
Adams-Bashforth ordem 4& 0,3032421  &  0,3678319 &   0,3346486  &  0,2706329  &  0,2051848  \\
\hline
\end{tabular}
\end{center}


\begin{center}
\begin{tabular}{|c|c|c|c|c|c|}
\hline
&0,5&1,0&1,5&2,0&2,5\\
\hline
Euler& 2,8e-2  &  2,9e-2  &  2,2e-2  &  1,4e-2 &   7,6e-3\\
\hline
Euler modificado& 7,0e-4  &  6,9e-4   & 4,7e-4 &   2,6e-4 &   1,1e-4\\
\hline
Runge-Kutta clássico& 4,6e-7 &   4,7e-7    &3,5e-7  &  2,2e-7 &   1,2e-7\\
\hline
Adams-Bashforth ordem 4&  2,3e-5 &   4,8e-5  &  4,7e-5  &  3,8e-5  &  2,8e-5 \\
\hline
\end{tabular}
\end{center}

\begin{center}
\begin{tabular}{|c|c|c|c|c|c|}
\hline
&0,1&0,05&0,01&0,005&0,001\\
\hline
Euler&2,9e-2  &  5,6e-3 &   2,8e-3 &   5,5e-4 &   2,8e-4\\
\hline
Euler modificado&6,9e-4 &   2,5e-5  &  6,2e-6 &   2,5e-7 &   6,1e-8   \\
\hline
Runge-Kutta clássico& 4,7e-7 &   6,9e-10 &   4,3e-11   & 6,8e-14  &  4,4e-15\\
\hline
Adams-Bashforth ordem 4&4,8e-5 &   9,0e-8 &   5,7e-9 &   9,2e-12 &   5,8e-13  \\
\hline
\end{tabular}
\end{center}

\end{resp}

%\end{document}


