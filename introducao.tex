Diversos problemas de interesse na engenharia e matemática aplicada, como  o cálculo de raízes de equações algébricas, soluções de equações diferenciais ou cálculo de integrais, podem se mostrar intratáveis se dispomos apenas de meios puramente analíticos, como aqueles estudados nos cursos de cálculo e álgebra linear. Por exemplo, o teorema de Abel-Ruffini nos garante que não existe uma fórmula algébrica, isto é, envolvendo apenas operações artiméticas e radicais, para calcular as raízes de uma equação polinomial de qualquer grau, mas apenas casos particulares:
\begin{itemize}
\item Simplesmente isolar a incógnita para encontrar a raíz de uma equação do primeiro grau;
\item Fórmula de Báskara para encontrar raízes de uma equação do segundo grau;
\item Fórmula de Cardano para encontrar raízes de uma equação do terceiro grau;
\item Existe expressão para equações de quarto grau;
\item Casos simplificados de equações de grau maior que 4 onde alguns coeficientes são nulos também podem ser resolvidos.
\end{itemize}
Equações não polinomiais podem ser ainda mais complicadas de resolver exatamente, por exemplo
$$
\cos(x)=x\qquad \hbox{e}\qquad xe^x= 10
$$

Para resolver o problema de valor inicial 
$$\left\{
\begin{array}{l}
y'+xy=x,\\
y(0)=2,
\end{array}\right.
$$
podemos usar o método de fator integrante e obtemos $y=1+e^{-x^2/2}$. Já o cálculo da solução exata para o problema 
$$\left\{
\begin{array}{l}
y'+xy=e^{-y},\\
y(0)=2,
\end{array}\right.
$$
não é possível.

Da mesma forma, resolvemos a integral
$$
\int_1^2xe^{-x^2}dx
$$
pelo método da substituição e obtemos $\frac{1}{2}(e^{-1}-e^{-2})$. Porém a integral
$$
\int_1^2 e^{-x^2} dx
$$
não pode ser resolvida analiticamente.

A maioria das modelagem de fenômenos reais chegam em problemas matemáticos onde a solução analítica é difícil (ou impossível) de ser encontrada, mesmo quando provamos que ela existe. Nesse curso propomos calcular aproximações numéricas para esses problemas, que apesar de, em geral, serem diferentes da solução exata, mostraremos que elas podem ser bem próximas.

Para entender a construção de aproximações é necessário estudar um pouco como funciona a aritmética de computador e erros de arredondamento. Como computadores, em geral, usam uma base binária para representar números, começaremos falando em mudança de base.
