%Este trabalho está licenciado sob a Licença Creative Commons Atribuição-CompartilhaIgual 3.0 Não Adaptada. Para ver uma cópia desta licença, visite http://creativecommons.org/licenses/by-sa/3.0/ ou envie uma carta para Creative Commons, PO Box 1866, Mountain View, CA 94042, USA.

\chapter*{Prefácio}
\addcontentsline{toc}{chapter}{Prefácio}

Este livro busca abordar os tópicos de um curso de introdução ao cálculo numérico moderno oferecido a estudantes de matemática, física, engenharias e outros. A ênfase é colocada na formulação de problemas, implementação em computador da resolução e interpretação de resultados. Pressupõe-se que o estudante domine conhecimentos e habilidades típicas desenvolvidas em cursos de graduação de cálculo, álgebra linear e equações diferenciais. Conhecimentos prévios em linguagem de computadores é fortemente recomendável, embora apenas técnicas elementares de programação sejam realmente necessárias.

%%%%%%%%%%%%%%%%%%%%
% scilab
%%%%%%%%%%%%%%%%%%%%
\ifisscilab
Nesta versão do livro, fazemos ênfase na utilização do \emph{software} livre \href{http://www.scilab.org/}{Scilab} para a implementação dos métodos numéricos abordados. Recomendamos ao leitor ter à sua disposição um computador com o \verb+Scilab+ instalado. Não é necessário estar familiarizado com esta linguagem, mas recomendamos a leitura do Apêndice~\ref{cap:scilab}, no qual apresentamos uma rápida introdução a este pacote computacional. Alternativamente, existem algumas soluções em nuvem que fornecem acesso ao \verb+Scilab+ via internet. Por exemplo, a plataforma virtual rollApp (\url{https://www.rollapp.com/app/scilab}) ou o Scilab on Cloud (\url{http://cloud.scilab.in/}).
\fi
%%%%%%%%%%%%%%%%%%%%
%%%%%%%%%%%%%%%%%%%%
% octave
%%%%%%%%%%%%%%%%%%%%
\ifisoctave
Nesta versão do livro, fazemos ênfase na utilização da linguagem computacional \href{https://www.gnu.org/software/octave/}{GNU Octave} para a implementação dos métodos numéricos abordados. Recomendamos ao leitor ter à sua disposição um computador com o \verb+GNU Octave+ (versão 4.0 ou superior) instalado. Não é necessário estar familiarizado com esta linguagem, mas recomendamos a leitura do Apêndice~\ref{cap:octave}, no qual apresentamos uma rápida introdução a esta linguagem com ênfase naquilo que é mais essencial para a leitura do livro. Alternativamente, existem algumas soluções em nuvem que fornecem acesso a consoles {\it online}. Veja, por exemplo, o \href{https://cocalc.com}{CoCalc} e o \href{https://octave-online.net/}{Octave Online}.
\fi
%%%%%%%%%%%%%%%%%%%%
%%%%%%%%%%%%%%%%%%%%
% python
%%%%%%%%%%%%%%%%%%%%
\ifispython
Nesta versão do livro, fazemos ênfase na utilização da linguagem computacional \href{https://www.python.org/}{Python} para a implementação dos métodos numéricos abordados. Recomendamos ao leitor ter à sua disposição um computador com o interpretador \verb+Python 2.7+ (ou superior) e o conjunto de biblioteca \href{https://www.scipy.org/}{SciPy} instalados. Não é necessário estar familiarizado com esta linguagem, mas recomendamos a leitura do Apêndice~\ref{cap:python}, no qual apresentamos uma rápida introdução a esta linguagem com ênfase naquilo que é mais essencial para a leitura do livro. Alternativamente, existem algumas soluções em nuvem que fornecem acesso a consoles {\it online} \href{https://www.python.org/}{Python}. Veja, por exemplo, o \href{https://cocalc.com}{CoCalc}.
\fi
%%%%%%%%%%%%%%%%%%%%

Os códigos computacionais dos métodos numéricos apresentados no livro são implementados em uma abordagem didática. Isto é, temos o objetivo de que a implementação em linguagem computacional venha a auxiliar o leitor no aprendizado das técnicas numéricas que são apresentadas no livro. Implementações computacionais eficientes de técnicas de cálculo numérico podem ser obtidas na série de livros ``Numerical Recipes'', veja \cite{numerical}.


