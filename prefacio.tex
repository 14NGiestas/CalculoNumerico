%Este trabalho está licenciado sob a Licença Creative Commons Atribuição-CompartilhaIgual 3.0 Não Adaptada. Para ver uma cópia desta licença, visite http://creativecommons.org/licenses/by-sa/3.0/ ou envie uma carta para Creative Commons, PO Box 1866, Mountain View, CA 94042, USA.

\chapter*{Prefácio}
\addcontentsline{toc}{chapter}{Prefácio}

Este livro busca abordar os tópicos de um curso de introdução ao cálculo numérico moderno oferecido a estudantes de matemática, física, engenharias e outros. A ênfase é colocada na formulação de problemas, implementação em computador da resolução e interpretação de resultados. Pressupõe-se que o estudante domine conhecimentos e habilidades típicas desenvolvidas em cursos de graduação de cálculo, álgebra linear e equações diferenciais. Conhecimentos prévios em linguagem de computadores é fortemente recomendável, embora apenas técnicas elementares de programação sejam realmente necessárias.

\ifisscilab
Nesta versão do livro, fazemos ênfase na utilização do \emph{software} livre \verb+Scilab+ para a implementação dos métodos numéricos abordados. Recomendamos ao leitor ter à sua disposição um computador com o \verb+Scilab+ instalado. Não é necessário estar familiarizado com esta linguagem, mas recomendamos a leitura do Apêndice~\ref{cap:scilab}, no qual apresentamos uma rápida introdução a este pacote computacional. Alternativamente, existem algumas soluções em nuvem que fornecem acesso ao \verb+Scilab+ via internet. Por exemplo, a plataforma virtual rollApp (\url{https://www.rollapp.com/app/scilab}) ou o Scilab on Cloud (\url{http://cloud.scilab.in/}).
\fi

