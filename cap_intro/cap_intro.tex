%Este trabalho está licenciado sob a Licença Creative Commons Atribuição-CompartilhaIgual 3.0 Não Adaptada. Para ver uma cópia desta licença, visite http://creativecommons.org/licenses/by-sa/3.0/ ou envie uma carta para Creative Commons, PO Box 1866, Mountain View, CA 94042, USA.

%\documentclass[main.tex]{subfiles}
%\begin{document}

\chapter{Introdução}

Cálculo numérico é a disciplina que estuda as técnicas para a solução aproximada de problemas matemáticos. Estas técnicas são de natureza analítica e computacional. As principais preocupações normalmente envolvem exatidão e perfórmance. 

Aliado ao aumento contínuo da capacidade de computação disponível, o desenvolvimento de métodos numéricos tornou a simulação computacional\index{simulação!computacional} de modelos matemáticos uma prática usual nas mais diversas áreas científicas e tecnológicas. As então chamadas simulações numéricas\index{simulação!numérica} são constituídas de um arranjo de vários esquemas numéricos dedicados a resolver problemas específicos como, por exemplo: resolver equações algébricas, resolver sistemas lineares, interpolar e ajustar pontos, calcular derivadas e integrais, resolver equações diferenciais ordinárias, etc.. Neste livro, abordamos o desenvolvimento, a implementação, utilização e aspectos teóricos de métodos numéricos para a resolução desses problemas.

Os problemas que discutiremos não formam apenas um conjunto de métodos fundamentais, mas são, também, problemas de interesse na engenharia, na física e na matemática aplicada. A necessidade de aplicar aproximações numéricas decorre do fato de que esses problemas podem se mostrar intratáveis se dispomos apenas de meios puramente analíticos, como aqueles estudados nos cursos de cálculo e álgebra linear. Por exemplo, o teorema de Abel-Ruffini nos garante que não existe uma fórmula algébrica, isto é, envolvendo apenas operações aritméticas e radicais, para calcular as raízes de uma equação polinomial de qualquer grau, mas apenas casos particulares:
\begin{itemize}
\item Simplesmente isolar a incógnita para encontrar a raiz de uma equação do primeiro grau;
\item Fórmula de Bhaskara para encontrar raízes de uma equação do segundo grau;
\item Fórmula de Cardano para encontrar raízes de uma equação do terceiro grau;
\item Existe expressão para equações de quarto grau;
\item Casos simplificados de equações de grau maior que 4 onde alguns coeficientes são nulos também podem ser resolvidos.
\end{itemize}
Equações não polinomiais podem ser ainda mais complicadas de resolver exatamente, por exemplo:
$$
\cos(x)=x\qquad \hbox{e}\qquad xe^x= 10
$$

Para resolver o problema de valor inicial 
$$
\begin{array}{l}
y'+xy=x,\\
y(0)=2,
\end{array}
$$
podemos usar o método de fator integrante e obtemos $y=1+e^{-x^2/2}$. Já o cálculo da solução exata para o problema 
$$
\begin{array}{l}
y'+xy=e^{-y},\\
y(0)=2,
\end{array}
$$
não é possível.

Da mesma forma, resolvemos a integral
$$
\int_1^2xe^{-x^2}dx
$$
pelo método da substituição e obtemos $\frac{1}{2}(e^{-1}-e^{-2})$. Porém a integral
$$
\int_1^2 e^{-x^2} dx
$$
não pode ser resolvida analiticamente.

A maioria dos modelos de fenômenos reais chegam em problemas matemáticos onde a solução analítica é difícil (ou impossível) de ser encontrada, mesmo quando provamos que ela existe. Nesse curso propomos calcular aproximações numéricas para esses problemas, que apesar de, em geral, serem diferentes da solução exata, mostraremos que elas podem ser bem próximas.

Para entender a construção de aproximações é necessário estudar um pouco como funciona a aritmética de computador e erros de arredondamento. Como computadores, em geral, usam uma base binária para representar números, começaremos falando em mudança de base.


%\end{document}
